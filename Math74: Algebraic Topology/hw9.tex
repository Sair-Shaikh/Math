\documentclass[12pt]{article}

\usepackage{fullpage}
\usepackage{mdframed}
\usepackage{colonequals}
\usepackage{algpseudocode}
\usepackage{algorithm}
\usepackage{tcolorbox}
\usepackage[all]{xy}
\usepackage{proof}
\usepackage{mathtools}
\usepackage{bbm}
\usepackage{amssymb}
\usepackage{amsthm}
\usepackage{amsmath}
\usepackage{amsxtra}
\newcommand{\bb}{\mathbb}


\newtheorem{theorem}{Theorem}[section]
\newtheorem{corollary}{Corollary}[theorem]
\newtheorem{lemma}{Lemma}

\newcommand{\mathcat}[1]{\textup{\textbf{\textsf{#1}}}} % for defined terms

\newenvironment{problem}[1]
{\begin{tcolorbox}\noindent\textbf{Problem #1}.}
{\vskip 6pt \end{tcolorbox}}

\newenvironment{enumalph}
{\begin{enumerate}\renewcommand{\labelenumi}{\textnormal{(\alph{enumi})}}}
{\end{enumerate}}

\newenvironment{enumroman}
{\begin{enumerate}\renewcommand{\labelenumi}{\textnormal{(\roman{enumi})}}}
{\end{enumerate}}

\newcommand{\defi}[1]{\textsf{#1}} % for defined terms

\theoremstyle{remark}
\newtheorem*{solution}{Solution}

\setlength{\hfuzz}{4pt}

\newcommand{\calC}{\mathcal{C}}
\newcommand{\calF}{\mathcal{F}}
\newcommand{\C}{\mathbb C}
\newcommand{\N}{\mathbb N}
\newcommand{\Q}{\mathbb Q}
\newcommand{\R}{\mathbb R}
\newcommand{\Z}{\mathbb Z}
\newcommand{\br}{\mathbf{r}}
\newcommand{\RP}{\mathbb{RP}}
\newcommand{\CP}{\mathbb{CP}}
\newcommand{\nbit}[1]{\{0, 1\}^{#1}}
\newcommand{\bits}{\{0, 1\}^{n}}
\newcommand{\bbni}{\bigbreak \noindent}
\newcommand{\norm}[1]{\left\vert\left\vert#1\right\vert\right\vert}

\let\1\relax
\newcommand{\1}{\mathbf{1}}
\newcommand{\fr}[2]{\left(\frac{#1}{#2}\right)}

\newcommand{\vecz}{\mathbf{z}}
\newcommand{\vecr}{\mathbf{r}}
\DeclareMathOperator{\Cinf}{C^{\infty}}
\DeclareMathOperator{\Id}{Id}

\DeclareMathOperator{\Alt}{Alt}
\DeclareMathOperator{\ann}{ann}
\DeclareMathOperator{\codim}{codim}
\DeclareMathOperator{\End}{End}
\DeclareMathOperator{\Hom}{Hom}
\DeclareMathOperator{\id}{id}
\DeclareMathOperator{\M}{M}
\DeclareMathOperator{\Mat}{Mat}
\DeclareMathOperator{\Ob}{Ob}
\DeclareMathOperator{\opchar}{char}
\DeclareMathOperator{\opspan}{span}
\DeclareMathOperator{\rk}{rk}
\DeclareMathOperator{\sgn}{sgn}
\DeclareMathOperator{\Sym}{Sym}
\DeclareMathOperator{\tr}{tr}
\DeclareMathOperator{\img}{img}
\DeclareMathOperator{\CandE}{CandE}
\DeclareMathOperator{\CandO}{CandO}
\DeclareMathOperator{\argmax}{argmax}
\DeclareMathOperator{\first}{first}
\DeclareMathOperator{\last}{last}
\DeclareMathOperator{\cost}{cost}
\DeclareMathOperator{\dist}{dist}
\DeclareMathOperator{\path}{path}
\DeclareMathOperator{\parent}{parent}
\DeclareMathOperator{\argmin}{argmin}
\DeclareMathOperator{\excess}{excess}
\let\Pr\relax
\DeclareMathOperator{\Pr}{\mathbf{Pr}}
\DeclareMathOperator{\Exp}{\mathbb{E}}
\DeclareMathOperator{\Var}{\mathbf{Var}}
\let\limsup\relax
\DeclareMathOperator{\limsup}{limsup}
%Paired Delims
\DeclarePairedDelimiter\ceil{\lceil}{\rceil}
\DeclarePairedDelimiter\floor{\lfloor}{ \rfloor}


\newcommand{\dagstar}{*}

\newcommand{\tbigwedge}{{\textstyle{\bigwedge}}}
\setlength{\parindent}{0pt}
\setlength{\parskip}{5pt}


\begin{document}

\title{CS 40: Computational Complexity}

\author{Sair Shaikh}
\maketitle

% Collaboration Notice: Talked to Henry Scheible '26 to discuss ideas.



\begin{problem}{1}(2.3.1) \\
    If $T_n(X,A)$ denotes the torsion subgroup of $H_n(X,A)$, show that the functors $(X,A) \mapsto T_n(X,A)$ with the obvious induced homomorphisms $T_n(X,A) \to T_n(Y,B)$ and boundary maps $T_n(X,A) \to T_{n-1}(A)$ do not satisfy a homology theory even if excluding the dimension axiom. Do the same for the `mod-torsion' functor $MT_n(X,A) = H_n(X,A)/T_n(X,A)$. 
\end{problem}

\begin{solution}
    Let $X = \RP^2$ and $A$ be a circle in $X$. The long exact sequence in homology gives us:
    \begin{align*}
        &\cdots \to H_2(X, A) \to H_1(A) \to H_1(X) \to H_1(X, A) \to H_0(A) \to H_0(X) \to \cdots \\
    \end{align*}
    Then, note that we have $H_1(X) = \Z/2\Z$, $H_1(A) = \Z$ and $H_0(X) = H_0(A) = \Z$.
    \[ \cdots \to H_2(X, A) \to \Z \to \Z/2\Z \to H_1(X, A) \to \Z \to \Z \to \cdots\]
    The generator of $H_1(A)$ maps to a boundary in $H_1(X)$, thus, the first map is $0$. Thus, the second map is injective. Moroever, the last map is induced by the inclusion of $A$ into $X$, both of which are path-connected, thus the last $H_0(A) \to H_0(X)$ is an isomorphism. Thus, the image of $H_1(X, A) \to H_0(A)$ is trivial, i.e. the map is $0$. Thus, $H_1(X, A) = 0$. Overall, we have:
    \[ \to \Z \to \Z/2\Z \to 0 \to \cdots\]
    Applying the torsion functors, we get:
    \[ T_1(A) = 0 \to T_1(X) = \Z/2\Z \to T_1(X,A) = 0 \to \cdots \]
    which is not exact. Thus, the torsion functor does not satisfy the exactness axiom.
\end{solution}
\newpage

\begin{problem}{2}(2.3.5, with $G= \Z$) 
    Regarding a cochain $\varphi \in C^1(X)$ as a function on paths in $X$ to $\Z$, show that if $\varphi$ is a cocycle, then 
    \begin{enumerate}
    \item $\varphi(f \cdot g) = \varphi(f) + \varphi(g)$,
    \item $\varphi$ takes the value $0$ on constant paths,
    \item $\varphi(f) = \varphi(g)$ if $f \simeq_p g$, and
    \item $\varphi$ is a coboundary if and only if $\varphi(f)$ depends only on the endpoints of $f$ for all paths $f$ in $X$.
    \end{enumerate}
\end{problem}

\begin{solution}

\end{solution}
\newpage

\begin{problem}{3}
Verify the remark in Hatcher after exercise 2.3.5: If $X$ is path-connected, the previous problem together with the universal coefficient theorem induces an isomorphism $H^1(X) \cong \mathrm{Hom}(\pi_1(X), \Z)$.
\end{problem}
\begin{solution}
\end{solution}
\newpage




\end{document}