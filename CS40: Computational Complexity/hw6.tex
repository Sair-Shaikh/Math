\documentclass[12pt]{article}


\usepackage{fullpage}
\usepackage{mdframed}
\usepackage{colonequals}
\usepackage{algpseudocode}
\usepackage{algorithm}
\usepackage{tcolorbox}
\usepackage[all]{xy}
\usepackage{proof}
\usepackage{mathtools}
\usepackage{bbm}
\usepackage{amssymb}
\usepackage{amsthm}
\usepackage{amsmath}
\usepackage{amsxtra}
\newcommand{\bb}{\mathbb}


\newtheorem{theorem}{Theorem}[section]
\newtheorem{corollary}{Corollary}[theorem]
\newtheorem{lemma}{Lemma}

\newcommand{\mathcat}[1]{\textup{\textbf{\textsf{#1}}}} % for defined terms

\newenvironment{problem}[1]
{\begin{tcolorbox}\noindent\textbf{Problem #1}.}
{\vskip 6pt \end{tcolorbox}}

\newenvironment{enumalph}
{\begin{enumerate}\renewcommand{\labelenumi}{\textnormal{(\alph{enumi})}}}
{\end{enumerate}}

\newenvironment{enumroman}
{\begin{enumerate}\renewcommand{\labelenumi}{\textnormal{(\roman{enumi})}}}
{\end{enumerate}}

\newcommand{\defi}[1]{\textsf{#1}} % for defined terms

\theoremstyle{remark}
\newtheorem*{solution}{Solution}

\setlength{\hfuzz}{4pt}

\newcommand{\calC}{\mathcal{C}}
\newcommand{\calF}{\mathcal{F}}
\newcommand{\C}{\mathbb C}
\newcommand{\N}{\mathbb N}
\newcommand{\Q}{\mathbb Q}
\newcommand{\R}{\mathbb R}
\newcommand{\Z}{\mathbb Z}
\newcommand{\F}{\mathbb F}
\newcommand{\br}{\mathbf{r}}
\newcommand{\RP}{\mathbb{RP}}
\newcommand{\CP}{\mathbb{CP}}
\newcommand{\nbit}[1]{\{0, 1\}^{#1}}
\newcommand{\bits}{\{0, 1\}^{n}}
\newcommand{\bbni}{\bigbreak \noindent}
\newcommand{\norm}[1]{\left\vert\left\vert#1\right\vert\right\vert}
\newcommand{\dbar}{\overline{\partial}}
\let\d\relax
\let\calF\relax
\newcommand{\d}{\partial}
\newcommand{\calO}{\mathcal{O}}
\newcommand{\calF}{\mathcal{F}}
\newcommand{\calG}{\mathcal{G}}
\newcommand{\calH}{\mathcal{H}}
\newcommand{\calE}{\mathcal{E}}

\let\1\relax
\newcommand{\1}{\mathbf{1}}
\newcommand{\fr}[2]{\left(\frac{#1}{#2}\right)}

\newcommand{\vecz}{\mathbf{z}}
\newcommand{\vecr}{\mathbf{r}}
\DeclareMathOperator{\Cinf}{C^{\infty}}
\DeclareMathOperator{\Id}{Id}

\DeclareMathOperator{\Alt}{Alt}
\DeclareMathOperator{\ann}{ann}
\DeclareMathOperator{\codim}{codim}
\DeclareMathOperator{\End}{End}
\DeclareMathOperator{\Hom}{Hom}
\DeclareMathOperator{\id}{id}
\DeclareMathOperator{\M}{M}
\DeclareMathOperator{\Mat}{Mat}
\DeclareMathOperator{\Ob}{Ob}
\DeclareMathOperator{\opchar}{char}
\DeclareMathOperator{\opspan}{span}
\DeclareMathOperator{\rk}{rk}
\DeclareMathOperator{\sgn}{sgn}
\DeclareMathOperator{\Sym}{Sym}
\DeclareMathOperator{\tr}{tr}
\DeclareMathOperator{\img}{img}
\DeclareMathOperator{\CandE}{CandE}
\DeclareMathOperator{\CandO}{CandO}
\DeclareMathOperator{\argmax}{argmax}
\DeclareMathOperator{\first}{first}
\DeclareMathOperator{\last}{last}
\DeclareMathOperator{\cost}{cost}
\DeclareMathOperator{\dist}{dist}
\DeclareMathOperator{\path}{path}
\DeclareMathOperator{\parent}{parent}
\DeclareMathOperator{\argmin}{argmin}
\DeclareMathOperator{\excess}{excess}
\let\Pr\relax
\DeclareMathOperator{\Pr}{\mathbf{Pr}}
\DeclareMathOperator{\Exp}{\mathbb{E}}
\DeclareMathOperator{\Var}{\mathbf{Var}}
\let\limsup\relax
\DeclareMathOperator{\limsup}{limsup}
%Paired Delims
\DeclarePairedDelimiter\ceil{\lceil}{\rceil}
\DeclarePairedDelimiter\floor{\lfloor}{ \rfloor}


\newcommand{\dagstar}{*}

\newcommand{\tbigwedge}{{\textstyle{\bigwedge}}}
\setlength{\parindent}{0pt}
\setlength{\parskip}{5pt}



\begin{document}


\title{CS 40: Computational Complexity}

\author{Sair Shaikh}
\maketitle

Collaboration Notice: Talked to Henry Scheible '26 to discuss ideas.



\begin{problem}{12}
Prove that \( \text{RP} \cap \text{coRP} = \text{ZPP} \) and that \( \text{RP} \cup \text{coRP} \subseteq \text{BPP} \).
\end{problem}

\begin{solution}
    We first show \( \text{RP} \cap \text{coRP} = \text{ZPP} \) by showing both containments. \bbni
    Let $L \in \text{RP} \cap \text{coRP}$. Then, using the definitions of $\text{RP}$ and $\text{coRP}$, there exist polynomial time bounded PTMs $M$ and $M'$ (corresponding to being in $\text{RP}$ and $\text{coRP}$ respectively) such that: 
    \begin{align*}
        x \in L &\implies \left(\Pr(M(x, r) \neq L(x)) \leq \frac{1}{2}\right) \land \left(\Pr(M'(x, r) \neq L(x)) = 0 \right) \\
        x \not \in L &\implies \left(\Pr(M(x, r) \neq L(x)) = 0 \right) \land \left(\Pr(M'(x, r) \neq L(x)) \leq \frac{1}{2}\right)
    \end{align*}
    Note that if $x \not \in L$, then $M$ is guaranteed to reject. By contrapositive, if $M$ accepts, we know that $x \in L$. Similarly, if $M'$ rejects, we know that $x \not \in L$. Based off of these observations, we construct a PTM $A$ as follows:
    \begin{itemize}
        \item Simulate $M$. If $M$ accepts, then accept. Otherwise, continue. 
        \item Simulate $M'$. If $M'$ rejects, then reject. Otherwise, continue.
        \item Repeat the last two steps with a different random seed (use further bits off the random tape).
    \end{itemize}
    We claim that $A$ is a zero-error PTM with expected run-time bounded by a polynomial. \bbni
    The zero-error claim is clear, as per our observation above. We only accept when $M$ accepts, which implies $x \in L$, and similarly reject when $M'$ rejects implying $x \not \in L$. Thus, we only need to analyze the time-bound. \bbni
    First, note that each iteration (running $M$ and $M'$ once each) runs in polynomial time. Concretely, let $i_1, i_2 \in \N$ be such that: 
    \[ \forall r \in \{0, 1\}^* : \text{TIMECOST}_{M}(n, r) \in O(n^{i_1}) \land \text{TIMECOST}_{M'}(n, r) \in O(n^{i_2})\]
    where $n$ represents the familiar abuse of notation, indicating a max over all inputs of length $n$. Then, running both in sequence is $O(n^{i})$ where $i := \max(i_1, i_2)$ (basic exercise from CS30/31). That is, each iteration, regardless of random seed, runs in $\leq c \cdot n^i$ for some $c > 0$. \bbni
    Now, let $x \in \Sigma^n$ be some input. Let $p$ be the probability (over the randomness) that the first iteration  returns on this input, i.e. accepts or rejects. Since we are using new independent random bits on each iteration, each iteration has the same probability of returning. Thus, the number of iterations until and including a successful answer is a random variable, $X \sim \text{Geom}(p)$ with expectation $\frac{1}{p}$. \bbni
    Overall, then the runtime for $A$ is bounded as: 
    \[ \text{TIMECOST}_A(n, r) \leq X \cdot (c \cdot n^i)\]
    Taking expectation over the randomness, we get:
    \[ \Exp_r[\text{TIMECOST}_A(n, r)] \leq \Exp_r[X] \cdot (c \cdot n^i) = \frac{1}{p} \cdot (c \cdot n^i)\]
    Thus, we are only left to show that $p$ is sufficiently large. We have two cases: 
    \begin{enumerate}
        \item If $x \in L$, then $M'$ always accepts and we wait till $M$ accepts to return. On each iteration, $M$ accepts with probability $\geq \frac{1}{2}$. Thus, $p \geq \frac{1}{2}$. 
        \item If $x \not \in L$, then $M$ always rejects and we wait for $M'$ to reject to return. On each iteration, $M'$ rejects with probability $\geq \frac{1}{2}$. Thus, $p \geq \frac{1}{2}$. 
    \end{enumerate}
    Thus, overall, $\frac{1}{p} \leq 2$. Thus, 
    \[ \Exp_r[\text{TIMECOST}_A(n, r)] \leq 2c \cdot n^i \]
    Thus, $L \in \text{ZPP}$. \bbni
    For the other containment, let $L \in \text{ZPP}$ and $M$ be the associated PTM. We will show that $L \in \text{RP}$ and $L \in \text{coRP}$ seperately. \bbni
    We know that whenever $M$ returns, it returns correctly. Moreover, there exists $c > 0$ and $i \in \N$ such that: 
    \[\forall x : \Exp_r[\text{TIMECOST}_M(x, r)] \leq c \cdot |x|^i\]
    Using Markov's Inequality, we note that:
    \begin{align*}
        &\Pr[\text{TIMECOST}_M(x,r) \geq 2\Exp_r[\text{TIMECOST}_M(x, r)]] \leq \frac{1}{2} \\
        \implies &\Pr[\text{TIMECOST}_M(x,r) \geq 2(c\cdot |x|^i)] \leq \frac{1}{2}
    \end{align*}
    where the 2nd line follows from the bound for the expectation above. Thus, we design two PTMs $A$ and $B$ as follows:
    \begin{itemize}
        \item Simulate $M$ for $2(c \cdot |x|^i)$ steps. 
        \item If $M$ returns in that time, return the same answer. Otherwise return False for $A$ and True for $B$. 
    \end{itemize}
    We claim that $A$ and $B$ satisfy the constraints for $L$ to be in $\text{RP}$ and $\text{coRP}$ respectively. Indeed, since they run for a fixed polynomial number of steps, both of them are polynomial time bounded, regardless of the randomness. Thus, we only need to show the error-bounds. \bbni
    Note that $A$ returns true if and only if $M$ returns true (within the time limit). Thus, $A$ does not error when $x \not \in L$, i.e.
    \[ x \not \in L \implies \Pr[A(x) \neq L(x)] = 0 \]
    Moreover, we know the probability that $M$ does not return in the given time-bounds is $\leq \frac{1}{2}$ from the last inequality above. The event that $A$ returns False erroneously (i.e. when $x \in L$) is a subset of the event where $M$ does not return within the time-limit. Thus, 
    \[ x \in L \implies \Pr[A(x) \neq L(x)] \leq \frac{1}{2}\]
    Thus, $L \in \text{RP}$. Similar analysis on $B$ shows that $L \in \text{coRP}$. \bbni
    Thus, $\text{ZPP} = \text{RP} \cap \text{coRP}$. \bbni
    For the 2nd part, we need to show that $\text{RP} \subseteq \text{BPP}$ and $\text{coRP} \subseteq \text{BPP}$. \bbni
    Let $L \in \text{RP}$ and $C$ be the associated polynomial time-bounded PTM. We construct a new machine $C'$ that executes $C$ twice (using fresh random bits off of the random tape), and rejects if both simulations reject, otherwise it accepts. Using our previous observations, we know if $x \not \in L$, then $C$ is guaranteed to reject $x$, thus, $C'$ is guaranteed to reject $x$, i.e.
    \[x \not \in L \implies \Pr(C'(x, r) \neq L(x)) = 0 \leq \frac{1}{3} \] 
    However, if $x \in L$, to make an error, we want both runs of $C$ to reject. Since we are using fresh random bits from the random tape, these runs are independent. Thus, we have:
    \[ x \in L \implies \Pr(C'(x, r) \neq L(x)) = \Pr(C(x, r) \neq L(x))^2 \leq \left(\frac{1}{2}\right)^{2} = \frac{1}{4} \leq \frac{1}{3} \]
    Since we only ran $C$ twice, $C'$ is also polynomial time bounded. Thus, we have shown:
    \[ L \in \text{BPP} \]
    Since $L$ was arbitrary, we have:
    \[ \text{RP} \in \text{BPP} \]
    The same argument works for $\text{coRP}$, where we modify the provided machine to only accept if both runs accept, and reject otherwise.
\end{solution}

\newpage

\begin{problem}{13}
Give a full formal proof that
\[
\text{BPP}_{\frac{1}{2}-\delta(n), \frac{1}{2}-\delta(n)} = \text{BPP} = \text{BPP}_{\gamma(n), \gamma(n)}
\]
for all error bounds where \(\delta(n)\) is polynomially small and \(\gamma(n)\) is exponentially small.
That is, there exist constants \(c, d > 0\) such that \(\delta(n) = 1/O(n^c)\) and \(\gamma(n) = 1/2^{O(n^d)}\).
\end{problem}

\begin{solution}
    It is easy to show that $\text{BPP}_{\frac{1}{2}-\delta(n), \frac{1}{2}-\delta(n)} \supseteq \text{BPP} \supseteq \text{BPP}_{\gamma(n), \gamma(n)}$ with at most a few repetitions. For instance, to show $\text{BPP}_{\frac{1}{2}-\delta(n), \frac{1}{2}-\delta(n)} \supseteq \text{BPP}$, we need to boose a machine with at most $\frac{1}{3}$ error guarantee to less than $\frac{1}{2} - \frac{k}{n^c}$ guarantee for some given $k > 0$. This is already true for large enough $n$, as $\frac{k}{n^c} \in \Omega(1)$, i.e. eventually grows smaller than the difference between $\frac{1}{2}$ and $\frac{1}{3}$. For the fininitely many remaining input lengths, $\delta(n)$ takes on some fixed value, thus, $\frac{1}{2}-\delta(n)$ is some fixed constant (dependent on $n$). Since there are finitely many such small $n$, we can take a max over them to get some constant (worse-case) value, and assume this is the error guarantee for all small $n$. We will show, in the remainder of the proof, that it is possible to boost a constant error guarantee to an exponentially small guarantee, while remaining polynomial bounded in time. Thus, in particular, we can boost it to $\leq \frac{1}{3}$. Similar argument shows that $\text{BPP} \supseteq \text{BPP}_{\gamma(n), \gamma(n)}$. \bbni
    Thus, for our main proof, we want to show that $\text{BPP}_{\frac{1}{2}-\delta(n), \frac{1}{2}-\delta(n)} \subseteq \text{BPP} \subseteq \text{BPP}_{\gamma(n), \gamma(n)}$, that is, we can ``boost'' to reduce the error probability from $\frac{1}{2}-\delta(n)$ to $\frac{1}{3}$ to $\gamma(n)$ with only at most a polynomial blowup in time. \bbni
    Let $L \in \text{BPP}_{1-\delta(n), 1-\delta(n)}$ and $M$ be the associated polynomial time bounded PTM. Our algorithm is to run $M$ $k$ times and take a majority vote. Note that each subsequent run reads and uses fresh random bits from the tape. We claim that $k = \frac{3}{4\delta(n)^2}$ suffices. \bbni
    Let $x \in \Sigma^n$ be some input. Let $X_1, \cdots, X_k$ be the indicator variables for whether the $i$th returned an erroneous answer, i.e. did not match $L(x)$. Note that these are iid as each run uses fresh random bits. Let 
    \[ p := \Pr[X_i] = \Pr_r[M(x, r) \neq L(x)]\] 
    From the definition of $\text{BPP}_{\frac{1}{2}-\delta(n), \frac{1}{2}-\delta(n)}$, we know that:
    \[ \forall i: \Exp[X_i] = \Pr[X_i = 1] = p \leq \frac{1}{2} - \delta(n)\]
    Let $S := \sum_{i = 1}^k X_i$. Then, by linearity of expectation, 
    \[ \Exp[S] = \sum_{i=1}^k \Exp[X_i] = k \cdot \Exp[X_1] \leq k \cdot \left(\frac{1}{2}-\delta(n)\right)\]
    Using the iid property, we can also find a bound for the variance: 
    \[ \Var[S] = k\cdot \Var[X_i] = k(p(1-p)) \leq \frac{k}{4} \]
    where the last inequality follows from $p(1-p) \leq \frac{1}{4}$ for $p \in [0, 1]$. \bbni
    Note that we only make an error when a majority of the runs are erroneous, i.e. $S \geq \frac{k}{2}$. We can bound the probability of error as follows: 
    \begin{align*}
        \Pr\left[S \geq \frac{k}{2}\right] &= \Pr\left[S-\frac{k}{2}+k\delta(n) \geq k\delta(n)\right]  \\
        &= \Pr\left[S-\Exp[S] \geq k\delta(n)\right] \\
        &\leq \Pr\left[|S-\Exp[S]| \geq k \delta(n)\right]
    \end{align*}
    Using Chebyshev's inequality, we note that: 
    \begin{align*}
        \Pr\left[|S-\Exp[S]| \geq k \delta(n)\right] \leq \frac{\Var[S]}{(k\delta(n))^2} \leq \frac{k}{4k^2\delta(n)^2} = \frac{1}{4k\delta(n)^2}
    \end{align*}
    Thus, since $k = \frac{3}{4\delta(n)^2}$, we get: 
    \[ \Pr\left[S \geq \frac{k}{2} \right] \leq \frac{1}{3}\]
    Since $\delta(n) = \frac{1}{O(n^c)}$, $k = O(n^{2c})$. Thus, we only have to repeat $M$ a polynomial number of times, thus, our algorithm is still polynomial time-bounded (the product of polynomials is a polynomial). Thus, we have shown that $L \in \text{BPP}$ (since our bound works for all inputs $x$). Since $L$ was arbitrary, this proves $\text{BPP}_{\frac{1}{2}-\delta(n), \frac{1}{2}-\delta(n)}$. \bbni
    Next, we use the same trick of taking a majority vote again to ``boost'' our algorithm to $\gamma(n)$ error.\bbni  
    Let $A$ be the machine described above (with at most symmetric $\frac{1}{3}$ error). Let $Y_1, \cdots, Y_m$ be random variables indicating if the $i$th run of $A$ produced an incorrect result. As before, these variables are iid. Let $q := \Exp[Y_i] = \Pr[X_i = 1] \leq \frac{1}{3}$ and $Y = \sum_{i=1}^m Y_i$ be the sum. Note that by linearity of expectation, $\Exp[Y] = m \cdot q$. We only accept erroneously if $Y \geq \frac{m}{2}$, thus, we try to bound the probability of that. 
    \begin{align*}
        \Pr\left[Y \geq \frac{m}{2}\right] &= \Pr\left[Y \geq mq\left(\frac{1}{2q}\right)\right]
    \end{align*}
    Note that as $q \leq \frac{1}{3}$, $\frac{1}{2q} \geq \frac{3}{2}$, thus, we get: 
    \begin{align*}
        \Pr\left[Y \geq mq\left(\frac{1}{2q}\right)\right] &= \Pr\left[Y \geq mq\left(1+\frac{1}{2}\right)\right]
    \end{align*}
    Then, using the Chernoff bound (with $\delta = \frac{1}{2}$), as $Y$ is the sum of independent bernoulli's, we get: 
    \[\Pr\left[Y \geq mq\left(1+\frac{1}{2}\right)\right] \leq \left(\frac{e^{\frac{1}{2}}}{\left(\frac{3}{2}\right)^{\frac{3}{2}}}\right)^{mq} \]
    Using Wolfram Alpha to evaluate, we note that: 
    \[  r := \frac{e^{\frac{1}{2}}}{\left(\frac{3}{2}\right)^{\frac{3}{2}}} \approx 0.897 < 1\]
    Let $a > 0$ be the constant such that $\gamma(n) \geq  \left(\frac{1}{2}\right)^{a \cdot n^d}$ (using basic manipulation of big-oh notation). Thus, it suffices to pick $m$ such that: 
    \[ r^{mq} \leq \left(\frac{1}{2}\right)^{a \cdot n^d}\]
    Working this out, we get: 
    \begin{align*}
        mq \log_2 r &\leq -an^d \\
        m &\geq -\frac{a}{q\log_2 r} \cdot n^d  \qquad  &&(\text{as } \log_2 r < \log_2 1 = 0) \\
        m & \geq -\frac{3a}{\log_2 r} \cdot n^d \qquad  &&(\text{as } q \leq \frac{1}{3})
    \end{align*}
    Thus, picking $m = \lceil \frac{-3a}{\log_2 r}\rceil \cdot n^d$ ($m$ here is positive as $\log_2 r < 0$) suffices to get:  
    \[ \Pr\left[Y \geq \frac{m}{2}\right] \leq \gamma(n)\]
    Since $m = O(n^d)$ is only polynomially large in $n$, we only have to run $A$ polynomially many times. Thus, overall, our run-time is a bounded as a polynomial in $n$. Thus, $L \in \text{BPP}_{\gamma(n), \gamma(n)}$ (since our bound works for all inputs $x$). Since $L$ was arbitrary, we get $\text{BPP} \subseteq \text{BPP}_{\gamma(n), \gamma(n)}$. That completes the proof. 
\end{solution}


\newpage

\begin{problem}{14}
Let \(X\) and \(Y\) be finite sets and let \(Y^X\) denote the set of all functions from \(X\) to \(Y\).
A family \( \mathcal{H} \subseteq Y^X \) is said to be 2-universal if the following property holds, with \(h \in_R \mathcal{H}\) picked uniformly at random:
\[
\forall x, x' \in X, \forall y, y' \in Y, \quad 
(x \neq x') \implies \Pr_h[h(x) = y \wedge h(x') = y'] = \frac{1}{|Y|^2}.
\]

Consider the sets \(X = \{0,1\}^n\) and \(Y = \{0,1\}^k\), with \(k \le n\).
Treat the elements of \(X\) and \(Y\) as column vectors with 0/1 entries.
For a matrix \(A \in \{0,1\}^{k \times n}\) and vector \(b \in \{0,1\}^k\), define
\[
h_{A,b}(x) = Ax + b,
\]
where all additions and multiplications are performed mod 2.

Now consider the family of functions
\[\mathcal{H} = \{h_{A,b} : A \in \{0,1\}^{k \times n}, b \in \{0,1\}^k\}\]
First, prove that
\[\forall x \in X, \forall y \in Y : \Pr_h[h(x) = y] = \frac{1}{|Y|}\]
Next, prove that \(\mathcal{H}\) is a 2-universal family of hash functions.
\end{problem}

\begin{solution}
    Fix some arbitrary $x \in X = \{0, 1\}^n$. For a given $A \in \{0,1\}^{k \times n}$ and $b \in \{0,1\}^k$, let $h_{A,b}(x)_i = (Ax+b)_i$ denote the $i$th bit of the output. With the notation set up, we claim that:
    \begin{enumerate}
        \item  $h(x)_i$ is equally likely to be $0$ or $1$ (over the choice of $h \in \mathcal{H}$), i.e.
        \[\forall i \in [k] : \Pr_h[h(x)_i = 0] = \Pr_h[h(x)_i = 1] = \frac{1}{2} \]
        \item The value of $h(x)_i$ is independendent of $h(x)_{i'}$ for $i \neq i'$ for all $h \in \mathcal{H}$. 
    \end{enumerate}
    To show these, first note that: 
    \[ (Ax+b)_i = b_i + \sum_j A_{ij}x_j \pmod{2} \]
    where $A_{ij}$ is the $ij$th entry of $A$ and $b_i$ is the $i$th entry of $b$. \bbni
    Then note that: 
    \[ (Ax+b)_i = 0 \iff \sum_{j}A_{ij}x_j \equiv b_i \pmod{2} \]
    However, for any given $A$, note that $\sum_{j}A_{ij}x_j$ has some fixed parity. Moreover, note that $\Pr_{b}[b_i = 1] = \frac{1}{2}$ as exactly half of our choices for $b$ have $1$ in the $i$th bit (one can set up a bijection from $b \in \{0,1\}^k$ with $b_i = 0$ to $b \in \{0,1\}^k$ with $b_i = 1$), and this is not dependent on our choice for $A$. Thus, 
    \[ \forall A : \Pr_{b}\left[\sum_{j}A_{ij}x_j \equiv b_i \pmod{2}\right] = \frac{1}{2}\] 
    as $b_i$ is equally likely to match or not match the parity of $\sum_{j}A_{ij}x_j$. Thus, we have that: 
    \[ \forall A : \Pr_b[(Ax+b)_i = 0] = \frac{1}{2} \]
    Since this is true for all $A$, we can write: 
    \[ \Pr_{A,b}[(Ax+b)_i = 0] = \Pr_{h}[h(x)_i = 0] = \frac{1}{2} \]
    Since, $(Ax+b)_i \in \{0,1\}$, we also have: 
    \[ \Pr_{h}[h(x)_i = 0] = 1- \frac{1}{2} = \frac{1}{2}\]
    That proves (1). For $(2)$, fix some $h \in \mathcal{H}$, i.e. fix some $A, b$. Now, note that $(Ax+b)_i$ only depends on the entries $A_{ij}$, for all $j$, of $A$ and $b_i$ of $b$. Thus, for $i \neq i'$, $(Ax+b)_i$ and $(Ax+b)_{i'}$ depend on disjoint set of entries of $A$ and $b$. Similar to our argument for $b_i$ before, for all $i, j$ $\Pr_A[A_{ij} = 1] = \Pr_b[b_i = 1] = \frac{1}{2}$ (we can set up similar bijections to show this), independently of other entries in $A$ and $b$. These two facts combined imply that the value of $(Ax+b)_i$ is independent of $(Ax+b)_{i'}$ for $i \neq i'$. That proves (2). \bbni  
    Now, fix some arbitrary $y \in \{0,1\}^k$. Note that: 
    \begin{align*}
        \Pr_h[h(x) = y] &= \Pr_{A,b}[Ax+b = y] \\
        &= \Pr_{A,b}\left[\bigwedge_i (Ax+b)_i = y_i\right] \\
        &= \prod_{i=1}^k \Pr_{A,b}\left[(Ax+b)_i = y_i\right] \qquad &&\text{(by (2))} \\
        &= \prod_{i=1}^k \frac{1}{2}  \qquad &&\text{(by (1), as $y_i$ is fixed)}\\
        &= \left(\frac{1}{2}\right)^k \\
        &= \frac{1}{|Y|}
    \end{align*}
    Since $x$ and $y$ were arbitrary, this suffices to show that: 
    \[\forall x \in X, \forall y \in Y: \Pr_h[h(x) = y] = \frac{1}{|Y|} \]
    Next, we need to show that $\mathcal{H}$ is a $2$-universal family of hash functions. Thus, let $y, y' \in Y$ and $x, x' \in X$ with $x \neq x'$. Note that: 
    \[ Ax+b = y \wedge Ax'+b = y' \iff b = y-Ax = y'-Ax' \iff A(x'-x) = y'-y\]
    where all operations are mod 2. Then, for a fixed $A$, the last equation is either consistent or not. If it is consistent, then there is only one $b$ that works, as all other variables in the equation are fixed. If it is inconsistent, no such $b$ exists. Thus, 
    \begin{align*}
        \Pr_b[(Ax+b = y \land Ax'+b = y')| A] = \begin{cases}
        \frac{1}{|Y|} \qquad &\text{ if } A(x'-x) = y'-y \\
        0 \qquad &\text{ otherwise }
        \end{cases}     
    \end{align*}
    Moreover, since $x'-x \neq 0$, we claim that there are an equal number of matrices $A$ such that $A(x'-x) = z$ for any $z$. One way to see this is as follows: 
    \begin{itemize}
        \item Since $x'-x \neq 0$, we can construct a basis for $X$ with $x'-x$ being the first basis vector.
        \item Then, expressed in this basis, the set of $A$ that send $x'-x$ to $z$ are precisely all the matrices that have $z$ as their first column, with the other entries ($(k-1) \times n)$ many)free to take on any values.
        \item Then, it is easy to see that there are equally as many matrices $A$ such that $A(x'-x) = z$ for every $z$ as there are exactly $2^{(k-1) \times n }$ such matrices for each $z$.
    \end{itemize}
    In particular, averaging over all $A$, we get:
    \[ \Pr_A[A(x'-x) = z] = \frac{1}{|Y|}\]
    Picking $z = y'-y$, and putting everything together, note that: 
    \begin{align*}
        \Pr_h[h(x) = y \land h(x') = y'] &= \Pr_{A,b}[Ax+b = y \land Ax'+b = y'] \\
        &= 0 + \Pr_{b}[Ax+b = y \land Ax'+b = y' | A]\Pr_A[A(x'-x) = y'-y] \\
        &= \frac{1}{|Y|} \cdot \frac{1}{|Y|} \\
        &= \frac{1}{|Y|^2} 
    \end{align*}
    This shows that $\mathcal{H}$ is a $2$-universal hash family.
\end{solution}

% \newpage
% \begin{problem}{15}
% Let \(k > 0\) be an integer. Construct a language in \(PH\) that is not in \(\text{SIZE}(n^k)\).
% \end{problem}






\end{document}