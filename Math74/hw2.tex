% \documentclass[12pt]{amsart}
\documentclass[12pt]{article}

\usepackage{fullpage}
\usepackage{mdframed}
\usepackage{colonequals}
\usepackage{algpseudocode}
\usepackage{algorithm}
\usepackage{tcolorbox}
\usepackage[all]{xy}
\usepackage{proof}
\usepackage{mathtools}
\usepackage{bbm}
\usepackage{amssymb}
\usepackage{amsthm}
\usepackage{amsmath}
\usepackage{amsxtra}
\newcommand{\bb}{\mathbb}


\newtheorem{theorem}{Theorem}[section]
\newtheorem{corollary}{Corollary}[theorem]
\newtheorem{lemma}{Lemma}

\newcommand{\mathcat}[1]{\textup{\textbf{\textsf{#1}}}} % for defined terms

\newenvironment{problem}[1]
{\begin{tcolorbox}\noindent\textbf{Problem #1}.}
{\vskip 6pt \end{tcolorbox}}

\newenvironment{enumalph}
{\begin{enumerate}\renewcommand{\labelenumi}{\textnormal{(\alph{enumi})}}}
{\end{enumerate}}

\newenvironment{enumroman}
{\begin{enumerate}\renewcommand{\labelenumi}{\textnormal{(\roman{enumi})}}}
{\end{enumerate}}

\newcommand{\defi}[1]{\textsf{#1}} % for defined terms

\theoremstyle{remark}
\newtheorem*{solution}{Solution}

\setlength{\hfuzz}{4pt}

\newcommand{\calC}{\mathcal{C}}
\newcommand{\calF}{\mathcal{F}}
\newcommand{\C}{\mathbb C}
\newcommand{\N}{\mathbb N}
\newcommand{\Q}{\mathbb Q}
\newcommand{\R}{\mathbb R}
\newcommand{\Z}{\mathbb Z}
\newcommand{\F}{\mathbb F}
\newcommand{\br}{\mathbf{r}}
\newcommand{\RP}{\mathbb{RP}}
\newcommand{\CP}{\mathbb{CP}}
\newcommand{\nbit}[1]{\{0, 1\}^{#1}}
\newcommand{\bits}{\{0, 1\}^{n}}
\newcommand{\bbni}{\bigbreak \noindent}
\newcommand{\norm}[1]{\left\vert\left\vert#1\right\vert\right\vert}
\newcommand{\dbar}{\overline{\partial}}
\let\d\relax
\let\calF\relax
\newcommand{\d}{\partial}
\newcommand{\calO}{\mathcal{O}}
\newcommand{\calF}{\mathcal{F}}
\newcommand{\calG}{\mathcal{G}}
\newcommand{\calH}{\mathcal{H}}
\newcommand{\calE}{\mathcal{E}}

\let\1\relax
\newcommand{\1}{\mathbf{1}}
\newcommand{\fr}[2]{\left(\frac{#1}{#2}\right)}

\newcommand{\vecz}{\mathbf{z}}
\newcommand{\vecr}{\mathbf{r}}
\DeclareMathOperator{\Cinf}{C^{\infty}}
\DeclareMathOperator{\Id}{Id}

\DeclareMathOperator{\Alt}{Alt}
\DeclareMathOperator{\ann}{ann}
\DeclareMathOperator{\codim}{codim}
\DeclareMathOperator{\End}{End}
\DeclareMathOperator{\Hom}{Hom}
\DeclareMathOperator{\id}{id}
\DeclareMathOperator{\M}{M}
\DeclareMathOperator{\Mat}{Mat}
\DeclareMathOperator{\Ob}{Ob}
\DeclareMathOperator{\opchar}{char}
\DeclareMathOperator{\opspan}{span}
\DeclareMathOperator{\rk}{rk}
\DeclareMathOperator{\sgn}{sgn}
\DeclareMathOperator{\Sym}{Sym}
\DeclareMathOperator{\tr}{tr}
\DeclareMathOperator{\img}{img}
\DeclareMathOperator{\CandE}{CandE}
\DeclareMathOperator{\CandO}{CandO}
\DeclareMathOperator{\argmax}{argmax}
\DeclareMathOperator{\first}{first}
\DeclareMathOperator{\last}{last}
\DeclareMathOperator{\cost}{cost}
\DeclareMathOperator{\dist}{dist}
\DeclareMathOperator{\path}{path}
\DeclareMathOperator{\parent}{parent}
\DeclareMathOperator{\argmin}{argmin}
\DeclareMathOperator{\excess}{excess}
\let\Pr\relax
\DeclareMathOperator{\Pr}{\mathbf{Pr}}
\DeclareMathOperator{\Exp}{\mathbb{E}}
\DeclareMathOperator{\Var}{\mathbf{Var}}
\let\limsup\relax
\DeclareMathOperator{\limsup}{limsup}
%Paired Delims
\DeclarePairedDelimiter\ceil{\lceil}{\rceil}
\DeclarePairedDelimiter\floor{\lfloor}{ \rfloor}


\newcommand{\dagstar}{*}

\newcommand{\tbigwedge}{{\textstyle{\bigwedge}}}
\setlength{\parindent}{0pt}
\setlength{\parskip}{5pt}


\begin{document}

\title{CS 40: Computational Complexity}

\author{Sair Shaikh}
\maketitle

Collaboration Notice: Talked to Henry Scheible '26 to discuss ideas.

\begin{problab}{1}
(0.10) A space is called contractible if the identity map is nullhomotopic. Show that \( X \) is contractible if and only if for every space \( Y \), every map \( f : X \to Y \) is nullhomotopic. Similarly, show that \( X \) is contractible if and only if for every \( Y \), every map \( f : Y \to X \) is nullhomotopic.
\end{problab}
\begin{solu}
    $(\impliedby)$ Assume for every $Y$ every map $f : X \to Y$ is nullhomotopic. Then, in particular, picking $Y = X$ and $f = \id_X$, we have that $\id_X$ is nullhomotopic. Thus, $X$ is contractible. \\
    $(\implies)$ Assume that $X$ is contractible and $Y$ be any space. Thus, there exists a homotopy $H: X \times I \to X$ from $\id_X$ to a constant map $\lambda_c : X \to X$ for some $c \in X$. Then, for any map $f : X \to Y$, we claim that $H': X \times I \to Y$ defined by $H'(x, t) = f(H(x, t))$ is a homotopy from $f$ to the constant map $f(c)$. Similarly, for any map $g$ from $Y$ to $X$, we claim that $H'': Y \times I \to X$ defined by $H''(y, t) = H(g(y), t)$ is a homotopy from $g$ to the constant map $g(c)$. Note that $H'$ and $H''$ are continous as they are compositions of continous maps. Moreover, we check, for all $x \in X$ and $y \in Y$,
    \begin{align*}
        H'(x, 0) &= f(H(x, 0)) = f(x) \\
        H'(x, 1) &= f(H(x, 1)) = f(\lambda_c(x)) = f(c) \\
        H''(y, 0) &= H(g(y), 0) = g(y) \\
        H''(y, 1) &= H(g(y), 1) = \lambda_c(g(y)) = c
    \end{align*}
    Thus, $H'$ is a homotopy from $f$ to the constant map valued at $f(c)$ and $H''$ is a homotopy from $g$ to the constant map valued at $c$.
\end{solu}
\newpage

\begin{problab}{2}
Let \( A \subset X \) and suppose \( r : X \to A \) is a continuous map such that \( r(a) = a \) for all \( a \in A \) (i.e., \( r \) is a retraction of \( X \) onto \( A \)). If \( a_0 \in A \), show that
\[
r_* : \pi_1(X, a_0) \to \pi_1(A, a_0)
\]
is surjective. (Hint: Consider also the inclusion map of \( A \) into \( X \).)
\end{problab}
\begin{solu}
    Let $\iota: A \to X$ be the inclusion map. Then, note that for all $a \in A$: 
    \[ r \circ \iota (a) = a\]    
    Thus, for any $f: I \to A$, we have $r \circ \iota \circ f = f$. Moreover, we have $\iota_*: \pi_1(A, a_0) \to \pi_1(X, a_0)$ with $\iota_*([f]) = [\iota \circ f]$. Let $p \in \pi_1(A, a_0)$ be arbitrary. Then, we claim that $\iota_*([p]) \in \pi_1(X, a_0)$ maps to $[p]$ under $r_*$. We compute:
    \begin{align*}
        r_*(\iota_*([p])) &= r_*([\iota \circ p]) \\
        &= [r \circ \iota \circ p] \\
        &= [p] \\
    \end{align*} 
    Thus, as $[p]$ was arbitrary, we have shown that $r_*$ is surjective.
\end{solu}
\newpage


\begin{problab}{3}
(1.1.3) If \( X \) is a path-connected space, show that \( \pi_1(X) \) is abelian if and only if every change-of-basepoint isomorphism (\( \hat{\alpha} \) in class or \( \beta_h \) in Hatcher) depends only on the endpoints of the path.
\end{problab}
\begin{solu}
    $(\implies)$ Assume $\pi_1(X)$ is abelian. Let $\alpha_1, \alpha_2: I \to X$ be two paths from $x_0$ to $x_1$. Then, $\hat{\alpha_i}: \pi_1(X, x_0) \to \pi_1(X, x_1)$ are given by $\hat{\alpha_i}([f]) = [\overline{\alpha_i} \cdot f \cdot \alpha_i]$. Note that $\pi_1(X, x_1)$ is abelian as $\pi_1(X)$ is. We show, for any $[f] \in \pi_1(X, x_0)$ that:
    \begin{align*}
        \hat{\alpha_1}([f]) &= [\overline{\alpha_1} \cdot f \cdot \alpha_1] \\
        &= [\overline{\alpha_1} \cdot f \cdot \alpha_2 \cdot \overline{\alpha_2} \cdot \alpha_1] \\
        &= [\overline{\alpha_1} \cdot f \cdot \alpha_2] \cdot [\overline{\alpha_2} \cdot \alpha_1] \\
        &= [\overline{\alpha_2} \cdot \alpha_1] \cdot [\overline{\alpha_1} \cdot f \cdot \alpha_2] \\
        &= [\overline{\alpha_2} \cdot \alpha_1 \cdot \overline{\alpha_1} \cdot f \cdot \alpha_2] \\
        &= [\overline{\alpha_2} \cdot f \cdot \alpha_2] \\
        &= \hat{\alpha_2}([f])
    \end{align*} 
    Thus, $\hat{\alpha_1}$ depends only on the endpoints of the path. \bbni
    $(\impliedby)$ Let $x_0 \in X$ and $[f], [g] \in \pi_1(X, x_0)$ be arbitary. Assume $\hat{f} = \hat{g}$, since $f$ and $g$ have the same endpoints (they are loops based at $x_0$). It suffices to show that $[f] = [\overline{g} \cdot f \cdot g]$. We compute:
    \begin{align*}
        [f] &= [\overline{f} \cdot f \cdot f] \\
        &= \hat{f}([f]) \\
        &= \hat{g}([f]) \\
        &= [\overline{g} \cdot f \cdot g]
    \end{align*}
    Thus, $\pi_1(X, x_0)$ is abelian. Since $x_0$ was arbitrary, we have shown that $\pi_1(X)$ is abelian.
\end{solu}
\newpage


\begin{problab}{4}
(1.1.6) Note that a loop based at \( x_0 \) can be regarded as a continuous map of pointed spaces
\[
(S^1, (1, 0)) \to (X, x_0).
\]
Let \( [S^1, X] \) be the set of homotopy classes of maps from \( S^1 \) to \( X \) without conditions on basepoints. Then, there is a map \( \Phi : \pi_1(X, x_0) \to [S^1, X] \) that forgets the base points.
\begin{itemize}
    \item[(a)] Show that \( \Phi([f]) = \Phi([g]) \) if and only if \( [f] \) and \( [g] \) are conjugate in \( \pi_1(X, x_0) \).
    \item[(b)] Deduce that if \( X \) is path connected, then \( [S^1, X] \) is in bijection with conjugacy classes of \( \pi_1(X, x_0) \).
\end{itemize}
\end{problab}
\begin{solu}

\end{solu}
\newpage

\begin{problab}{5}
Suppose that \( p : E \to B \) is a covering map where \( B \) is connected. Show that if \( p^{-1}(b_0) \) has \( k \) elements for some \( b_0 \in B \), then \( p^{-1}(b) \) has \( k \) elements for every \( b \in B \).
\end{problab}
\begin{solu}

\end{solu}
\newpage

\begin{problab}{6}
Let \( q : X \to Y \) and \( r : Y \to Z \) be covering maps such that \( r \) has finite degree. Show that \( p = r \circ q \) is a covering map.
\end{problab}
\begin{solu}

\end{solu}

\end{document}