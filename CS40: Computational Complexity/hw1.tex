\documentclass[12pt]{article}


\usepackage{fullpage}
\usepackage{mdframed}
\usepackage{colonequals}
\usepackage{algpseudocode}
\usepackage{algorithm}
\usepackage{tcolorbox}
\usepackage[all]{xy}
\usepackage{proof}
\usepackage{mathtools}
\usepackage{bbm}
\usepackage{amssymb}
\usepackage{amsthm}
\usepackage{amsmath}
\usepackage{amsxtra}
\newcommand{\bb}{\mathbb}


\newtheorem{theorem}{Theorem}[section]
\newtheorem{corollary}{Corollary}[theorem]
\newtheorem{lemma}{Lemma}

\newcommand{\mathcat}[1]{\textup{\textbf{\textsf{#1}}}} % for defined terms

\newenvironment{problem}[1]
{\begin{tcolorbox}\noindent\textbf{Problem #1}.}
{\vskip 6pt \end{tcolorbox}}

\newenvironment{enumalph}
{\begin{enumerate}\renewcommand{\labelenumi}{\textnormal{(\alph{enumi})}}}
{\end{enumerate}}

\newenvironment{enumroman}
{\begin{enumerate}\renewcommand{\labelenumi}{\textnormal{(\roman{enumi})}}}
{\end{enumerate}}

\newcommand{\defi}[1]{\textsf{#1}} % for defined terms

\theoremstyle{remark}
\newtheorem*{solution}{Solution}

\setlength{\hfuzz}{4pt}

\newcommand{\calC}{\mathcal{C}}
\newcommand{\calF}{\mathcal{F}}
\newcommand{\C}{\mathbb C}
\newcommand{\N}{\mathbb N}
\newcommand{\Q}{\mathbb Q}
\newcommand{\R}{\mathbb R}
\newcommand{\Z}{\mathbb Z}
\newcommand{\br}{\mathbf{r}}
\newcommand{\RP}{\mathbb{RP}}
\newcommand{\CP}{\mathbb{CP}}
\newcommand{\nbit}[1]{\{0, 1\}^{#1}}
\newcommand{\bits}{\{0, 1\}^{n}}
\newcommand{\bbni}{\bigbreak \noindent}
\newcommand{\norm}[1]{\left\vert\left\vert#1\right\vert\right\vert}

\let\1\relax
\newcommand{\1}{\mathbf{1}}
\newcommand{\fr}[2]{\left(\frac{#1}{#2}\right)}

\newcommand{\vecz}{\mathbf{z}}
\newcommand{\vecr}{\mathbf{r}}
\DeclareMathOperator{\Cinf}{C^{\infty}}
\DeclareMathOperator{\Id}{Id}

\DeclareMathOperator{\Alt}{Alt}
\DeclareMathOperator{\ann}{ann}
\DeclareMathOperator{\codim}{codim}
\DeclareMathOperator{\End}{End}
\DeclareMathOperator{\Hom}{Hom}
\DeclareMathOperator{\id}{id}
\DeclareMathOperator{\M}{M}
\DeclareMathOperator{\Mat}{Mat}
\DeclareMathOperator{\Ob}{Ob}
\DeclareMathOperator{\opchar}{char}
\DeclareMathOperator{\opspan}{span}
\DeclareMathOperator{\rk}{rk}
\DeclareMathOperator{\sgn}{sgn}
\DeclareMathOperator{\Sym}{Sym}
\DeclareMathOperator{\tr}{tr}
\DeclareMathOperator{\img}{img}
\DeclareMathOperator{\CandE}{CandE}
\DeclareMathOperator{\CandO}{CandO}
\DeclareMathOperator{\argmax}{argmax}
\DeclareMathOperator{\first}{first}
\DeclareMathOperator{\last}{last}
\DeclareMathOperator{\cost}{cost}
\DeclareMathOperator{\dist}{dist}
\DeclareMathOperator{\path}{path}
\DeclareMathOperator{\parent}{parent}
\DeclareMathOperator{\argmin}{argmin}
\DeclareMathOperator{\excess}{excess}
\let\Pr\relax
\DeclareMathOperator{\Pr}{\mathbf{Pr}}
\DeclareMathOperator{\Exp}{\mathbb{E}}
\DeclareMathOperator{\Var}{\mathbf{Var}}
\let\limsup\relax
\DeclareMathOperator{\limsup}{limsup}
%Paired Delims
\DeclarePairedDelimiter\ceil{\lceil}{\rceil}
\DeclarePairedDelimiter\floor{\lfloor}{ \rfloor}


\newcommand{\dagstar}{*}

\newcommand{\tbigwedge}{{\textstyle{\bigwedge}}}
\setlength{\parindent}{0pt}
\setlength{\parskip}{5pt}



\begin{document}


\title{CS 40: Computational Complexity}

\author{Sair Shaikh}
\maketitle

% Collaboration Notice: Talked to Henry Scheible '26 to discuss ideas.
 


\begin{problem}{1}
    For a complexity class $\mathcal{C}$, define two new complexity classes $\exists \mathcal{C}$ and $\forall \mathcal{C}$ as follows.
    \begin{align*}
        \exists \mathcal{C} &= \{ \{ x \in \Sigma^* : \exists y \in \Sigma^* \text{ with } |y| = \text{poly}(|x|) \text{ such that } \langle x, y \rangle \in L_0 \} : L_0 \in \mathcal{C} \} \\
        \forall \mathcal{C} &= \{ \{ x \in \Sigma^* : \forall y \in \Sigma^* \text{ with } |y| = \text{poly}(|x|) \text{ we have } \langle x, y \rangle \in L_0 \} : L_0 \in \mathcal{C} \}
    \end{align*}
    The notation $|y| = \text{poly}(|x|)$ means that there exist fixed constants $c, k > 0$ such that $|y| \leq c |x|^k$ for all $| x | \geq 1$. Prove, rigorously, that $\exists \text{P} = \text{NP}$ and $\forall \text{P} = \text{coNP}$. \\ 
    Use only the basic definitions, where NP is defined using NDTMs and 
    \[ \text{coNP} = \{L \subseteq \Sigma^* : \overline{L} \in \text{NP}\}\] 
    In your proofs, make sure you precisely defined appropriate languages $L_0$ used in the definitions above.
\end{problem}

\begin{solution}
    We handle each part seperately. 
    \begin{enumerate}
        \item[(a)]
        First, we show that $\text{NP} \subseteq \exists P$. Let $L$ be an arbitrary language in $\text{NP}$. We need to show that $L \in \exists P$. \bbni 
        Since $L \in \text{NP}$, for each input string $x \in L$, there exists a computation path, with number of transitions polynomial in $|x|$, that an NDTM can take to reach an accept state. Moreover, the description of each configuration reached in this computational path is also upper-bounded by a polynomial in $x$ (the tape contents are at most polynomial as otherwise the NDTM will not be able to read/write this in polynomial time). Thus, the description of the computation path to reach an accept state is also polynomial in $x$. \bbni 
        Define $y(x)$ to be this description if $x \in L$ and $y(x) = \bot$ if $x \not \in L$. ($\bot$ is a stand-in for any character that is not in the alphabet for $L$). \bbni
        Next, we can design a deterministic TM, $M$, that takes as input $\langle x', y'\rangle$ (with appropriate delimiters) and does the following: 
        \begin{itemize}
            \item If $y' = \bot$, reject. 
            \item Otherwise, mimic the computation conducted by the NDTM on input $x'$, using $y'$ as a description of the computation path to take deterministically.  
        \end{itemize}
        The language that $M$ decides is $L_0 := \{ \langle x, y(x) \rangle : x \in L\}$. Since this computation is polynomial-time in the input, we have $L_0 \in P$. We have shown that: 
        \[ x \in L \iff \exists y(x), |y| = poly(|x|), \langle x, y \rangle \in L_0 \in P\]
        Thus, $L \in \exists P$. Thus, $\text{NP} \subseteq \exists P$. \bbni 
        Next, we need to show that $\exists P \subseteq \text{NP}$. Let $L \in \exists P$ be arbitrary. We need to show that $L \in \text{NP}$. \bbni
        Let $L_0 \in \text{P}$ be the language associated to $L$ in the definition of $\exists P$. Let $M'$ be the deterministic polynomial-time TM that decides $L_0$. We design a NDTM $N$ as follows: Given current tape contents $a$,
        \begin{enumerate}
            \item First run $M'$ on the input $a$. Accept if $M'$ accepts.
            \item If not, transition to the configurations corresponding to the starting configurations of $M'$ on $a \circ c$ for each $c \in \Sigma$.
        \end{enumerate}
        Since $\Sigma$ is finite, the number of states $N$ transitions to is also finite. Moreover, for any input $x$, $N$ will continue adding characters to $x$, running $M'$ on each string non-deterministically. Then, if there exist $y_x$ with $|y_x| \in poly(|x|)$ such that $\langle x, y \rangle \in L_0$, $N$ will find (by adding) $y_x$ in polynomial time, run $M'$ in polynomial time, and accept. If there is no such $y_x$, $N$ will not halt and continue adding characters. Thus, $L(N) = L$. \bbni
        Thus, $L \in \text{NP}$ (not sure if this is true as $N$ recognizes but doesn't decide $L$), thus $\exists \text{P} \subseteq \text{NP}$. Thus, $\text{NP} = \exists \text{P}$. 
        
        \item[(b)] To show that $\text{coNP} = \forall P$, we will show that for any $L$, $L \in \text{coNP} \iff L \in \forall \text{P}$. \bbni
        We know $L \in \text{coNP}$ if and only if $\overline{L} \in \text{NP}$. \bbni        
        By the previous part, $\overline{L} \in \text{NP}$ if and only if $\overline{L} \in \exists \text{P}$. \bbni
        Next, we will show that $\overline{L} \in \exists \text{P} \iff L \in \forall \text{P}$. Note that $\overline{L} \in \exists P$ if and only if there exists $L_0 \in \text{P}$, such that: 
        \[  x \in \overline{L} \iff \exists y_x, |y_x| \in poly(|x|) \text{ such that } \langle x, y_x\rangle \in L_0 \]
        Taking the complement, we have: 
        \[ x \in L \iff \forall y_x, |y_x| \in poly(|x|), \langle x, y_x \rangle \in \overline{L_0} \]  
        Thus, to claim that $L \in \forall \text{P}$, we only need to show $\overline{L_0} \in \text{P} \iff L_0 \in \text{P}$ (i.e. $\text{coP} = \text{P}$). Let $M$ be a polynomial-time TM that decides $L_0$. Then, we can design a machine $M'$ that runs the same computation as $M$, but flips the result before returning. By construction, $M'$ decides $\overline{L_0}$. Moreover, the run-time of $M'$ is only a constant time slower than $M$, and is thus polynomial in the input. Thus, $L_0 \in \text{P} \implies \overline{L_0} \in \text{P}$. For the other direction, note that complements are involutions, so the same arguement applies. Thus, $\overline{L_0} \in \text{P} \iff L_0 \in \text{P}$. \bbni 
        Thus, $\overline{L} \in \exists P \iff L \in  \forall P$. \bbni
        To summarize, we have shown: 
        \[ L \in \text{coNP} \iff \overline{L} \in \text{NP} \iff \overline{L} \in \exists \text{P} \iff L \in \forall \text{P}\]
        Thus, $\text{coP} = \forall \text{P}$. 
    \end{enumerate}
\end{solution}

\newpage

\begin{problem}{2}
    Define the following two complexity classes:
    \begin{align*}
        \text{EXP} &= \bigcup_{i=1}^{\infty} \text{DTIME}\left(2^{n^i}\right) \\
        \text{NEXP} &= \bigcup_{i=1}^{\infty} \text{NTIME}\left(2^{n^i}\right)
    \end{align*}
    Prove that P = NP implies EXP = NEXP. This proof requires one creative idea, namely ``padding'': given a language $L$,
    think about designing a new language wherein each input instance is constructed by starting with an instance of $L$ and
    ``padding it out'' by appending a long string of extra symbols. 
\end{problem}
\begin{solution}
    Assume $\text{P} = \text{NP}$. We know that $\text{EXP} \subseteq \text{NEXP}$. Thus, we need to show $\text{NEXP} \subseteq \text{EXP}$. Towards that goal, let $L \in \text{NEXP}$ be an arbitrary language. It suffices to show $L \in \text{EXP}$. \bbni
    Since $L \in \text{NEXP}$, there exists a positive integer $i$ such that $\text{TimeCost}_M(n) \in O(2^{n^i})$ where $M$ is the TM that decides $L$. We construct a new language $L'$ in the following manner: 
    \begin{itemize}
        \item Take any string $x \in L$
        \item Create string $x'$ by appending $\bot$ (assumed to be not in the alphabet for $L$) to the end of the string until $|x'| = 2^{n^i}$ where $n = |x|$. 
        \item These strings $x'$ are the strings in $L'$. 
    \end{itemize}
    Similarly, we can use a modification of $M$ that treats $\bot$ the same as an empty character to decide $L'$. Call this machine $M'$. \bbni 
    The computational paths of $M'$ look identical to that of $M$, thus, on input $x'$ derived from $x \in L$ with $|x| = n$, it runs in $O(2^{n^i})$ time. However, since $|x'| = 2^{n^i}$, $\text{TimeCost}_{M'}(k) \in O(k)$ is linear. Thus, we conclude that $L' \in \text{NP}$ (as $M'$ is an NDTM). \bbni 
    Using our hypothesis, $L' \in \text{P}$. Thus, there exists a deterministic Turing machine $T'$ that decides $L'$ in time polynomial in the input. Since this machine is derived from $M'$, which treats $\bot$ the same as empty characters, so does this machine (i.e. its internal computation do not depend on our padding). Thus, this machine also decides $L$. \bbni 
    Since the run-time of this machine is polynomial in $2^{n^i}$, it is exponential in $n$. Thus, $L \in \text{EXP}$. This suffices to show $\text{NEXP} = \text{EXP}$. 

\end{solution}

\end{document}