\documentclass[12pt]{article}

\usepackage{fullpage}
\usepackage{mdframed}
\usepackage{colonequals}
\usepackage{algpseudocode}
\usepackage{algorithm}
\usepackage{tcolorbox}
\usepackage[all]{xy}
\usepackage{proof}
\usepackage{mathtools}
\usepackage{bbm}
\usepackage{amssymb}
\usepackage{amsthm}
\usepackage{amsmath}
\usepackage{amsxtra}
\newcommand{\bb}{\mathbb}


\newtheorem{theorem}{Theorem}[section]
\newtheorem{corollary}{Corollary}[theorem]
\newtheorem{lemma}{Lemma}

\newcommand{\mathcat}[1]{\textup{\textbf{\textsf{#1}}}} % for defined terms

\newenvironment{problem}[1]
{\begin{tcolorbox}\noindent\textbf{Problem #1}.}
{\vskip 6pt \end{tcolorbox}}

\newenvironment{enumalph}
{\begin{enumerate}\renewcommand{\labelenumi}{\textnormal{(\alph{enumi})}}}
{\end{enumerate}}

\newenvironment{enumroman}
{\begin{enumerate}\renewcommand{\labelenumi}{\textnormal{(\roman{enumi})}}}
{\end{enumerate}}

\newcommand{\defi}[1]{\textsf{#1}} % for defined terms

\theoremstyle{remark}
\newtheorem*{solution}{Solution}

\setlength{\hfuzz}{4pt}

\newcommand{\calC}{\mathcal{C}}
\newcommand{\calF}{\mathcal{F}}
\newcommand{\C}{\mathbb C}
\newcommand{\N}{\mathbb N}
\newcommand{\Q}{\mathbb Q}
\newcommand{\R}{\mathbb R}
\newcommand{\Z}{\mathbb Z}
\newcommand{\F}{\mathbb F}
\newcommand{\br}{\mathbf{r}}
\newcommand{\RP}{\mathbb{RP}}
\newcommand{\CP}{\mathbb{CP}}
\newcommand{\nbit}[1]{\{0, 1\}^{#1}}
\newcommand{\bits}{\{0, 1\}^{n}}
\newcommand{\bbni}{\bigbreak \noindent}
\newcommand{\norm}[1]{\left\vert\left\vert#1\right\vert\right\vert}
\newcommand{\dbar}{\overline{\partial}}
\let\d\relax
\let\calF\relax
\newcommand{\d}{\partial}
\newcommand{\calO}{\mathcal{O}}
\newcommand{\calF}{\mathcal{F}}
\newcommand{\calG}{\mathcal{G}}
\newcommand{\calH}{\mathcal{H}}
\newcommand{\calE}{\mathcal{E}}

\let\1\relax
\newcommand{\1}{\mathbf{1}}
\newcommand{\fr}[2]{\left(\frac{#1}{#2}\right)}

\newcommand{\vecz}{\mathbf{z}}
\newcommand{\vecr}{\mathbf{r}}
\DeclareMathOperator{\Cinf}{C^{\infty}}
\DeclareMathOperator{\Id}{Id}

\DeclareMathOperator{\Alt}{Alt}
\DeclareMathOperator{\ann}{ann}
\DeclareMathOperator{\codim}{codim}
\DeclareMathOperator{\End}{End}
\DeclareMathOperator{\Hom}{Hom}
\DeclareMathOperator{\id}{id}
\DeclareMathOperator{\M}{M}
\DeclareMathOperator{\Mat}{Mat}
\DeclareMathOperator{\Ob}{Ob}
\DeclareMathOperator{\opchar}{char}
\DeclareMathOperator{\opspan}{span}
\DeclareMathOperator{\rk}{rk}
\DeclareMathOperator{\sgn}{sgn}
\DeclareMathOperator{\Sym}{Sym}
\DeclareMathOperator{\tr}{tr}
\DeclareMathOperator{\img}{img}
\DeclareMathOperator{\CandE}{CandE}
\DeclareMathOperator{\CandO}{CandO}
\DeclareMathOperator{\argmax}{argmax}
\DeclareMathOperator{\first}{first}
\DeclareMathOperator{\last}{last}
\DeclareMathOperator{\cost}{cost}
\DeclareMathOperator{\dist}{dist}
\DeclareMathOperator{\path}{path}
\DeclareMathOperator{\parent}{parent}
\DeclareMathOperator{\argmin}{argmin}
\DeclareMathOperator{\excess}{excess}
\let\Pr\relax
\DeclareMathOperator{\Pr}{\mathbf{Pr}}
\DeclareMathOperator{\Exp}{\mathbb{E}}
\DeclareMathOperator{\Var}{\mathbf{Var}}
\let\limsup\relax
\DeclareMathOperator{\limsup}{limsup}
%Paired Delims
\DeclarePairedDelimiter\ceil{\lceil}{\rceil}
\DeclarePairedDelimiter\floor{\lfloor}{ \rfloor}


\newcommand{\dagstar}{*}

\newcommand{\tbigwedge}{{\textstyle{\bigwedge}}}
\setlength{\parindent}{0pt}
\setlength{\parskip}{5pt}


\begin{document}

\title{CS 40: Computational Complexity}

\author{Sair Shaikh}
\maketitle

Collaboration Notice: Talked to Henry Scheible '26 to discuss ideas.



\begin{problab}{1}
Let $X$ be a differentiable manifold. Prove that $H^k_{\text{dR}}(X,\mathbb{C}) \simeq H^k_{\text{dR}}(X,\mathbb{R}) \otimes_{\mathbb{R}} \mathbb{C}$.
\end{problab}
\begin{solu}
    We note that the de Rham cohomology with complex coefficients is defined as: 
    \begin{align*}
        H_{dR}^k(X, \mathbb{C}) &= \frac{\ker(d_\C : \Omega^k(X) \otimes_\R \C \to \Omega^{k+1}(X) \otimes_\R \C)}{\text{im}(d_\C : \Omega^{k-1}(X) \otimes_\R \C\to \Omega^k(X) \otimes_\R \C)}
    \end{align*}
    where $d_\C$ is the complexified map. From the last problem set (Problem 3.4), we know that this is equivalent to: 
    \begin{align*}
        H_{dR}^k(X, \mathbb{C}) &= \frac{\ker(d : \Omega^k(X) \to \Omega^{k+1}(X))\otimes_\R \C}{\text{im}(d : \Omega^{k-1}(X) \to \Omega^k(X) ) \otimes_\R \C}
    \end{align*}
    However, as tensoring with a vector space is exact, we note that for any real-vector spaces $A, B \subseteq A$: 
    \begin{align*}
        &0 \to B \to A \to A/B \to 0 \\
        \implies& 0 \to B \otimes_\R \C \to A \otimes_\R \C \to (A/B) \otimes_\R \C \to 0
    \end{align*}
    Thus, we have:     
    \[(A/B) \otimes_\R \C \equiv (A \otimes_\R \C) / (B \otimes_\R \C) \]
    Applying this to the cohomology groups, we get:
    \begin{align*}
        H_{dR}^k(X, \mathbb{C}) &= \frac{\ker(d : \Omega^k(X) \to \Omega^{k+1}(X))}{\text{im}(d : \Omega^{k-1}(X) \to \Omega^k(X) )}\otimes_\R \C \\
        &= H_{dR}^k(X, \mathbb{R}) \otimes_\R \C
    \end{align*}

\end{solu}
\newpage

\begin{problab}{2}
This exercise is taken from HW1 as, unfortunately, the hint for question 1 was missing. As the techniques and the result are important, I put it back.

Let $U$ be an open subset of $\mathbb{C}$ and $D \subset \Omega$ be a closed disk.

\begin{enumerate}
    \item Let $f : U \to \mathbb{C}$ be a $C^1$ function. Show that for all $z \in D$, we have:
    \[
    f(z) = \frac{1}{2i\pi} \int_{\partial D} \frac{f(\xi)}{\xi - z} d\xi + \frac{1}{2i\pi} \int_D \frac{\partial f}{\partial \bar{z}}(\xi) \frac{d\xi \wedge d\bar{\xi}}{\xi - z}.
    \]
    \textit{Hint: You can apply the Stokes formula to $\frac{f(\xi)}{\xi-z} d\xi$ on $D \setminus B(z, \epsilon)$ and let $\epsilon \to 0$.}

    \item Let $g$ be a $C^1$ function on $\mathbb{C}$ with compact support and let:
    \[
    f(z) = \frac{1}{2i\pi} \int_{\mathbb{C}} \frac{g(\xi)}{\xi - z} d\xi \wedge d\bar{\xi}.
    \]
    Show that $f$ is a $C^1$ function and $\frac{\partial f}{\partial \bar{z}} = g$.

    \textit{Hint: you can differentiate under the integral sign after the change of variable $\xi' = \xi - z$, then change back and conclude using the formula from the first question.}

    \item Show that for any function $g$ on $U$ which is $C^1$, there exists $f$ which is $C^1$ on $U$ such that $\partial f/\partial \bar{z} = g$ on $D$.

    \item In the last question, show that if $g$ is $C^\infty$, then $f$ can be chosen $C^\infty$. Show also that if $g$ depends smoothly (or holomorphically) on other parameters, then so does $f$.
\end{enumerate}
\end{problab}
\begin{solu}

\end{solu}
\newpage

\begin{problab}{3}
\textbf{Holomorphic $\bar{\partial}$-Dolbeault Lemma.} Let $U$ be an open subset of $\mathbb{C}^n$ and $D$ an open polydisk with closure contained in $U$. Let $0 \le p \le n$, $1 \le q \le n$. The goal of this exercise is to prove that any $(p, q)$-form $\bar{\partial}$-closed on $U$ has a restriction to $D$ which is $\bar{\partial}$-exact.

\begin{enumerate}
    \item Prove that we can reduce to the case where $p = 0$. \textit{Hint: show that each form $\alpha \in \mathcal{A}^{p,q}(U)$ can be written as $\alpha = \sum_{|I|=p} \alpha_I \wedge dz^I$ with $\alpha_I \in \mathcal{A}^{0,q}(U)$ uniquely determined by $\alpha$.}

    \item Let $\alpha \in \Omega^{0,q}(U)$. Show that there exists $1 \le k \le n$ such that $\alpha = dz^k \wedge \gamma + \delta$ and $\gamma, \delta$ are forms in the subalgebra generated by $dz^i$, $1 \le i \le k - 1$.

    \item Prove the result by induction on $k$. \textit{Hint: you can consider a form $\mu \in \mathcal{A}^{0,q-1}$ obtained from $\gamma$ by replacing each coefficient $f \in C^\infty(D)$ by a function $g \in C^\infty$ such that $\partial g/\partial z^k = f$ on $D$. Show that if $\bar{\partial}\alpha = 0$, then we can choose $\mu$ such that $\bar{\partial}\mu = dz^k \wedge \gamma + \nu$ where $\nu$ can be expressed only in terms of $dz^1, \dots, dz^{k-1}$ and $C^\infty(U)$.}
\end{enumerate}
\end{problab}
\begin{solu}

\end{solu}
\newpage

\begin{problab}{4}
\textbf{Dolbeault cohomology of the open disk.} Let $D$ be an open disk in $\mathbb{C}$ or $D = \mathbb{C}$.

\begin{enumerate}
    \item Let $g \in C^\infty(D)$. Show that there exists $f \in C^\infty(D)$ such that $\partial f/\partial \bar{z} = g$.

    \textit{Hint: choose a sequence of disks $D_n \subset D$ such that $D_n \subset D_{n+1}$ and $\bigcup_n D_n = D$. Construct $f_n \in C^\infty(D)$ such that $\partial f_n/\partial \bar{z} = g$ on $D_n$ and such that $|f_{n+1} - f_n| \le 2^{-n}$ on $D_{n-1}$. Show that $f_n$ converges to a function $f$ that solves the problem.}

    \item Compute the Dolbeault cohomology groups of $D$.
\end{enumerate}
\end{problab}
\begin{solu}

\end{solu}
\newpage
\begin{problab}{5}
Let $\mathbb{P}^3(\mathbb{C})$ denote the complex projective 3-space with homogeneous coordinates $x_0, x_1, x_2, x_3$. Consider the complex submanifold
\[
X := \{x \in \mathbb{P}^3(\mathbb{C}) \mid x_0^4 + x_1^4 + x_2^4 + x_3^4 = 0\}.
\]
Let $M$ be the underlying $C^\infty$ manifold of $X$ and let $I$ denote the corresponding complex structure. Show that $(M, I)$ and $(M, -I)$ are isomorphic as complex manifolds. How can you generalize this example?
\end{problab}
\begin{solu}
    \bbni
    Define the map $\phi: (M, I) \to (M, I)$ by: 
    \[ \phi([x_0 : \cdots : x_3 ]) \to [\overline{x_0} : \cdots : \overline{x_3}]\]
    Which we see as picking a representative in $\C^4$, complex conjugating, and then quotienting back into $\mathbb{P}^3(C)$. We claim that this is an isomorphism between $(M, I)$ and $(M, -I)$. We need to show that $\phi$ is well-defined, a diffeomorphism, and holomorphic with respect to the complex structure on the image. \bbni
    To show that $\phi$ is well-defined, we need to show that $\phi(X) \subset X$ and that is it well-defined with respect to the choince of representative in $\C^4$. First, notice that $(\overline{z})^4 = \overline{z^4}$. Thus, by conjugating the equation, we have:
    \[ x_0^4 + x_1^4 + x_2^4 + x_3^4 = 0 \implies \overline{x_0}^4 + \overline{x_1}^4 + \overline{x_2}^4 + \overline{x_3}^4 = 0  \]
    Thus, $\phi(X) \subset X$. Moreover, for any $\lambda \in \C$, 
    \begin{align*}
        \phi([\lambda x_0 : \cdots : \lambda x_3]) &= [\overline{\lambda x_0}: \cdots : \overline{\lambda x_3}] \\
        &= [\overline{\lambda}\overline{x_0}: \cdots : \overline{\lambda} \overline{x_3}] \\
        &= [\overline{x_0}: \cdots : \overline{x_3}]
    \end{align*}
    Thus, $\phi$ is independent of the choice of representative. Thus, $\phi$ is well-defined. \bbni
    Next, we know that complex conjugation is smooth and bijective, as it is a linear map on $\R^8 \cong \C^4$. Moreover, it is self-inverse, thus is a diffeomorphism. Since it is well-defined with respect to quotienting, we conclude $\phi$ is a diffeomorphism. \bbni
    Next, we claim that taking the complex conjugate is anti-holomorphic. Let $\psi(z) = \overline{z}$, be the complex conjugation map, where $z = x + iy$. Then, we have:
    \begin{align*}
        \frac{\partial \overline{z}}{\partial x} = 1 & \qquad \frac{\partial \overline{z}}{\partial y} = -i \\
    \end{align*}
    Thus, we have:
    \begin{align*}
        \frac{\partial \overline{z}}{\partial z} &= \frac{1}{2}(1-i(-i)) = 0 \\
        \frac{\partial \overline{z}}{\partial \overline{z}} &= \frac{1}{2}(1+i(-i)) = 1
    \end{align*}
    Thus, $\psi$ is anti-holomorphic. Since (anti-)holomorphicity for higher dimensions is defined coordinate-wise, we note that complex conjugation is anti-holomorphic. Thus, complex conjugation is holomorphic with respect to the flipped complex structure on the target space. Thus, $\phi$ is holomorphic with respect to the complex structure on the target space (using the same atlas). \bbni 
    In general, every part of this argument applies to any complex submanifold of $\mathbb{P}^n(\C)$, so long as the submanifold is defined by a polynomial equation with real coefficients, as this guarantees that the complex conjugate of a solution is also a solution. The rest of the argument stays the same in those cases.
\end{solu}



\end{document}