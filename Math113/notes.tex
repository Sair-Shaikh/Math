\documentclass[12pt]{article}

\usepackage{fullpage}
\usepackage{mdframed}
\usepackage{colonequals}
\usepackage{algpseudocode}
\usepackage{algorithm}
\usepackage{tcolorbox}
\usepackage[all]{xy}
\usepackage{proof}
\usepackage{mathtools}
\usepackage{bbm}
\usepackage{amssymb}
\usepackage{amsthm}
\usepackage{amsmath}
\usepackage{amsxtra}
\newcommand{\bb}{\mathbb}


\newtheorem{theorem}{Theorem}[section]
\newtheorem{corollary}{Corollary}[theorem]
\newtheorem{lemma}{Lemma}

\newcommand{\mathcat}[1]{\textup{\textbf{\textsf{#1}}}} % for defined terms

\newenvironment{problem}[1]
{\begin{tcolorbox}\noindent\textbf{Problem #1}.}
{\vskip 6pt \end{tcolorbox}}

\newenvironment{enumalph}
{\begin{enumerate}\renewcommand{\labelenumi}{\textnormal{(\alph{enumi})}}}
{\end{enumerate}}

\newenvironment{enumroman}
{\begin{enumerate}\renewcommand{\labelenumi}{\textnormal{(\roman{enumi})}}}
{\end{enumerate}}

\newcommand{\defi}[1]{\textsf{#1}} % for defined terms

\theoremstyle{remark}
\newtheorem*{solution}{Solution}

\setlength{\hfuzz}{4pt}

\newcommand{\calC}{\mathcal{C}}
\newcommand{\calF}{\mathcal{F}}
\newcommand{\C}{\mathbb C}
\newcommand{\N}{\mathbb N}
\newcommand{\Q}{\mathbb Q}
\newcommand{\R}{\mathbb R}
\newcommand{\Z}{\mathbb Z}
\newcommand{\br}{\mathbf{r}}
\newcommand{\RP}{\mathbb{RP}}
\newcommand{\CP}{\mathbb{CP}}
\newcommand{\nbit}[1]{\{0, 1\}^{#1}}
\newcommand{\bits}{\{0, 1\}^{n}}
\newcommand{\bbni}{\bigbreak \noindent}
\newcommand{\norm}[1]{\left\vert\left\vert#1\right\vert\right\vert}

\let\1\relax
\newcommand{\1}{\mathbf{1}}
\newcommand{\fr}[2]{\left(\frac{#1}{#2}\right)}

\newcommand{\vecz}{\mathbf{z}}
\newcommand{\vecr}{\mathbf{r}}
\DeclareMathOperator{\Cinf}{C^{\infty}}
\DeclareMathOperator{\Id}{Id}

\DeclareMathOperator{\Alt}{Alt}
\DeclareMathOperator{\ann}{ann}
\DeclareMathOperator{\codim}{codim}
\DeclareMathOperator{\End}{End}
\DeclareMathOperator{\Hom}{Hom}
\DeclareMathOperator{\id}{id}
\DeclareMathOperator{\M}{M}
\DeclareMathOperator{\Mat}{Mat}
\DeclareMathOperator{\Ob}{Ob}
\DeclareMathOperator{\opchar}{char}
\DeclareMathOperator{\opspan}{span}
\DeclareMathOperator{\rk}{rk}
\DeclareMathOperator{\sgn}{sgn}
\DeclareMathOperator{\Sym}{Sym}
\DeclareMathOperator{\tr}{tr}
\DeclareMathOperator{\img}{img}
\DeclareMathOperator{\CandE}{CandE}
\DeclareMathOperator{\CandO}{CandO}
\DeclareMathOperator{\argmax}{argmax}
\DeclareMathOperator{\first}{first}
\DeclareMathOperator{\last}{last}
\DeclareMathOperator{\cost}{cost}
\DeclareMathOperator{\dist}{dist}
\DeclareMathOperator{\path}{path}
\DeclareMathOperator{\parent}{parent}
\DeclareMathOperator{\argmin}{argmin}
\DeclareMathOperator{\excess}{excess}
\let\Pr\relax
\DeclareMathOperator{\Pr}{\mathbf{Pr}}
\DeclareMathOperator{\Exp}{\mathbb{E}}
\DeclareMathOperator{\Var}{\mathbf{Var}}
\let\limsup\relax
\DeclareMathOperator{\limsup}{limsup}
%Paired Delims
\DeclarePairedDelimiter\ceil{\lceil}{\rceil}
\DeclarePairedDelimiter\floor{\lfloor}{ \rfloor}


\newcommand{\dagstar}{*}

\newcommand{\tbigwedge}{{\textstyle{\bigwedge}}}
\setlength{\parindent}{0pt}
\setlength{\parskip}{5pt}


\begin{document}

\title{CS 40: Computational Complexity}

\author{Sair Shaikh}
\maketitle

% Collaboration Notice: Talked to Henry Scheible '26 to discuss ideas.


\begin{itemize}
    \item[Defn.] Let $\{U_\alpha\}_{\alpha \in A}$ be an open cover of $(X, \rho)$. We say that $d > 0$ is a Lebesgue number for the cover if given any $d$-ball $B_d(x_0)$ with $x_0 \in X$, there exists $a_0 \in A$ such that $B_d(x_0) \subseteq U_{a_0}$. 
    \item[Ex] $X = \R$. $U_1 = (-\infty, 1)$, $U_2 = (0, 2)$, and $U_3 = (1, \infty)$. Here $d = 1/2$ is a LN for $\{U_1, U_2, U_3\}$. This is clear if $x_0 \in (1/2, 3/2)$. 
    \item[Ex. (Hwk.)] Given $x \in (0, 1)$, $\exists \delta_x > 0$ such that:
    \[ y \in B_{\delta_x}(x) = \{y \in (0, 1): |y-x| \leq \delta_x\} \]
    \[ \implies |1/x - 1/y| < 1 \]
    Then, 
    \[ (0, 1) = \bigcup_{x \in (0, 1)} B_{\delta_x}(x) \]
    has no Lebesgue number.
    \item[Lemma] (Lebesgue Covering Lemma) Every open cover of a compact metric space has a Lebesgue number. \\ 
    Proof. Pictures. Apr 9.
    \item[Thm.] Suppose $(X, \rho)$ is compact, and $F: (X, \rho) \to (Y, \sigma)$ is continuous. Then $F$ is uniformly continuous.
    % Proof. Let $\epsilon > 0$. We need to find $\delta > 0$ such that $\forall x, y \in X$: 
    % \[ \rho(x, y) < \delta \implies \sigma(F(x), F(y)) < \epsilon \]
    % Since $F$ is continous, $\forall z \in X$, $\exists \delta_z > 0$ such that:
    % \[ \rho(x, z) < \delta_z \implies \sigma(F(x), F(z)) < \epsilon/2 \]
    % That is, 
    % \[ F(B_{\delta_z}(z)) \subseteq B_{\epsilon/2}(F(z)) \]
    % Let $\delta > 0$ be a Lebesgue number for the cover $\{B_{\delta_z}(z)\}_{z \in X}$. \\
    % Now supposed $\rho(x,y) < \delta$. Then $\exists z \in X$ such that:
    % \[ B_\delta(x) \subseteq B_{\delta_z}(z) \]
    % and 
    % \[\sigma(F(x), F(y)) \leq \sigma(F(x), F(y)) + \sigma(F(z), F(y)) < \epsilon/2 + \epsilon/2 = \epsilon \]
    \item[Defn. ] Let $(X, \rho)$ be a metric space and $C(X)$ the $\C$-vector space of continous functions on $X$. We say $\mathcal{J} \subset C$ is equicontinous at $x \in X$ if $\forall \epsilon > 0, \exists \delta > 0$ such that:
    \[ \forall F \in \mathcal{J}, F(B_\delta(x)) \subseteq B_\epsilon(F(x)) \]
    We say $\mathcal{J}$ is equicontinous on $X$ if $\forall x \in X$, $\mathcal{J}$ is equicontinous at $x$.
    \item[Ex.] Let $X = [0, 1] \subset \R$. Let $F_n(x) = x^n \forall n \geq 1$. Let:
    \[\mathcal{J} = \{F_n: n \in \mathbb{N}\}\]
    Let $x_n = \frac{1}{2}^{1/n}$. Then, $x_n$ arrow up to $1$. Then, 
    \[ |F_n(x_n) - F_n(1)| = |1/2 -1| = 1/2\]
    Thus, $\mathcal{J}$ is not equicontinous at $1$.
    \item[Ex. (Hwk)] Show that $\mathcal{J}$ is equicontinous on $[0, 1)$. 
    \item[Defn.] Let $(F_n)$ be a sequence of ($\C$-valued) functions on $X$. Then, $(F_n)$ is uniformly bounded if $\exists M > 0$ such that $\forall n \geq 1, \forall x \in X$:
    \[ |F_n(x)| < M \]
    We say that $(F_n)$ is pointwise bounded if $\forall x \in X$, $\exists M_x > 0$ such that:
    \[ |F_n(x)| < M_x \]
    \item[Defn.] A metric space (top. space) is seperable if there is a countable dense subset $D \subset X$. 
    \item[Ex.] Since $\Q^n \subset \R^n$ is dense, $(\R^n, ||\cdot||_p)$ is separable.  
    \item[Lemma.] (Arzelà-Ascoli) Let $(X, \rho)$ is a seperable metric space and that $(F_n)$ is pointwise bounded and equicontinous in $C(X)$. Then, there is subsequence $(F_{n_k})$ such that:
        \[ \lim_{x\to\infty} F_{n_k}(x)\]
    exists $\forall x \in X$.
    \item[Yap.] Given a sequence $(x_n)$, we get subsequence by finding $n_k \in \mathbb{N}$ such that $n_{k+1} > n_k$ and $(x_{n_k})_{k=1}^\infty \to x$ is a sequence. \\
    A subsubsequence is determined by finding $n_{k_1} < n_{k_2} < \cdots$ and then we write: 
    \[ (x_{n_{k_j}})_{j=1}^\infty\]
    A subsequence is determined by an infinite subset $S_1 = \{n_1 \leq n_2 \leq \cdots\} \subset \mathbb{N}$. A subsubsequence is determined by an infinite subset $S_2 \subset S_1$, 
    \[ S_2 = \{n_{k_1} < n_{k_2} < \cdots \} \subset S_1\]
    Now, we write: 
    \[ \lim_{n \in S_1} x_n = a \text{ instead of } \lim_{k \to \infty} x_{n_k}\]
    Note that $\lim_{n \in S_1} x_n = a$ if:
    \[ \forall \epsilon > 0\, \exists N: n \geq N, n \in S' \implies |x_n - a| < \epsilon\] 
    \item[Rmk.] Suppose $S_1 \subset \mathbb{N}$ determines a subsequence as above. Suppose $S' \subset \mathbb{N}$ is infinite and:
    \[\{n \in S' : n \not \in S_1\}\]
    is finite. The $\lim_{n \in S_1} x_n = a$ then $\lim_{n \in S'} x_n = a$ as well. \\
    Proof of the AA Lemma: Pictures. Apr 9 and 10.
    \item[Rmk.] If $X$ is compact, then $C(x) = C_b(X)$ is a complete metric space with respect to the uniform norm $||\cdot||_\infty$. 
    \item[Thm.] (Arzelà-Ascoli) Let $(X, \rho)$ be a compact metric space and $(F_n) \subset C(X)$ be a sequence of functions that are point-wise bounded and equicontinous. Then $(F_n)$ has a subsequence converging uniformly to some function $F \in C(X)$.
    Proof. Pictures.
    \item[Lemma.] Suppose $X$ is compact and that $\mathcal{J} \subset C(X)$ is equicontinous on $X$. Then, $\mathcal{J}$ is uniformly equicontinous on $X$, in that for all $\epsilon > 0 \exists \delta > 0$ such that for all $x, y \in X$ and all $F  \in \mathcal{J}$, 
        \[ \rho(x,y) < \delta \implies |F(x) - F(y)| < \epsilon \]
    Rewriting, 
    \[ F(B_\delta(x)) \subseteq B_\epsilon(F(x))\]
    Proof left as homework.  
    \item[Corr.] Let $X$ be a compact metric space. Let $\mathcal{J} \subset C(X)$ be a closed subset such that $\mathcal{J}$ is equicontinous and pointwise bounded. Then $\mathcal{J}$ is compact and uniformly bounded.
    \item[Thm.] Suppose $X$ is a compact metric space. Then $\mathcal{J} \subset C(X)$ is compact if and only if $\mathcal{J}$ is closed, uniformly bounded, and equicontinous on $X$.
    Proof. Pictures.   
    \item[Defn.] A topological space is called a Baire space if the countable intersection of dense open sets is dense. 
    \item[Rmk.] If $\rho$ and $\sigma$ are equivalent metrics on $X$ then $(X, \rho)$ is a Baire space if and only if $(X, \sigma)$ is a Baire space.
    \item[Defn.] If $S \subset X$ then the interior of $S$ is: 
    \[ Int(S) = \bigcup \{U \subset S : U \text{ is open in } X\} \]
    \item[Rmk.] One can write $Int_X(S)$ as interior of $S$ in $X$. Where you take the interior matters. 
    \item[Lemma.] A space $X$ is a Baire space if and only if given a countable $\{F_n\}_{n = 1}^\infty$ of closed sets, such that:
    \[ \bigcup_{n = 1}^\infty F_n\]    
    has non-empty interior, then at least one $F_n$ has (non-empty) interior. 
    \item[Hwk. ] Write $O_n = F_n^C = X \setminus F_n$. $O_n$ dense $\iff Int(F_n) = \emptyset$. 
    \item[Thm.] Barise Category Thm. Every complete metric space is a Baire space.
    \item[Rmk.] Note that $(0, 1)$ is homeomorphic to $\R$. Hence, $(0, 1)$ is a Baire space. 
    \item[Hwk.] More generally, every non-empty open subset $V$ of a complete metric space admits an equivalent complete metric (Hwk). Hence $V$ is a Baire space. Thus, if:
    \[ V = \bigcup_{n=1}^\infty V \cap F_n \]
    of each $F_n$ closed in $X$, thus $V \cap F_n$ is closed in $V$. Hence, at least one $V \cap F_n$ has non-empty interior in $V$. But $V$ is open, so it has interior in $X$.
    \item[Thm.] Suppose that $X$ is a Baire space and that $(F_n) \subset C(X)$ such that $F_n \to F$ pointwise. Then,
    \[ A = \{x \in X : F \text{ is cts at $x$}\}\]
    is dense in $X$. \\
    Proof. Pictures. 
    \item[Rmk.] If $X = \R$, then we can assume: 
    \[ A = \{x \in X: F \text{ is cts at $x$}\}\]
    is uncountable. 
    \item[Defn.] A normed vector space $(V, ||\cdot||)$ is called a Banach space if $V$ is complete in in the induced metric. Recall that $\mathbb{F}$ is always $\C$ or $\R$.  
    \item[Yap.] Normed vector spaces are special. 
    \begin{enumerate}
        \item $B_r(x) = x + B_r(0)$. The topology is homogenous. 
        \item $| ||v|| - ||w|| | \leq ||v - w|| \implies v \to ||v||$ is continous.
        \item $\overline{B_r(v)} = \{x \in V : ||x-v|| \leq r\}$. 
        \item $\epsilon B_r(0) = B_{\epsilon r}(0)$. 
        \item $\epsilon \overline{B_r(0)} = \overline{B_{\epsilon r}(0)}$.
    \end{enumerate} 
    \item[Thm.] Supposed that $X$ and $Y$ are normed vector spaces and $T: X \to Y$ is linear. Then, the following are equivalent: 
    \begin{enumerate}
        \item $T$ is continuous.
        \item $T$ is continous at a single point.
        \item $\exists \alpha \geq 0$ such that $||T(x)|| \leq \alpha ||x||$ for all $v \in X$. 
    \end{enumerate}
    Proof. \\
    (1) $\implies$ (2) is trivial. \\
    (2) $\implies$ (3). Since $T$ is continous at $x_0$, $\exists \delta > 0$ such that: 
    \[ T(\overline{B_\delta(x_0)}) \subset B_1(T(x_0))\]
    LHS $ = T(B_\delta(x_0)) + T(x_0)$. \\
    RHS $ = B_1(0) + T(x)$. \\
    This implies, $T(\overline{B_\delta(0)}) \subset B_1(0)$. \\
    Now if $z \neq 0$, 
    \[  ||T(z) || = || \frac{||z||}{\delta} T(\delta \cdot \frac{z}{||z||})|| \leq \frac{||z||}{\delta}\]
    Let $\alpha = \frac{1}{\delta}$. \\
    $(3) \implies (1)$. Erased. 
    \item[Rmk.] Suppose that $||\cdot||_1$ and $||\cdot ||_2$ (two different norms) induce equivalent metrics on $V$. Then, take the $\id: (V, ||\cdot||_1) \to (V, ||\cdot||_2)$ is continous (same topology). Hence, $\exists c \geq 0$ such that: 
    \[ ||x_2|| \leq c \cdot ||x_1||\]  
    Clearly, that means $c > 0$. By symmetry, there exists a $d > 0$, such that:
    \[ ||x_1|| \leq d \cdot ||x_2||\]
    Thus, the metrics and the norms are strongly equivalent.
    \item[Defn.] If $X$ and $Y$ are normed vector spaces, then $\mathcal{L}(X, Y)$ is the vector space of continous linear maps $T: X \to Y$. Define $||T|| = \sup_{||x|| \leq 1} ||T(x)||$. If $X = Y$, then we write $\mathcal{L}(X)$ instead of $\mathcal{L}(X, X)$.
    \item[Lem.] With $||T||$ as above, $\mathcal{L}(X,Y)$ is a normed vector space with:
    \[ ||T(x)|| \leq ||T||||x|| \forall x \in X\]
    If $S \in \mathcal{L}(Y, Z)$, then we write $ST$ in place of $S \circ T$ and: 
    \[ ||ST|| \leq ||S||||T||\]
    \item[Defn.] An algebra over $\mathbb{F}$ is a vector space $A$ over $\mathbb{F}$ with a ring structure, with $\lambda(xy) = (\lambda x)y = x(\lambda y)$ for all $\lambda \in \mathbb{F}$ and $x, y \in A$.
    \item[Ex. ] $M_n(\mathbb{F})$, $\R[x]$, $\C[x]$, $C(X)$. 
    \item[Defn.] If $||\cdot||$ is a norm on an algebra $A$, then we call $(A, ||\cdot||)$ a normed algebra if $\forall x,y \in A$: 
    \[ ||xy|| \leq ||x||||y|| \]
    We call $(A, ||\cdot||)$ a Banach algebra if $(A, ||\cdot||)$ is a normed algebra and $A$ is complete with respect to $||\cdot||$.
    \item[Prop.] If $X$ and $Y$ are normed vector spaces, and $Y$ is a Banach space, then $\mathcal{L}(X,Y)$ is a Banach space. If $X$ is a Banach space, then $\mathcal{L}(X)$ is a Banach algebra. \\
    Proof. Suppose that $(T_n)$ is a Cauchy sequence in $\mathcal{L:}(X,Y)$. Then, for each $z \in X$, $T_n(X)$ is a Cauchy sequence in $Y$. Hebcem $\exists T(x) \in Y$ such that:
    \[ T_n(X) \to T(X) \]
    It is not hard to see that $T: X \to Y$ is linear. \\
    Since $(T_n)$ is Cauchy in norm, its bounded i.e. $\exists m > 0$ such that:
    \[||T_n|| \leq M \forall n \geq 1 \]
    Now if $||x|| \leq 1$, then:
    \[ ||T(x)|| = \lim_{n\to \infty} ||T_n(x)|| \leq \limsup_{n} ||T_n|| ||x|| \leq M||x||\]
    This implies $T \in \mathcal{L}(X,Y)$. \\
    Let $\epsilon > 0$. Let $N$ be such that $m, n \geq N$. Thus, 
    \[ ||T_n - T_m|| \leq \frac{\epsilon}{2} \]
    Now if $||x|| \leq 1$, and if $n \geq N$, 
    \begin{align*}
        ||(T-T_n)(x)|| & = ||T(x) - T_n(x)|| \\
        &= \lim_{m\to \infty} ||T_m(x) - T_n(x)|| \\
        &\leq \limsup_{m} ||T_m - T_n|| \cdot ||x|| \\
        &\leq \frac{\epsilon}{2} < \epsilon
    \end{align*}
    \item[Yap.] Let $Y \subset X$ be a subsoace of a normed vector space $X$. Then we can form the quotient vector space $X / Y = \{x +Y : x\in X\}$ with $q: X \to X/Y$ the quotient map. \\
    If $x \in X$, then: 
     \[ \inf \{||x-y|| : y \in Y\} \] 
    depends only on $q(x)$. We call: 
    \[||q(x)|| := \inf\{||x-y|| : y \in Y\}\]
    the quotient norm on $X/Y$.
    \item[Rmk.] As $Y$ is a subspace, 
    \begin{align*}
        ||q(x)|| &= \inf\{||x+y|| : y \in Y\} \\
        &= \inf\{||x+\alpha y|| : y \in Y\} \, \forall \alpha \in \mathbb{F}\setminus\{0\}\\
    \end{align*}
    \item[Thm.] If $Y$ be a subspace of $(X, ||\cdot||)$. Then, 
    \[ ||q(x)|| = \inf\{||x-y|| : y \in Y\} \]
    is a seminorm on $X/Y$ which is a norm exactly when $Y$ is closed. If $X$ is a Banach space, and $Y$ is closed in $X$, then $X/Y$ is a Banach space. \\
    Proof. Note that for $\alpha \neq 0$: 
    \[ ||\alpha q(x) || = || q(\alpha x) || = \inf \{||\alpha x + \alpha y|| : y \in Y\} = |\alpha| \inf\{||x+y||: y \in Y\} = |\alpha|||q(x)||\]
    Next, fix $x_1, x_2 \in X$. Then given $\epsilon > 0$, $\exists y_1, y_2 \in Y$ such that:
    \begin{align*}
        ||q(x_1)|| + ||q(x_2)|| + \epsilon &\geq ||x_1 - y_1|| + ||x_2 - y_2|| \\
        &\geq ||x_1 - y_1 + x_2 - y_2|| \\
        &\geq ||q(x_1 + x_2)|| \\
        &=  ||q(x_1)|| + ||q(x_2)|| 
    \end{align*}
    Since $\epsilon > 0$ is arbitrary, $||q(x_1) + q(x_2)|| \leq ||q(x_1)|| + ||q(x_2)||$. Thus, it is a semi-norm. \\
    Sketch for closed. If quotient norm is $0$, there exists a sequence $y_n$ converging to $x$. If $Y$ closed, $x \in Y$ and $g(x) = 0$. Conversely, if quotient norm is a norm and $y_n \to x$. Then, $||q(x)|| = 0$. Then, $q(x) = 0$ and $x \in X$. \\
    Now suppose $X$ is a Banach space and $Y$ is closed in $X$. Let $(q(z_0))$ be Cauchy in $X/Y$. Then we can pass to a subsequence and assume: 
    \[ ||q(z_{n+1}) - q(z_0)|| \leq 1/2^n\]
    Let $x_1 = z_1$. Since: 
    \[ ||q(z_2) - q(x_1)|| < 1/2 \]
    Thus, 
    \[\inf\{||z_2 - x_1 - y || : y \in Y \} < 1/2\]
    Hence, we can find $x_2$ such that $q(x_1) = q(z_2)$ and 
    \[||x_2-x_1|| < 1/2 \]
    Continuing, we get $(x_n)$ such that $q(x_n) = q(z_n)$ and: 
    \[ ||x_{n+1} - x_n|| < 1/2^n\]
    (Hwk) $(x_n)$ is Cauchy, thus, $x_n \to x$ in $X$. \\
    But $||q(z)|| \leq ||z||$. Hence, $q$ is continous. Thus, 
    \[q(z_n) = q(x_n) \to q(x)\]
    \item[Rmk.] Note that $q : X \to X/Y$ is a bounded linear map of norm at most $1$. 
    \item[Thm.] Suppose that $Y$ is a closed subspace of $(X, ||\cdot||)$. Then $X$ is a Banach space if and only if both $Y$ and $X/Y$ are Banach spaces. \\
    Proof. Pictures. 4/16
    \item[Thm.] Every finite-dimensinal subspace $Y$ of a normed vector space $X$ is a Banach space and hence closed in $X$. If $\dim(Y) = n$, then every linear isomorphism $\Phi: \mathbb{F}^n \to Y$ is a homeomorphism. \\
    Proof. Pictures. 4/16.
    
\end{itemize}




\end{document}