% \documentclass[12pt]{amsart}
\documentclass[12pt]{article}

\usepackage{fullpage}
\usepackage{mdframed}
\usepackage{colonequals}
\usepackage{algpseudocode}
\usepackage{algorithm}
\usepackage{tcolorbox}
\usepackage[all]{xy}
\usepackage{proof}
\usepackage{mathtools}
\usepackage{bbm}
\usepackage{amssymb}
\usepackage{amsthm}
\usepackage{amsmath}
\usepackage{amsxtra}
\newcommand{\bb}{\mathbb}


\newtheorem{theorem}{Theorem}[section]
\newtheorem{corollary}{Corollary}[theorem]
\newtheorem{lemma}{Lemma}

\newcommand{\mathcat}[1]{\textup{\textbf{\textsf{#1}}}} % for defined terms

\newenvironment{problem}[1]
{\begin{tcolorbox}\noindent\textbf{Problem #1}.}
{\vskip 6pt \end{tcolorbox}}

\newenvironment{enumalph}
{\begin{enumerate}\renewcommand{\labelenumi}{\textnormal{(\alph{enumi})}}}
{\end{enumerate}}

\newenvironment{enumroman}
{\begin{enumerate}\renewcommand{\labelenumi}{\textnormal{(\roman{enumi})}}}
{\end{enumerate}}

\newcommand{\defi}[1]{\textsf{#1}} % for defined terms

\theoremstyle{remark}
\newtheorem*{solution}{Solution}

\setlength{\hfuzz}{4pt}

\newcommand{\calC}{\mathcal{C}}
\newcommand{\calF}{\mathcal{F}}
\newcommand{\C}{\mathbb C}
\newcommand{\N}{\mathbb N}
\newcommand{\Q}{\mathbb Q}
\newcommand{\R}{\mathbb R}
\newcommand{\Z}{\mathbb Z}
\newcommand{\br}{\mathbf{r}}
\newcommand{\RP}{\mathbb{RP}}
\newcommand{\CP}{\mathbb{CP}}
\newcommand{\nbit}[1]{\{0, 1\}^{#1}}
\newcommand{\bits}{\{0, 1\}^{n}}
\newcommand{\bbni}{\bigbreak \noindent}
\newcommand{\norm}[1]{\left\vert\left\vert#1\right\vert\right\vert}

\let\1\relax
\newcommand{\1}{\mathbf{1}}
\newcommand{\fr}[2]{\left(\frac{#1}{#2}\right)}

\newcommand{\vecz}{\mathbf{z}}
\newcommand{\vecr}{\mathbf{r}}
\DeclareMathOperator{\Cinf}{C^{\infty}}
\DeclareMathOperator{\Id}{Id}

\DeclareMathOperator{\Alt}{Alt}
\DeclareMathOperator{\ann}{ann}
\DeclareMathOperator{\codim}{codim}
\DeclareMathOperator{\End}{End}
\DeclareMathOperator{\Hom}{Hom}
\DeclareMathOperator{\id}{id}
\DeclareMathOperator{\M}{M}
\DeclareMathOperator{\Mat}{Mat}
\DeclareMathOperator{\Ob}{Ob}
\DeclareMathOperator{\opchar}{char}
\DeclareMathOperator{\opspan}{span}
\DeclareMathOperator{\rk}{rk}
\DeclareMathOperator{\sgn}{sgn}
\DeclareMathOperator{\Sym}{Sym}
\DeclareMathOperator{\tr}{tr}
\DeclareMathOperator{\img}{img}
\DeclareMathOperator{\CandE}{CandE}
\DeclareMathOperator{\CandO}{CandO}
\DeclareMathOperator{\argmax}{argmax}
\DeclareMathOperator{\first}{first}
\DeclareMathOperator{\last}{last}
\DeclareMathOperator{\cost}{cost}
\DeclareMathOperator{\dist}{dist}
\DeclareMathOperator{\path}{path}
\DeclareMathOperator{\parent}{parent}
\DeclareMathOperator{\argmin}{argmin}
\DeclareMathOperator{\excess}{excess}
\let\Pr\relax
\DeclareMathOperator{\Pr}{\mathbf{Pr}}
\DeclareMathOperator{\Exp}{\mathbb{E}}
\DeclareMathOperator{\Var}{\mathbf{Var}}
\let\limsup\relax
\DeclareMathOperator{\limsup}{limsup}
%Paired Delims
\DeclarePairedDelimiter\ceil{\lceil}{\rceil}
\DeclarePairedDelimiter\floor{\lfloor}{ \rfloor}


\newcommand{\dagstar}{*}

\newcommand{\tbigwedge}{{\textstyle{\bigwedge}}}
\setlength{\parindent}{0pt}
\setlength{\parskip}{5pt}


\begin{document}

% \title{CS 40: Computational Complexity}

\author{Sair Shaikh}
\maketitle

% Collaboration Notice: Talked to Henry Scheible '26 to discuss ideas.



\begin{problab}{1}
Prove the pasting lemma: Suppose $X = A \cup B$ is a topological space with $A$, $B$ closed in $X$. If $f \colon X \to Y$ is a map such that the restrictions $f|_A$ and $f|_B$ are continuous, then $f$ is continuous. 
\end{problab}
\begin{solu}
    This follows from the definition of continuity. Let $U \subseteq Y$ be open. Then $f^{-1}(U) = (f|_A)^{-1}(U) \cup (f|_B)^{-1}(U)$. Since $f|_A$ and $f|_B$ are continuous, $(f|_A)^{-1}(U)$ and $(f|_B)^{-1}(U)$ are open in $A$ and $B$, respectively. Because $A$ and $B$ are closed in $X$, these sets are also open in $X$. Hence, $f^{-1}(U)$ is open in $X$, and $f$ is continuous.
\end{solu}
\newpage

\begin{problab}{2}
In a connected space $X$, a point $x \in X$ is called a \emph{cut point} if $X \setminus \{ x \}$ is disconnected. 
    \begin{enumerate}
        \item Suppose that $f: X \to Y$ is a homeomorphism of connected spaces. Show that $x \in X$ is a cut point if and only if $f(x) \in Y$ is a cut point. 
        \item Show that none of the spaces $(0,1), (0,1], [0,1],$ and $S^1 = \{ (x,y) \in \R^2 : x^2 + y^2 =1 \} $ are homeomorphic to each other. 
        \item Show that $\R$ is not homeomorphic to $\R^n$ for any $n \geq 2$. 
        \item The bouquet $B_n$ of $n$ circles is the space obtained by gluing $n$ disjoint copies of $S^1$ at a single point in each circle. Show that $B_n$ and $B_m$ are not homeomorphic for $n \neq m$. 
    \end{enumerate}
\end{problab}
\begin{solu}

\end{solu}
\newpage

\begin{problab}{3}
Define $\bb{RP}^n$ to be the quotient space of $\R^{n+1}\setminus \{0\}$ by $(x_1, \hdots, x_{n+1}) \simeq (a x_1, \hdots, a x_{n+1})$ for all nonzero scalars $a$. Let $[x_1: \hdots : x_{n+1}] \in \bb{RP}^n$ denote the image of $(x_1, \hdots, x_{n+1}) \in \bb{R}^{n+1} \setminus \{ 0 \}$ under the quotient map. Show that $i \colon \bb{R}^n \to \bb{RP}^n$ given by 
$$ i (x_1, \hdots, x_n) = [1: x_1 : \hdots : x_n ] $$
is a topological embedding (i.e., a homeomorphism onto its image) and the complement of $i(\bb{R}^n)$ is homeomorphic to $\bb{RP}^{n-1}$.
\end{problab}

\begin{solu}

\end{solu}
\newpage


\begin{problab}{4}
Suppose that $\mathcal{C}$ is a category, $A, B, C$ are objects of $\mathcal{C}$, and $f \in \hom(A,B)$ and $g \in \hom(B,C)$ are isomorphisms. 
    \begin{enumerate}
        \item Show that $f$ has a unique inverse in $\hom(B,A)$. 
        \item Show that $\id_A$ is an isomorphism.
        \item Show that the inverse of $f$ is an isomorphism.
        \item Show that $g \circ f$ is an isomorphism.
        \item Show that $\text{Aut}(A)$ and $\text{Aut}(B)$ are isomorphic groups.
        \item Show that if $F: \mathcal{C} \to \mathcal{D}$ is a functor, then $F(f) \in \hom_\mathcal{D} (F(A), F(B))$ is an isomorphism.
    \end{enumerate}
\end{problab}

\begin{solu}

\end{solu}
\newpage

\begin{problab}{5}
For a space $X$, let $\pi_0(X)$ be the space of path components of $X$. Recall that the image of a path-connected space is path-connected. Thus, for a continuous map $f: X \to Y$, there is an induced map $\pi_0(f): \pi_0(X) \to \pi_0(Y)$ taking a path component $A$ to the path component containing $f(A)$. Show that this makes $\pi_0$ a functor from the category of topological spaces to the category of sets.
\end{problab}

\begin{solu}
    First, we understand what the induced morphism $\pi_0(f)$ is. For a subset $A \subseteq X$, we let $[A] \in \pi_0(X)$ denote the path component of $A$. Then,
    \[ \pi_0(f)([A]) = [f(A)] \]   
    To show that $\pi_0$ is a functor, we need to show that it preserves identities and compositions.
    \begin{itemize}
        \item Let $X$ be a topological space and $\id_X : X \to X$ be the identity map on $X$. Then, for any path component $A\subseteq X$, $\pi_0([\id_X])(A) = [\id_X(A)] = [A]$. Thus, $\pi_0(\id_X) = \id_{\pi_0(X)}$.
        \item Let $X, Y, Z$ be topological spaces and $f: X \to Y$, $g: Y \to Z$ be continuous maps. Then, for any path component $A \subseteq X$, we have: 
        \begin{align*}
            \pi_0(g) \circ \pi_0(f)([A]) &= \pi_0(g)([f(A)]) \\
            &= [g(f(A))] \\
            &= [g \circ f(A)] \\
            &= \pi_0(g \circ f)([A])
        \end{align*}
        Thus, 
        \[ \pi_0(g \circ f) = \pi_0(g) \circ \pi_0(f) \]
    \end{itemize}
    Thus, $\pi_0$ preserves identities and compositions, and hence is a functor.
\end{solu}
\newpage

\begin{problab}{6} (Munkres, \emph{Topology}, 51.2) Given spaces $X$ and $Y$, let $[X,Y]$ denote the set of homotopy classes of maps of $X$ into $Y$.
\begin{enumerate}
\item Let $I = [0,1]$. Show that for any $X$, the set $[X,I]$ has a single element.
\item Show that if $Y$ is path connected, then $[I,Y]$ has a single element.
\end{enumerate}
\end{problab}

\begin{solu} .
    \begin{enumerate}
        \item Let $X$ be a topological space and let $A, B \in [X, I]$. We need to show that $A = B$, i.e. $A$ and $B$ are homotopic. 
    \end{enumerate}

\end{solu}
\newpage

\begin{problab}{7}
(Munkres, \emph{Topology}, 51.3) A space $X$ is said to be \emph{contractible} if the identity map $i_X: X \to X$ is nullhomotopic. 
\begin{enumerate}
\item Show that $I$ and $\R$ are contractible.
\item Show that a contractible space is path connected.
\item Show that if $Y$ is contractible, then for any $X$, the set $[X,Y]$ has a single element.
\item Show that if $X$ is contractible and $Y$ is path connected then $[X,Y]$ has a single element. 
\end{enumerate}
\end{problab}



\end{document}