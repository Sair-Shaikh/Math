\documentclass[12pt]{article}

\usepackage{fullpage}
\usepackage{mdframed}
\usepackage{colonequals}
\usepackage{algpseudocode}
\usepackage{algorithm}
\usepackage{tcolorbox}
\usepackage[all]{xy}
\usepackage{proof}
\usepackage{mathtools}
\usepackage{bbm}
\usepackage{amssymb}
\usepackage{amsthm}
\usepackage{amsmath}
\usepackage{amsxtra}
\newcommand{\bb}{\mathbb}


\newtheorem{theorem}{Theorem}[section]
\newtheorem{corollary}{Corollary}[theorem]
\newtheorem{lemma}{Lemma}

\newcommand{\mathcat}[1]{\textup{\textbf{\textsf{#1}}}} % for defined terms

\newenvironment{problem}[1]
{\begin{tcolorbox}\noindent\textbf{Problem #1}.}
{\vskip 6pt \end{tcolorbox}}

\newenvironment{enumalph}
{\begin{enumerate}\renewcommand{\labelenumi}{\textnormal{(\alph{enumi})}}}
{\end{enumerate}}

\newenvironment{enumroman}
{\begin{enumerate}\renewcommand{\labelenumi}{\textnormal{(\roman{enumi})}}}
{\end{enumerate}}

\newcommand{\defi}[1]{\textsf{#1}} % for defined terms

\theoremstyle{remark}
\newtheorem*{solution}{Solution}

\setlength{\hfuzz}{4pt}

\newcommand{\calC}{\mathcal{C}}
\newcommand{\calF}{\mathcal{F}}
\newcommand{\C}{\mathbb C}
\newcommand{\N}{\mathbb N}
\newcommand{\Q}{\mathbb Q}
\newcommand{\R}{\mathbb R}
\newcommand{\Z}{\mathbb Z}
\newcommand{\F}{\mathbb F}
\newcommand{\br}{\mathbf{r}}
\newcommand{\RP}{\mathbb{RP}}
\newcommand{\CP}{\mathbb{CP}}
\newcommand{\nbit}[1]{\{0, 1\}^{#1}}
\newcommand{\bits}{\{0, 1\}^{n}}
\newcommand{\bbni}{\bigbreak \noindent}
\newcommand{\norm}[1]{\left\vert\left\vert#1\right\vert\right\vert}
\newcommand{\dbar}{\overline{\partial}}
\let\d\relax
\let\calF\relax
\newcommand{\d}{\partial}
\newcommand{\calO}{\mathcal{O}}
\newcommand{\calF}{\mathcal{F}}
\newcommand{\calG}{\mathcal{G}}
\newcommand{\calH}{\mathcal{H}}
\newcommand{\calE}{\mathcal{E}}

\let\1\relax
\newcommand{\1}{\mathbf{1}}
\newcommand{\fr}[2]{\left(\frac{#1}{#2}\right)}

\newcommand{\vecz}{\mathbf{z}}
\newcommand{\vecr}{\mathbf{r}}
\DeclareMathOperator{\Cinf}{C^{\infty}}
\DeclareMathOperator{\Id}{Id}

\DeclareMathOperator{\Alt}{Alt}
\DeclareMathOperator{\ann}{ann}
\DeclareMathOperator{\codim}{codim}
\DeclareMathOperator{\End}{End}
\DeclareMathOperator{\Hom}{Hom}
\DeclareMathOperator{\id}{id}
\DeclareMathOperator{\M}{M}
\DeclareMathOperator{\Mat}{Mat}
\DeclareMathOperator{\Ob}{Ob}
\DeclareMathOperator{\opchar}{char}
\DeclareMathOperator{\opspan}{span}
\DeclareMathOperator{\rk}{rk}
\DeclareMathOperator{\sgn}{sgn}
\DeclareMathOperator{\Sym}{Sym}
\DeclareMathOperator{\tr}{tr}
\DeclareMathOperator{\img}{img}
\DeclareMathOperator{\CandE}{CandE}
\DeclareMathOperator{\CandO}{CandO}
\DeclareMathOperator{\argmax}{argmax}
\DeclareMathOperator{\first}{first}
\DeclareMathOperator{\last}{last}
\DeclareMathOperator{\cost}{cost}
\DeclareMathOperator{\dist}{dist}
\DeclareMathOperator{\path}{path}
\DeclareMathOperator{\parent}{parent}
\DeclareMathOperator{\argmin}{argmin}
\DeclareMathOperator{\excess}{excess}
\let\Pr\relax
\DeclareMathOperator{\Pr}{\mathbf{Pr}}
\DeclareMathOperator{\Exp}{\mathbb{E}}
\DeclareMathOperator{\Var}{\mathbf{Var}}
\let\limsup\relax
\DeclareMathOperator{\limsup}{limsup}
%Paired Delims
\DeclarePairedDelimiter\ceil{\lceil}{\rceil}
\DeclarePairedDelimiter\floor{\lfloor}{ \rfloor}


\newcommand{\dagstar}{*}

\newcommand{\tbigwedge}{{\textstyle{\bigwedge}}}
\setlength{\parindent}{0pt}
\setlength{\parskip}{5pt}


\begin{document}

\title{CS 40: Computational Complexity}

\author{Sair Shaikh}
\maketitle

Collaboration Notice: Talked to Henry Scheible '26 to discuss ideas.


\begin{problem}{1}
    \begin{enumerate}
        \item Let $X$ be a compact hermitian manifold and let $E \to X$ be a holomorphic vector bundle of rank $d$ endowed with a Hermitian metric $h$. Prove that the cohomology groups $H^q(X, E)$ are finite-dimensional vector spaces. In particular, the vector space of holomorphic sections of $E$ is finite dimensional. (Hint provided).
        \item Let $M$ be a differentiable manifold, $E, F, G \to M$ differentiable vector bundle and $P: \mathcal{C}^\infty(E) \to \mathcal{C}^\infty(F)$ and $Q: \mathcal{C}^\infty(F) \to \mathcal{C}^\infty(G)$ be differential operators. Show that the symbol of $Q \circ P$ is equal to $\sigma_Q \circ \sigma_P$ and compute its degree in terms of the degrees of $P$ and $Q$.
        \item Show that a form $\alpha$ is privitive if and only if $\Lambda \alpha = 0$, where $\Lambda$ is the adjoint of the Lefschetz operator.
        \item Let $\omega_{FS}$ the Fubini-Study metric on $\mathbb{P}^n\C$. Show that $\int_{\mathbb{P}^n\C} \omega_{FS}^n = 1$. (Hint provided.)
        \item Conclude that $H^{2k}(\mathbb{P}^n\C, \Z) = \Z[\omega_{FS}^k]$. 
        \item Solve Exercise 2, Chapter 5 of Voisin. 
    \end{enumerate}
\end{problem}

\begin{solution}
    Noting down ideas here. 
    \begin{enumerate}
        \item First note that we have the decomposition:
        \[ H^k(X, E) = \bigoplus_{p+q=k} H^{p, q}(X, E)\]
        Thus, it suffices to show that $H^{p, q}(X, E)$ is finite dimensional. This is the cohomology of the Dobault complex. 
        \item First, trivialize the bundles $E, F, G$ over $U$. Then, locally for coordinates $(x_1, \ldots, x_n)$ on $M$, and $(\alpha_1, \cdots, \alpha_p)$ on $\Gamma(X, E)$, we have: 
        
    \end{enumerate}
\end{solution}
\newpage

\begin{problem}{2}
    Let $X$ be a Kahler compact manifold and $\omega$ a differential form which is $\partial$ and $\overline{\partial}$ closed. Prove that if $\omega$ is either $d$, $\partial$, or $\overline{\partial}$ exact, then there exists a differential form $\chi$ such that $\omega = \partial \overline{\partial} \chi$.
\end{problem}

\begin{solution}
    
\end{solution}
\newpage

\begin{problem}{x.}
\end{problem}

\begin{solution}

\end{solution}




\end{document}