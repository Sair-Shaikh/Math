% \documentclass[12pt]{amsart}
\documentclass[12pt]{article}

\usepackage{fullpage}
\usepackage{mdframed}
\usepackage{colonequals}
\usepackage{algpseudocode}
\usepackage{algorithm}
\usepackage{tcolorbox}
\usepackage[all]{xy}
\usepackage{proof}
\usepackage{mathtools}
\usepackage{bbm}
\usepackage{amssymb}
\usepackage{amsthm}
\usepackage{amsmath}
\usepackage{amsxtra}
\newcommand{\bb}{\mathbb}


\newtheorem{theorem}{Theorem}[section]
\newtheorem{corollary}{Corollary}[theorem]
\newtheorem{lemma}{Lemma}

\newcommand{\mathcat}[1]{\textup{\textbf{\textsf{#1}}}} % for defined terms

\newenvironment{problem}[1]
{\begin{tcolorbox}\noindent\textbf{Problem #1}.}
{\vskip 6pt \end{tcolorbox}}

\newenvironment{enumalph}
{\begin{enumerate}\renewcommand{\labelenumi}{\textnormal{(\alph{enumi})}}}
{\end{enumerate}}

\newenvironment{enumroman}
{\begin{enumerate}\renewcommand{\labelenumi}{\textnormal{(\roman{enumi})}}}
{\end{enumerate}}

\newcommand{\defi}[1]{\textsf{#1}} % for defined terms

\theoremstyle{remark}
\newtheorem*{solution}{Solution}

\setlength{\hfuzz}{4pt}

\newcommand{\calC}{\mathcal{C}}
\newcommand{\calF}{\mathcal{F}}
\newcommand{\C}{\mathbb C}
\newcommand{\N}{\mathbb N}
\newcommand{\Q}{\mathbb Q}
\newcommand{\R}{\mathbb R}
\newcommand{\Z}{\mathbb Z}
\newcommand{\br}{\mathbf{r}}
\newcommand{\RP}{\mathbb{RP}}
\newcommand{\CP}{\mathbb{CP}}
\newcommand{\nbit}[1]{\{0, 1\}^{#1}}
\newcommand{\bits}{\{0, 1\}^{n}}
\newcommand{\bbni}{\bigbreak \noindent}
\newcommand{\norm}[1]{\left\vert\left\vert#1\right\vert\right\vert}

\let\1\relax
\newcommand{\1}{\mathbf{1}}
\newcommand{\fr}[2]{\left(\frac{#1}{#2}\right)}

\newcommand{\vecz}{\mathbf{z}}
\newcommand{\vecr}{\mathbf{r}}
\DeclareMathOperator{\Cinf}{C^{\infty}}
\DeclareMathOperator{\Id}{Id}

\DeclareMathOperator{\Alt}{Alt}
\DeclareMathOperator{\ann}{ann}
\DeclareMathOperator{\codim}{codim}
\DeclareMathOperator{\End}{End}
\DeclareMathOperator{\Hom}{Hom}
\DeclareMathOperator{\id}{id}
\DeclareMathOperator{\M}{M}
\DeclareMathOperator{\Mat}{Mat}
\DeclareMathOperator{\Ob}{Ob}
\DeclareMathOperator{\opchar}{char}
\DeclareMathOperator{\opspan}{span}
\DeclareMathOperator{\rk}{rk}
\DeclareMathOperator{\sgn}{sgn}
\DeclareMathOperator{\Sym}{Sym}
\DeclareMathOperator{\tr}{tr}
\DeclareMathOperator{\img}{img}
\DeclareMathOperator{\CandE}{CandE}
\DeclareMathOperator{\CandO}{CandO}
\DeclareMathOperator{\argmax}{argmax}
\DeclareMathOperator{\first}{first}
\DeclareMathOperator{\last}{last}
\DeclareMathOperator{\cost}{cost}
\DeclareMathOperator{\dist}{dist}
\DeclareMathOperator{\path}{path}
\DeclareMathOperator{\parent}{parent}
\DeclareMathOperator{\argmin}{argmin}
\DeclareMathOperator{\excess}{excess}
\let\Pr\relax
\DeclareMathOperator{\Pr}{\mathbf{Pr}}
\DeclareMathOperator{\Exp}{\mathbb{E}}
\DeclareMathOperator{\Var}{\mathbf{Var}}
\let\limsup\relax
\DeclareMathOperator{\limsup}{limsup}
%Paired Delims
\DeclarePairedDelimiter\ceil{\lceil}{\rceil}
\DeclarePairedDelimiter\floor{\lfloor}{ \rfloor}


\newcommand{\dagstar}{*}

\newcommand{\tbigwedge}{{\textstyle{\bigwedge}}}
\setlength{\parindent}{0pt}
\setlength{\parskip}{5pt}


\begin{document}

\title{CS 40: Computational Complexity}

\author{Sair Shaikh}
\maketitle

% Collaboration Notice: Talked to Henry Scheible '26 to discuss ideas.

\begin{problem}{1}
(0.10) A space is called contractible if the identity map is nullhomotopic. Show that \( X \) is contractible if and only if for every space \( Y \), every map \( f : X \to Y \) is nullhomotopic. Similarly, show that \( X \) is contractible if and only if for every \( Y \), every map \( f : Y \to X \) is nullhomotopic.
\end{problem}
\begin{solution}
    $(\impliedby)$ Assume for every $Y$ every map $f : X \to Y$ is nullhomotopic. Then, in particular, picking $Y = X$ and $f = \id_X$, we have that $\id_X$ is nullhomotopic. Thus, $X$ is contractible. \\
    $(\implies)$ Assume that $X$ is contractible and $Y$ be any space. Thus, there exists a homotopy $H: X \times I \to X$ from $\id_X$ to a constant map $\lambda_c : X \to X$ for some $c \in X$. Then, for any map $f : X \to Y$, we claim that $H': X \times I \to Y$ defined by $H'(x, t) = f(H(x, t))$ is a homotopy from $f$ to the constant map $f(c)$. Similarly, for any map $g$ from $Y$ to $X$, we claim that $H'': Y \times I \to X$ defined by $H''(y, t) = H(g(y), t)$ is a homotopy from $g$ to the constant map $g(c)$. Note that $H'$ and $H''$ are continous as they are compositions of continous maps. Moreover, we check, for all $x \in X$ and $y \in Y$,
    \begin{align*}
        H'(x, 0) &= f(H(x, 0)) = f(x) \\
        H'(x, 1) &= f(H(x, 1)) = f(\lambda_c(x)) = f(c) \\
        H''(y, 0) &= H(g(y), 0) = g(y) \\
        H''(y, 1) &= H(g(y), 1) = \lambda_c(g(y)) = c
    \end{align*}
    Thus, $H'$ is a homotopy from $f$ to the constant map valued at $f(c)$ and $H''$ is a homotopy from $g$ to the constant map valued at $c$.
\end{solution}
\newpage

\begin{problem}{2}
Let \( A \subset X \) and suppose \( r : X \to A \) is a continuous map such that \( r(a) = a \) for all \( a \in A \) (i.e., \( r \) is a retraction of \( X \) onto \( A \)). If \( a_0 \in A \), show that
\[
r_* : \pi_1(X, a_0) \to \pi_1(A, a_0)
\]
is surjective. (Hint: Consider also the inclusion map of \( A \) into \( X \).)
\end{problem}
\begin{solution}
    Let $\iota: A \to X$ be the inclusion map. Then, note that for all $a \in A$: 
    \[ r \circ \iota (a) = a\]    
    Thus, for any $f: I \to A$, we have $r \circ \iota \circ f = f$. Moreover, we have $\iota_*: \pi_1(A, a_0) \to \pi_1(X, a_0)$ with $\iota_*([f]) = [\iota \circ f]$. Let $p \in \pi_1(A, a_0)$ be arbitrary. Then, we claim that $\iota_*([p]) \in \pi_1(X, a_0)$ maps to $[p]$ under $r_*$. We compute:
    \begin{align*}
        r_*(\iota_*([p])) &= r_*([\iota \circ p]) \\
        &= [r \circ \iota \circ p] \\
        &= [p] \\
    \end{align*} 
    Thus, as $[p]$ was arbitrary, we have shown that $r_*$ is surjective.
\end{solution}
\newpage


\begin{problem}{3}
(1.1.3) If \( X \) is a path-connected space, show that \( \pi_1(X) \) is abelian if and only if every change-of-basepoint isomorphism (\( \hat{\alpha} \) in class or \( \beta_h \) in Hatcher) depends only on the endpoints of the path.
\end{problem}
\begin{solution}
    $(\implies)$ Assume $\pi_1(X)$ is abelian. Let $\alpha_1, \alpha_2: I \to X$ be two paths from $x_0$ to $x_1$. Then, $\hat{\alpha_i}: \pi_1(X, x_0) \to \pi_1(X, x_1)$ are given by $\hat{\alpha_i}([f]) = [\overline{\alpha_i} \cdot f \cdot \alpha_i]$. Note that $\pi_1(X, x_1)$ is abelian as $\pi_1(X)$ is. We show, for any $[f] \in \pi_1(X, x_0)$ that:
    \begin{align*}
        \hat{\alpha_1}([f]) &= [\overline{\alpha_1} \cdot f \cdot \alpha_1] \\
        &= [\overline{\alpha_1} \cdot f \cdot \alpha_2 \cdot \overline{\alpha_2} \cdot \alpha_1] \\
        &= [\overline{\alpha_1} \cdot f \cdot \alpha_2] \cdot [\overline{\alpha_2} \cdot \alpha_1] \\
        &= [\overline{\alpha_2} \cdot \alpha_1] \cdot [\overline{\alpha_1} \cdot f \cdot \alpha_2] \\
        &= [\overline{\alpha_2} \cdot \alpha_1 \cdot \overline{\alpha_1} \cdot f \cdot \alpha_2] \\
        &= [\overline{\alpha_2} \cdot f \cdot \alpha_2] \\
        &= \hat{\alpha_2}([f])
    \end{align*} 
    Thus, $\hat{\alpha_1}$ depends only on the endpoints of the path. \bbni
    $(\impliedby)$ Let $x_0 \in X$ and $[f], [g] \in \pi_1(X, x_0)$ be arbitary. Assume $\hat{f} = \hat{g}$, since $f$ and $g$ have the same endpoints (they are loops based at $x_0$). It suffices to show that $[f] = [\overline{g} \cdot f \cdot g]$. We compute:
    \begin{align*}
        [f] &= [\overline{f} \cdot f \cdot f] \\
        &= \hat{f}([f]) \\
        &= \hat{g}([f]) \\
        &= [\overline{g} \cdot f \cdot g]
    \end{align*}
    Thus, $\pi_1(X, x_0)$ is abelian. Since $x_0$ was arbitrary, we have shown that $\pi_1(X)$ is abelian.
\end{solution}
\newpage


\begin{problem}{4}
(1.1.6) Note that a loop based at \( x_0 \) can be regarded as a continuous map of pointed spaces
\[
(S^1, (1, 0)) \to (X, x_0).
\]
Let \( [S^1, X] \) be the set of homotopy classes of maps from \( S^1 \) to \( X \) without conditions on basepoints. Then, there is a map \( \Phi : \pi_1(X, x_0) \to [S^1, X] \) that forgets the base points.
\begin{itemize}
    \item[(a)] Show that \( \Phi([f]) = \Phi([g]) \) if and only if \( [f] \) and \( [g] \) are conjugate in \( \pi_1(X, x_0) \).
    \item[(b)] Deduce that if \( X \) is path connected, then \( [S^1, X] \) is in bijection with conjugacy classes of \( \pi_1(X, x_0) \).
\end{itemize}
\end{problem}
\begin{solution}
    \bbni
    \begin{itemize}
        \item[(a)] Let $f, g: S^1 \to X$ be two loops based at $x_0$. \\
        $(\implies)$ Assume $\Phi([f]) = \Phi([g])$. Then there exists a homotopy $H: S^1 \times I \to X$ such that $H(x, 0) = f(x)$ and $H(x, 1) = g(x)$ (we can pick the representative for the conjugacy classes as there exists homotopies between any two representatives that we can compose). \bbni
        We define $\alpha_t: I \to X$ be the path that $x_0$ takes under the homotopy $H$ in the interval $[0, t]$ (appropriately rescaled). We then define the homotopy $H': S^1  \times I: \to X$ as: 
        \[ H'(x, t) = \alpha_t \cdot H(x, t) \cdot \overline{\alpha_t} \]
        Clearly, $\alpha_t = \overline{\alpha_t}= H(0, t)$ by definition. $H'$ is also continuous as it is a concatenation of continuous paths. Moreover, we check:
            \begin{align*}
                H'(x, 0) &= \alpha_0 \cdot H(x, 0) \cdot \overline{\alpha_0} \\
                &= f(x) \\
                H'(x, 1) &= \alpha_1 \cdot H(x, 1) \cdot \overline{\alpha_1} \\
                &= \alpha_1 \cdot g(x) \cdot \overline{\alpha_1} \\
                H'((1,0), t) &= \alpha_t \cdot H((1,0), t) \cdot \overline{\alpha_t} \\
                &= \alpha_t \cdot \overline{\alpha_t} \\
                &= x_0 
            \end{align*}
        Thus, $H'$ is a homotopy from $f$ to $\alpha_1 \cdot g \cdot \overline{\alpha_1}$. Note that $\alpha_1$ is the path of $x_0$ under the homotopy $H$. Thus, $\alpha_1(0) = H((0, 1), 0) = f(0) = x_0$ and $\alpha_1(1) = H((0, 1), 1) = g(0) = x_0$. Thus, $\alpha_1$ is a loop based at $x_0$. We have shown that $[f] = [\alpha_1 \cdot g \cdot \overline{\alpha_1}]$, thus $[f]$ and $[g]$ are conjugate in $\pi_1(X, x_0)$. \bbni  
        $(\impliedby)$ Assume $[f]$ and $[g]$ are conjugate in $\pi_1(X, x_0)$. Then, there exists a path $\alpha: I \to X$ such that $\alpha(0) = x_0$ and $\alpha(1) = x_0$ such that $[f] = [\overline{\alpha} \cdot g \cdot \alpha]$, via a path homotopy $H$. Let $\beta_t$ be the part of the loop $\alpha$ on the interval $[1-t, 1]$ (rescaled appropriately). Note that $\beta_1 = \alpha_1$. We define the homotopy $H': S^1 \times I \to X$ as: 
        \[H'(x, t) = \beta_t \cdot H(x, t) \cdot \overline{\beta_t }\]
        $H'$ is a well-defined continous map as its a concatenation of paths (which are compatible by definition). Finally, we check: 
        \begin{align*}
            H'(x, 0) &= \beta_0 \cdot H(x, 0) \cdot \overline{\beta_0} \\
            &= f(x) \\
            H'(x, 1) &= \beta_1 \cdot H(x, 1) \cdot \overline{\beta_1} \\
            &= \beta_1 \cdot \overline{\alpha_1 } \cdot g(x) \cdot \alpha_1 \cdot \overline{\beta_1}\\
            &= g(x)
        \end{align*}
        Thus, $H'$ is a homotopy from $f$ to $g$. Thus, $\Phi([f]) = \Phi([g])$.
        \item[(b)] In the previous part, we already showed that $\Phi$ provides an injection from the conjugacy classes of $\pi_1(X, x_0)$ to $[S^1, X]$. We need to show that $\Phi$ provides a surjection from the conjugacy classes of $\pi_1(X, x_0)$ to $[S^1, X]$. Since every element in a conjugacy class has the same image, it is sufficient to show that $\Phi$ is surjective. \bbni
        Let $f: S^1 \to X$ be a loop. By path-connectedness, we can pick a path $\alpha: I \to X$ such that $\alpha(0) = x_0$ and $\alpha(1) = f((1,0))$. Then, 
        \[\alpha \cdot f \cdot \overline{\alpha} \]
        is a loop based at $x_0$. We claim that $\Phi([\alpha \cdot f \cdot \overline{\alpha}]) = [f]$. Similar to before, define $\alpha_t$ to be the part of $\alpha$ on the interval $[1-t, 1]$. Define the homotopy $H: S^1 \times I \to X$ as follows:
        \[ H(x, t) = \alpha_t \cdot f(x) \cdot \overline{\alpha_t} \]
        Similar to before, $H$ is a well-defined continous map. We check: 
        \begin{align*}
            H(x, 0) &= \alpha_0 \cdot f(x) \cdot \overline{\alpha_0} \\
            &= f(x) \\
            H(x, 1) &= \alpha_1 \cdot f(x) \cdot \overline{\alpha_1} \\
            &= \alpha \cdot f(x) \cdot \overline{\alpha}
        \end{align*}
        Thus, $\Phi([\alpha \cdot f \cdot \overline{\alpha}]) = [f]$. Thus, $\Phi$ is surjective. Therefore, there is a bijection between the conjugacy classes of $\pi_1(X, x_0)$ and $[S^1, X]$.
    \end{itemize}
\end{solution}
\newpage

\begin{problem}{5}
Suppose that \( p : E \to B \) is a covering map where \( B \) is connected. Show that if \( p^{-1}(b_0) \) has \( k \) elements for some \( b_0 \in B \), then \( p^{-1}(b) \) has \( k \) elements for every \( b \in B \).
\end{problem}
\begin{solution}
    Let $b \in B$ be arbitrary. Since $p$ is a covering map, there exists open $U$ such that $b \in U$ and $p^{-1}(U) = \bigcup_{\alpha \in A} U_\alpha$ where $U_\alpha$ are disjoint open subsets in $E$ such that $p|_{U_\alpha}: U_\alpha \to U$ is a homeomorphism, where $A$ is some space. Since each $p|_{U_\alpha}$ is a homeomorphism onto $U$, we must have $p^{-1}(b) \cap U_\alpha$ be a singleton. Thus, $|p^{-1}(b)| \leq |A|$. Since the $U_\alpha$ are disjoint, each $U_\alpha$ must contain a distinct element, thus, $|p^{-1}(b)| = |A|$. Similarly, $\forall b' \in U$, the same arguments hold, thus, $|p^{-1}(b')| = |A|$. Thus, the size of the pre-image is constant for any two points within the same evenly covered open. \bbni
    Let $G_i = \{b \in B : p^{-1}(b) = i\}$ for $i \in \mathbb{N} \cup \{\infty\}$. For every $b \in G_i$, we know that there exists an evenly covered open $U_b$ such that every point in $U_b$ has $i$ pre-images. Thus, $U_b \in G_i$. Thus, $G_i$ is open for all $i$. Moreover, as $B \setminus G_i = \bigcup_{j \neq i} G_j$, is a union of opens, hence open. Thus, $G_i$ is also closed for all $i$. \bbni
    Since $B$ is connected, the only non-empty clopen set is $B$. We know that $G_k$ is non-empty as $b_0 \in G_k$. Since $G_k$ is clopen, $G_k = B$. Thus, $\forall b \in B$, $p^{-1}(b) = k$.
\end{solution}
\newpage

\begin{problem}{6}
Let \( q : X \to Y \) and \( r : Y \to Z \) be covering maps such that \( r \) has finite degree. Show that \( p = r \circ q \) is a covering map.
\end{problem}
\begin{solution}
    Let $z \in Z$ be some point. We need to show there exists a $U \subseteq Z$ with $z \in U$ such that $U$ is evenly covered by $p$. Since $r$ is a covering map of finite degree, call it $d$, there exists an evenly covered $U \subset Z$ such that: 
    \[ r^{-1}(U) = \bigcup_{i = 1}^d V_i \]
    where $V_i$ are disjoint open sets in $Y$ such that $r|_{V_i}: V_i \to U$ is a homeomorphism. \bbni
    We know that each $V_i$ has a unique point $y_i$ such that $r(y_i) = z$. Since $q$ is a covering map, there exists an evenly covered open $W_i \subseteq Y$ such that $y_i \in W_i$. Let $U' = \bigcap_{i = 1}^d r(V_i \cap W_i)$. Then, $z \in U'$ as $y_i \in V_i \cap W_i$ for all $i$. We will show that $U'$ is evenly covered by $p$. \bbni
    As $U' \subseteq U$ it is evenly covered by $r$ by restricting all homomorphisms to the pre-image of $U'$ in each $V_i \cap W_i$. Thus, we have:
    \[r^{-1}(U') = \bigcup_{i = i}^d B_{i}\]
    where $B_i \subset V_i \cap W_i$ are disjoint open sets in $Y$ such that $r|_{B_i}: B_i \to U'$ is a homeomorphism. \bbni
    Since $B_i \subseteq W_i$, and $W_i$ is evenly covered by $q$, $B_i$ is also evenly covered by restricting all the homeomorphisms onto $W_i$ to the pre-image of $B_i$. Thus, we have:
    \[q^{-1}(B_i) = \bigcup_{\alpha \in A_i} C_{\alpha}\]
    where $C_{\alpha}$ are disjoint open sets in $X$ such that $q|_{C_{\alpha}}: C_{\alpha} \to B_i$ is a homeomorphism. Thus, each $C_\alpha$ is homeomorphic to $U'$ under $p = r \circ q$. \bbni  
    Moreover, since the $B_i$ are disjoint, for $C_\alpha$ for $\alpha \in A_i$ is disjoint from $C_\beta$ if $\beta \not \in A_i$ as their images are disjoint. $W_\alpha$ already disjoint of $W_\beta$ for $\beta \in A_i$ by construction. \bbni 
    Let $A = \bigcup_{i = 1}^d A_i$. Then, we have:
    \[p^{-1}(U') = \bigcup_{\alpha \in A} C_\alpha\]
    where $C_\alpha$ are disjoint open sets in $X$ such that $p|_{C_\alpha}: C_\alpha \to U'$ is a homeomorphism for all $\alpha \in A$, where $U'$ is an open set containing $z$. Since $z$ was arbitrary, this shows that $p$ is a covering map.
\end{solution}

\end{document}