% \documentclass[12pt]{amsart}
\documentclass[12pt]{article}

\usepackage{fullpage}
\usepackage{mdframed}
\usepackage{colonequals}
\usepackage{algpseudocode}
\usepackage{algorithm}
\usepackage{tcolorbox}
\usepackage[all]{xy}
\usepackage{proof}
\usepackage{mathtools}
\usepackage{bbm}
\usepackage{amssymb}
\usepackage{amsthm}
\usepackage{amsmath}
\usepackage{amsxtra}
\newcommand{\bb}{\mathbb}


\newtheorem{theorem}{Theorem}[section]
\newtheorem{corollary}{Corollary}[theorem]
\newtheorem{lemma}{Lemma}

\newcommand{\mathcat}[1]{\textup{\textbf{\textsf{#1}}}} % for defined terms

\newenvironment{problem}[1]
{\begin{tcolorbox}\noindent\textbf{Problem #1}.}
{\vskip 6pt \end{tcolorbox}}

\newenvironment{enumalph}
{\begin{enumerate}\renewcommand{\labelenumi}{\textnormal{(\alph{enumi})}}}
{\end{enumerate}}

\newenvironment{enumroman}
{\begin{enumerate}\renewcommand{\labelenumi}{\textnormal{(\roman{enumi})}}}
{\end{enumerate}}

\newcommand{\defi}[1]{\textsf{#1}} % for defined terms

\theoremstyle{remark}
\newtheorem*{solution}{Solution}

\setlength{\hfuzz}{4pt}

\newcommand{\calC}{\mathcal{C}}
\newcommand{\calF}{\mathcal{F}}
\newcommand{\C}{\mathbb C}
\newcommand{\N}{\mathbb N}
\newcommand{\Q}{\mathbb Q}
\newcommand{\R}{\mathbb R}
\newcommand{\Z}{\mathbb Z}
\newcommand{\br}{\mathbf{r}}
\newcommand{\RP}{\mathbb{RP}}
\newcommand{\CP}{\mathbb{CP}}
\newcommand{\nbit}[1]{\{0, 1\}^{#1}}
\newcommand{\bits}{\{0, 1\}^{n}}
\newcommand{\bbni}{\bigbreak \noindent}
\newcommand{\norm}[1]{\left\vert\left\vert#1\right\vert\right\vert}

\let\1\relax
\newcommand{\1}{\mathbf{1}}
\newcommand{\fr}[2]{\left(\frac{#1}{#2}\right)}

\newcommand{\vecz}{\mathbf{z}}
\newcommand{\vecr}{\mathbf{r}}
\DeclareMathOperator{\Cinf}{C^{\infty}}
\DeclareMathOperator{\Id}{Id}

\DeclareMathOperator{\Alt}{Alt}
\DeclareMathOperator{\ann}{ann}
\DeclareMathOperator{\codim}{codim}
\DeclareMathOperator{\End}{End}
\DeclareMathOperator{\Hom}{Hom}
\DeclareMathOperator{\id}{id}
\DeclareMathOperator{\M}{M}
\DeclareMathOperator{\Mat}{Mat}
\DeclareMathOperator{\Ob}{Ob}
\DeclareMathOperator{\opchar}{char}
\DeclareMathOperator{\opspan}{span}
\DeclareMathOperator{\rk}{rk}
\DeclareMathOperator{\sgn}{sgn}
\DeclareMathOperator{\Sym}{Sym}
\DeclareMathOperator{\tr}{tr}
\DeclareMathOperator{\img}{img}
\DeclareMathOperator{\CandE}{CandE}
\DeclareMathOperator{\CandO}{CandO}
\DeclareMathOperator{\argmax}{argmax}
\DeclareMathOperator{\first}{first}
\DeclareMathOperator{\last}{last}
\DeclareMathOperator{\cost}{cost}
\DeclareMathOperator{\dist}{dist}
\DeclareMathOperator{\path}{path}
\DeclareMathOperator{\parent}{parent}
\DeclareMathOperator{\argmin}{argmin}
\DeclareMathOperator{\excess}{excess}
\let\Pr\relax
\DeclareMathOperator{\Pr}{\mathbf{Pr}}
\DeclareMathOperator{\Exp}{\mathbb{E}}
\DeclareMathOperator{\Var}{\mathbf{Var}}
\let\limsup\relax
\DeclareMathOperator{\limsup}{limsup}
%Paired Delims
\DeclarePairedDelimiter\ceil{\lceil}{\rceil}
\DeclarePairedDelimiter\floor{\lfloor}{ \rfloor}


\newcommand{\dagstar}{*}

\newcommand{\tbigwedge}{{\textstyle{\bigwedge}}}
\setlength{\parindent}{0pt}
\setlength{\parskip}{5pt}


\begin{document}

\title{CS 40: Computational Complexity}

\author{Sair Shaikh}
\maketitle

% Collaboration Notice: Talked to Henry Scheible '26 to discuss ideas.


\begin{problab}{1}
    Let $p: \R^{n+1} \setminus \{0\} \to \bb{RP}^n$ be the quotient map from HW1. For $n \geq 2$, show that $p|_{S^n}$ is a degree two cover and deduce that $\pi_1(\bb{RP}^n) \simeq \bb{Z}/2\bb{Z}$. (The fundamental group is $\bb{Z}$ for $n=1$ as $S^1 \cong \bb{RP}^1$.)
\end{problab}
\begin{solu}
    Recall the quotient map identifies points in $\R^{n+1} \setminus \{0\}$ as follows: 
    \[ \forall \lambda \neq 0 \in \R: \, (x_0, \cdots, x_n) \sim \lambda (x_0, \cdots, x_n)\]
    Let $(x_0, \cdots, x_n) \in S^n$ be a point on the sphere. Then, we additionally know that:
    \[ x_0^2 + x_1^2 + \cdots + x_n^2 = 1\]
    The equivalence class of this point on the sphere contains all points such that $\lambda(x_0, \cdots, x_n)$ such that:
    \[ (\lambda x_0)^2 + \cdots + (\lambda x_n)^2 = \lambda^2 (x_0^2 + \cdots + x_n^2) = 1\]
    Thus, we have $\lambda = \pm 1$. Thus, restricting $p$ to $S^n$, we get a quotient map that identifies antipodal points on the sphere, i.e. $x \sim -x$. \\
    Thus, each $[p] \in \RP^n$ has exactly two pre-images in $S^n$, i.e. $p|_{S^n}$ is a two to one mapping. \bbni
    Moreover, let $[x] \in \RP^n$ be a point. Then, $p|_{S^n}^{-1}([x]) = \{x, -x\}$. Let $V$ and $V'$ be two disjoint open sets in $S^n$ around $x$ and $-x$ ($S^n$ is Hausdorff). Let $W = V \cap -V'$ and $W' = V' \cap -V$. Then, $W$ and $W'$ are still disjoint. Moreover, $W$ and $W'$ are such that they contain antipodal points (by definition). \\
    Let $U = p(W)$. By the definition of $p$, we note that:
    \[ \rho|_{S^n}^{-1}(U) = W \bigsqcup W'\]
    Since $W$ and $W'$ are open, $U$ is open in $\RP^n$. Moreover, $p|_W(W) = U$ is a continous bijection of compact Hausdroff spaces, and is thus a homeomorphism. Similarly for $W'$. Thus, as $[x]$ was arbitrary, we have found an open set $U$ around $x$ whose pre-image is a disjoint union of two isomorphic copies of $U$. Thus, $p|_{S^n}$ is a degree $2$ covering map. \bbni 
    Moreover, we note that for $n \geq 2$, $S^n$ is simply connected. Thus, for any point $x \in \RP^n$, the map: 
    \[ \pi_1(\RP^n, x) \to p|_{S^n}^{-1}(x)\]
    is a bijection. Thus, $|\pi_1(\RP^n, x)| = 2$. Thus, 
    \[ \pi_1(\RP^n, x) = \Z/2\Z\]

\end{solu}
\newpage

\begin{problab}{2}
    Let $g\colon S^1 \to S^1$ be the covering map $g(z) = z^n$ for $n \in \bb{Z}$. Under the isomorphism $\pi_1(S^1, (1,0)) \cong \bb{Z}$ that we proved, compute $g_* \colon \bb{Z} \to \bb{Z}$ and the map $\phi_g: \bb{Z} \to g^{-1}(1,0)$ defined by lifting loops so that they still start at $(1,0)$.
\end{problab}
\begin{solu}
    Let $\gamma_k(s) = e^{2\pi i ks} \in \pi_1(S^1, (1, 0))$ be the loop that goes around the circle $k$ times. Recall the isomprhism $f: \pi_1(S^1, (1,0)) \to \bb{Z}$, which maps $f(\gamma_k) = k$. \bbni
    Next, we calculate $g_*([\gamma_k])$. We have:
    \begin{align*}
        g_*([\gamma_k]) &= [g \circ \gamma_k] \\
        &= [\gamma_k^n] \\
        &= [s \mapsto e^{2\pi i nk s}] \\
        &= [\gamma_{nk}]
    \end{align*}
    Thus, under the identification through the isomorphism, we have:
    \[g_*(k) = nk\]
    That is, $g_*$ is just multiplication by $n$.\bbni
    Next, we calculate $\phi_g: \pi_1(S^1, (1, 0)) \to g^{-1}((1,0))$. Notice that $(1,0)$ is identified to $z = 1 = e^0$. Thus, we have: 
    \[ g^{-1}(1) = \{z \in S^1: z^n = 1\} = \{e^{2\pi i k/n}: k = 0, \cdots, n-1\}\]
    For $\gamma_k \in \pi_1(S^1, (1,0))$, we have the lift: 
    \[\tilde{\gamma}_k = s \mapsto e^{2\pi i k/n}\]
    as we can verify: 
    \[ g \circ \tilde{\gamma}_k = s \mapsto (e^{2\pi i k/n})^n = \gamma_k\]
    Moroever, as $\tilde{\gamma}_k(0) = \gamma_k(0) = 1$, this is unique. Thus, we can evaluate $\phi_g$ as:
    \begin{align*}
        \phi_g(\gamma_k) &= \tilde{\gamma}_k(1) \\
        &= e^{2\pi i k/n}
    \end{align*}
    Beyond the requirements of the problem, it is also interesting to show that $g^{-1}(1)$ is the cyclic group of order $n$ under multiplication, since it contains the $n$th roots of unity. If we identify $e^{2i\pi k/n} \to k$, then we get an isomorphism from $C_n \to \Z/n\Z$ (written additively). This is a standard exercise. Under this identification, as well as the identification of $\pi_1(S^1, (1,0))$ with $\Z$, we have that: $\phi_g: \Z \to \Z/n\Z$ is given by:
    \[ \phi_g(k) = k\pmod n\]
\end{solu}
\newpage

\begin{problab}{3}
    Show that there are no retractions $r \colon X \to A$ in the following cases:
    \begin{enumerate}
    \item $X = \R^3$ and $A$ is any subspace homeomorphic to $S^1$.
    \item $X = S^1 \times D^2$ and $A$ is its boundary torus $S^1 \times S^1$.
    \item $X$ is the M{\" o}bius band and $A$ is its boundary circle.
    \end{enumerate}
\end{problab}
\begin{solu}
    \bbni
    \begin{enumerate}
        \item If we had such a retraction $r$, the push-forwards on the fundamental group will yield the following commutative diagram:
        \[ \xymatrix{ \pi_1(S^1) \ar[rr]^{\id_*} \ar[rd]^{\iota_*} & & \pi_1(S^1) \\ 
        & \pi_1(\R^3) \ar[ur]^{r_*}} \]
        that is:
        \[ \xymatrix{ \Z \ar[rr]^{\id_*} \ar[rd]^{\iota_*} & & \Z \\ 
        & \{1\} \ar[ur]^{r_*}} \]
        This implies that $\iota_*$ is injective, as $r_* \circ \iota_*$ is bijective. However, this is clearly impossible.
        \item Similar to the case above, noting that taking the product commutes with taking the fundamental group, a retraction $r$ would yield the following commutative diagram:
        \[ \xymatrix{ \Z \times \Z \ar[rr]^{\id_*} \ar[rd]^{\iota_*} & & \Z \times \Z\\ 
        & \Z \ar[ur]^{r_*}} \]       
        Similar to before, there is no injective homomorphism from $\Z \times \Z$ to $\Z$, since $\Z \times \Z$ is not cyclic and $\Z$ (and all its subgroups) are cyclic. Thus, $r$ cannot exist.
        \item Define the Mobius strip as:
        \[ M = [0,1] \times [0, 1] / \sim\]
        with the equivalence relation:
        \[ (0, y) \sim (1, 1-y)\]
        Next, let $C = \{[(s, 1/2)]: s\in [0, 1]\}$ be the central circle and $\iota_C: C \to M$ be the inclusion map. We define a retraction $r_C: M \to C$ as:
        \[r_C( [(s, t)]) = [(s, 1/2)]\]
        This is well-defined as: 
        \[r_C([(0, y)]) = [(0, 1/2)] = [(1, 1/2)] = r_C([1, 1-y])\]
        as is clearly continous as (topologically) its a projection onto the first coordinate. Thus,
        \[r_C \circ \iota_C = \id_C \]
        Moreover, we show that $\iota_C \circ r_C$ is homotopic to $\id_M$. We define the straight-line homotopy $H: M \times [0, 1] \to M$ as:
        \[H([(x, y)], t) = [(x, (1-t)y + t/2)]\]
        Similarly notice that this is well-defined as:
        \begin{align*}
            H([(0, y)], t) &= [(0, (1-t)y + t/2)] \\
            &= [(1, 1-(t/2+(1-t)y))] \\
            &= [(1, 1-t -(1-t)y + t/2)] \\
            &= [(1, (1-t)(1-y) + t/2)] \\
            &= H([(1, 1-y)], t)
        \end{align*}
        and is clearly continous as it is a sum of products of continous functions. Moroever, 
        \[H([x,y], 0) = [x,y] = \id_M([x,y]) \qquad H([x,y]) = [x,1/2] = r_C([x,y]))\] 
        Thus, we have: 
        \[[\iota_c \circ r_C] = [\id_M]\]
        Since $\pi_1$ is functorial and homotopy invariant, this implies that:
        \[ r_{C*} \circ \iota_{C*} = \id_{\pi_1(C)} \qquad \iota_{C*} \circ r_{C*} = \id_{\pi_1(M)}\]
        Thus, $\iota_{C*}$ is an isomorphism. \bbni 
        Let $B = \{[(s, t)]: t \in \{0, 1\} , s \in [0, 1]\}$ be the boundary circle of the Mobius strip and $\iota_B: B \to M$ be the inclusion map. Then, note that we have the continous map: $r_C \circ \iota_B: B \to C$. Taking the push-forward, if there was a retraction $r_B: M \to B$, we have the following commutative diagram: 
        \[ \xymatrix{ \pi_1(B) \ar[r]^{\id_*} \ar[d]^{r_{C*}} & \pi_1(B)\\ 
        \pi_1(C) \ar[r]^{\iota_{C*}} & \pi_1(M) \ar[u]^{r_{B*}}} \]   
        Finally, let $\gamma: I \to B$ be the generator of $\pi_1(B)$, i.e.:
        \[\gamma(t) = \begin{cases}
            [(2t, 0)] & t \in [0, 1/2] \\
            [(2t-1, 1)] & t \in [1/2, 1]
        \end{cases}\]  
        Note that $\gamma$ is well-defined loop as $\gamma(1/2) = [(1, 0)] = [(0, 1)]$ and $\gamma(0) = [(0, 0)] = [(1, 1)] = \gamma(1)$. Moreover, it is easy to see that $\gamma$ loops around the boundary circle once, and thus is a generator. \bbni 
        Finally, we calculate $\iota_{*C} \circ r_{*C}(\gamma)$. We have:
        \begin{align*}
            \iota_{*C} \circ r_{*C}(\gamma) &= [\iota_C \circ r_C \circ \gamma(t)] \\
            &= \begin{cases}
                [(2t, 1/2)] & t \in [0, 1/2] \\
                [(2t-1, 1/2)] & t \in [1/2, 1]
            \end{cases}
        \end{align*} 
        Thus, we notice that $\iota_{*C} \circ r_{*C}(\gamma)$ is a loop that goes around the center circle twice. In terms of the identification of the fundamental group with $\Z$, $\iota_{*C}\circ r_{*C}(1) = 2$. Thus, by our commutative diagram, if $r_B$ existed, we would have:
        \[ r_{B*}(2) = 1\]
        However, $r_{B*}$ is a group homomorphism, this implies: 
        \[ 1 = r_{B*}(2) = r_{B*}(1+1) = r_{B*}(1)+r_{B*}(1) \] 
        However, there is no integer whose sum with itself is $1$. Thus, $r_B$ cannot exist.
    \end{enumerate}
\end{solu}
\newpage

\begin{problab}{4}
    Use the intermediate value theorem to prove the $1$-dimensional version of the Brouwer fixed point theorem: If $f \colon I \to I$ is continuous, there is a point $x \in I$ such that $f(x) = x$. 
\end{problab}
\begin{solu}
    Notice first that if $f(1) = 1$ or $f(0) = 0$ we are done. Thus, we can assume that $f(0) > 0$ and $f(1) < 1$. Define $g: I \to [-1, 1]$ as $g(x) = f(x) - x$. $g$ is continous as it is the difference of two continous functions. Then, notice: 
    \[g(0) = f(0)-0 > 0 \qquad g(1) = f(1) - 1 < 0 \] 
    Thus, by the intermediate value theorem, there exists a point $c \in (0, 1)$ such that $g(c) = 0$. Thus, we have:
    \[f(c) - c = 0 \implies f(c) = c\] 
    and we are done.
\end{solu}
\newpage

\begin{problab}{5}
    Use the intermediate value theorem to prove the $1$-dimensional version of the Borsuk-Ulam theorem: If $f: S^1 \to \R$ is continuous, there is a point $x \in S^1$ such that $f(x) = f(-x)$.
\end{problab}
\begin{solu}
    Define $g: S^1 \to \R$ as:
    \[ g(x) = f(x) - f(-x)\]
    Notice that $g$ is continous as it is the sum of two continous functions. Moreover, notice that: 
    \[ g(-x) = f(-x)-f(x) = -g(x)\]
    We claim that $g(c) = 0$ for some $c \in S^1$. First, if $g$ is identically $0$, we pick any point to be $c$. Otherwise, we can assume that there exists a point $x \in S^1$ such that $g(x) \neq 0$. Then, since $g(-x) = -g(y)$, we have that $g$ takes both a positive and a negative value. Thus, there exists a point $c \in S^1$ (more specifically, on either arc from $x$ to $-x$) such that $g(c) = 0$. Thus, we have:
    \[g(c) = 0 \implies f(c) = f(-c)\]
    and we are done. 
\end{solu}

\end{document}