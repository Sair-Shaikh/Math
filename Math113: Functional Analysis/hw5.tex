\documentclass[12pt]{article}

\usepackage{fullpage}
\usepackage{mdframed}
\usepackage{colonequals}
\usepackage{algpseudocode}
\usepackage{algorithm}
\usepackage{tcolorbox}
\usepackage[all]{xy}
\usepackage{proof}
\usepackage{mathtools}
\usepackage{bbm}
\usepackage{amssymb}
\usepackage{amsthm}
\usepackage{amsmath}
\usepackage{amsxtra}
\newcommand{\bb}{\mathbb}


\newtheorem{theorem}{Theorem}[section]
\newtheorem{corollary}{Corollary}[theorem]
\newtheorem{lemma}{Lemma}

\newcommand{\mathcat}[1]{\textup{\textbf{\textsf{#1}}}} % for defined terms

\newenvironment{problem}[1]
{\begin{tcolorbox}\noindent\textbf{Problem #1}.}
{\vskip 6pt \end{tcolorbox}}

\newenvironment{enumalph}
{\begin{enumerate}\renewcommand{\labelenumi}{\textnormal{(\alph{enumi})}}}
{\end{enumerate}}

\newenvironment{enumroman}
{\begin{enumerate}\renewcommand{\labelenumi}{\textnormal{(\roman{enumi})}}}
{\end{enumerate}}

\newcommand{\defi}[1]{\textsf{#1}} % for defined terms

\theoremstyle{remark}
\newtheorem*{solution}{Solution}

\setlength{\hfuzz}{4pt}

\newcommand{\calC}{\mathcal{C}}
\newcommand{\calF}{\mathcal{F}}
\newcommand{\C}{\mathbb C}
\newcommand{\N}{\mathbb N}
\newcommand{\Q}{\mathbb Q}
\newcommand{\R}{\mathbb R}
\newcommand{\Z}{\mathbb Z}
\newcommand{\br}{\mathbf{r}}
\newcommand{\RP}{\mathbb{RP}}
\newcommand{\CP}{\mathbb{CP}}
\newcommand{\nbit}[1]{\{0, 1\}^{#1}}
\newcommand{\bits}{\{0, 1\}^{n}}
\newcommand{\bbni}{\bigbreak \noindent}
\newcommand{\norm}[1]{\left\vert\left\vert#1\right\vert\right\vert}

\let\1\relax
\newcommand{\1}{\mathbf{1}}
\newcommand{\fr}[2]{\left(\frac{#1}{#2}\right)}

\newcommand{\vecz}{\mathbf{z}}
\newcommand{\vecr}{\mathbf{r}}
\DeclareMathOperator{\Cinf}{C^{\infty}}
\DeclareMathOperator{\Id}{Id}

\DeclareMathOperator{\Alt}{Alt}
\DeclareMathOperator{\ann}{ann}
\DeclareMathOperator{\codim}{codim}
\DeclareMathOperator{\End}{End}
\DeclareMathOperator{\Hom}{Hom}
\DeclareMathOperator{\id}{id}
\DeclareMathOperator{\M}{M}
\DeclareMathOperator{\Mat}{Mat}
\DeclareMathOperator{\Ob}{Ob}
\DeclareMathOperator{\opchar}{char}
\DeclareMathOperator{\opspan}{span}
\DeclareMathOperator{\rk}{rk}
\DeclareMathOperator{\sgn}{sgn}
\DeclareMathOperator{\Sym}{Sym}
\DeclareMathOperator{\tr}{tr}
\DeclareMathOperator{\img}{img}
\DeclareMathOperator{\CandE}{CandE}
\DeclareMathOperator{\CandO}{CandO}
\DeclareMathOperator{\argmax}{argmax}
\DeclareMathOperator{\first}{first}
\DeclareMathOperator{\last}{last}
\DeclareMathOperator{\cost}{cost}
\DeclareMathOperator{\dist}{dist}
\DeclareMathOperator{\path}{path}
\DeclareMathOperator{\parent}{parent}
\DeclareMathOperator{\argmin}{argmin}
\DeclareMathOperator{\excess}{excess}
\let\Pr\relax
\DeclareMathOperator{\Pr}{\mathbf{Pr}}
\DeclareMathOperator{\Exp}{\mathbb{E}}
\DeclareMathOperator{\Var}{\mathbf{Var}}
\let\limsup\relax
\DeclareMathOperator{\limsup}{limsup}
%Paired Delims
\DeclarePairedDelimiter\ceil{\lceil}{\rceil}
\DeclarePairedDelimiter\floor{\lfloor}{ \rfloor}


\newcommand{\dagstar}{*}

\newcommand{\tbigwedge}{{\textstyle{\bigwedge}}}
\setlength{\parindent}{0pt}
\setlength{\parskip}{5pt}


\begin{document}

\title{CS 40: Computational Complexity}

\author{Sair Shaikh}
\maketitle

% Collaboration Notice: Talked to Henry Scheible '26 to discuss ideas.


\begin{problem}{54}
    Suppose that $X$ is a reflexive Banach space. Show that the unit ball $B = \{x \in X: \norm{x} \leq 1\}$ is weakly compact. (Hint provided). 
\end{problem}
\begin{solution}
    Let $X^*$ be the dual of $X$. By Alaoglu's theorem, we know that $B^{**} = \{f \in X^{**}: \norm{f} \leq 1\}$ is compact in $\sigma(X^{**}, X^*)$. Thus, it suffices to show that $B$ under the weak topology on $X$ is homeomorphic to $B^{**}$ in the weak-* topology on $X^{**}$ under the natural map $\iota$. \bbni
    Since $X$ is reflexive, we already know that $\iota: X \to X^{**}$ is a bijection. Recall the definition of an basic open set in the weak* topology on $X^{**}$, for $f_0 \in X^{**}$, $\phi_1, \cdots, \phi_k \in X^*$ and $\epsilon > 0$:
    \[ U = \{f \in X^{**}: |f(\phi_i)-f_0(\phi_i)| < \epsilon, 1 \leq i\leq k\}\]
    Noting that $\iota$ is bijective, we may rewrite this as:
    \begin{align*}
       U &= \{\iota(x) \in X^{**}: |\iota(x)(\phi_i)-\iota(x_0)(\phi_i)| < \epsilon, 1 \leq i\leq k\}  \\
         &= \{\iota(x) \in X^{**}: |\phi_i(x)-\phi_i(x_0)| < \epsilon, 1 \leq i\leq k\}
    \end{align*}
    The preimage of $U$ under $\iota$ then is precisely:
    \begin{align*}
        \iota^{-1}(U) &= \{x \in X: |\phi_i(x)-\phi_i(x_0)| < \epsilon, 1 \leq i\leq k\}    
    \end{align*}
    which is a basic open set in the weak topology on $X$. Thus, $\iota$ is continous. \bbni
    Similarly, let $V$ be a basic open set in the weak topology on $X$, written, for some $x_0 \in X$, $\phi_1, \cdots \phi_k \in X^*$ and $\epsilon > 0$ as:
    \[ V = \{x \in X: |\phi_i(x)-\phi_i(x_0)| < \epsilon, 1 \leq i\leq k\}\]
    Then, we have that:
    \begin{align*}
        \iota(V) &= \{\iota(x) \in X^{**}: |(\phi_i(x))-\phi_i(x_0)| < \epsilon, 1 \leq i\leq k\} \\
        &= \{\iota(x) \in X^{**}: |\iota(x)(\phi_i)-\iota(x_0)(\phi_i)| < \epsilon, 1 \leq i\leq k\}
    \end{align*}
    which is a basic open set in the weak* topology on $X^{**}$. Thus, $\iota^{-1}$ is also continous. \bbni
    Thus, $\iota$ is a homeomorphism from $X$ with the weak topology to $X^{**}$ with the weak* topology. Finally, note that as $\iota$ is isometric,
    \begin{align*}
        \iota(B) &= \{\iota(x): ||\iota(x)|| \leq 1 \} = B^{**}
    \end{align*}
    Thus, as $B^{**}$ is compact in the weak* topology, we have that $B$ is weakly compact in the weak topology on $X$.
\end{solution}
\newpage

\begin{problem}{59}
    Let $E$ be a nonempty subset of a Hilbert space $H$. Let $Y$ be a subspace spanned by $E$. Then $E^{\perp\perp}$ is the closure of $Y$ in $H$.
\end{problem}
\begin{solution}
    We show both inclusions separately. \bbni
    Let $y \in Y$ be arbitrary. Then, we have that $y = \sum_{i=1}^n a_i e_i$ for some $e_i \in E$ and $a_i \in \F$ ($\R$ or $\C$). Then, for all $w \in E^\perp$, we have that: 
    \begin{align*}
        (y \mid w) &= \left(\sum_{i=1}^n a_i e_i \mid w\right) \\
        &= \sum_{i=1}^n a_i (e_i \mid w) \\
        &= 0
    \end{align*}
    Thus, $y \in E^{\perp\perp}$. Therefore, $Y \subseteq E^{\perp\perp}$ (as $y$ was arbitrary). As noted in class, since the inner-product is continous, $E^{\perp\perp}$ is closed. Thus, we have that $\overline{Y} \subseteq E^{\perp\perp}$. \bbni
    Next, let $z \in E^{\perp\perp}$. Since $\overline{Y}$ is closed, we have that $H = \overline{Y} \oplus \overline{Y}^\perp$. Thus, we can write $z = y + y'$ where $y \in \overline{Y}$ and $y' \in \overline{Y}^\perp$. It suffices to show $y' = 0$ and thus, $z = y \in \overline{Y}$, implying $E^{\perp\perp} \subseteq \overline{Y}$ (as $z$ was arbitrary). \bbni
    Let $w \in \overline{Y}^\perp$. Then, as $E \subset \overline{Y}$, we have that:
    \[ (w \mid e) = 0 \qquad \forall e \in E\] 
    Thus, we have $w \in E^\perp$, thus, $\overline{Y}^\perp \subseteq E^\perp$. In particular, $y' \in E^\perp$. Moreover, $(y \mid y') = 0$ as $y \in \overline{Y}$ and $y' \in \overline{Y}^\perp$. Since $z \in E^{\perp\perp}$, we have:
    \begin{align*}
        0 &= (z \mid y') \\
        &= (y \mid y') + (y' \mid y') \\
        &= (y' \mid y')
    \end{align*} 
    Thus, $y' = 0$ by positive definiteness of the inner product. Therefore, $z = y \in \overline{Y}$. Thus, we have that $E^{\perp\perp} \subseteq \overline{Y}$. \bbni
    Therefore, we have that $E^{\perp\perp} = \overline{Y}$.
\end{solution}
\newpage

\begin{problem}{60}
    Let $X = l^2$. Show that the sequence $\{e_n\}$ of standard basis vectors converges weakly to $0$. 
\end{problem}
\begin{solution}
    Let $\phi \in (l^2)^*$ and $\epsilon > 0$. We need to show that there exists an $N$ such that for all $n \geq N$, $e_n \in U := \{x \in l^2: |\phi(x)| < \epsilon\}$. Since $(l^2) \cong (l^2)^*$ (isometric isomorphism) via the map $y \to \Phi_y$ defined by: 
    \[ \Phi_y(x) = \sum_{n=1}^\infty x_ny_n\]
    Thus, there exists a $y \in l^2$ such that $\phi = \Phi_y$. Then, we have:
    \[ \phi(e_n) = y_n \]
    Then, since $y \in l^2$, we have that $\sum_{n=1}^\infty |y_n|^2 < \infty$. Thus, there exists an $N$, such that $\sum_{n=N+1}^\infty |y_n|^2 < \epsilon^2$. Thus, for all $n \geq N$, we have that: 
    \[ |\phi(e_n)| = |y_n| < \epsilon\]
    Thus, $e_n \in U$ for all $n \geq N$. Therefore, $\{e_n\}$ converges weakly to $0$. \bbni
    % If $l^2$ is over the field of complex numbers, then we can use the same argument, but take the conjugate of $y_n$ in the definition of $\Phi_y$.  
\end{solution}
\newpage

\begin{problem}{61}
    Let $H$ be a hilbert space. If $x, y \in H$, define $\Theta_{x,y}: H \to H$ by:
    \[ \Theta_{x,y}(z) = (z \mid y)x\]
    Compute the norm of $\Theta_{x,y}$ and its adjoint $\Theta_{x,y}^*$.
\end{problem}
\begin{solution}
    First, note that if $x = 0$ or $y = 0$, then $\Theta_{x,y}(z) = (z\mid y)x = 0$ for all $z \in H$. Thus, $\norm{\Theta_{x,y}} = 0$. Thus, assume $x \ne 0$ and $y \ne 0$.
    % Moreover, we also have for $z, z' \in H$: 
    % \begin{align*}
    %     0 = (\Theta_{x,y} z \mid z') &= (z \mid \Theta^*_{x,y}(z'))
    % \end{align*}
    For $z \in H$ by homogeneity of the norm and Cauchy-Schwarz we have:
    \begin{align*}
        \norm{\Theta_{x,y}(z)} &= \norm{(z \mid y)x} \\
        &= |(z \mid y)| \norm{x} \\
        &\leq \norm{z} \norm{y} \norm{x}
    \end{align*}
    Thus, $\norm{\Theta_{x,y}} \leq \norm{y} \norm{x}$. Let $z = \frac{y}{\norm{y}}$. Then, $\norm{z} = 1$. Moreover, we have:
    \begin{align*}
        \norm{\Theta_{x,y}\left(\frac{y}{\norm{y}}\right)} &= \left|\left(\frac{y}{\norm{y}} \mid y\right)\right| \norm{x} \\
        &= \frac{1}{\norm{y}}\norm{y}^2 \norm{x} \\
        &= \norm{y}\norm{x}
    \end{align*}
    Thus, the bound is achieved. Therefore, $\norm{\Theta_{x,y}} = \norm{y}\norm{x}$. This also captures the $x = 0$ or $y=0$ case. \bbni
    Next, let $z, w \in H$. Then, we have that: 
    \begin{align*}
        (\Theta_{x,y} z \mid w) &= ((z \mid y)x \mid w) \\
        &= (z \mid y)(x \mid w) \\
        &= (z \mid y) \overline{(w \mid x)} \\
        &= (z \mid (w \mid x)y)
    \end{align*}
    Thus, we can define $\Theta_{x,y}^*(w) = (w \mid x)y$. Notice that:
    \[ \Theta^*_{x,y} = \Theta_{y,x} \]
    Thus, by the previous argument, we have that: 
    \[ \norm{\Theta_{x,y}^*} = \norm{\Theta_{y,x}} = \norm{y}\norm{x} \]
    
\end{solution}
\newpage

\begin{problem}{64}
    Let $P \in \mathcal{L}(H)$ be the orthogonal projection onto a nonzero subspace $W$. Show that $P = P^* = P^2$ and that $\norm{P} = 1$. Conversely, show that if $P \in \mathcal{L}(H)$ and $P = P^* = P^2$, then $P$ is an orthogonal projection onto its range.
\end{problem}
\begin{solution}
    We assume $W$ is closed as the definition of $P$ requires it. Let $P^*$ be the adjoint of $P$. Note that $x - Px \in W^{\perp}$ for all $x \in H$. Let $h, k \in H$. Note that: 
    \begin{align*}
        (Ph \mid k) &= (Ph \mid Pk + (k - Pk)) \\
        &= (Ph \mid Pk) + (Ph \mid k-Pk) \\
        &= (Ph \mid Pk) + 0 \\
        &= (Ph \mid Pk) + (h-Ph \mid Pk) \\
        &= (Ph+(h-Ph) \mid Pk) \\
        &= (h \mid Pk)
    \end{align*}
    Thus, $P$ is self-adjoint. Moreover, note that: 
    \begin{align*}
        (P^2 h \mid k) &= (Ph \mid Pk) \\
        &= (Ph \mid Pk) + (Ph \mid k-Pk) \\
        &= (Ph \mid k)
    \end{align*}
    for all $h, k \in H$. Thus, $P^2 = P$. \bbni
    First, note that by the Pythagorean theorem, for all $h \in H$,
    \begin{align*}
        \norm{Ph}^2 &\leq \norm{Ph}^2 + \norm{h - Ph}^2 \\
        &= \norm{h}^2
    \end{align*}
    Thus, $\norm{Ph} \leq \norm{h}$. Therefore, $\norm{P} \leq 1$. \bbni
    Next, pick $h \in W$ with $\norm{h} = 1$ (we can do this as $W$ is a subspace and we can divide any non-zero vector by its norm to get norm $1$). Then, $Ph = h$, thus $\norm{Ph} = \norm{h}$. Thus, $\norm{P} \geq 1$ (by the sup definition of the operator norm). Therefore, $\norm{P} = 1$. \bbni
    Conversely, let $P \in \mathcal{L}(H)$ such that $P = P^* = P^2$. We need to show that $P$ is an orthogonal projection onto $V := \img(P)$. Thus, we need to show that $V$ is closed and $H = V \oplus V^\perp$. \bbni
    Let $x \in H$ and $Py \in V$ be arbitrary. Then note: 
    \begin{align*}
        (x-Px \mid Py) &= (x \mid Py) - (Px \mid Py) \\
        &= (x \mid Py) - (x \mid P^*Py) \\
        &= (x \mid Py) - (x \mid P^2y) \\
        &= (x \mid Py) - (x \mid Py) \\
        &= 0
    \end{align*}
    Thus, as $Py$ was arbitrary, $x - Px \in V^\perp$. Therefore, we have that $H = V + V^\perp$. Next, let $v \in V \cap V^\perp$. Then, we have $(v \cap v) = 0$ thus $v = 0$. Thus, $V \cap V^\perp = \{0\}$. Therefore, we have that $H = V \oplus V^\perp$. \bbni
    Then, note that $V^\perp$ is closed as the inner product is continous (as we noted in class). Then, we note that $H = V^\perp \oplus V^{\perp\perp}$ by the orthgonal projections theorem we proved in class. Moreover, $V^{\perp\perp} = \overline{V}$ (Problem 59). Thus, we have:
    \[ H = V \oplus V^\perp = \overline{V} \oplus V^\perp \]
    Then, let $x \in \overline{V}\setminus V$. Then, as $H = V \oplus V^\perp$, we must have $x \in V^\perp$. However, $\overline{V} \cap V^\perp = \{0\}$, thus $x = 0$. But $0 \in V$ as $V$ is a subspace. Thus, we have a contradiction, and $\overline{V} \subseteq V$ is empty. Thus, $V = \overline{V}$, i.e. $V$ is closed. \bbni
    Thus, we have shown that $P$ is an orthogonal projection onto its range $V = \img(P)$. \bbni
    % Check: Is this what the definition of orthogonal projection is?
    % Then, let $x \in \overline{V}$. Then, the orthogonal projection of $x$ onto $V^\perp$ is $0$. 
    %  $x = v + v'$ for some $v \in V$ and $v' \in V^\perp$.  
    % Thus, it follows that $V = \overline{V}$, i.e. $V$ is closed. 
\end{solution}
\newpage

\begin{problem}{65}(Dini's Theorem)
    Suppose that $X$ is a compact metric space and that $C(X)$ is the Banach space of real-valued functions on $X$. Show that if $(f_n) \subset C(X)$ is such that there is a $f \in C(X)$ such that $f_n(X) \nearrow f(x)$ for all $x \in X$, then $f_n \to f$ in $C(X)$. Equivalently, show that $f_n \to f$ uniformly on $X$. (Hint provided).
\end{problem}
\begin{solution}
    Let $(f_n) \subset C(X)$ be such that $f_n(x) \nearrow f(x)$ for all $x \in X$. We need to show that $f_n \to f$ uniformly on $X$. \bbni
    Let $\epsilon > 0$ be arbitrary. 
    % For each $x \in X$, there exists an $N_x$ such that for all $n \geq N_x$, we have:
    % \[ |f(x) - f_n(x)| < \epsilon\]
    Let $E_n = \{ x \in X : |f(x)-f_n(x)| < \epsilon\}$. Then, as $f_{n}(x) \leq f_{n+1}(x)$, we have that $|f(x) - f_{n+1}(x)| \leq |f(x) - f_n(x)|$ for all $x \in X$ for all $n$. Thus, we have that $E_n \subseteq E_{n+1}$ for all $n$. Moreover, for each $x \in X$, as $f_n(x) \to f(x)$, there exists $N_x$ such that for all $n \geq N_x$,
    \[ |f(x) -f_n(x)| < \epsilon\]
    Thus, $x \in E_n$ for all $n \geq N_x$. Therefore, we have that $\{E_n\}$ is a cover for $X$. \bbni
    Moreover, since $f$ and $f_n$ are continous, so is $f-f_n$ for all $n$. As $\{c \in \R : |c| < \epsilon\}$ is open, we have that $E_n$ is the continous preimage of an open set, thus is open. Thus, $\{E_n\}$ is an open cover for $X$. \bbni
    Since $X$ is compact, there exista finite subcover $\{E_{n_1}, \ldots, E_{n_k}\}$ such that:
    \[ X = \bigcup_{i=1}^k E_{n_i}\]
    Let $N = \max\{n_1, \ldots, n_k\}$. Then, as noted previously, $E_{n_i} \subseteq E_N$ for all $i$. Thus for any $x \in X$, there exists an $i$ such that $x \in E_{n_i} \subseteq E_N$. Thus, $X = E_N$. \bbni
    Moreover, we also note that the containments $E_N \subseteq E_{N+1} \subseteq \cdots$ imply that $E_m = X$ for all $m \geq N$. Thus, we have that for $m \geq N$, for all $x \in X$,:
    \[ x\in E_m \iff |f(x) - f_m(x)| < \epsilon\]
    Thus, $(f_n) \to f$ uniformly on $X$.
\end{solution}
\newpage

\begin{problem}{66}
    A linear map $V: H \to H$ is called an isometry if $\norm{V(x)} = \norm{x}$ for all $x \in H$. Show that the following are equivalent:
    \begin{enumerate}
        \item $V$ is an isometry.
        \item $(V(x) \mid V(y)) = (x \mid y)$ for all $x, y \in H$.
        \item $V^*V = I$.
    \end{enumerate}
\end{problem}
\begin{solution}
    We show $(1) \implies (2)$, $(2) \implies (3)$, and $(3) \implies (1)$. \bbni
    ($(1) \implies (2)$). Let $x, y \in H$ be arbitrary. Then, by the polarization identity (for $\F = \C$), we have:
    \begin{align*}
        (Vx \mid Vy) &= \frac{1}{4}\left(\norm{Vx+Vy}^2 - \norm{Vx-Vy}^2 + i\norm{Vx+iVy}^2 - i\norm{Vx-iVy}^2\right) \\
        &= \frac{1}{4}\left(\norm{V(x+y)}^2 - \norm{V(x-y)}^2 + i\norm{V(x+iy)}^2 - i\norm{V(x-iy)}^2\right) \\
        &= \frac{1}{4}\left(\norm{x+y}^2 - \norm{x-y}^2 + i\norm{x+iy}^2 - i\norm{x-iy}^2\right) \\
        &= (x \mid y) 
    \end{align*}
    and for $\F = \R$, we have:
    \begin{align*}
        (Vx \mid Vy) &= \frac{1}{4}\left(\norm{Vx+Vy}^2 - \norm{Vx-Vy}^2\right) \\
        &=  \frac{1}{4}\left(\norm{V(x+y)}^2 - \norm{V(x-y)}^2\right) \\
        &= \frac{1}{4}\left(\norm{x+y}^2 - \norm{x-y}^2\right) \\
        &= (x \mid y)
    \end{align*}
    Thus, we have that $(Vx \mid Vy) = (x \mid y)$ for all $x, y \in H$. \bbni
    ($(2) \implies (3)$). Let $y \in H$ be arbitrary. Then, for all $x \in H$, we have:
    \begin{align*}
       (x \mid y) &= (Vx \mid Vy) \\
       &= (x \mid V^*Vy) 
    \end{align*}
    Thus, $(x \mid (y-V^*Vy)) = 0$ for all $x \in H$. Letting $x = y - V^*Vy$, we have that $y - V^*Vy = 0$ by the positive definiteness of the inner product. Thus, $V^*Vy = y$. Since $y$ was arbitrary, $V^*V = I$. \bbni
    ($(3) \implies (1)$). Let $x \in H$ be arbitrary. Then, we have:
    \begin{align*}
        \norm{Vx}^2 &= (Vx \mid Vx) \\
        &= (x \mid V^*Vx) \\
        &= (x \mid Ix) \\
        &= (x \mid x) \\
        &= \norm{x}^2
    \end{align*}
    Thus, $\norm{Vx} = \norm{x}$ for all $x \in H$. Therefore, $V$ is an isometry. \bbni
    Thus, we have shown that the three statements are equivalent.
\end{solution}
\newpage

\begin{problem}{67}
    A surjective isometry $U: H \to H$ is called a unitary. Show that the following are equivalent for $U \in \mathcal{H}$. 
    \begin{enumerate}
        \item $U$ is a unitary.
        \item $U$ is invertible with $U^{-1} = U^*$.
        \item If $\{e_n\}$ an orthonormal basis for $H$, then $\{U(e_n)\}$ is an orthonormal basis for $H$.
    \end{enumerate}
    (Remark: $(c)$ implies $(a)$ is not true unless $U$ is both linear and bounded.)
\end{problem}
\begin{solution}
    We show $(1) \implies (2)$, $(2) \implies (3)$, and $(3) \implies (1)$. \bbni
    ($(1) \implies (2)$). Let $U$ be a unitary. Then, as $U$ is an isometry, we have that: 
    \[ \norm{Ux} = 0 \iff \norm{x} = 0 \iff x = 0\]
    Thus, $U$ is injective. Since $U$ is surjective, we have that $U$ is a bijection. By a direct corollary of the Open Mapping theorem, we have that $U$ is invertible (proved in class). Then, as $U^*U = I$ (Problem 66), we have that $U^* = U^{-1}$ by the uniqueness of the inverse. \bbni 
    ($(2) \implies (3)$). Let $\{e_n\}$ be an orthonormal basis for $H$. Let $h \in H$ be such that:
    \[ (h \mid U(e_n)) = 0\]
    for all $n$. It suffices to show that this implies $h = 0$ by one of the alternative characterizations of an orthonormal basis (proved in class). \bbni 
    As $U$ is inveritble, hence surjective, there exists $h' \in H$ such that $U(h') = h$. Moreover, as $U^* = U^{-1}$ implies $U^*U = I$, we use the results of Problem 66 to get:
    \[ (U(h') \mid U(e_n)) = (h' \mid e_n) = 0\]
    for all $n$. Then, as $\{e_n\}$ is an orthonormal basis for $H$, we have that $h' = 0$. Then, $h = U(h') = 0$. Thus, $\{U(e_n)\}$ is an orthonormal basis.  \bbni
    ($(3) \implies (1)$). Assume that if $\{e_n\}$ is an orthonormal basis for $H$, then $\{U(e_n)\}$ is an orthonormal basis for $H$. Since $\{Ue_n\}$ is an orthonormal basis, we can write $x \in H$ as:
    \[ x = \sum_{n=1}^\infty (x \mid Ue_n)Ue_n\]
    Then, as $U$ is linear and continous, we can write:
    \begin{align*}
        x &= \sum_{n=1}^\infty (x \mid Ue_n)Ue_n \\
        &= \lim_{N \to \infty} \sum_{n=1}^N (x \mid Ue_n)Ue_n \\
        &= U\left(\lim_{N \to \infty} \sum_{n=1}^N (x \mid Ue_n)e_n\right) \\
        &= U\left(\sum_{n=1}^\infty (x \mid Ue_n)e_n\right)
    \end{align*}
    Thus, $U$ is surjective. Now, let $x, y \in H$ be arbitrary such that:
    \[ x = \sum_{n=1}^\infty (x \mid e_n)e_n \qquad y = \sum_{n=1}^\infty (y \mid e_n)e_n\]
    Then, we have that:
    \[ Ux = \sum_{n=1}^\infty (x \mid e_n)Ue_n \qquad Uy = \sum_{n=1}^\infty (y \mid e_n)Ue_n\]
    using the linearity and boundedness of $U$ (as above). Thus, we have:
    \[ (Ux \mid Ue_n) = (x \mid e_n)\]
    and similarly for $Uy$. Then, using alternative characterizations of an orthnormal basis, we have:
    \begin{align*}
        (Ux \mid Uy) &= \sum_{n=1}^\infty (Ux \mid Ue_n)(Ue_n \mid Uy) \\
        &=  \sum_{n=1}^\infty (x \mid e_n)(e_n \mid y) \\
        &= (x \mid y)
    \end{align*}
    Thus, by Problem 66, we have that $U$ is an isometry. Thus, $U$ is a unitary. \bbni
    Thus, we have shown that the three statements are equivalent.
\end{solution}
\newpage
\end{document}
