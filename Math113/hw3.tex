\documentclass[12pt]{article}

\usepackage{fullpage}
\usepackage{mdframed}
\usepackage{colonequals}
\usepackage{algpseudocode}
\usepackage{algorithm}
\usepackage{tcolorbox}
\usepackage[all]{xy}
\usepackage{proof}
\usepackage{mathtools}
\usepackage{bbm}
\usepackage{amssymb}
\usepackage{amsthm}
\usepackage{amsmath}
\usepackage{amsxtra}
\newcommand{\bb}{\mathbb}


\newtheorem{theorem}{Theorem}[section]
\newtheorem{corollary}{Corollary}[theorem]
\newtheorem{lemma}{Lemma}

\newcommand{\mathcat}[1]{\textup{\textbf{\textsf{#1}}}} % for defined terms

\newenvironment{problem}[1]
{\begin{tcolorbox}\noindent\textbf{Problem #1}.}
{\vskip 6pt \end{tcolorbox}}

\newenvironment{enumalph}
{\begin{enumerate}\renewcommand{\labelenumi}{\textnormal{(\alph{enumi})}}}
{\end{enumerate}}

\newenvironment{enumroman}
{\begin{enumerate}\renewcommand{\labelenumi}{\textnormal{(\roman{enumi})}}}
{\end{enumerate}}

\newcommand{\defi}[1]{\textsf{#1}} % for defined terms

\theoremstyle{remark}
\newtheorem*{solution}{Solution}

\setlength{\hfuzz}{4pt}

\newcommand{\calC}{\mathcal{C}}
\newcommand{\calF}{\mathcal{F}}
\newcommand{\C}{\mathbb C}
\newcommand{\N}{\mathbb N}
\newcommand{\Q}{\mathbb Q}
\newcommand{\R}{\mathbb R}
\newcommand{\Z}{\mathbb Z}
\newcommand{\br}{\mathbf{r}}
\newcommand{\RP}{\mathbb{RP}}
\newcommand{\CP}{\mathbb{CP}}
\newcommand{\nbit}[1]{\{0, 1\}^{#1}}
\newcommand{\bits}{\{0, 1\}^{n}}
\newcommand{\bbni}{\bigbreak \noindent}
\newcommand{\norm}[1]{\left\vert\left\vert#1\right\vert\right\vert}

\let\1\relax
\newcommand{\1}{\mathbf{1}}
\newcommand{\fr}[2]{\left(\frac{#1}{#2}\right)}

\newcommand{\vecz}{\mathbf{z}}
\newcommand{\vecr}{\mathbf{r}}
\DeclareMathOperator{\Cinf}{C^{\infty}}
\DeclareMathOperator{\Id}{Id}

\DeclareMathOperator{\Alt}{Alt}
\DeclareMathOperator{\ann}{ann}
\DeclareMathOperator{\codim}{codim}
\DeclareMathOperator{\End}{End}
\DeclareMathOperator{\Hom}{Hom}
\DeclareMathOperator{\id}{id}
\DeclareMathOperator{\M}{M}
\DeclareMathOperator{\Mat}{Mat}
\DeclareMathOperator{\Ob}{Ob}
\DeclareMathOperator{\opchar}{char}
\DeclareMathOperator{\opspan}{span}
\DeclareMathOperator{\rk}{rk}
\DeclareMathOperator{\sgn}{sgn}
\DeclareMathOperator{\Sym}{Sym}
\DeclareMathOperator{\tr}{tr}
\DeclareMathOperator{\img}{img}
\DeclareMathOperator{\CandE}{CandE}
\DeclareMathOperator{\CandO}{CandO}
\DeclareMathOperator{\argmax}{argmax}
\DeclareMathOperator{\first}{first}
\DeclareMathOperator{\last}{last}
\DeclareMathOperator{\cost}{cost}
\DeclareMathOperator{\dist}{dist}
\DeclareMathOperator{\path}{path}
\DeclareMathOperator{\parent}{parent}
\DeclareMathOperator{\argmin}{argmin}
\DeclareMathOperator{\excess}{excess}
\let\Pr\relax
\DeclareMathOperator{\Pr}{\mathbf{Pr}}
\DeclareMathOperator{\Exp}{\mathbb{E}}
\DeclareMathOperator{\Var}{\mathbf{Var}}
\let\limsup\relax
\DeclareMathOperator{\limsup}{limsup}
%Paired Delims
\DeclarePairedDelimiter\ceil{\lceil}{\rceil}
\DeclarePairedDelimiter\floor{\lfloor}{ \rfloor}


\newcommand{\dagstar}{*}

\newcommand{\tbigwedge}{{\textstyle{\bigwedge}}}
\setlength{\parindent}{0pt}
\setlength{\parskip}{5pt}


\begin{document}

\title{CS 40: Computational Complexity}

\author{Sair Shaikh}
\maketitle

% Collaboration Notice: Talked to Henry Scheible '26 to discuss ideas.


\begin{problem}{30}
    Suppose that $X$ and $Y$ are normed vector spaces. 
    \begin{enumerate}
        \item Show that $\mathcal{L}(X, Y)$ is a normed vector space with respect to the the operator norm defined in lecture such that: 
        \[ ||T(x)|| \leq ||T|| ||x|| \]
        \item Show that if $S \in \mathcal{L}(Y, Z)$. Then, 
        \[  ||ST|| \leq ||S||||T|| \]
        \item Show that: 
        \[ ||T|| = \inf\{ a \geq 0 : ||T(x)|| \leq a||x|| \quad \forall x \in X\}\]
    \end{enumerate}
\end{problem}
\begin{solution} 
    \bbni 
    \begin{enumerate}
        \item First, note that $\mathcal{L}(X, Y)$ is a vector space, through pointwise addition and scalar multiplication defined in $Y$, i.e. for $T, S \in \mathcal{L}(X, Y)$ and $\alpha \in \mathbb{F}$ we let:
        \[(\alpha T + S)(x) := \alpha T(x) + S(x)\]
        for all $x \in X$. Thus, we only need to show that the operator norm is a norm and satisfies the given property. Recall the definition of the operator norm for $T \in \mathcal{L}(X, Y)$:
        \[ ||T|| = \sup_{||x|| \leq 1} ||T(x)||\] 
        \begin{enumerate}
            \item[Non-Neg.] Let $T \in \mathcal{L}(X, Y)$ be arbitrary. Then, for all $x \in X$ with $||x|| \leq 1$, $||T(x)|| \geq 0$ by the non-negativity of the norm on $Y$. Thus, $||T|| \geq 0$. 
            \item[Homogeneity.] Let $T \in \mathcal{L}(X, Y)$ and $\alpha \in \mathbb{F}$. Then, for every $x \in X$, with $||x|| \leq 1$, we have that: 
            \begin{align*}
                ||(\alpha T)(x)|| &= ||\alpha T(x)|| \\
                &= |\alpha|\cdot ||T(x)|| \\
                &\leq |\alpha| \cdot ||T||
            \end{align*} 
            using the homogeneity of the norm on $Y$. Thus, 
            \[ ||\alpha T|| \leq |\alpha| \cdot ||T(x)||\]
            Similarly, we also have: 
            \begin{align*}
                |\alpha| \cdot ||T(x)|| &= ||\alpha T(x) || \\
                &= ||(\alpha T)(x)|| \\
                &\leq ||\alpha T||
            \end{align*}
            Thus, we have:
            \[ |\alpha| \cdot ||T|| \leq ||\alpha T||\]
            Thus, we have shown that:
            \[ ||\alpha T|| = |\alpha| \cdot ||T||\]
            \item[$\triangle$ ineq.] Let $T, S \in \mathcal{L}(X, Y)$. For every $x \in X$, with $||x|| \leq 1$, we have: 
            \begin{align*}
                ||(T+S)(x)|| &= ||T(x) + S(x)||  \\
                &\leq ||T(x)|| + ||S(x)|| \\
                &\leq ||T|| + ||S||
            \end{align*}
            using the triangle inequality for the norm in $Y$. Thus, 
            \[ ||T+S|| \leq ||T|| + ||S||\]
            \item[Pos. Def.] To prove this, we first will show that $||T(x)|| \leq ||T||||x||$ for all $x \in X$. For $x \in X$, by homogeneity (and non-negativity) of the norm on $X$: 
            \[  \left\lvert\left\lvert\frac{1}{||x||} x\right\rvert\right\rvert = \frac{1}{||x||}\cdot ||x|| = 1 \]
            Thus, we note that:
            \[  \left\lvert\left\lvert T\left(\frac{1}{||x||} x\right)\right\rvert\right\rvert \leq ||T|| \]
            By the linearity of $T$ and the homogeneity of the norm in $Y$, this implies:
            \begin{align*}
                \frac{1}{||x||} ||T(x)||  &= 
                \left\lvert\left\lvert \frac{1}{||x||} T(x)\right\rvert\right\rvert \\
                &=  \left\lvert\left\lvert T\left(\frac{1}{||x||} x\right)\right\rvert\right\rvert \\
                &\leq ||T|| \\
            \end{align*}
            Thus, 
            \[ ||T(x)|| \leq ||T|| \cdot ||x||\]
            Now, let $T \in \mathcal{L}(X, Y)$ be such that $||T|| = 0$. Then, for all $x \in X$, we have that: 
            \begin{align*}
                ||T(x)|| &\leq ||T|| \cdot ||x|| \\
                &= 0
            \end{align*}
            However, by the non-negativity of the norm in $Y$, we must have that $||T(x)|| = 0$ for all $x \in X$. Then, by positive definiteness of the norm in $Y$, we have that $T(x) = 0$ for all $x \in X$. Thus, $T$ is the zero map. \\
            Conversely, if $T$ is the zero map, then for all $x \in X$ with $||x|| \leq 1$, we have that:
            \[ ||T|| = ||0 \cdot T|| = 0 \cdot ||T|| = 0\]
            by homogeneity. Thus, the norm is positive definite.
        \end{enumerate}
        Therefore, we have shown that the operator norm is a norm on $\mathcal{L}(X, Y)$ and satisfies for all $T \in \mathcal{L}(X, Y)$ and $x \in X$:
        \[ ||T(x)|| \leq ||T|| \cdot ||x||\]
        \item For any $x \in X$, with $||x|| = 1$, by applying the property from part 1 twice, we have that: 
        \begin{align*}
            ||ST(x)|| &= ||S(T(x))|| \\
            &\leq ||S|| \cdot ||T(x)|| \\
            &\leq ||S|| \cdot ||T|| \cdot ||x|| \\
            &= ||S|| \cdot ||T||
        \end{align*}
        Thus,
        \[ ||ST|| \leq ||S|| \cdot ||T||\]
        \item Let $\alpha(T)$ be the defined infimum. \bbni
        Since we have that $||T(x)|| \leq ||T||\cdot ||x||$ for all $x \in X$, $||T||$ is in the set we are taking the infimum over. Thus, $\alpha(T) \geq ||T||$. \bbni
        Moreover, by the definition of $\alpha(T)$, we have that for all $x \in X$ with $||x|| \leq 1$,
        \[ ||T(x)|| \leq \alpha(T)||x|| = \alpha(T) \]
        Thus, $\alpha(T)$ is an upperbound on $||T(x)||$ with $||x|| \leq 1$. Therefore, by the definition of the supremum, we have that:
        \[ ||T|| \leq \alpha(T)\]
        Thus, we have shown that:
        \[ ||T|| = \alpha(T) = \inf\{ a \geq 0 : ||T(x)|| \leq a||x|| \quad \forall x \in X\}\]


    \end{enumerate}
\end{solution}
\newpage 

\begin{problem}{31}
    Suppose that $X$ and $Y$ are Banach spaces with $T \in \mathcal{L}(X, Y)$. Suppose that $E$ is a closed proper subspace of $X$ such that $E \subset \ker(T)$. Show that there is a unique operator $\overline{T} \in \mathcal{L}(X/E, Y)$ such that $\overline{T}(q(x)) = T(x)$ for all $x \in X$ where $q: X \to X/E$ is the quotient map. Moreover, $||\overline{T}|| = ||T||$.
\end{problem}
\begin{solution} 
    We claim the map $\overline{T}: X/E \to Y$ given by: 
    \[ \overline{T}([x]) = T(x)\]
    satisfies the desired properties. We need to show that $\overline{T}$ is well-defined, linear, satisfies $\overline{T}(q(x)) = T(x)$ for all $x \in X$, and satisfies $||\overline{T}|| = ||T||$ (hence is bounded/continous). \bbni
    Let $x, y \in X$ be such that $[x] = [y]$. Then, note that $x - y \in E \subset \ker(T)$. Thus, $T(x)-T(y) = T(x-y) = 0$. Thus, $T(x) = T(y)$. Then, by the definition of $\overline{T}$, we have $\overline{T}([x]) = \overline{T}([y])$. Thus, $\overline{T}$ is well-defined. \bbni
    Next, let $[x], [y] \in X/E$ and $\alpha \in \mathbb{F}$ be arbitrary. Then, noting the linearity of $q$ and $T$, we have that:
    \begin{align*}
        \overline{T}(\alpha[x] + [y]) &= \overline{T}([\alpha x + y]) \\
        &= T(\alpha x + y) \\
        &= \alpha T(x) + T(y) \\
        &= \alpha \overline{T}([x]) + \overline{T}([y])
    \end{align*}
    Thus, $\overline{T}$ is linear. \bbni
    
\end{solution}
\newpage 


\begin{problem}{33}
    Let $E$ and $X$ be Banach spaces with $E$ finite dimensional. 
    \begin{enumerate}
        \item Show that every linear map $S: E \to X$ is bounded. 
        \item Show that a linear map $T: X \to E$ is bounded if and only if $\ker(T)$ is closed. 
    \end{enumerate}
\end{problem}
\begin{solution} 

\end{solution}
\newpage 

\begin{problem}{34}
    Supposed that $E$ and $M$ are closed subspaces of a Banach space $X$. If $E$ is finite dimensional, show that $E+M = \{x+y: x \in E \, y \in M\}$ is closed. 
\end{problem}
\begin{solution} 

\end{solution}
\newpage 

\begin{problem}{35}
    Suppose that $X$ and $Y$ are Banach spaces for $T \in \mathcal{L}(X, Y)$. Show that $T$ is injective with closed range if and only if: 
    \[ \inf \{||T(x)|| : ||x|| = 1\} > 0 \]
\end{problem}
\begin{solution} 

\end{solution}
\newpage 


\begin{problem}{38}
    Let $X$ be a normed vector space. A Banach space $\tilde{X}$ is called a completion of $X$ is there is an isometric isomorphism $\iota: X \to \tilde{X}$ onto a dense subspace of $\tilde{X}$. Show that any two completions $(tilde{X}_1, \iota_1)$ and $(\tilde{X}_2, \iota_2)$ are isometrically isomorphic by an isomorphism: 
    \[ \Phi: \tilde{X}_1 \to \tilde{X}_2\]
    such that $\Phi(\iota_1(x)) = \iota_2(x)$ for all $x \in X$. 
\end{problem}
\begin{solution} 

\end{solution}
\newpage 

\begin{problem}{39}
    Lets find a use for a genuine Minkowski functional. In this problem, we'll let $l_\R^\infty$ be the real Banach space of bounded sequences in $\R$. Define $m$ on $l_\R^\infty$: 
    \[ m(x) = \limsup_n x_n\]
    We clearly have $m(tx) = tm(x)$ if $t \geq 0$ and it is not hard to check that $m(x+y) \leq m(x) + m(y)$ for all $x, y \in l_\R^\infty$. We want to show that there are Banach limits or what I prefer to call a generalized limit on $l_\R^\infty$. This is we want to show that there is a functional $L \in l_\R^{\infty^*}$ such that:
    \[ L(S(x)) = L(x)\] 
    where $S \in \mathcal{L}(l_\R^\infty)$ is given by $S(x)_n = x_{n+1}$ and such that $\liminf_n x_n \leq L(x) \leq \limsup_n x_n$. (Hint provided).
\end{problem}
\begin{solution} 

\end{solution}
\newpage 

\begin{problem}{40}
    Prove the following Lemma from lecture. Let $X$ be a complex vector space. Every real linear functional of $X$ is the real part of a complex linear functional on $X$. In fact, if $\phi = \Re(\psi)$ then $\psi(x) = \phi(x) - i\phi(ix)$. 
\end{problem}
\begin{solution} 

\end{solution}
\newpage 

\begin{problem}{41}
    Suppose that $X$ is a normed vector space such that $X^*$ is seperable. Show that $X$ is seperable. (Hint provided).
\end{problem}
\begin{solution} 

\end{solution}
\newpage 

\end{document}