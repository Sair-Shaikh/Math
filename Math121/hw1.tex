% \documentclass[12pt]{amsart}
\documentclass[12pt]{article}

\usepackage{fullpage}
\usepackage{mdframed}
\usepackage{colonequals}
\usepackage{algpseudocode}
\usepackage{algorithm}
\usepackage{tcolorbox}
\usepackage[all]{xy}
\usepackage{proof}
\usepackage{mathtools}
\usepackage{bbm}
\usepackage{amssymb}
\usepackage{amsthm}
\usepackage{amsmath}
\usepackage{amsxtra}
\newcommand{\bb}{\mathbb}


\newtheorem{theorem}{Theorem}[section]
\newtheorem{corollary}{Corollary}[theorem]
\newtheorem{lemma}{Lemma}

\newcommand{\mathcat}[1]{\textup{\textbf{\textsf{#1}}}} % for defined terms

\newenvironment{problem}[1]
{\begin{tcolorbox}\noindent\textbf{Problem #1}.}
{\vskip 6pt \end{tcolorbox}}

\newenvironment{enumalph}
{\begin{enumerate}\renewcommand{\labelenumi}{\textnormal{(\alph{enumi})}}}
{\end{enumerate}}

\newenvironment{enumroman}
{\begin{enumerate}\renewcommand{\labelenumi}{\textnormal{(\roman{enumi})}}}
{\end{enumerate}}

\newcommand{\defi}[1]{\textsf{#1}} % for defined terms

\theoremstyle{remark}
\newtheorem*{solution}{Solution}

\setlength{\hfuzz}{4pt}

\newcommand{\calC}{\mathcal{C}}
\newcommand{\calF}{\mathcal{F}}
\newcommand{\C}{\mathbb C}
\newcommand{\N}{\mathbb N}
\newcommand{\Q}{\mathbb Q}
\newcommand{\R}{\mathbb R}
\newcommand{\Z}{\mathbb Z}
\newcommand{\br}{\mathbf{r}}
\newcommand{\RP}{\mathbb{RP}}
\newcommand{\CP}{\mathbb{CP}}
\newcommand{\nbit}[1]{\{0, 1\}^{#1}}
\newcommand{\bits}{\{0, 1\}^{n}}
\newcommand{\bbni}{\bigbreak \noindent}
\newcommand{\norm}[1]{\left\vert\left\vert#1\right\vert\right\vert}

\let\1\relax
\newcommand{\1}{\mathbf{1}}
\newcommand{\fr}[2]{\left(\frac{#1}{#2}\right)}

\newcommand{\vecz}{\mathbf{z}}
\newcommand{\vecr}{\mathbf{r}}
\DeclareMathOperator{\Cinf}{C^{\infty}}
\DeclareMathOperator{\Id}{Id}

\DeclareMathOperator{\Alt}{Alt}
\DeclareMathOperator{\ann}{ann}
\DeclareMathOperator{\codim}{codim}
\DeclareMathOperator{\End}{End}
\DeclareMathOperator{\Hom}{Hom}
\DeclareMathOperator{\id}{id}
\DeclareMathOperator{\M}{M}
\DeclareMathOperator{\Mat}{Mat}
\DeclareMathOperator{\Ob}{Ob}
\DeclareMathOperator{\opchar}{char}
\DeclareMathOperator{\opspan}{span}
\DeclareMathOperator{\rk}{rk}
\DeclareMathOperator{\sgn}{sgn}
\DeclareMathOperator{\Sym}{Sym}
\DeclareMathOperator{\tr}{tr}
\DeclareMathOperator{\img}{img}
\DeclareMathOperator{\CandE}{CandE}
\DeclareMathOperator{\CandO}{CandO}
\DeclareMathOperator{\argmax}{argmax}
\DeclareMathOperator{\first}{first}
\DeclareMathOperator{\last}{last}
\DeclareMathOperator{\cost}{cost}
\DeclareMathOperator{\dist}{dist}
\DeclareMathOperator{\path}{path}
\DeclareMathOperator{\parent}{parent}
\DeclareMathOperator{\argmin}{argmin}
\DeclareMathOperator{\excess}{excess}
\let\Pr\relax
\DeclareMathOperator{\Pr}{\mathbf{Pr}}
\DeclareMathOperator{\Exp}{\mathbb{E}}
\DeclareMathOperator{\Var}{\mathbf{Var}}
\let\limsup\relax
\DeclareMathOperator{\limsup}{limsup}
%Paired Delims
\DeclarePairedDelimiter\ceil{\lceil}{\rceil}
\DeclarePairedDelimiter\floor{\lfloor}{ \rfloor}


\newcommand{\dagstar}{*}

\newcommand{\tbigwedge}{{\textstyle{\bigwedge}}}
\setlength{\parindent}{0pt}
\setlength{\parskip}{5pt}


\begin{document}

% \title{CS 40: Computational Complexity}

\author{Sair Shaikh}
\maketitle

% Collaboration Notice: Talked to Henry Scheible '26 to discuss ideas.




\begin{problab}{1}
    Let $U \subset \mathbb{C}^n$ be a domain (i.e., a non-empty connected subset) and let $f : U \to \mathbb{C}$ be a holomorphic function.
    \begin{enumerate}
        \item Prove that $f$ satisfies analytic continuation: if it vanishes on an open subset of $U$, then it vanishes everywhere.
        \item Prove that $f$ satisfies the maximum principle: if $|f|$ admits a local maximum in $U$, then $f$ is constant.
        \item Let $M$ be a compact complex manifold. Prove that any holomorphic function on $M$ is constant.
    \end{enumerate}    
\end{problab}
\newpage


\begin{problab}{2}
    \begin{enumerate}
        \item Show that the assignment $L \mapsto (L_{\mathbb{R}}, \text{mult}(i))$ gives an equivalence between the category of complex vector spaces and the category of pairs $(V, J)$, where $V$ is a real vector space, $J : V \to V$ is an $\mathbb{R}$-linear operator satisfying $J^2 = -\mathrm{Id}_{L_\mathbb{R}}$, and a morphism $(V, J) \to (W, K)$ is defined as an $\mathbb{R}$-linear map $f : V \to W$ that intertwines $J$ and $K$, i.e., such that $K \circ f = f \circ J$.

        \item Let $(V, J)$ be a pair as above:
        \begin{enumerate}
            \item Let $\iota : v \mapsto v$ be the $\mathbb{R}$-linear automorphism of $V \otimes_{\mathbb{R}} \mathbb{C}$ induced from conjugation on the second factor. Show that $V$, seen inside $V \otimes_{\mathbb{R}} \mathbb{C}$ via $v \mapsto v \otimes 1$, is isomorphic to the fixed locus of $\iota$.
            \item By diagonalizing $J$ over $V \otimes_{\mathbb{R}} \mathbb{C}$, show that we have a decomposition $V_{\mathbb{C}} = W \oplus \bar{W}$ where $J$ acts on $W$ by multiplication by $i$ and $\bar{W} = \iota(W)$.
            \item Show that the projection map $V \to W$ is an $\mathbb{R}$-linear isomorphism that intertwines $J$ and multiplication by $i$ on $W$.
            \item Each element $v \in V_{\mathbb{C}}$ can be written as $v = v_1 + v_2$ along the above decomposition. Show that $v \in V$ if and only if $v_2 = \bar{v}_1$.
        \end{enumerate}
    \end{enumerate}
\end{problab}
\newpage


\begin{problab}{3}
    Let $E, F$ be real vector spaces and let $G$ be a complex vector space, all finite-dimensional.
    \begin{enumerate}
        \item Show that $E \otimes_{\mathbb{R}} G$ admits a natural structure of a complex vector space. Construct a basis in terms of bases of $E$ and $G$. Show that $E \otimes_{\mathbb{R}} G = (E \otimes_{\mathbb{R}} \mathbb{C}) \otimes_{\mathbb{C}} G$.
        \item Show that $\mathrm{Hom}_{\mathbb{R}}(E, G) = \mathrm{Hom}_{\mathbb{C}}(E \otimes \mathbb{C}, G)$.
        \item Show that
        \[
            (E \oplus F) \otimes_{\mathbb{R}} \mathbb{C} = (E \otimes_{\mathbb{R}} \mathbb{C}) \oplus (F \otimes_{\mathbb{R}} \mathbb{C}), \quad (E \otimes_{\mathbb{R}} F) \otimes_{\mathbb{R}} \mathbb{C} = (E \otimes_{\mathbb{R}} \mathbb{C}) \otimes_{\mathbb{C}} (F \otimes_{\mathbb{R}} \mathbb{C})
        \]
        and
        \[
            \Lambda^n E \otimes_{\mathbb{R}} \mathbb{C} = \Lambda^n(E \otimes_{\mathbb{R}} \mathbb{C}).
        \]
        \item Let $f : E \to F$ be a linear map and let $f_{\mathbb{C}} = f \otimes_{\mathbb{R}} \mathbb{C} : E \otimes_{\mathbb{R}} \mathbb{C} \to F \otimes_{\mathbb{R}} \mathbb{C}$ be the induced map. Show that
        \[
            \ker(f_{\mathbb{C}}) = \ker(f) \otimes_{\mathbb{R}} \mathbb{C}, \quad \mathrm{Im}(f_{\mathbb{C}}) = \mathrm{Im}(f) \otimes_{\mathbb{R}} \mathbb{C}.
        \]
    \end{enumerate}
\end{problab}
\begin{solu}
    \bbni
    \begin{enumerate}
        \item Let $e \in E$ and $g \in G$. Then, we define the scalar multiplication by $z \in \C$ by using the complex structure of $G$: 
        \[ z\cdot (e \otimes g) = e \otimes zg \]
        and extend linearly. For $z \in \R$, this definition respects the tensor product over $\R$ as:
        \[ z \cdot (e \otimes g ) = e \otimes zg = ze \otimes g\]
        If $\{e_i\}_i$ is a basis of $E$ and $\{g_j\}_j$ is a basis of $G$, then the set $\{e_i \otimes_\R g_j\}_{i,j}$ is a $\R$-linear basis of $E \otimes_{\mathbb{R}} G$ (before we introduce the complex structure). With respect to the complex structure, note that this set is $\C$-linearly independent as if: 
        \[\sum_{i,j} z_{ij} (e_i \otimes g_j)= 0\]
        then, 
        \[ \sum_{i,j}  e_i \otimes z_{ij}g_j = 0\]
        which contradicts the $\R$-linear independence of $\{e_i \otimes_\R g_j\}_{i,j}$. Moreover, the set clearly $\C$-spans $E \otimes_\R G$ as it $\R$-spans it. Thus, the set is a $\C$-linear basis of $E \otimes_\R G$. \bbni
        Next, we define the homomorphisms $\Phi: E \otimes_{\mathbb{R}} G \to (E \otimes_{\mathbb{R}} \mathbb{C}) \otimes_{\mathbb{C}} G$ and $\Psi: (E \otimes_{\mathbb{R}} \mathbb{C}) \otimes_{\mathbb{C}} G \to E \otimes_{\mathbb{R}} G$ as follows:
        \begin{align*}
            \Phi(e \otimes g) &= (e \otimes_\R 1) \otimes_\C g \\
            \Psi((e \otimes_\R z) \otimes_\C g) &= e \otimes_\R zg
        \end{align*}
        and extending linearly. We show that these are inverses as follows: 
        \begin{align*}
            \Psi \circ \Phi((e \otimes_\R g)) &= \Psi((e \otimes_\R 1) \otimes_\C g) \\
            &= e \otimes_\R g \\
            \Phi \circ \Psi((e \otimes_\R z) \otimes_\C g) &= \Phi(e \otimes_\R zg) \\
            &= (e \otimes_\R 1) \otimes_\C zg \\
            &= z\cdot (e \otimes_\R 1) \otimes_\C g \\
            &= (e \otimes_\R z) \otimes_\C g
        \end{align*}
        where we use the same natural complex structure on $E \otimes_\R \C$. Thus, $\Phi$ and $\Psi$ are inverses. Thus, 
        \[E \otimes_\R G \cong (E \otimes_\R \C) \otimes_\C G\]
        \item We show this by constructing the linear maps $\Phi: \Hom_\R(E, G) \to \Hom_\C(E \otimes_\R \C, G)$ and $\Psi: \Hom_\C(E \otimes_\R \C, G) \to \Hom_\R(E, G)$ as follows, for $f \in \Hom_\R(E, G)$ and $g \in \Hom_\C(E \otimes_\R \C, G)$:
        \begin{align*}
            \Phi(f)(e \otimes_\R z) &= zf(e) \in G \\
            \Psi(g)(e) &= g(e \otimes_\R 1) \in G
        \end{align*}
        and extending linearly. We show that $\Phi(g)$ is $\C$-linear and $\Psi(g)$ is $\R$-linear. 
        \begin{align*}
            \Phi(f)(z \cdot e_1 \otimes_\R z_1 + e_2 \otimes_\R z_2) &= \Phi(f)(e_1 \otimes_\R zz_1 + e_2 \otimes_\R z_2) \\
            &= zz_1f(e_1) + z_2f(e_2) \\
            &= z\cdot \Phi(f)(e_1 \otimes_\R z_1)  + \Phi(f)(e_2 \otimes_\R z_2)
        \end{align*}
        Thus, $\Phi(f)$ is $\C$-linear.
        \begin{align*}
            \Psi(g)(ze_1 + e_2) &= g((ze_1+e_2) \otimes_\R 1) \\
            &= g((e_1 \otimes_\R z) + (e_2 \otimes_\R 1)) \\
            &= \Phi(g)(ze_1) + \Phi(g)(e_2)
        \end{align*}
        Thus, $\Psi(g)$ is $\R$-linear. Finally, we wanna show that $\Phi$ and $\Psi$ are inverses, for $e\in E$ and $z \in \C$. 
        \begin{align*}
            \Psi(\Phi(f))(e) &= \Phi(f)(e \otimes_\R 1) \\
            &= f(e) \\
            \Phi(\Psi(g))(e \otimes_\R z) &= z\Psi(g)(e) \\
            &= zg(e \otimes_\R 1) \\
            &= g(e \otimes_\R z)
        \end{align*}
        Thus, $\Phi$ and $\Psi$ are inverses. Thus, we have the isomorphism:
        \[\Hom_\R(E, G) \cong \Hom_\C(E \otimes_\R \C, G)\]
        \item We do this by constructing explicit maps. 
        \begin{itemize}
            \item Let $e \in E$, $f\in F$ and $z \in \C$. Define the two maps: 
            \begin{align*}
                \Phi: (E \oplus F) \otimes_\R \C &\to (E \otimes_\R \C) \oplus (F \otimes_\R \C) \\
                (e, f) \otimes_\R z &\mapsto (e \otimes_\R z, f \otimes_\R z) \\
                \Psi: (E \otimes_\R \C) \oplus (F \otimes_\R \C) &\to (E \oplus F) \otimes_\R \C \\
                (e \otimes z_1, f \otimes z_2) &\mapsto (e, 0) \otimes_\R z_1 + (0, f) \otimes_\R z_2
            \end{align*} 
            and extend linearly. We show that these are inverses as follows:
            \begin{align*}
                \Phi \circ \Psi(e \otimes_\R z_1, f \otimes_\R z_2) &= \Phi((e, 0) \otimes_\R z_1 + (0, f) \otimes_\R z_2) \\ 
                &= \Phi((e, 0) \otimes_\R z_1) + \Phi((0, f) \otimes_\R z_2) \\
                &= (e \otimes_\R z_1, 0) + (0, f \otimes_\R z_2) \\
                &= (e \otimes_\R z_1, f \otimes_\R z_2) \\
                \Psi \circ \Phi((e, f) \otimes_\R z) &= \Psi(e \otimes_\R z, f \otimes_\R z) \\
                &= (e, 0) \otimes_\R z + (0, f) \otimes_\R z \\
                &= (e, f) \otimes_\R z
            \end{align*}
            Thus, $\Phi$ and $\Psi$ are inverses. Thus, we have the isomorphism:
            \[(E \oplus F) \otimes_\R \C \cong (E \otimes_\R \C) \oplus (F \otimes_\R \C)\]
            \item Let $e \in E$, $f\in F$ and $z \in \C$. Note that a simple tensor in $(E \otimes_\R \C) \otimes_\C (F \otimes_\R \C)$ is $(e \otimes_\R z) \otimes_\C (f \otimes_\R 1)$ as we are tensoring over $\C$. Define the two maps: 
            \begin{align*}
                \Phi: (E \otimes_\R F) \otimes_\R \C &\to (E \otimes_\R \C) \otimes_\C (F \otimes_\R \C) \\
                e \otimes f \otimes_\R z &\mapsto (e \otimes_\R z) \otimes_\C (f \otimes_\R 1) \\
                \Psi: (E \otimes_\R \C) \otimes_\C (F \otimes_\R \C) &\to (E \otimes F) \otimes_\R \C \\
                (e \otimes z) \otimes (f \otimes 1) &\mapsto e \otimes_\R f \otimes_\R z
            \end{align*} 
            and extend linearly. We show that these are inverses as follows:
            \begin{align*}
                \Phi \circ \Psi( (e \otimes_\R z) \otimes_\C (f \otimes_\R 1)) &=  \Phi(e \otimes_\R f \otimes_\R z) \\ 
                &= (e \otimes_\R z) \otimes_\C (f \otimes_\R 1)\\ 
                \Psi \circ \Phi(e \otimes_\R f \otimes_\R z) &= \Psi(e \otimes_\R z) \otimes_\C (f \otimes_\R 1) \\
                &= e \otimes_\R f \otimes_\R z
            \end{align*}
            Thus, $\Phi$ and $\Psi$ are inverses. Thus, we have the isomorphism:
            \[ (E \otimes_\R F) \otimes_\R \C \cong  (E \otimes_\R \C) \otimes_\C (F \otimes_\R \C) \]
            \item Let $e_1, \cdots, e_n \in E$ and $z_1, \cdots, z_n \in \C$. Assuming that $\bigwedge^n (E \otimes_\R \C)$ is wedging over $\C$. Define the two maps: 
            \begin{align*}
                \Phi: \bigwedge^n E \otimes_\R \C &\to \bigwedge^n (E \otimes_\R \C)\\
                e_1 \wedge \cdots \wedge e_n \otimes_\R z_1 &\mapsto (e_1 \otimes_\R z_1) \wedge (e_2 \otimes_\R 1) \wedge \cdots \wedge (e_n \otimes_\R 1) \\
                \Psi: \bigwedge^n (E \otimes_\R \C)  &\to \bigwedge^n E \otimes_\R \C \\
                (e_1 \otimes_\R z_1) \wedge \cdots \wedge (e_n \otimes_\R z_n) &\mapsto e_1 \wedge \cdots \wedge e_n \otimes_\R z_1 \cdots z_n
            \end{align*} 
            and extend linearly. We show that these are inverses as follows:
            \begin{align*}
                \Phi \circ \Psi( (e_1 \otimes_\R z_1) \wedge \cdots \wedge (e_n \otimes_\R z_n) ) &=  \Phi(e_1 \wedge \cdots \wedge e_n \otimes_\R z_1 \cdots z_n) \\ 
                &= (e_1 \otimes_\R z_1\cdots z_n) \wedge (e_2 \otimes_\R 1) \wedge  \cdots \wedge (e_n \otimes_\R 1)\\ 
                &= (e_1 \otimes_\R z_1) \wedge \cdots \wedge (e_n \otimes_\R z_n)\\ 
                \Psi \circ \Phi(e_1 \wedge \cdots \wedge e_n \otimes_\R z_1) &= \Psi((e_1 \otimes_\R z_1) \wedge (e_2 \otimes_\R 1) \wedge  \cdots \wedge (e_n \otimes_\R 1)) \\
                &= e_1 \wedge \cdots \wedge e_n \otimes_\R z_1
            \end{align*}
            Thus, $\Phi$ and $\Psi$ are inverses. Thus, we have the isomorphism:
            \[  \bigwedge^n E \otimes_\R \C \cong \bigwedge^n (E \otimes_\R \C)\]
        \end{itemize}
        \item Let $e \in \ker(f)$. Then, for any $e \otimes_\R z \in \ker(f) \otimes_\R \C$, we have:
        \[f_\C(e \otimes_\R z) = f(e) \otimes_\R z = 0 \otimes_\R z = 0 \]
        Thus, $e \otimes_\R z \in \ker(f_\C)$ for any $z \in \C$. Thus, $\ker(f) \otimes \C \subseteq \ker(f_\C)$. \\
        Similarly, let $e \otimes_\R z \in \ker(f_\C)$. If $z = 0$, then, $e \otimes_\R z = 0 \otimes 0$ and $0 \in \ker(f)$. Thus, assume $z \neq 0$. Then, $f_\C(e \otimes_\R z) = 0$ implies that $f(e) \otimes_\R z = 0$. Since $z \neq 0$, we have $f(e) = 0$. Thus, $e \in \ker(f)$. Thus, $\ker(f_\C) \subseteq \ker(f) \otimes_\R \C$. Hence, we conclude $\ker(f_\C) = \ker(f) \otimes_\R \C$. \bbni
        Let $p \in \img(f)$. Then, there exists $e \in E$ such that $f(e) = p$. Then, we have $f_\C(e \otimes_\R z) = f(e) \otimes_\R z = p \otimes_\R z$. Thus, $\img(f) \otimes_\R \C \subseteq \img(f_\C)$. \\
        Conversely, let $p \otimes_\R z \in \img(f_\C)$. Then, there exists $e \otimes_\R z' \in E \otimes_\R \C$ such that $f_\C(e \otimes_\R z') = p \otimes_\R z$. Then, by the definition of $f_\C$, we have $z = z'$ and $p = f(e)$. Thus, $p \in \img(f)$. Thus, $\img(f_\C) \subseteq \img(f) \otimes_\R \C$. Hence, we conclude $\img(f_\C) = \img(f) \otimes_\R \C$.
    \end{enumerate}
\end{solu}

\newpage



\begin{problab}{4}
    Let $U \subset \mathbb{C}$ be an open subset and let $D \subset \Omega$ be a closed disk.

    \begin{enumerate}
        \item Let $f : U \to \mathbb{C}$ be a $\mathcal{C}^1$ function. Show that for all $z \in D$, we have:
        \[
            f(z) = \frac{1}{2\pi i} \int_{\partial D} \frac{f(\xi)}{\xi - z} \, d\xi + \frac{1}{2\pi i} \int_D \frac{\partial f}{\partial \bar{z}}(\xi) \frac{d\xi \wedge d\bar{\xi}}{\xi - z}.
        \]
        \textit{Hint:} Apply Stokes’ theorem to $\frac{f(\xi)}{\xi - z} d\xi$ on $D \setminus B(z, \varepsilon)$ and let $\varepsilon \to 0$.
        
        \item Let $g$ be a $\mathcal{C}^1$ function on $\mathbb{C}$ with compact support and define
        \[
            f(z) = \frac{1}{2\pi i} \int_{\mathbb{C}} \frac{g(\xi)}{\xi - z} d\xi \wedge d\bar{\xi}.
        \]
        Show that $f$ is $\mathcal{C}^1$ and $\frac{\partial f}{\partial \bar{z}} = g$. \textit{Hint:} Differentiate under the integral sign after substituting $\xi' = \xi - z$.
        
        \item Show that for any function $g$ on $U$ which is $\mathcal{C}^1$, there exists a function $f$ on $U$, also $\mathcal{C}^1$, such that $\frac{\partial f}{\partial \bar{z}} = g$ on $D$.
        
        \item In the previous question, show that if $g$ is $\mathcal{C}^\infty$, then $f$ can also be chosen to be $\mathcal{C}^\infty$.
    \end{enumerate}
\end{problab}
\newpage

\begin{problab}{5}
    Let $E$ and $F$ be two holomorphic vector bundles on a complex manifold $X$. Given an open cover $\{ U_\alpha \}$ of $X$ that trivializes $E$, the vector bundle $E$ is described on overlaps $U_\alpha \cap U_\beta$ by holomorphic transition functions:
    \[
        \rho_{\alpha \beta} : U_\alpha \cap U_\beta \to \mathrm{GL}_n(\mathbb{C}).
    \]
    \begin{enumerate}
        \item Prove the cocycle condition: $\rho_{\alpha \gamma} = \rho_{\beta \gamma} \circ \rho_{\alpha \beta}$.
        \item Let $E'$ be the quotient of $\bigsqcup_\alpha U_\alpha \times \mathbb{C}^n$ by the equivalence relation on $U_\alpha \cap U_\beta \times \mathbb{C}^n$ given by
        \[
            (x, v) \sim (x, \rho_{\alpha \beta}(x)(v)).
        \]
        Prove that $E'$ is a holomorphic vector bundle and that it is isomorphic to $E$ as vector bundles over $X$, i.e., there exists a biholomorphism $f : E \to E'$ commuting with projection to $X$.
        
        \item Conversely, assume that $E$ and $F$ are isomorphic as holomorphic vector bundles. How are their transition functions related?
        
        \item Using the transition maps of $E$ and $F$, construct the following vector bundles by writing down explicitly their transition functions: $E \otimes_{\mathbb{C}} F$, $E \oplus F$, $\Lambda^n E$.
    \end{enumerate}
\end{problab}
\begin{solu}
    \bbni
    \begin{enumerate}
        \item Let $\{U_\alpha\}$ be an open cover of $X$ that trivializes $E$, with respective maps $\tau_\alpha: \pi^{-1}(U_\alpha) \to U_\alpha \times \C^n$. Then, note that $\rho_{\alpha\beta}: \tau_\alpha(\pi^{-1}(U_\alpha \cap U_\beta)) \to U_\beta \times \C^n$ is given by: 
        \[\rho_{\alpha\beta} = \tau_\beta \circ \tau_\alpha^{-1}  \] 
        with others defined similarly. Let $x \in U_\alpha \cap U_\beta \cap U_\gamma$ and $E_x = \pi^{-1}(a)$ be the fiber over $x$ in $E$. Then, we calculate: 
        \begin{align*}
            \rho_{\beta\gamma} \circ \rho_{\alpha\beta}(E_x) &= \tau_\gamma \circ \tau_\beta^{-1} \circ \tau_\beta \circ \tau_\alpha^{-1}(E_x) \\
            &= \tau_\gamma \circ \tau_\alpha^{-1}(E_x) \\
            &= \rho_{\alpha\gamma}(E_x)
        \end{align*}
        Thus, the cocycle condition holds on every such fiber. Thus, it holds in general. We can also realize this by picking a basis for the image of $E_x$ in each trivialization, and noticing that $\rho$ are just change of basis matrices, which satisfy the cocycle condition. 
        \item We first check that $E'$ is a holomorphic vector bundle. We write an element of $E'$ as $(x, [v])$, for $x \in U_\alpha$, where $[v]$ is the equivalence class of $\{\rho_{\alpha\beta}(v): x \in U_\beta\}$. \bbni 
        Defining $\pi'(x, [v]) = x$, we note that the projection $\pi'$ is clearly well-defined and can be made holomorphic by inheriting the holomorphic structure from the base space. \bbni
        Furthermore, we define the trivialization maps $\tau'_\beta: \pi'^{-1}(U_\beta) \to U_\beta \times \C^n$ as follows:
        \[
            \tau'_\beta(x, [v]) = (x, v)
        \]
        where $v$ is the representative of $[v]$ such that $(x, v) \in U_\beta \times \C^n$. Then, $\tau'^{-1}_\beta$ is given by taking the equivalence class: 
        \[ \tau'^{-1}_\beta(x, v) = (x, [v]) \]
        Thus, the transition maps $\rho'_{\alpha\beta}$ are given by, for $x \in U_\alpha \cap U_\beta$:
        \begin{align*}
            \rho'_{\alpha\beta}(x, v) &= \tau'_\beta \circ \tau'^{-1}_\alpha(x, v) \\
            &= (x, [v])) \\
            &= (x, [\rho_{\alpha\beta}(v)])  \\
            &= (x, \rho_{\alpha\beta}(v))         
        \end{align*}
        Thus, the transition functions $\rho'_{\alpha\beta}$ are holomorphic as $\rho_{\alpha\beta}$ are holomorphic. Thus, $E'$ is a holomorphic vector bundle. \bbni
        Next, to show that $E'$ is isomorphic to $E$ as a vector bundle, we need to find a biholomorphism $f: E \to E'$ that respects the fibers. We can define $f$ as follows: for $x \in U_\alpha$, and $\epsilon \in \pi^{-1}(x) = E_x$ , 
        \[ f(\epsilon) = \tau_\alpha'^{-1} \circ \tau_\alpha(\epsilon)\]
        To see that $f$ is well-defined, let $x \in U_\alpha \cap U_\beta$. Then, we have: 
        \begin{align*}
            f(\epsilon) &= \tau_\alpha'^{-1} \circ \tau_\alpha(\epsilon) \\
            &= \tau'^{-1}_\alpha(x, v) \\
            &= (x, [v]) \\
            &= (x, [\rho_{\alpha\beta}(v)]) \\
            &= \tau'^{-1}_\beta(x, \rho_{\alpha\beta}(v)) \\
            &= \tau'^{-1}_\beta \circ \tau_{\beta}(\epsilon) \\
            &= f(\epsilon)
        \end{align*}
        Since it is a composition of holomorphic maps, $f$ is holomorphic. Next, we define the inverse $f^{-1}: E' \to E$ as follows: for $x \in U_\alpha$ and $(x, [v]) \in E'$, with $v$ being the representative that came from $U_\alpha \times \C^n$, we have: 
        \begin{align*}
            f^{-1}(x, [v]) &= \tau^{-1}_\alpha \circ \tau'_\alpha(x, [v]) 
        \end{align*}
        To see that this is well-defined, let $x \in U_\alpha \cap U_\beta$. Then, we have:
        \begin{align*}
            f^{-1}(x, [v]) &= \tau^{-1}_\alpha \circ \tau'_\alpha(x, [v]) \\
            &= \tau^{-1}_\alpha(x, v) \\
            &= \tau_\beta^{-1}(x, \rho_{\alpha\beta}(v)) \\
            &= \tau_\beta^{-1} \circ \tau'_\beta(x, [\rho_{\alpha\beta}(v)]) \\
            &= \tau_\beta^{-1} \circ \tau'_\beta(x, [v]) \\
            &= f^{-1}(x, [v])
        \end{align*}
        Since $f^{-1}$ is also a composition of holomorphic maps, $f^{-1}$ is holomorphic. Moreover, it is easy to see that $f^{-1}$ and $f$ are inverses. Thus, $f$ is a biholomorphism. \bbni
        Finally, we can see that $f$ and $f^{-1}$ respect the fibers as they are compositons of fiber-preserving maps. Thus, they commute with the projection maps. Moreover, we note that on a particular fiber, $E_x$, with respect to the trivializations, $f$ is a vector space isomorphism. That is, $\tau'_{\alpha} \circ f \circ \tau^{-1}_{\alpha}$ is an isomorphism on $\{x\} \times \C^n$ as it is linear and has an inverse.
        \item Since $E$ and $F$ are isomorphic as holomorphic vector bundles, there exists a biholomorphism $f: E \to F$ that respects the fibers, and is a vector space isomorphism when restricted to each fiber. \bbni
        By refining the covers over which the trivializations for $E$ and $F$ are defined, choose a cover $\{U_\alpha\}$ such that both $E$ and $F$ are trivialized over it. Let $x \in U_\alpha \cap U_\beta$. Let $\tau_{\alpha, E}$ and $\tau_{\alpha, F}$ be trivializations and $\rho_E$ and $\rho_F$ be the transition functions from $\tau_\alpha(\pi^{-1}(U_\alpha)) \to U_\beta \times \C^n$ (sloppily identifying the two projections $\pi$). Let $E_{x, \alpha}$, $E_{x, \beta}$ be the trivilized fibers over $x$ and similarly for $f$. Then, $f$ is such that:
        \begin{align*}
            \tau_{\alpha, F} \circ f \circ \tau_{\alpha, E}^{-1} \qquad \tau_{\beta, F} \circ f \circ \tau_{\alpha, E}^{-1}  \\
            \tau_{\alpha, F} \circ f \circ \tau_{\beta, E}^{-1} \qquad \tau_{\beta, F} \circ f \circ \tau_{\beta, E}^{-1}
        \end{align*} 
        are all isomorphisms. Composition by $\rho_E$ and $\rho_F$ and their inverses permutes them. Write $f_{\alpha} := \tau_{\alpha, F} \circ f \circ \tau_{\alpha, E}^{-1}$. Thus, we have the commutative diagram connecting the two transition functions (where every arrow is invertible):
        \[
            \xymatrix{ E_{x, \alpha} \ar[r]^{\rho_E} \ar[d]_{f_\alpha} & E_{x, \beta} \ar[d]^{f_\beta} \\
            F_{x, \alpha} \ar[r]_{\rho_F} & F_{x, \beta}}
        \]
        % Extending this to three intersecting opens, the commutative diagram:
        % \[ 
        %     \xymatrix{ E_{x, \alpha} \ar[r]^{\rho_{E, \alpha\beta}} \ar[d]^{f_\alpha}  & E_{x, \beta} \ar[r]^{\rho_{E, \beta\gamma}} \ar[d]^{f_\beta}  & E_{x, \gamma} \ar[d]^{f_\gamma}  \\
        %     F_{x, \alpha} \ar[r]^{\rho_{F, \alpha\beta}} & F_{x, \beta} \ar[r]^{\rho_{F, \beta\gamma}} & F_{x, \gamma} }
        % \]
        \item We can pick a cover $\{U_\alpha\}$ that trivializes both $E$ and $F$. Then, pick $x \in U_\alpha \cap U_\beta$ and $\rho_E$ and $\rho_F$ be the transiton functions from the $U_\alpha$ trivialization to the $U_\beta$ trivialization. Then, 
        \begin{enumerate}
            \item $\rho_E \otimes_\C \rho_F$ is a transition function for $E \otimes_\C F$. 
            \item $\rho_E \oplus \rho_F$ be a transition function for $E \oplus F$.
            \item $\bigwedge^n \rho_E$ be a transition function for $\bigwedge^n E$.
        \end{enumerate}
        These are all holomorphic, as they are holomorphic on each component. 
        
    \end{enumerate}
\end{solu}


\subsection*{Questions:}
\begin{enumerate}
    \item In 5b), is it okay to just take a representative that comes from a particular $U_\alpha \times \C^n$ pre-quotienting? What's a better way to write this?
\end{enumerate}


\end{document}