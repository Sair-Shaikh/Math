\documentclass[12pt]{article}


\usepackage{fullpage}
\usepackage{mdframed}
\usepackage{colonequals}
\usepackage{algpseudocode}
\usepackage{algorithm}
\usepackage{tcolorbox}
\usepackage[all]{xy}
\usepackage{proof}
\usepackage{mathtools}
\usepackage{bbm}
\usepackage{amssymb}
\usepackage{amsthm}
\usepackage{amsmath}
\usepackage{amsxtra}
\newcommand{\bb}{\mathbb}


\newtheorem{theorem}{Theorem}[section]
\newtheorem{corollary}{Corollary}[theorem]
\newtheorem{lemma}{Lemma}

\newcommand{\mathcat}[1]{\textup{\textbf{\textsf{#1}}}} % for defined terms

\newenvironment{problem}[1]
{\begin{tcolorbox}\noindent\textbf{Problem #1}.}
{\vskip 6pt \end{tcolorbox}}

\newenvironment{enumalph}
{\begin{enumerate}\renewcommand{\labelenumi}{\textnormal{(\alph{enumi})}}}
{\end{enumerate}}

\newenvironment{enumroman}
{\begin{enumerate}\renewcommand{\labelenumi}{\textnormal{(\roman{enumi})}}}
{\end{enumerate}}

\newcommand{\defi}[1]{\textsf{#1}} % for defined terms

\theoremstyle{remark}
\newtheorem*{solution}{Solution}

\setlength{\hfuzz}{4pt}

\newcommand{\calC}{\mathcal{C}}
\newcommand{\calF}{\mathcal{F}}
\newcommand{\C}{\mathbb C}
\newcommand{\N}{\mathbb N}
\newcommand{\Q}{\mathbb Q}
\newcommand{\R}{\mathbb R}
\newcommand{\Z}{\mathbb Z}
\newcommand{\br}{\mathbf{r}}
\newcommand{\RP}{\mathbb{RP}}
\newcommand{\CP}{\mathbb{CP}}
\newcommand{\nbit}[1]{\{0, 1\}^{#1}}
\newcommand{\bits}{\{0, 1\}^{n}}
\newcommand{\bbni}{\bigbreak \noindent}
\newcommand{\norm}[1]{\left\vert\left\vert#1\right\vert\right\vert}

\let\1\relax
\newcommand{\1}{\mathbf{1}}
\newcommand{\fr}[2]{\left(\frac{#1}{#2}\right)}

\newcommand{\vecz}{\mathbf{z}}
\newcommand{\vecr}{\mathbf{r}}
\DeclareMathOperator{\Cinf}{C^{\infty}}
\DeclareMathOperator{\Id}{Id}

\DeclareMathOperator{\Alt}{Alt}
\DeclareMathOperator{\ann}{ann}
\DeclareMathOperator{\codim}{codim}
\DeclareMathOperator{\End}{End}
\DeclareMathOperator{\Hom}{Hom}
\DeclareMathOperator{\id}{id}
\DeclareMathOperator{\M}{M}
\DeclareMathOperator{\Mat}{Mat}
\DeclareMathOperator{\Ob}{Ob}
\DeclareMathOperator{\opchar}{char}
\DeclareMathOperator{\opspan}{span}
\DeclareMathOperator{\rk}{rk}
\DeclareMathOperator{\sgn}{sgn}
\DeclareMathOperator{\Sym}{Sym}
\DeclareMathOperator{\tr}{tr}
\DeclareMathOperator{\img}{img}
\DeclareMathOperator{\CandE}{CandE}
\DeclareMathOperator{\CandO}{CandO}
\DeclareMathOperator{\argmax}{argmax}
\DeclareMathOperator{\first}{first}
\DeclareMathOperator{\last}{last}
\DeclareMathOperator{\cost}{cost}
\DeclareMathOperator{\dist}{dist}
\DeclareMathOperator{\path}{path}
\DeclareMathOperator{\parent}{parent}
\DeclareMathOperator{\argmin}{argmin}
\DeclareMathOperator{\excess}{excess}
\let\Pr\relax
\DeclareMathOperator{\Pr}{\mathbf{Pr}}
\DeclareMathOperator{\Exp}{\mathbb{E}}
\DeclareMathOperator{\Var}{\mathbf{Var}}
\let\limsup\relax
\DeclareMathOperator{\limsup}{limsup}
%Paired Delims
\DeclarePairedDelimiter\ceil{\lceil}{\rceil}
\DeclarePairedDelimiter\floor{\lfloor}{ \rfloor}


\newcommand{\dagstar}{*}

\newcommand{\tbigwedge}{{\textstyle{\bigwedge}}}
\setlength{\parindent}{0pt}
\setlength{\parskip}{5pt}



\begin{document}


\title{CS 40: Computational Complexity}

\author{Sair Shaikh}
\maketitle

% Collaboration Notice: Talked to Henry Scheible '26 to discuss ideas.


\begin{problem}{1}
    Recall that two metrics $\rho_1$ and $\rho_2$ are \emph{equivalent} on $X$ if they generate the same topology on $X$, and \emph{strongly equivalent} on $X$ if there are strictly positive numbers $c$ and $d$ such that
    \[
    c \rho_1(x, y) \leq \rho_2(x, y) \leq d \rho_1(x, y)
    \quad \text{for all } x, y \in X.
    \]
    Let $\| \cdot \|_1$ and $\| \cdot \|_2$ be norms on a vector space $X$. Let $\rho_1(x, y) = \|x - y\|_1$ and $\rho_2(x, y) = \|x - y\|_2$ be the corresponding metrics. Show that $\rho_1$ and $\rho_2$ are equivalent if and only if they are strongly equivalent.
\end{problem}
\begin{solution}
    First assume that $\rho_1$ and $\rho_2$ are strongly equivalent. Then, there exist positive constants $c$ and $d$ such that:
    \[c \|x - y\|_1 \leq \|x - y\|_2 \leq d \|x - y\|_1\quad \text{for all } x, y \in X \]
    Using Problem 5, we note that $\rho_1$ and $\rho_2$ are equivalent if and only if they have the same convergent sequences. Thus, let $(x_n) \subset (X, \rho_2)$ converge to $x$. Let $\epsilon > 0$. Then, there exists $N$ such that for all $n \geq N$, we have: 
    \[ \rho_2(x_n, x) < c \cdot \epsilon\]
    Then, since $c > 0$, we have for all $n \geq N$:
    \[ \rho_1(x_n, x) \leq \frac{1}{c}\rho_2(x_n, x) < \epsilon\]
    Thus, $(x_n) \to x$ in $(X, \rho_1)$. Moreover, manipulating the inequalities above, we also have: 
    \[\frac{1}{d} \|x - y\|_2 \leq \|x - y\|_1 \leq \frac{1}{c} \|x - y\|_2 \quad \text{for all } x, y \in X \]
    Then, as $\frac{1}{d} > 0$, we can use the same argument to show that a sequence converging in $(X, \rho_1)$ also converges in $(X, \rho_2)$. Thus, $\rho_1$ and $\rho_2$ have the same convergent sequences and thus, by Problem 5, they are equivalent. \bbni
    Next, assume that $\rho_1$ and $\rho_2$ are equivalent, i.e. they generate the same topology on $X$. Then, by Problem 5, they have the same convergent sequences. Assume to the contrary that there is no such $c > 0$ such that $\norm{x-y}_1 \leq c\cdot \norm{x-y}_2$ for all $x, y \in X$. Thus, for each $n \in \N$, there exists $x_n, y_n \in X$ such that:
    \[ \norm{x_n-y_n}_1 > n \cdot \norm{x_n-y_n}_2\]
    Thus, noting that $\norm{x_n-y_n}_1 > n > 0$ by the previous inequality, we get:
    \[\frac{\norm{x_n-y_n}_2}{\norm{x_n-y_n}_1} < \frac{1}{n}\]
    Let $z_n := \frac{x_n - y_n}{\norm{x_n-y_n}_1}$. Then, we have $\norm{z_n}_1 = 1$ for all $n \in \N$. Moreover, by homogeneity, we have: 
    \[ \norm{z_n}_2 = \frac{1}{\norm{x_n-y_n}_1}\norm{x_n-y_n}_2 < \frac{1}{n}\]
    Thus, consider the sequence $(z_n) \subset (X, \rho_2)$. For $\epsilon > 0$, there exists $N \in \N$ such that $\frac{1}{N} < \epsilon$. Then, for all $n \geq N$, we have:
    \[ \norm{z_n}_2 < \frac{1}{n} \leq \frac{1}{N}< \epsilon\]
    Thus, $(z_n) \to 0$ in $(X, \rho_2)$. However, since $\norm{z_n}_1 = 1$ for all $n$, we have that $(z_n)$ does not converge to $0$ in $(X, \rho_1)$. This is a contradiction, thus there exists a positive constant $c$ such that:
    \[ \norm{x-y}_1 \leq c\cdot \norm{x-y}_2\]
    for all $x, y \in X$. \bbni
    Using the same argument, swapping $\rho_1$ and $\rho_2$, we can show that there exists a positive constant $d$ such that:
    \[ \norm{x-y}_2 \leq d\cdot \norm{x-y}_1\]
    for all $x, y \in X$. Then, we have:
    \[ \frac{1}{c}\norm{x-y}_1 \leq \norm{x-y}_2 \leq d\cdot\norm{x-y}_1\]
    for all $x, y \in X$. Thus, $\rho_1$ and $\rho_2$ are strongly equivalent.
\end{solution}
\newpage

\begin{problem}{2}
    Let $X$ be a Banach space and $Y$ a closed subspace. We say that $Y$ is \emph{complemented} in $X$ if there is a closed subspace $Z$ such that $X = Y \oplus Z$. Show that $Y$ is complemented in $X$ if and only if there is a bounded linear map $P : X \to X$ such that $P = P^2$ and $P(X) = Y$. That is, $Y$ is complemented if and only if there is a continuous projection $P$ with range $Y$. (To show $P$ is continuous, I used the Closed Graph Theorem.)
\end{problem}
\begin{solution}
    First, assume that $Y$ is complemented in $X$, i.e. there exists a subspace $Z$ such that $X = Y \oplus Z$. Then, every $x \in X$ can be written uniquely as $x = y+z$ for $y \in Y$ and $z \in Z$. Define the map $P: X \to X$ by $P(x) = y$. We show that $P$ is a bounded linear map satisfying $P = P^2$ and $P(X) = Y$. \bbni
    First, we show linearity. Let $x_1, x_2 \in X$ and $\alpha \in \F$. Let $x_1 = y_1 + z_1$ and $x_2 = y_2 + z_2$ for $y_1, y_2 \in Y$ and $z_1, z_2 \in Z$. Then, we have:
    \begin{align*}
        P(\alpha x_1 + x_2) &= P(\alpha(y_1 + z_1) + (y_2 + z_2)) \\
        &= \alpha P(y_1) + \alpha P(z_1) + P(y_2) + P(z_2) \\
        &= \alpha y_1 + y_2 \\
        &= \alpha P(x_1) + P(x_2)
    \end{align*}
    Hence, $P$ is linear. \bbni
    Next, we use the Closed Graph Theorem to show that $P$ is bounded. Let $x_n \to x$ in $X$ and $P(x_n) \to y'$ in $X$. Let $x = y+z$ where $y \in Y, z \in Z$. We need to show that $y' = P(x) = y$. Since $x_n \to x$, we can write $x_n = y_n + z_n$ for $y_n \in Y$ and $z_n \in Z$. Then, note that:
    \[ P(x_n) = y_n \]
    Thus, by uniqueness of limits (Hausdorff space), it suffices to show that $y_n \to y$. Notice that since $X = Y \oplus Z$, we have an (vector space) isomorphism from $X \to Y \times Z$ given by $x' \mapsto (y', z')$ where $x' = y'+z'$, $y' \in Y, z' \in Z$ is the unique decomposition. Bijectivity and linearity are both immediate. \todo{What?} Since $Y$ and $Z$ are closed, they are Banach spaces with the subspace topology. Using the results of Problem 27, we conclude that $Y \times Z$ is a Banach space under the product metric $\delta((y_1, z_1), (y_2, z_2)) = \norm{y_1 - y_2} + \norm{z_1 - z_2}$. Moreover, since $(x_n) \to x$ in $X$, we must have $(y_n) \to y$ in $Y$ and $(z_n) \to z$ in $Z$ (Problem 27). Thus, we have shown that: 
    \[ (P(x_n)) = (y_n) \to y = P(x)\]
    Thus, $P$ is continuous and bounded by the Closed Graph Theorem. \\
    \todo{Check this argument!} \bbni
    By definition of $P$, we have that $P(x) \in Y$ for all $x \in X$. Thus, $P(X) \subset Y$. Moreover, also by definition, we have $P(y) = y$ for $y \in Y$. This implies $Y = P(Y) \subset P(X)$. Thus, $P(X) = Y$. \bbni
    Finally, for any $x \in X$, as $P(x) \in Y$, we have that $P$ acts as the identity on $Y$. Thus, we have:
    \[ P^2(x) = P(x)\]
    Hence, $P = P^2$. \bbni
    Conversely, assume that there exists a bounded linear map $P: X \to X$ with $P(X) = Y$ and $P = P^2$. We need to show that there exists a closed subspace $Z$ such that $X = Y \oplus Z$. \bbni
    First, we show that $P$ is the identity on $Y$. Let $y \in Y$. Then, there exists $x \in X$ such that $P(x) = y$ as $P(X) = Y$. Then, 
    \[ P(y) = P^2(x) = P(x) = y\]
    Thus, $P$ acts as the identity on $Y$. \bbni 
    Let $Z = \ker(P)$. Since $\{0\}$ is closed, $Z = P^{-1}(0)$ is closed as $P$ is continous. Moreover, since $P$ is the identity on $Y$, for $y \in Y$, $P(y) = 0$ implies $y = 0$. Thus, $Y \cap Z = \{0\}$. Hence, we only need to show that $X = Y \oplus Z$.  \bbni    
    Consider the map $I-P \in \mathcal L(X)$, where $I$ is the identity map. We claim that that $(I-P)^2 = I-P$ and $(I-P)(X) = Z$. To see this, note that for $x = y+z \in X$, with $y \in Y$ and $z \in Z$, we have:
    \[ (I-P)(x) = (I-P)(y+z) = (y-P(y))+(z-P(z)) = (y-y)+(z-0) = z\]
    Thus, $(I-P)(X) = Z$. Moreover, if $z \in Z$, we have: 
    \[ (I-P)(z) = z-P(z) = z\]
    Thus, $(I-P)$ acts as the identity on $Z$. Thus, $Z = (I-P)(Z) \subset (I-P)(X)$. Additionally, for $x \in X$, as $(I-P)(x) \in Z$, as $(I-P)$ acts as the identity on $Z$, we have: 
    \[ (I-P)^2(x) = (I-P)(x)\]
    Finally, as we can write the identity map as $I = P + (I-P)$, we have a unique way to write, for every $x \in X$, 
    \[ x = P(x) + (I-P)(x)\]
    with $P(x) \in Y$ and $(I-P)(x) \in Z$. Thus, we have $X = Y \oplus Z$ and $Y$ is complemented by $\ker(P)$. \bbni






    % Noting that $x_n$ is Cauchy, we claim that $y_n$ and $z_n$ are also Cauchy. Assume $y_n$ was not Cauchy, there there exists $\epsilon > 0$ such that for all $N$, there exists $n, m \geq N$ such that $\norm{y_n - y_m} \geq \epsilon$.
    
    
    % We claim that for $x = y+z \in X$, with $y \in Y$, $z \in Z$, that we must have $P(y) = y$. For any $y \in Y$, we have that $y = P(x)$ for some $x \in X$, as $P(X) = Y$. Then, 
    % \[ P(y) = P^2(x) = P(x) = y\]
    % Thus, $P$ is the identity on $Y$. Next, write $x = P(x) + (x - P(x))$. 
    
\end{solution}
\newpage

\begin{problem}{3}
    Suppose that $X$ is a Banach space and that $X = Y \oplus Z$ for closed subspaces $Y$ and $Z$. Let $X/Y$ be the quotient Banach space and $q : X \to X/Y$ the quotient map. Show that there is a continuous isomorphism $\Psi : X/Y \to Z$. (Therefore the Open Mapping Theorem implies that $\Psi^{-1}(z) = q(z)$ is a continuous isomorphism of $Z$ onto $X/Y$.)
\end{problem}
\begin{solution}
    We will write $[x]$ for $q(x)$ throughout this solution. \bbni
    First assume that $Y = 0$, then $X = X/Y = Z$. In this case, let $\Psi$ be the identity map, which is clearly a continous isomorphism. Next, assume $Z = \{0\}$. Then, $X = Y$ and thus, $X/Y = \{0\}$. Thus, let $\Psi$ be the identity map. This is also a continous isomorphism. \bbni
    Now, we assume that both $Y$ and $Z$ are proper subspaces of $X$. By the previous problem, as $Z$ is closed and complemented, we know that there exists a continous projection $P: X \to Z$. Additionally, we know that $\ker(P) = Y$ from the proof of the previous problem. \todo{Check this!} \bbni
    Then, using Problem 31, since $Y \subset \ker(P)$ is a proper closed subspace, there exists a unique bounded linear map $\Psi: X/Y \to X$ satisfying $\Psi([x]) = P(x)$ for all $x \in X$ and satisfying $\norm{\Psi} = \norm{P}$. Since $P(X) = Z$, we actually have $\Psi: X/Y \to Z$. Thus, we only need to show that $\Psi$ is a bijection. \bbni
    To show injectivity, let $[x]\in X/Y$ be such that $\Psi([x]) = 0$. Then, $P(x) = 0$, which implies that $x \in Y = \ker(P)$. Thus, $[x] = [0]$. Hence, $\Psi$ is injective. \bbni
    To show surjectivity, let $z \in Z \subset X$. Then, note that: 
    \[ \Psi([z]) = P(z) = z\]
    as $P$ acts as the identity on $Z$ (previous problem). Thus, $\Psi$ is surjective. \bbni
    Thus, we have constructed $\Psi$ as a bounded (hence continuous) isomorphism $X/Y \to Z$.
\end{solution}
\newpage

\begin{problem}{4}
    Let $c_0$ be the subspace of $\ell^\infty$ of sequences $(x_n)$ such that $\lim_n x_n = 0$.
    \begin{enumerate}
        \item Show that $c_0$ is closed in $\ell^\infty$.
        \item Let $q : \ell^\infty \to \ell^\infty / c_0$ be the quotient map. Show that the quotient norm is given by
        \[
        \|q(x)\| = \limsup_n |x_n|.
        \]
    \end{enumerate}
\end{problem}
\begin{solution}
    \bbni
    \begin{enumerate}
        \item Let $(x^{(k)}) \subset c_0$ be a sequence converging to $x \in \ell^\infty$. We need to show that $x \in c_0$. \bbni
        Let $\epsilon > 0$. Since $x^{(k)} \to x$, there exists $N$ such that for all $k \geq N$, we have:
        \[ \norm{x^{(k)} - x}_\infty < \frac\epsilon 2\]
        Moreover, since $x^{(N)} \in c_0$, there exists $M$ such that for all $n \geq M$, we have: 
        \[ |x^{(N)}_n| < \frac\epsilon 2\]
        Thus, for all $n \geq M$, we have: 
        \begin{align*}
            |x_n| &\leq |x_n - x^{(N)}_n| + |x^{(N)}_n| \\
            &\leq \norm{x-x^{(N)}}_\infty + |x^{(N)}_n| \\
            &< \frac\epsilon 2 + \frac\epsilon 2 \\
            &= \epsilon
        \end{align*}
        Thus, $\lim_n |x_n| = 0$. Therefore, $x \in c_0$, and hence $c_0$ is closed in $\ell^\infty$.
        \item Let $x \in \ell^\infty$. Recall the quotient norm is defined as: 
        \[ \norm{q(x)} = \inf\{ \norm{x-y}_\infty: y \in c_0 \} = \inf\{ \norm{x+y}_\infty: y \in c_0 \}\]
        Pick $(y^{(k)}) \subset c_0$ such that $y_i^{(k)} = x_i$ for $i < k$ and $y_i^{(k)} = 0$ for $i \geq k$. Note that by definition, $\lim_n y^{(k)}_n = 0$. Notice that since the first $k$ entries of $x-y^{(k)}$ are $0$, we have:
        \[ \norm{x-y^{(k)}} = \sup_{n\geq k} |x_n-y_n| = \sup_{n\geq k} |x_n|\]
        Using this, we compute: 
        \begin{align*}
            \norm{q(x)} &= \inf\{\norm{x-y}_\infty: y \in c_0\} \\
            &\leq \inf_k \norm{x-y^{(k)}}_\infty \\
            &= \inf_k \sup_{n \geq k} |x_n| \\
            &= \limsup_n |x_n|
        \end{align*}
        Where the last equality follows as $\sup_{n \geq k'} |x_n| \leq \sup_{n \geq k} |x_n|$ for all $k' \geq k$, i.e., since it is non-increasing, the infimum is the limit. \todo{Check this!} \bbni
        To show the other direction, we need to show that $\norm{q(x)}$ gets arbitrarily close to $\limsup_n |x_n|$, i.e. for $\epsilon > 0$, we have: 
        \[ \norm{q(x)} \geq \limsup_n |x_n| - \epsilon\]
        Let $y \in c_0$ be arbitrary. Then, there exists $N$ such that $|y_n| < \epsilon$. Then, we have:
        \begin{align*}
            \norm{x+y}_\infty &\geq \limsup_n |x_n + y_n| \\
            &\geq \limsup_n |x_n| - \limsup_n |y_n| \\
            &\geq \limsup_n |x_n| - \epsilon
        \end{align*}
        Thus,
        \[ \norm{q(x)} \geq \limsup_n |x_n| - \epsilon\]
        Therefore, we conclude:
        \[ \norm{q(x)} = \limsup_n |x_n|\]
    \end{enumerate}
\end{solution}
\newpage

\begin{problem}{5}
    Let $E$ and $F$ be closed subspaces of a Hilbert space $H$ with $\dim E < \infty$ and $\dim E < \dim F$. Show that $E^\perp \cap F \neq \{0\}$.
\end{problem}
\begin{solution}
    Assume for the sake of contradiction that $E^\perp \cap F = \{0\}$. Since $E$ is closed, we can write $H = E \oplus E^\perp$. Then, consider the projection map $P : H \to E$. For $f \neq 0 \in F$, we have $f \not\in E^\perp$. Thus, $P(f) \neq 0$. Taking the contrapositive, this implies for $f \in F$, $P(f) = 0$ if and only if $f = 0$. Thus, $P|_F$ is injective. Since $\dim E < \infty$ and $\dim E < \dim F$, this is a contradiction. Thus, $E^\perp \cap F \neq \{0\}$.
\end{solution}
\newpage

\begin{problem}{6}
    Suppose that $H$ is a Hilbert space and that $T : H \to H$ is linear and norm-weak continuous. Show that $T$ is bounded.
\end{problem}
\begin{solution}
    Let $\omega$ be the weak topology on $H$. We are given that: 
    \[ T: (H, \norm{\cdot}) \to (H, \omega)\]
    is continuous. For $h \in H$, let $\phi_h: (H, \omega) \to \F$ be the linear functional defined by: 
    \[ \phi_h(x) = (x \mid h) \]
    Note that by the definition of the weak topology, $\phi_h$ is continuous for all $h \in H$. Thus, $\phi_h \circ T: (H, \norm{\cdot}) \to \F$ is a composition of continuous maps, and hence is continuous. Thus, $\phi_h \circ T$ is bounded for all $h \in H$. \bbni 
    Now, let $x_\lambda \to x \subset (H, \norm{\cdot})$ be a convergent net. Then, since $\phi_h \circ T$ is continuous, by Problem 48, we have that:
    \[ \phi_h(T(x_\lambda))  \to \phi_h(T(x))\]
    Then, note that $T(x_\lambda)$ is a net, such that for each $\phi_h$, we have $\phi_h(T(x_\lambda)) \to \phi_h(T(x))$. Moreover, as we proved in class, every functional in $H^*$ is of the form $\phi_h$. Thus, by Problem 49, we have that $T(x_\lambda) \to T(x)$ in the weak topology. \bbni
    Thus, we have shown that $T$ takes a convergent net to a convergent net. By Problem 48, this implies that $T$ is continuous. Thus, $T$ is bounded.    
    \todo{Check this whole thing!}
\end{solution}




\end{document}