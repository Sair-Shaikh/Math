\documentclass[12pt]{article}


\usepackage{fullpage}
\usepackage{mdframed}
\usepackage{colonequals}
\usepackage{algpseudocode}
\usepackage{algorithm}
\usepackage{tcolorbox}
\usepackage[all]{xy}
\usepackage{proof}
\usepackage{mathtools}
\usepackage{bbm}
\usepackage{amssymb}
\usepackage{amsthm}
\usepackage{amsmath}
\usepackage{amsxtra}
\newcommand{\bb}{\mathbb}


\newtheorem{theorem}{Theorem}[section]
\newtheorem{corollary}{Corollary}[theorem]
\newtheorem{lemma}{Lemma}

\newcommand{\mathcat}[1]{\textup{\textbf{\textsf{#1}}}} % for defined terms

\newenvironment{problem}[1]
{\begin{tcolorbox}\noindent\textbf{Problem #1}.}
{\vskip 6pt \end{tcolorbox}}

\newenvironment{enumalph}
{\begin{enumerate}\renewcommand{\labelenumi}{\textnormal{(\alph{enumi})}}}
{\end{enumerate}}

\newenvironment{enumroman}
{\begin{enumerate}\renewcommand{\labelenumi}{\textnormal{(\roman{enumi})}}}
{\end{enumerate}}

\newcommand{\defi}[1]{\textsf{#1}} % for defined terms

\theoremstyle{remark}
\newtheorem*{solution}{Solution}

\setlength{\hfuzz}{4pt}

\newcommand{\calC}{\mathcal{C}}
\newcommand{\calF}{\mathcal{F}}
\newcommand{\C}{\mathbb C}
\newcommand{\N}{\mathbb N}
\newcommand{\Q}{\mathbb Q}
\newcommand{\R}{\mathbb R}
\newcommand{\Z}{\mathbb Z}
\newcommand{\F}{\mathbb F}
\newcommand{\br}{\mathbf{r}}
\newcommand{\RP}{\mathbb{RP}}
\newcommand{\CP}{\mathbb{CP}}
\newcommand{\nbit}[1]{\{0, 1\}^{#1}}
\newcommand{\bits}{\{0, 1\}^{n}}
\newcommand{\bbni}{\bigbreak \noindent}
\newcommand{\norm}[1]{\left\vert\left\vert#1\right\vert\right\vert}
\newcommand{\dbar}{\overline{\partial}}
\let\d\relax
\let\calF\relax
\newcommand{\d}{\partial}
\newcommand{\calO}{\mathcal{O}}
\newcommand{\calF}{\mathcal{F}}
\newcommand{\calG}{\mathcal{G}}
\newcommand{\calH}{\mathcal{H}}
\newcommand{\calE}{\mathcal{E}}

\let\1\relax
\newcommand{\1}{\mathbf{1}}
\newcommand{\fr}[2]{\left(\frac{#1}{#2}\right)}

\newcommand{\vecz}{\mathbf{z}}
\newcommand{\vecr}{\mathbf{r}}
\DeclareMathOperator{\Cinf}{C^{\infty}}
\DeclareMathOperator{\Id}{Id}

\DeclareMathOperator{\Alt}{Alt}
\DeclareMathOperator{\ann}{ann}
\DeclareMathOperator{\codim}{codim}
\DeclareMathOperator{\End}{End}
\DeclareMathOperator{\Hom}{Hom}
\DeclareMathOperator{\id}{id}
\DeclareMathOperator{\M}{M}
\DeclareMathOperator{\Mat}{Mat}
\DeclareMathOperator{\Ob}{Ob}
\DeclareMathOperator{\opchar}{char}
\DeclareMathOperator{\opspan}{span}
\DeclareMathOperator{\rk}{rk}
\DeclareMathOperator{\sgn}{sgn}
\DeclareMathOperator{\Sym}{Sym}
\DeclareMathOperator{\tr}{tr}
\DeclareMathOperator{\img}{img}
\DeclareMathOperator{\CandE}{CandE}
\DeclareMathOperator{\CandO}{CandO}
\DeclareMathOperator{\argmax}{argmax}
\DeclareMathOperator{\first}{first}
\DeclareMathOperator{\last}{last}
\DeclareMathOperator{\cost}{cost}
\DeclareMathOperator{\dist}{dist}
\DeclareMathOperator{\path}{path}
\DeclareMathOperator{\parent}{parent}
\DeclareMathOperator{\argmin}{argmin}
\DeclareMathOperator{\excess}{excess}
\let\Pr\relax
\DeclareMathOperator{\Pr}{\mathbf{Pr}}
\DeclareMathOperator{\Exp}{\mathbb{E}}
\DeclareMathOperator{\Var}{\mathbf{Var}}
\let\limsup\relax
\DeclareMathOperator{\limsup}{limsup}
%Paired Delims
\DeclarePairedDelimiter\ceil{\lceil}{\rceil}
\DeclarePairedDelimiter\floor{\lfloor}{ \rfloor}


\newcommand{\dagstar}{*}

\newcommand{\tbigwedge}{{\textstyle{\bigwedge}}}
\setlength{\parindent}{0pt}
\setlength{\parskip}{5pt}



\begin{document}


\title{CS 40: Computational Complexity}

\author{Sair Shaikh}
\maketitle

Collaboration Notice: Talked to Henry Scheible '26 to discuss ideas.



\begin{problem}{27}
Let $\text{UNSAT}$ be the language $\{\langle \varphi \rangle : \varphi \text{ is an unsatisfiable CNF formula}\}$. 
Note that I said “CNF” and not “3CNF” (not that this will matter much, but please stick to the given definition).

Describe an interactive proof (IP) protocol for $\text{UNSAT}$ that has perfect completeness, a soundness error of at most $1/3$, 
and where the prover and verifier exchange only $O(n / \log n)$ messages in total, $n$ being the number of variables in the input
formula $\varphi$. As usual, the verifier needs to run in time polynomial in the length of the input.

\end{problem}


\begin{solution}
    The protocol follows a structure similar to the protocol for $\#\text{SAT}$ described in the class and in Sipser 10.4, except instead of removing one variable at each step, we will remove $\log n$ variables. \bbni

    First, the verifier arithmetizes the CNF formula in the input $w$ to obtain a polynomial $f(x_1, \cdots, x_n)$ over some finite field $\F_{q}$ with $q > 2^{|w|}$. As the arithmetization equivalents for $\land$ and $\lor$ result in summing the degrees of their operands, the degree of each term in $f$ (expanded out) is at most the input length, $|w|$. Call the total degree of this polynomial $d \leq |w|$. \bbni
    Let 
    \[ f_i(x_1, \cdots, x_i) := \sum_{x_{i+1}, \cdots, x_n \in \{0, 1\}} f(x_1, \cdots, x_n) \]
    To show that the CNF has no satisfying assignments, the prover needs to prove that:
    \[ f_0() = 0\]
    in $O(n/\log n)$ interactions. In the $i$th phase of interaction, the Prover persuades the verifier that $f_{(i-1)\log n}(r_1, \cdots, r_{(i-1)\log n})$ is correct if $f_{i\log n}(r_1, \cdots, r_{i\log n})$ is correct for $1 \leq i \leq n/\log n$ where $r_i$ are random elements in $\F_q$ chosen by the verifier. This is done as follows: 
    \begin{enumerate}
        \item The prover sends the coefficients of $f_{i\log n}(r_1, \cdots, r_{(i-1)\log n}, z_1, \cdots, z_{\log n})$ (a polynomial in $\log n$ variables) over to the prover. Note that each coefficient is in $\F_q$, which is finite. Moreover, the number of coefficients is polynomial in the input length as each term of the polynomial must have degree at most $d \leq |w|$ (reject if not), thus we can use stars and bars to determine that there are only polynomially many monomials possible. Thus this is a polynomial-sized message.
        \item The verifier evaluates the provided polynomial at all values $s_1, \cdots, s_{\log n} \in \{0, 1\}^{\log n}$ (which is polynomially many) and adds up the results (mod $q$). If this does not agree with $f_{(i-1)\log n}(r_1, \cdots, r_{(i-1)\log n})$ (take $f_0() = 0$ if $i=1$) from the previous interaction step, reject. 
        \item Finally, the verifier sends the prover $r_{(i-1)\log n +1}, \cdots, r_{i\log n}$ uniformly from $\F_q$ and computes $f_{i \log n}(r_1, \cdots, r_{i \log n})$ for the next interaction step.     
    \end{enumerate}
    Finally, the verifier checks $f_n(r_1, \cdots, r_n) = f(r_1, \cdots, r_n)$ himself. \bbni

    It is clear that the verifier operates in polynomial time. We argue the correctness of this protocol similarly to in class. If $f_0() = 0$, then a prover sending the correct $f_{i}$s at every step can clearly make the verifier accept, thus we have perfect completeness. Thus, we only need to the analyze soundness error. \bbni 

    If $f_0() \neq 0$ and we do not reject in the first round, then the polynomial the prover sent in the first round, $\tilde f_{\log n}$ must disagree with the ``true'' polynomial $f_{\log n}$ at at least one value for $(z_1, \cdots, z_{\log n}) \in \{0, 1\}^n$, i.e. $f_{\log n} \not \equiv \tilde{f}_{\log n}$. Thus, by the Schwartz-Zippel lemma,        
    \[ \Pr_{\vec{r}}[\tilde{f}_{\log n}(r_1, \cdots, r_{\log n})-f_{\log n}(r_1, \cdots, r_{\log n})= 0 ] \leq \frac{d}{|\F^q|} \]
    
    The upshot of this is that the value of $\tilde{f}_{\log n}(r_1, \cdots, r_{\log n})$ that the verifier uses in the subsequent step will be wrong with high probability (we'll show $\frac{d}{|\F^q|}$ is small). We can repeat the same logic in general: If the value $\tilde f_{(i-1)\log n}(r_1, \cdots, r_{(i-1)\log n}) \neq f_{(i-1)\log n}(r_1, \cdots, r_{(i-1)\log n})$ (is not correct) and we do not reject in the $i$th step, then it must be that $\tilde f_{i\log n} \not \equiv f_{i\log n}$. Thus, we have two cases: 
    \begin{itemize}
        \item If $\forall i: \tilde f_{(i-1)\log n}(r_1, \cdots, r_{(i-1)\log n}) \neq f_{(i-1)\log n}(r_1, \cdots, r_{(i-1)\log n})$, then at the final step, the verifier will directly evaluate $f(r_1, \cdots, r_n)$ and catch this. Thus, we accept with probability $0$ in this case. 
        \item Otherwise there exists some step $k$ where 
        \[ \tilde f_{(i-1)\log n}(r_1, \cdots, r_{(i-1)\log n}) = f_{(i-1)\log n}(r_1, \cdots, r_{(i-1)\log n})\] 
        Taking the union bound, and applying Schwarz-Zippel, 
        \begin{align*}
            \Pr\left[P * V(w, r) = 1 \right] &\leq \Pr\left[\bigcup_{i=1}^{n/\log n} \tilde{f}_{i \log n}(r_1, \cdots, r_{i \log n}) - f_{i \log n}(r_1, \cdots, r_{i \log n}) = 0 \right] \\
            &\leq \sum_{i=1}^{n/\log n} \Pr\left[ \tilde{f}_{i \log n}(r_1, \cdots, r_{i \log n}) = f_{i \log n}(r_1, \cdots, r_{i \log n}) \right] \\
            &\leq \frac{n}{\log n} \cdot \frac{d}{|\F_q|} \\ 
            &\leq \frac{|w|}{\log |w|} \cdot \frac{|w|}{2^{|w|}} = o(1)
        \end{align*}
    \end{itemize}
    Thus, for large enough messages, this probability is $< \frac{1}{3}$. If we want to improve this, we could pick an even bigger field $\F_q$. The question of finding a big prime of that form is not answered in this problem, but can be done to a high accuracy by the verifier using probabilisitic primality testing methods seen before. 
\end{solution}



\end{document}