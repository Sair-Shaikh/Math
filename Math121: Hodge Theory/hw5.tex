\documentclass[12pt]{article}

\usepackage{fullpage}
\usepackage{mdframed}
\usepackage{colonequals}
\usepackage{algpseudocode}
\usepackage{algorithm}
\usepackage{tcolorbox}
\usepackage[all]{xy}
\usepackage{proof}
\usepackage{mathtools}
\usepackage{bbm}
\usepackage{amssymb}
\usepackage{amsthm}
\usepackage{amsmath}
\usepackage{amsxtra}
\newcommand{\bb}{\mathbb}


\newtheorem{theorem}{Theorem}[section]
\newtheorem{corollary}{Corollary}[theorem]
\newtheorem{lemma}{Lemma}

\newcommand{\mathcat}[1]{\textup{\textbf{\textsf{#1}}}} % for defined terms

\newenvironment{problem}[1]
{\begin{tcolorbox}\noindent\textbf{Problem #1}.}
{\vskip 6pt \end{tcolorbox}}

\newenvironment{enumalph}
{\begin{enumerate}\renewcommand{\labelenumi}{\textnormal{(\alph{enumi})}}}
{\end{enumerate}}

\newenvironment{enumroman}
{\begin{enumerate}\renewcommand{\labelenumi}{\textnormal{(\roman{enumi})}}}
{\end{enumerate}}

\newcommand{\defi}[1]{\textsf{#1}} % for defined terms

\theoremstyle{remark}
\newtheorem*{solution}{Solution}

\setlength{\hfuzz}{4pt}

\newcommand{\calC}{\mathcal{C}}
\newcommand{\calF}{\mathcal{F}}
\newcommand{\C}{\mathbb C}
\newcommand{\N}{\mathbb N}
\newcommand{\Q}{\mathbb Q}
\newcommand{\R}{\mathbb R}
\newcommand{\Z}{\mathbb Z}
\newcommand{\F}{\mathbb F}
\newcommand{\br}{\mathbf{r}}
\newcommand{\RP}{\mathbb{RP}}
\newcommand{\CP}{\mathbb{CP}}
\newcommand{\nbit}[1]{\{0, 1\}^{#1}}
\newcommand{\bits}{\{0, 1\}^{n}}
\newcommand{\bbni}{\bigbreak \noindent}
\newcommand{\norm}[1]{\left\vert\left\vert#1\right\vert\right\vert}
\newcommand{\dbar}{\overline{\partial}}
\let\d\relax
\let\calF\relax
\newcommand{\d}{\partial}
\newcommand{\calO}{\mathcal{O}}
\newcommand{\calF}{\mathcal{F}}
\newcommand{\calG}{\mathcal{G}}
\newcommand{\calH}{\mathcal{H}}
\newcommand{\calE}{\mathcal{E}}

\let\1\relax
\newcommand{\1}{\mathbf{1}}
\newcommand{\fr}[2]{\left(\frac{#1}{#2}\right)}

\newcommand{\vecz}{\mathbf{z}}
\newcommand{\vecr}{\mathbf{r}}
\DeclareMathOperator{\Cinf}{C^{\infty}}
\DeclareMathOperator{\Id}{Id}

\DeclareMathOperator{\Alt}{Alt}
\DeclareMathOperator{\ann}{ann}
\DeclareMathOperator{\codim}{codim}
\DeclareMathOperator{\End}{End}
\DeclareMathOperator{\Hom}{Hom}
\DeclareMathOperator{\id}{id}
\DeclareMathOperator{\M}{M}
\DeclareMathOperator{\Mat}{Mat}
\DeclareMathOperator{\Ob}{Ob}
\DeclareMathOperator{\opchar}{char}
\DeclareMathOperator{\opspan}{span}
\DeclareMathOperator{\rk}{rk}
\DeclareMathOperator{\sgn}{sgn}
\DeclareMathOperator{\Sym}{Sym}
\DeclareMathOperator{\tr}{tr}
\DeclareMathOperator{\img}{img}
\DeclareMathOperator{\CandE}{CandE}
\DeclareMathOperator{\CandO}{CandO}
\DeclareMathOperator{\argmax}{argmax}
\DeclareMathOperator{\first}{first}
\DeclareMathOperator{\last}{last}
\DeclareMathOperator{\cost}{cost}
\DeclareMathOperator{\dist}{dist}
\DeclareMathOperator{\path}{path}
\DeclareMathOperator{\parent}{parent}
\DeclareMathOperator{\argmin}{argmin}
\DeclareMathOperator{\excess}{excess}
\let\Pr\relax
\DeclareMathOperator{\Pr}{\mathbf{Pr}}
\DeclareMathOperator{\Exp}{\mathbb{E}}
\DeclareMathOperator{\Var}{\mathbf{Var}}
\let\limsup\relax
\DeclareMathOperator{\limsup}{limsup}
%Paired Delims
\DeclarePairedDelimiter\ceil{\lceil}{\rceil}
\DeclarePairedDelimiter\floor{\lfloor}{ \rfloor}


\newcommand{\dagstar}{*}

\newcommand{\tbigwedge}{{\textstyle{\bigwedge}}}
\setlength{\parindent}{0pt}
\setlength{\parskip}{5pt}


\begin{document}

\title{CS 40: Computational Complexity}

\author{Sair Shaikh}
\maketitle

Collaboration Notice: Talked to Henry Scheible '26 to discuss ideas.


\begin{problem}{1}
    \bbni
    \begin{enumerate}
        \item[(a)] Check the equivalence between the two definitions of the Hodge structure of weight $k$ given in class. 
        \item[(b)] Check the a morphism Hodge structures is strict for the Hodge filtration. 
        \item[(c)] Show that the kernel, cokernel, and image of a morphism of Hodge structures are Hodge structures.
        \item[(d)] Let $\phi: X \to Y$ a surjective holomorphic map of complex compact manifolds such that $X$ is k\"ahlerian. Show that $\phi^*$ is injective.
    \end{enumerate}
\end{problem}
\begin{solution}
    \begin{enumerate}
        \item[(a)] Assume first that we have a Hodge structure of weight $k$ $V$ such that: 
        \[ V_\C  = \bigoplus V^{p,q} \qquad \qquad V^{p, q} = \overline{V}^{q, p}\]
        Define the Hodge filtration by: 
        \[ F^p V_\C = \bigoplus_{i \geq p} V^{i, k-i}\]
        Then note that if $a > b$, we have that:
        \begin{align*}
            F^aV_\C  &= \bigoplus_{i \geq a} V^{i, k-i} \\
            &\subseteq \bigoplus_{i \geq b} V^{i, k-i} \\
            &= F^bV_\C
        \end{align*} 
        Moreover, we have that: 
        \begin{align*}
            F^pV_\C \cap \overline{F^{k-p+1}V_\C} &= \bigoplus_{i \geq p} V^{i, k-i} \cap \bigoplus_{j \geq k-p+1} \overline{V^{j, k-j}} \\
            &=  \bigoplus_{i \geq p} V^{i, k-i} \cap \bigoplus_{j \geq k-p+1} V^{k-j, j} \\
            &=  \bigoplus_{i \geq p} V^{i, k-i} \cap \bigoplus_{j' \leq p-1} V^{j', k-j'} \\
        \end{align*}
        
        
        % Assume first that we have a filtration $F^pV_\C$ that satisfies $F^pV_\C \oplus \overline{F^{k-p+1}V_\C} = V_\C$ for all $p$. This looks like: 
        % \[ F^k V_\C \subseteq F^{k-1}V_\C \subseteq \cdots \subseteq F^0 V_\C\]
        % Then, we define for $p + q = k$, we define: 
        % \[ V_{p, q} = F^pV_\C \cap \overline{F^qV_\C}\]


    \end{enumerate}
\end{solution}


\newpage



% \begin{problem}{2.1}

% \end{problem}
% \begin{solution}
% \end{solution}
% \newpage

% \begin{problem}{2.2}

% \end{problem}
% \begin{solution}
% \end{solution}
% \newpage


\begin{problem}{3}(The Hodge Decomposition for Curves)
    Let $X$ be a compact connected complex curve. We have the differential:
    \[d: \calO \to \Omega_X\]
    between the sheaf of homolormphic functions and the sheaf of holomorphic differentials.
    \begin{enumerate}
        \item[(a)] Show that $d$ is surjective with kernel equal to the constant sheaf $\C$. Hence, we have an exact sequence: 
        \[ 0 \to \C \to \calO \rightarrow \to \Omega_X \to 0\]
        \item[(b)] Deduce from Serre duality that $H^1(X, \Omega_X) \cong \C$. Deduce from Poincare duality that $H^2(X, \C) = \C$. 
        \item[(c)] Show that $(6.15)$ induces a short exact sequence: 
        \[ 0 \to H^0(X, \Omega_X) \to H^1(X, \C) \to H^1(X, \calO_X) \to 0\]
        \item[(d)] Show that the map which to a holomorphic form $\alpha$ associates the class of $\overline{\alpha}$ in $H^1(X, \calO)$ is injective.
        \item[(e)] Deduce from Serre duality that it is also surjective and that we have the decomposiiton: 
        \[ H^1(X, \C) = H^0(X, \Omega_X) \oplus \overline{H^0(X, \Omega_X)}\]
        with 
        \[ \overline{H^0(X, \Omega_X)} \cong H^1(X, \calO)\]
    \end{enumerate}

\end{problem}
\begin{solution}
\end{solution}
\newpage

\begin{problem}{4}

\end{problem}
\begin{solution}
\end{solution}
\newpage

\end{document}