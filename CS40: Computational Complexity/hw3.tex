\documentclass[12pt]{article}


\usepackage{fullpage}
\usepackage{mdframed}
\usepackage{colonequals}
\usepackage{algpseudocode}
\usepackage{algorithm}
\usepackage{tcolorbox}
\usepackage[all]{xy}
\usepackage{proof}
\usepackage{mathtools}
\usepackage{bbm}
\usepackage{amssymb}
\usepackage{amsthm}
\usepackage{amsmath}
\usepackage{amsxtra}
\newcommand{\bb}{\mathbb}


\newtheorem{theorem}{Theorem}[section]
\newtheorem{corollary}{Corollary}[theorem]
\newtheorem{lemma}{Lemma}

\newcommand{\mathcat}[1]{\textup{\textbf{\textsf{#1}}}} % for defined terms

\newenvironment{problem}[1]
{\begin{tcolorbox}\noindent\textbf{Problem #1}.}
{\vskip 6pt \end{tcolorbox}}

\newenvironment{enumalph}
{\begin{enumerate}\renewcommand{\labelenumi}{\textnormal{(\alph{enumi})}}}
{\end{enumerate}}

\newenvironment{enumroman}
{\begin{enumerate}\renewcommand{\labelenumi}{\textnormal{(\roman{enumi})}}}
{\end{enumerate}}

\newcommand{\defi}[1]{\textsf{#1}} % for defined terms

\theoremstyle{remark}
\newtheorem*{solution}{Solution}

\setlength{\hfuzz}{4pt}

\newcommand{\calC}{\mathcal{C}}
\newcommand{\calF}{\mathcal{F}}
\newcommand{\C}{\mathbb C}
\newcommand{\N}{\mathbb N}
\newcommand{\Q}{\mathbb Q}
\newcommand{\R}{\mathbb R}
\newcommand{\Z}{\mathbb Z}
\newcommand{\br}{\mathbf{r}}
\newcommand{\RP}{\mathbb{RP}}
\newcommand{\CP}{\mathbb{CP}}
\newcommand{\nbit}[1]{\{0, 1\}^{#1}}
\newcommand{\bits}{\{0, 1\}^{n}}
\newcommand{\bbni}{\bigbreak \noindent}
\newcommand{\norm}[1]{\left\vert\left\vert#1\right\vert\right\vert}

\let\1\relax
\newcommand{\1}{\mathbf{1}}
\newcommand{\fr}[2]{\left(\frac{#1}{#2}\right)}

\newcommand{\vecz}{\mathbf{z}}
\newcommand{\vecr}{\mathbf{r}}
\DeclareMathOperator{\Cinf}{C^{\infty}}
\DeclareMathOperator{\Id}{Id}

\DeclareMathOperator{\Alt}{Alt}
\DeclareMathOperator{\ann}{ann}
\DeclareMathOperator{\codim}{codim}
\DeclareMathOperator{\End}{End}
\DeclareMathOperator{\Hom}{Hom}
\DeclareMathOperator{\id}{id}
\DeclareMathOperator{\M}{M}
\DeclareMathOperator{\Mat}{Mat}
\DeclareMathOperator{\Ob}{Ob}
\DeclareMathOperator{\opchar}{char}
\DeclareMathOperator{\opspan}{span}
\DeclareMathOperator{\rk}{rk}
\DeclareMathOperator{\sgn}{sgn}
\DeclareMathOperator{\Sym}{Sym}
\DeclareMathOperator{\tr}{tr}
\DeclareMathOperator{\img}{img}
\DeclareMathOperator{\CandE}{CandE}
\DeclareMathOperator{\CandO}{CandO}
\DeclareMathOperator{\argmax}{argmax}
\DeclareMathOperator{\first}{first}
\DeclareMathOperator{\last}{last}
\DeclareMathOperator{\cost}{cost}
\DeclareMathOperator{\dist}{dist}
\DeclareMathOperator{\path}{path}
\DeclareMathOperator{\parent}{parent}
\DeclareMathOperator{\argmin}{argmin}
\DeclareMathOperator{\excess}{excess}
\let\Pr\relax
\DeclareMathOperator{\Pr}{\mathbf{Pr}}
\DeclareMathOperator{\Exp}{\mathbb{E}}
\DeclareMathOperator{\Var}{\mathbf{Var}}
\let\limsup\relax
\DeclareMathOperator{\limsup}{limsup}
%Paired Delims
\DeclarePairedDelimiter\ceil{\lceil}{\rceil}
\DeclarePairedDelimiter\floor{\lfloor}{ \rfloor}


\newcommand{\dagstar}{*}

\newcommand{\tbigwedge}{{\textstyle{\bigwedge}}}
\setlength{\parindent}{0pt}
\setlength{\parskip}{5pt}



\begin{document}


\title{CS 40: Computational Complexity}

\author{Sair Shaikh}
\maketitle

% Collaboration Notice: Talked to Henry Scheible '26 to discuss ideas.




\begin{problem}{6}
    Prove that 
    \[ \text{STRONGCON} := \{ \langle G\rangle : G \text{ is a strongly connected directed graph } \}\] 
    is NL-complete.
\end{problem}

\begin{proof}
    To show that STRONGCON is NL-complete, we need to show that $\text{STRONGCON} \in \text{NL}$ and STRONGCON is NL-hard. We do each of these seperately. \bbni
    First, we show that STRONGCON is in NL. It suffices to show that $\text{STRONGCON} \in \text{coNL}$ as NL = coNL. Thus, we consider the complement: 
    \[\overline{\text{STRONGCON}} = \{ \langle G \rangle : \exists u, v: \text{ there is no path } u \to v\}\]
    We need to show that this is in NL. We do this in the following way: 
    \begin{enumerate}
        \item Guess $u$ and $v$ non-deterministically. Concretely we can do this by counting up from $1$ to $|V|$ (we can figure out $|V|$ by parsing the input and maintaining a counter) in a binary counter and choosing whether to stop at each particular point. Thus, this can be done in $O(\log(|V|))$ space. 
        \item Simulate the NDTM associated to $\overline{\text{STCONN}}$ on $\langle G, u, v \rangle$. We have $G$ on our input tape and $u, v$ on the work-tape, so this requires only a slight modification and can be done in log-space. Note that since NL = coNL, $\overline{\text{STCONN}} \in \text{NL}$ thus this NDTM exists. 
        \item If the simulation ACCEPTS, then there is no path from $u$ to $v$, thus, $G$ is not strongly connected, and we ACCEPT. Otherwise, we REJECT. This takes no space.
    \end{enumerate}
    Thus, we conclude that $\overline{\text{STRONGCON}} \in \text{NL}$. Thus, $\text{STRONGCON} \in \text{coNL} = \text{NL}$. \bbni
    Finally, we need to show STRONGCON is NL-hard. Since STCONN is NL-hard, it suffices to show a log-space reduction from STCONN to STRONGCON. \bbni 
    Let $\langle G, s, t \rangle$ be some encoding of a graph and two vertices. We first create a graph $G'$ with the property that: 
    \[ \text{there is a path } s \to t \text{ in } G \iff G' \text{ is strongly connected} \]
    Let $G'$ have the same vertex set as $G$, but with the following edges added: 
    \begin{itemize}
        \item Edge $(u, s)$ for all $u \in V(G)$ if they do not exist
        \item Edge $(t, v)$ for all $v \in V(G)$ if they do not exist
    \end{itemize}
    We show that $G'$ satisfies the given property. First assume there is a path from $s \to t$ in $G$. Let $u, v \in V(G')$ be any vertices. We know there exists an edge $(u, s)$, path $s \to t$, and edge $(t, v)$. Thus, there is a path from $u \to v$. Since $u, v$ were arbitrary, $G'$ is strongly connected. \\
    Next, assume that $G'$ is strongly connected. Then, there is at least one path from $s \to t$ in $G'$. Let $p$ be (one of) the shortest path(s) from $s$ to $t$. We claim that $p$ only uses edges already present in $G$. Indeed, if any edge $(u, s)$ appeared in $p$, the path $p$ can be shortened by taking the section after this edge. By the minimality of $p$, no such edge exists in $p$. Similarly, we see that no edge $(t, v)$ can exist in $p$. In particular, $p$ avoids all the edges we appended to $G$ to create $G'$. Thus, $p$ only uses edges in $G$, hence there is a path from $s \to t$ in $G$. \bbni
    Note that in terms of our languages, we can write this property as: 
    \[ \langle G, s, t \rangle \in \text{STCONN} \iff \langle G' \rangle \in \text{STRONGCON}\]
    We only need to show that we can construct $G'$ from $G$ in log-space. We only need to stream the list of vertices and the list of edges for $G'$. Streaming the list of vertices and the edges already contained in $G$ requires no additional space, as we just read the input and regurgitate it. Thus, we only need to worry about outputting $(u, s)$ and $(t, u)$ for all $u \in V(G)$. But this can be done in logspace as it only requires one to maintain a counter to track which vertex we are on as we loop over the vertex set, which is of size $O(\log(|V|))$, which is at most logarithmic in the description of the entire graph ($s$ and $t$ can be read from the input tape each time). \bbni 
    Thus, we have shown a logspace reduction of STCONN to STRONGCON. Hence, STRONGCONN is NL-hard, therefore NL-complete. \bbni 
    
    
    
\end{proof}





\end{document}