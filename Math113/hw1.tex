% \documentclass[12pt]{amsart}
\documentclass[12pt]{article}

\usepackage{fullpage}
\usepackage{mdframed}
\usepackage{colonequals}
\usepackage{algpseudocode}
\usepackage{algorithm}
\usepackage{tcolorbox}
\usepackage[all]{xy}
\usepackage{proof}
\usepackage{mathtools}
\usepackage{bbm}
\usepackage{amssymb}
\usepackage{amsthm}
\usepackage{amsmath}
\usepackage{amsxtra}
\newcommand{\bb}{\mathbb}


\newtheorem{theorem}{Theorem}[section]
\newtheorem{corollary}{Corollary}[theorem]
\newtheorem{lemma}{Lemma}

\newcommand{\mathcat}[1]{\textup{\textbf{\textsf{#1}}}} % for defined terms

\newenvironment{problem}[1]
{\begin{tcolorbox}\noindent\textbf{Problem #1}.}
{\vskip 6pt \end{tcolorbox}}

\newenvironment{enumalph}
{\begin{enumerate}\renewcommand{\labelenumi}{\textnormal{(\alph{enumi})}}}
{\end{enumerate}}

\newenvironment{enumroman}
{\begin{enumerate}\renewcommand{\labelenumi}{\textnormal{(\roman{enumi})}}}
{\end{enumerate}}

\newcommand{\defi}[1]{\textsf{#1}} % for defined terms

\theoremstyle{remark}
\newtheorem*{solution}{Solution}

\setlength{\hfuzz}{4pt}

\newcommand{\calC}{\mathcal{C}}
\newcommand{\calF}{\mathcal{F}}
\newcommand{\C}{\mathbb C}
\newcommand{\N}{\mathbb N}
\newcommand{\Q}{\mathbb Q}
\newcommand{\R}{\mathbb R}
\newcommand{\Z}{\mathbb Z}
\newcommand{\F}{\mathbb F}
\newcommand{\br}{\mathbf{r}}
\newcommand{\RP}{\mathbb{RP}}
\newcommand{\CP}{\mathbb{CP}}
\newcommand{\nbit}[1]{\{0, 1\}^{#1}}
\newcommand{\bits}{\{0, 1\}^{n}}
\newcommand{\bbni}{\bigbreak \noindent}
\newcommand{\norm}[1]{\left\vert\left\vert#1\right\vert\right\vert}
\newcommand{\dbar}{\overline{\partial}}
\let\d\relax
\let\calF\relax
\newcommand{\d}{\partial}
\newcommand{\calO}{\mathcal{O}}
\newcommand{\calF}{\mathcal{F}}
\newcommand{\calG}{\mathcal{G}}
\newcommand{\calH}{\mathcal{H}}
\newcommand{\calE}{\mathcal{E}}

\let\1\relax
\newcommand{\1}{\mathbf{1}}
\newcommand{\fr}[2]{\left(\frac{#1}{#2}\right)}

\newcommand{\vecz}{\mathbf{z}}
\newcommand{\vecr}{\mathbf{r}}
\DeclareMathOperator{\Cinf}{C^{\infty}}
\DeclareMathOperator{\Id}{Id}

\DeclareMathOperator{\Alt}{Alt}
\DeclareMathOperator{\ann}{ann}
\DeclareMathOperator{\codim}{codim}
\DeclareMathOperator{\End}{End}
\DeclareMathOperator{\Hom}{Hom}
\DeclareMathOperator{\id}{id}
\DeclareMathOperator{\M}{M}
\DeclareMathOperator{\Mat}{Mat}
\DeclareMathOperator{\Ob}{Ob}
\DeclareMathOperator{\opchar}{char}
\DeclareMathOperator{\opspan}{span}
\DeclareMathOperator{\rk}{rk}
\DeclareMathOperator{\sgn}{sgn}
\DeclareMathOperator{\Sym}{Sym}
\DeclareMathOperator{\tr}{tr}
\DeclareMathOperator{\img}{img}
\DeclareMathOperator{\CandE}{CandE}
\DeclareMathOperator{\CandO}{CandO}
\DeclareMathOperator{\argmax}{argmax}
\DeclareMathOperator{\first}{first}
\DeclareMathOperator{\last}{last}
\DeclareMathOperator{\cost}{cost}
\DeclareMathOperator{\dist}{dist}
\DeclareMathOperator{\path}{path}
\DeclareMathOperator{\parent}{parent}
\DeclareMathOperator{\argmin}{argmin}
\DeclareMathOperator{\excess}{excess}
\let\Pr\relax
\DeclareMathOperator{\Pr}{\mathbf{Pr}}
\DeclareMathOperator{\Exp}{\mathbb{E}}
\DeclareMathOperator{\Var}{\mathbf{Var}}
\let\limsup\relax
\DeclareMathOperator{\limsup}{limsup}
%Paired Delims
\DeclarePairedDelimiter\ceil{\lceil}{\rceil}
\DeclarePairedDelimiter\floor{\lfloor}{ \rfloor}


\newcommand{\dagstar}{*}

\newcommand{\tbigwedge}{{\textstyle{\bigwedge}}}
\setlength{\parindent}{0pt}
\setlength{\parskip}{5pt}


\begin{document}

% \title{CS 40: Computational Complexity}

\author{Sair Shaikh}
\maketitle

Collaboration Notice: Talked to Henry Scheible '26 to discuss ideas.



\begin{problab}{1}
    We want to show that the metrics induced on $\R^n$ by the $p$-norms are all strongly equivalent for $1 \leq p \leq \infty$.
\end{problab}
\begin{solu}
    First, note that strong equivalence of the norms implies strong equivalence of the metrices induced by the norms. Indeed, let $\alpha$ and $\beta$ be two norms on $\R^n$. Then, if there exist $C, D > 0 \in \R$ such that:
    \[ \forall x \in \R^n : C \alpha(x) \leq \beta(x) \leq D \alpha(x) \]
    Then, we have:
    \[ \forall x, y \in \R^n : C \alpha(x-y, x-y) \leq \beta(x-y, x-y) \leq D \alpha(x-y,x-y) \]
    which is precisely the statement for the strong equivalence on the induced metrics. Thus, we only need to show the strong equivalence of the norms. \bbni
    Next, we show that being strongly equivalent is an equivalence relation on norms (although we will only use transitivity). Let $\alpha$, $\beta$, and $\gamma$ be norms on $\R^n$. We prove the following: 
    \begin{itemize}
        \item Reflexivity: Note that:
        \[ \forall x \in \R^n : \alpha(x) \leq \alpha(x) \leq \alpha(x) \] 
        Thus, $\alpha$ is strongly equivalent to itself.
        \item Symmetry: Assume $\alpha$ is strongly equivalent to $\beta$. Then, there exist $C, D > 0 \in \R$ such that: 
        \[ \forall x \in \R^n : C \alpha(x) \leq \beta(x) \leq D \alpha(x) \]
        Then, we have:
        \[ \forall x \in \R^n : \frac{1}{D} \beta(x) \leq \alpha(x) \leq \frac{1}{C} \beta(x) \]
        where $\frac{1}{D}, \frac{1}{C} > 0$. Thus, $\beta$ is strongly equivalent to $\alpha$.
        \item Transitivity: Assume $\alpha$ is strongly equivalent to $\beta$ and $\beta$ is strongly equivalent to $\gamma$. Then, there exist $C, D, E, F > 0 \in \R$ such that:
        \[ \forall x \in \R^n : C \alpha(x) \leq \beta(x) \leq D \alpha(x) \]
        \[ \forall x \in \R^n : E \beta(x) \leq \gamma(x) \leq F \beta(x) \]
        Then, we have:
        \[ \forall x \in \R^n: CE \alpha(x) \leq E \beta(x) \leq \gamma(x) \leq F\beta(x) \leq DF\alpha(x) \]
        where $CE, DF > 0$. Thus, $\alpha$ is strongly equivalent to $\gamma$.
    \end{itemize}
    Thus, strong equivalence is an equivalence relation on norms. \bbni
    Thus, by transitivity, it sufficies to prove that for an arbitrary $1 \leq p \leq \infty$, the $p$-norm is strongly equivalent to the $2$-norm, i.e. we need to show that there exist $C, D > 0 \in \R$ such that:
    \[ \forall x \in \R^n : C ||x||_2 \leq ||x||_p \leq D_p ||x||_2 \]
    We claim that it is sufficient to show the result on the subset $S = \{x \in \R^n \mid ||x||_2 = 1\}$. First, note that if $||x||_2 = 0$, we know that $x = 0$ (by definiteness), thus, $||x||_p = 0$ as well, and the metrics are strongly equivalent. Thus, we can assume $||x||_2 > 0$.
    Indeed, using homogeneity of the $p$-norm, we have:
    \begin{align*}
        &\forall x \neq 0 \in X: C ||x||_2 \leq ||x||_p \leq D ||x||_2 \\
        \iff &\forall x \neq 0 \in X: C \leq \frac{1}{||x||_2} ||x||_p \leq D \\
        \iff &\forall x \neq 0 \in X: C \leq  || (x/||x||_2)||_p \leq D \\
        \iff &\forall w \in S: C \leq ||w||_p \leq D
    \end{align*}
    Next, note that the $p$-norm is a continous function on $\R^n$ for $p \neq \infty$ as it is a $p$-th root of a finite sum of polynomial functions of the coordinates, thus is a composition of continous functions. Similarly, for $p = \infty$, the $p$-norm is a continous function on $\R^n$ as it is the maximum of a finite number of continous functions of the coordinates. Since $S$ is closed and bounded, we can apply the extreme value theorem to let $C, D > 0 \in \R$ be minimum and maximum of $||x||_p$ attained on $S$. Thus, we have:
    \[ \forall x \in S: C \leq ||x||_p \leq D \]
    This sufficies to prove that all $p$-norms are strongly equivalent.
\end{solu}

\newpage

\begin{problab}{2}
    Give an example of a metric on $\R^n$ which is not equivalent to any of the metrics in problem 1.
\end{problab}
\begin{solu}
    Take the discrete metric on $\R^n$ given by:
    \[ d(x,y) = \begin{cases}
        0 & x = y \\
        1 & x \neq y
    \end{cases} \]
    Under this metric, the singleton ${x}$ is open for all $x \in \R^n$ as $\{x\} = B_{1/2}(x)$. However, for the $p$-norm, the singleton $\{x\}$ is not open for any $p \in 1\leq p \leq \infty$. Thus, the discrete metric is not equivalent to any of the metrics in problem 1 as they do not generate the same topologies.
\end{solu}

\newpage

\begin{problab}{3}
    Let $(X, \rho)$ be a metric space. If $A \subseteq X$, then define: 
    \[ \rho(x, A) = \inf\{\rho(x,y) : y \in A\}\]
    \begin{enumerate}
        \item Show that $\rho(x,A) = 0$ if and only if $x \in \overline{A}$.
        \item Show that $x \mapsto \rho(x,A)$ is continuous.
        \item Show that if $A$ and $B$ are disjoint, non-empty, closed subsets of $X$, then there is a $f \in C_b(X)$ such that:
        \begin{itemize}
            \item $0 \leq f(x) \leq 1$ for all $x \in X$.
            \item $f(x) = 1$ if and only if $x \in A$.
            \item $f(x) = 0$ if and only if $x \in B$. 
        \end{itemize}
        (Hint: Use the function $\rho(x,B)/(\rho(x, A)+\rho(x, B))$.)
    \end{enumerate} 
\end{problab}

\begin{solu}
    \bbni
    \begin{enumerate}
        \item Unpacking the definitions, and using a lemma we proved in class, we note that:
        \begin{align*}
            &\rho(x, A) = 0  \\
            \iff &\forall \epsilon > 0, \exists x' \in A: \rho(x,x') < \epsilon \\
            \iff &\forall \epsilon > 0, \exists x' \in A: x' \in B_\epsilon(x) \\
            \iff &\forall \epsilon > 0, B_\epsilon(x) \cap A \neq \emptyset \\
            \iff &x \in \overline{A}
        \end{align*}
        \item Let $\epsilon > 0$ and $x \in X$ be arbitrary. We need to show that there exists a $\delta > 0$ such that for all $x' \in X$ with $\rho(x, x') < \delta$, we have:
        \[|\rho(x, A) - \rho(x', A)| < \epsilon\]
        Using the triangle inequality, we have:
        \begin{align*}
            \rho(x', A) &= \inf\{ \rho(x', y) : y\in A \} \\
            &\leq \inf\{ \rho(x, x') + \rho(x, y) : y \in A \} \\
            &\leq \rho(x, x') + \inf\{ \rho(x, y) : y \in A \} \\
            &= \rho(x, x') + \rho(x, A)
        \end{align*}
        Thus, 
        \[ \rho(x', A) - \rho(x, A) \leq \rho(x, x') \]
        Similarly, by an analogous calculation switching $x$ with $x'$, we have:
        \[ \rho(x, A) - \rho(x', A) \leq \rho(x, x') \]
        Combining these, we conclude: 
        \[ |\rho(x, A) - \rho(x', A)| \leq \rho(x, x')\]
        Thus, for $\epsilon > 0$, we can take $\delta = \epsilon$. Then, for $\rho(x, x') < \delta$, we have:
        \[ |\rho(x, A) - \rho(x', A)| \leq \rho(x, x') < \epsilon \]
        \item Consider the function $f$ mapping $x \to \frac{\rho(x, B)}{\rho(x, A) + \rho(x, B)}$. Since $A$ and $B$ are disjoint, using part 1., we conclude that the denominator is never $0$ and the function is well-defined for all $x \in X$. This function is continous as it is a composition of continous functions.  We show it is bounded as a part of this question. Thus, $f \in C_b(X)$. We show that $f$ satifies these properties:  
        \begin{itemize}
            \item $0 \leq f(x) \leq 1$ for all $x \in X$: For any $x \in X$, $\rho(x, B)$ and $\rho(x, A)$ are the infimums of a set of non-negatives, and thus are non-negative. Thus, both the numerator and denominator, being sums of non-negative numbers, are non-negative. Thus, $f(x) \geq 0$. Additionally, by the non-negativity of $\rho(x, A)$, we have $\rho(x, B) \leq \rho(x, A) + \rho(x, B) \implies f(x) \leq 1$ for all $x \in X$. Thus,
            \[ \forall x \in X: 0 \leq f(x) \leq 1 \]
            \item Note that since $A$ is closed, $A = \overline{A}$. Thus, from part 1. we note that: 
            \[ x \in A \iff \rho(x, A) = 0 \iff f(x) = \frac{\rho(x, B)}{0 + \rho(x, B)} = 1\]
            \item Similarly, since $B$ is closed, $B = \overline{B}$. Thus, from part 1. we note that:
            \[ x \in B \iff \rho(x, B) = 0 \iff f(x) = \frac{0}{\rho(x, A) + 0} = 0\]
        \end{itemize}
    \end{enumerate}
\end{solu}
\newpage


\begin{problab}{4}
    Show that a Cauchy sequence in a metric space with a convergent subsequence is necessarily convergent. 
\end{problab}

\begin{solu}
    Let $(x_n)$ be the Cauchy sequence in a metric space $(X, \rho)$. Let $(x_{n_k})$ be a convergent subsequence of $(x_n)$ with limit $x \in X$. We need to show that $(x_n)$ converges to $x$. \bbni
    Since $(x_{n_k})$ converges to $x$, there exists an $N_0$, such that for all $k \geq N_0$, we have:
    \[ \rho(x_{n_k}, x) < \frac{\epsilon}{2} \]
    Since $(x_n)$ is Cauchy, there exists an $N_1$ such that for all $m, n \geq N_1$, we have:
    \[ \rho(x_m, x_n) < \frac{\epsilon}{2} \]
    Let $N = \max(N_0, N_1)$. Then for all $k \geq N$, noting $n_k \geq k$, we have:
    \begin{align*}
        \rho(x_{k}, x) &\leq \rho(x_{k}, x_{n_k}) + \rho(x_{n_k}, x) \\
        &< \frac{\epsilon}{2} + \frac{\epsilon}{2} \\
        &= \epsilon
    \end{align*}
    Thus, $(x_n)$ converges to $x$.
\end{solu}
\newpage

\begin{problab}{5}
    Let $\rho$ and $\sigma$ be metrices on $X$. Show that $\rho$ and $\sigma$ are equivalent if and only if they have the same convergent sequences. That is, show that $x_n \to x$ in $(X, \rho)$ if and only if $x_n \to x$ in $(X, \sigma)$.
\end{problab}

\begin{solu}
    \bbni
    \begin{itemize}
        \item[($\implies$)] Assume $\rho$ and $\sigma$ are equivalent. Let $x_n \to x$ be a convergent sequence in $(X, \rho)$. To show that $x_n \to x$ in $(X, \sigma)$, we need to show that for all $\epsilon > 0$, there exists an $N$ such that for all $n \geq N$, we have:
        \[ x_n \in B_\epsilon^\sigma(x) \]
        Since $\rho$ and $\sigma$ are equivalent, there exists $\epsilon' > 0$ such that: 
        \[ B_{\epsilon'}^\rho(x) \subset B_{\epsilon}^\sigma(x)\]
        Since $x_n \to x$ in $(X, \rho)$, there exists an $N$ such that for all $n \geq N$, we have:
        \[ x_n \in B_{\epsilon'}^\rho(x)\]
        Thus, 
        \[ x_n \in B_{\epsilon}^\sigma(x) \]
        Hence, the sequence converges in $(X, \sigma)$ as well.

        \item[($\impliedby$)] Assume $(X, \sigma)$ and $(X, \rho)$ have the same convergent sequences. To show that $\rho$ and $\sigma$ are equivalent, we need to show that they generate the same metric topologies. Let $U \subseteq X$ be open in $(X, \rho)$. We need to show that $U$ in open in $(X, \sigma)$. \bbni
        We show instead that $X \setminus U$ is closed in $(X, \sigma)$. Let $(x_n) \subseteq X \setminus U$ be any convergent sequence and call its limit $x$. Then, by assumption, $(x_n)$ converges to $x$ in $(X, \rho)$. However, since $X \setminus U$ is closed in $(X, \rho)$, we conclude that $x \in X \setminus U$. Thus, $X \setminus U$ contains all its limit points, thus is closed in $(X, \sigma)$. Thus, $U$ is open in $(X, \sigma)$. Hence, $\rho$ and $\sigma$ generate the same metric topology and are equivalent. 
    \end{itemize}
\end{solu}
\newpage 


\begin{problab}{6} 
    Let $X$ be a metric space. Prove that the uniform limit of continous function $f_n: X \to \C$ is continuous.
\end{problab}

\begin{solu}
    Let $d$ be the metric. Let $\epsilon > 0$ and $x_0 \in X$ be arbitrary. To prove that $f$ is continous, we need to show that there exists $\delta > 0$ such that $|f(x) - f(x_0)| < \epsilon$ for all $x \in X$ such that $d(x,x_0) < \delta$. \bbni
    Since $f_n$ is continous, there exists $\delta > 0$ such that for all $x \in X$ with $d(x, x_0) < \delta$, we have:
    \[ |f_n(x) - f_n(x_0)| < \frac{\epsilon}{3} \]
    we claim this $\delta$ works. 
    Since $(f_n)$ converges uniformly to $f$, there exists $N$ such that for all $n \geq N$, we have:
    \[ \forall x \in X: |f(x) - f_n(x)| < \frac{\epsilon}{3} \]
    Picking an $n \geq N$, for all $x \in X$ with $d(x, x_0) < \delta$, we have:
    \begin{align*}
        |f(x) -f(x_0)| &= |f(x) - f_n(x) + f_n(x) - f_n(x_0) + f_n(x_0) - f(x_0)| \\
        &\leq |f(x)-f_n(x)| + |f_n(x) -f_n(x_0)| + |f_n(x_0) - f(x_0)| \\
        &< \frac{\epsilon}{3} + \frac{\epsilon}{3} + \frac{\epsilon}{3} \\
        &= \epsilon
    \end{align*}
    Thus, $f$ is continous. 
\end{solu}
\newpage

\begin{problab}{7}
    Let $X$ be a metric space. Recall that we say $f: X \to C$ is bounded if $||f||_\infty \leq \infty$. A sequence of functions $f_n: X \to D$ is uniformly bounded if there is a $M$ such that $||f_n||_\infty \leq M$ for all $n$. Also, $(f_n)$ is called uniformly Cauchy if for all $\epsilon > 0$ there is an $N$ such that $m, n \geq N$, implies $|f_n - f_m| < \epsilon$ for all $x \in X$. Show that a uniformly Cauchy sequence $(f_n)$ of bounded functions is uniformly bounded. In particular, $(f_n)$ converges to a bouned function.
\end{problab}

\begin{solu}
    Let $(f_n)$ be a uniformly Cauchy sequence of bounded functions. We need to show that $(f_n)$ is uniformly bounded. Thus, we need to find an $M$ such that $||f||_\infty \leq M$ for all $n$. \bbni
    Since $(f_n)$ is uniformly Cauchy, there exists an $N \in \mathbb{N}$ such that for all $m, n \geq N$, we have for all $x \in X$:
    \[ |f_n(x) - f_m(x)| < 1 \]
    In terms of the $\infty$-norm (the supremum of over all $x \in X$), we have for all $x \in X$:
    \[ ||f_n(x) - f_m(x)||_\infty \leq 1 \]
    Fix $m = N$. Then, for all $n \geq N$, we have (by the triangle inequality):
    \[ ||f_n||_\infty \leq ||f_n-f_N||_\infty + ||f_N||_\infty \leq ||f_N(x)||_\infty + 1\]
    Thus, let $M = \max(||f_1||_\infty, ||f_2||_\infty, \ldots, ||f_N||_\infty)+1$. Then, 
    \begin{itemize}
        \item For $n < N$, we have $||f_n||_\infty \leq M$ by definition.
        \item For $n \geq N$, we have $||f_n||_\infty \leq ||f_N||_\infty + 1 \leq M$. 
    \end{itemize}
    Thus, $f_n$ is uniformly bounded. We already know that $f_n$ converges to $f$ uniformly. Thus, $f$ is bounded by $M$ as well. 
\end{solu}

\end{document}