\documentclass[12pt]{article}


\usepackage{fullpage}
\usepackage{mdframed}
\usepackage{colonequals}
\usepackage{algpseudocode}
\usepackage{algorithm}
\usepackage{tcolorbox}
\usepackage[all]{xy}
\usepackage{proof}
\usepackage{mathtools}
\usepackage{bbm}
\usepackage{amssymb}
\usepackage{amsthm}
\usepackage{amsmath}
\usepackage{amsxtra}
\newcommand{\bb}{\mathbb}


\newtheorem{theorem}{Theorem}[section]
\newtheorem{corollary}{Corollary}[theorem]
\newtheorem{lemma}{Lemma}

\newcommand{\mathcat}[1]{\textup{\textbf{\textsf{#1}}}} % for defined terms

\newenvironment{problem}[1]
{\begin{tcolorbox}\noindent\textbf{Problem #1}.}
{\vskip 6pt \end{tcolorbox}}

\newenvironment{enumalph}
{\begin{enumerate}\renewcommand{\labelenumi}{\textnormal{(\alph{enumi})}}}
{\end{enumerate}}

\newenvironment{enumroman}
{\begin{enumerate}\renewcommand{\labelenumi}{\textnormal{(\roman{enumi})}}}
{\end{enumerate}}

\newcommand{\defi}[1]{\textsf{#1}} % for defined terms

\theoremstyle{remark}
\newtheorem*{solution}{Solution}

\setlength{\hfuzz}{4pt}

\newcommand{\calC}{\mathcal{C}}
\newcommand{\calF}{\mathcal{F}}
\newcommand{\C}{\mathbb C}
\newcommand{\N}{\mathbb N}
\newcommand{\Q}{\mathbb Q}
\newcommand{\R}{\mathbb R}
\newcommand{\Z}{\mathbb Z}
\newcommand{\br}{\mathbf{r}}
\newcommand{\RP}{\mathbb{RP}}
\newcommand{\CP}{\mathbb{CP}}
\newcommand{\nbit}[1]{\{0, 1\}^{#1}}
\newcommand{\bits}{\{0, 1\}^{n}}
\newcommand{\bbni}{\bigbreak \noindent}
\newcommand{\norm}[1]{\left\vert\left\vert#1\right\vert\right\vert}

\let\1\relax
\newcommand{\1}{\mathbf{1}}
\newcommand{\fr}[2]{\left(\frac{#1}{#2}\right)}

\newcommand{\vecz}{\mathbf{z}}
\newcommand{\vecr}{\mathbf{r}}
\DeclareMathOperator{\Cinf}{C^{\infty}}
\DeclareMathOperator{\Id}{Id}

\DeclareMathOperator{\Alt}{Alt}
\DeclareMathOperator{\ann}{ann}
\DeclareMathOperator{\codim}{codim}
\DeclareMathOperator{\End}{End}
\DeclareMathOperator{\Hom}{Hom}
\DeclareMathOperator{\id}{id}
\DeclareMathOperator{\M}{M}
\DeclareMathOperator{\Mat}{Mat}
\DeclareMathOperator{\Ob}{Ob}
\DeclareMathOperator{\opchar}{char}
\DeclareMathOperator{\opspan}{span}
\DeclareMathOperator{\rk}{rk}
\DeclareMathOperator{\sgn}{sgn}
\DeclareMathOperator{\Sym}{Sym}
\DeclareMathOperator{\tr}{tr}
\DeclareMathOperator{\img}{img}
\DeclareMathOperator{\CandE}{CandE}
\DeclareMathOperator{\CandO}{CandO}
\DeclareMathOperator{\argmax}{argmax}
\DeclareMathOperator{\first}{first}
\DeclareMathOperator{\last}{last}
\DeclareMathOperator{\cost}{cost}
\DeclareMathOperator{\dist}{dist}
\DeclareMathOperator{\path}{path}
\DeclareMathOperator{\parent}{parent}
\DeclareMathOperator{\argmin}{argmin}
\DeclareMathOperator{\excess}{excess}
\let\Pr\relax
\DeclareMathOperator{\Pr}{\mathbf{Pr}}
\DeclareMathOperator{\Exp}{\mathbb{E}}
\DeclareMathOperator{\Var}{\mathbf{Var}}
\let\limsup\relax
\DeclareMathOperator{\limsup}{limsup}
%Paired Delims
\DeclarePairedDelimiter\ceil{\lceil}{\rceil}
\DeclarePairedDelimiter\floor{\lfloor}{ \rfloor}


\newcommand{\dagstar}{*}

\newcommand{\tbigwedge}{{\textstyle{\bigwedge}}}
\setlength{\parindent}{0pt}
\setlength{\parskip}{5pt}



\begin{document}


\title{CS 40: Computational Complexity}

\author{Sair Shaikh}
\maketitle

% Collaboration Notice: Talked to Henry Scheible '26 to discuss ideas.




\begin{problem}{18}
In the proof of the Valiant-Vazirani theorem from the lecture, the CNF formula produced by the randomized mapping reduction is rather non-explicit: we invoked the Cook–Levin reduction to argue that the condition 
\[ h(x_1, \ldots, x_n) = 0^k\]
(where \(h\) is a function from a suitable 2-universal hash family), being polynomial-time checkable, admits a polynomial-sized CNF formula using some auxiliary variables; these auxiliary variables are the ones introduced by Cook–Levin, to encode the computational tableau. This non-explicitness is unsatisfactory. (Ha ha!)

Remedy this by choosing the specific hash family from Problem 14 (the \(A x + b\) family) to make the reduction explicit. That is, write out the predicate
\[
\psi(x_1, \ldots, x_n) := \varphi(x_1, \ldots, x_n) \wedge (h(x_1, \ldots, x_n) = 0^k)
\]
as a CNF formula \(\tilde{\psi}(x_1, \ldots, x_n, y_1, \ldots, y_m)\) for some \(m = \mathrm{poly}(n)\), with the property that if \(\psi\) has a unique satisfying assignment, so does \(\tilde{\psi}\). Prove this property of \(\tilde{\psi}\).
\end{problem}

\begin{solution}
    Note that since $\varphi(x_1, \cdots, x_n)$ is a SAT instance, it is already a CNF. Thus, as $\psi$ is an and of two clauses, it suffices to express the second, i.e. $h(x_1, \cdots, x_n) = 0^k$ as a polynomial sized CNF for a given $h$. \bbni 
    Write $x = [x_1, \cdots, x_n]^T$ as a vector. Using the $2$-Universal hash family from before, there exists a matrix \(A \in \{0,1\}^{k \times n}\) and vector \(b \in \{0,1\}^k\), with: 
    \[ h(x) = Ax + b \]
    Note that: 
    \[ Ax+b = 0^k \iff \bigwedge_{i=1}^k (Ax+b)_i = 0 \]
    Thus, as $k \in \text{poly}(n)$ in the original proof, it suffices to express $(Ax+b)_i = 0$ as a polynomial sized CNF as this gives a CNF for $h(x) = 0^k$. Thus, that's what we will show in the rest of the proof. \bbni
    Noting the matrix product expansion, this means: 
    \begin{align*}
        (Ax+b)_i = b_i + \sum_{j=1}^n A_{ij}x_j = 0 \iff \sum_{j=1}^n A_{ij}x_j = b_i
    \end{align*}
    Let $I$ be the set of indices for which the entry in the $i$th row of $A$ is $1$, i.e.: 
    \[ I := \{ j \in [n] : A_{ij} = 1 \}\]
    Then, by the definition of products mod 2, the previous equation reduces to: 
        \begin{align*}
        \sum_{j=1}^n A_{ij}x_j = b_i \iff \sum_{j \in I} x_j = b_i
    \end{align*}
    By relabelling the variables, we will write $\sum_{j \in I} x_j$ as $\sum_{j = 1}^t x_j$ for $t = |I|$. Next, we introduce additional variables $y_1, \cdots, y_t$ and claim the following: 
    \begin{align*}
        \sum_{j = 1}^t x_j = b_i \iff \exists y_1, \cdots, y_t : (y_1 = x_1) \land (y_2 = y_1 + x_2) \land \cdots \land (y_t = y_{t-1} + x_t) \land (y_t = b_i) 
    \end{align*} 
    Clearly, if $\sum_{j = 1}^t x_j = b_i$ then there exist $y_i$s, representing the partial sums of $x_i$s, that satisfy the clauses on the left, with the full sum, $y_t = b_i$. Moreover, if such an assignment of $y_i$s exists, then expanding out the partial sums that hold by the truth of the clauses, we get: 
    \begin{align*}
        b_1 &= y_t  \\
        &= y_{t-1} + x_t \\
        &= y_{t-2} + x_{t-1} + x_{t} \\
        &= \cdots \\
        &= y_1 + x_2 + \cdots + x_t \\
        &= x_1 + \cdots + x_t 
    \end{align*}
    Thus, we only need to write the expression on the RHS as a CNF. We do this as follows: 
    \begin{enumerate}
        \item $(y_1 = x_1)$ can be written as $(y_1 \lor \overline{x_1}) \land (\overline{y_1} \lor x_1)$. Note that if $x_1 = 1$ (second clause already satisfied), then, $y_1 = 1$ is forced to satisfy the first clause. Instead, if $x_1 = 0$ (first clause already satisfied), $\overline{y_1} = 1$ is forced to satisfy the second clause, thus, $y_1 = 0$. In both cases, the only satisfying assignment forces $x_1 = y_1$.
        \item We claim that we can write $y_r = y_{r-1} + x_r$ as the following CNF: 
        \begin{align*}
            (x_r \lor y_{r-1} \lor \overline{y_r}) \land (x_r \lor \overline{y_{r-1}} \lor y_r) \land (\overline{x_r} \lor \overline{y_{r-1}} \lor \overline{y_r}) \land (\overline{x_r} \lor y_{r-1} \lor y_r)
        \end{align*} 
        Note that if $x_r = 0$, then the last two clauses are already satisfied, and the first two clauses, after substituting $x_r = 0$, reduce to $y_r = y_{r-1}$ (notice similarity to (1)). \\
        Instead, if $x_r = 1$, then the first two clauses are already satisfied, and the last two clauses, after substituting $x_r = 1$, reduce to $y_r = \overline{y_{r-1}}$. \\
        Thus, in both cases, the only satisfying assignments (two, for a given value of $x_r$) satisfy $y_{r} = y_{r-1}+x_r$. 
        \item Finally, the argument for $y_t = b_i$ is identical to (1). 
    \end{enumerate}
    Note that each of these required at most $4$ clauses with at most $3$ variables each, i.e. a constant number. Since we had $t+1$ of these expressions, and $t = |I| \leq n $, we only used a linear number of additional clauses and additional variables to express $(Ax+b)_i = 0$ as a CNF. Thus, we are done.
\end{solution}


\newpage


\begin{problem}{2}
Let \textsc{FIND-SAT} be the following problem: given a CNF formula \(\varphi(x_1, \ldots, x_n)\), either output any one Boolean assignment \((\alpha_1, \ldots, \alpha_n) \in \{0, 1\}^n\) that satisfies \(\varphi\), or else report that no such assignment exists.

Suppose you are given an oracle for \textsc{SAT}. As you know, you can invoke this oracle repeatedly to solve \textsc{FIND-SAT} in polynomial time. However, the standard way of doing this requires adaptive queries to the oracle, i.e., each question you pose to the oracle depends on its answers to your previous questions. 

Prove that there exists a randomized polynomial-time algorithm for \textsc{FIND-SAT} that uses only non-adaptive queries to the oracle, i.e., the algorithm prepares a (possibly randomized) batch of questions to pose to the oracle, asks them all at once, and based on the answers (and perhaps using further computation of its own), produces an output that is correct with probability at least \(\frac{2}{3}\).

\end{problem}

\begin{solution}
    Our algorithm, $A$, is as follows: 
    \begin{enumerate}
        \item Use the Valiant-Vazirani reduction $k$ times on $\varphi(x_1, \cdots, x_n)$ to get formulae $\psi_1, \cdots, \psi_k$ (in $x_i$'s and some auxillary variables), where $k := \lceil 8\ln(3)n\rceil$.
        \item Use the oracle for SAT to find whether $\varphi$ is satisfiable and the following are satisfiable, for each $i \in [k]$: 
        \begin{enumerate}
            \item $\psi_i$
            \item $\psi_i \land x_j$ for all $j \in [n]$
        \end{enumerate}
        \item If $\varphi$ is unsatisfiable, report no such assignment exists. 
        \item Iterate over the $\psi_i$. For each $\psi_i$ that is satisfiable (i.e. the oracle accepted $(a)$ for this formula), use the answers to all the queries in $(b)$ to come up with a candidate assignment, i.e. set $x_j = 1$ if $\psi_i \land x_j$ was accepted, and $0$ otherwise. Verify whether the candidate assignment satisfies $\varphi$. If the candidate assignment for any $\psi_i$ works, return this assignment as the solution. Otherwise, report that no such assignment exists.  
    \end{enumerate}
    Since $k \in \text{poly}(n)$, we only query the oracle $O(kn) \subseteq \text{poly}(n)$ times in (2), and we derive and verify at most $k$ assignments each of which takes polynomial time, the algorithm overall runs in polynomial time. Thus, we only need to prove that the algorithm returns a correct output with probability at least $\frac{2}{3}$. \bbni

    First, we claim and prove the following lemma: 
    \begin{lemma}
        If $\psi_i \in \text{USAT}$ for any $i \in [k]$, we return correctly with certainity.
    \end{lemma}
    \begin{proof}
        Note that we always verify an assignment before returning it, thus, if an assignment is returned, it is correct with certainty. Thus, the only mistakes are made when we report that no assignments exist. Thus, it suffices to show that if any $\psi_i \in \text{USAT}$, then an assignment is returned. \bbni 

        If $\psi_i \in \text{USAT}$, the result of each query $\psi_i \land x_j$ uniquely determines the value of $x_j$ in the unique satisfying assignment for $\psi_i$, which we pick for our candidate assignment in step (3). Then, this candidate assignment satisfies $\varphi$, assuming the particular form of the Valiant-Vazirani reduction presented in class and in the question above. Then, either we return a correct assignment before the loop reaches $\psi_i$, or we return this assignment. In particular, we would not return that there are no assignments to $\varphi$ and thus would not make an error. That concludes the proof.
    \end{proof}
    Using this lemma, $A$ returning incorrectly implies that (i) $\varphi$ is satisfiable (otherwise we'd catch this in Step (3)), and (ii) that none of the $\psi_i$ were in $\text{USAT}$. That is, 
    \[ \Pr[\text{A returns incorrectly}] \leq \Pr\left[\bigwedge_{i=1}^k \psi_i \not \in \text{USAT}\right] \]
    Since we run the Valiant-Vazirani reductions using independent random bits, we note that $\psi_i \not \in \text{USAT}$ are all mutually independent. Thus, using the Valiant-Vazirani bounds (knowing that $\varphi \in \text{SAT}$), we write: 
    \[ \Pr\left[\bigwedge_{i=1}^k \psi_i \not \in \text{USAT}\right] = \Pr\left[\psi_i \not \in \text{USAT}\right]^k  \leq \left(1-\frac{1}{8n}\right)^k\] 
    We want this probability to be at most $\frac{1}{3}$. Thus, calculating, we get: 
    \begin{align*}
        &\left(1-\frac{1}{8n}\right)^k \leq \frac{1}{3} \\
        \implies&k \ln\left(1-\frac{1}{8n}\right) \leq -\ln(3) \\
        \implies& k \geq -\frac{\ln(3)}{\ln\left(1-\frac{1}{8n}\right)} \qquad \left( \text{as }\ln\left(1-\frac{1}{8n}\right) < 0\right)
    \end{align*}
    Using the Taylor expansion for $\ln(1-x)$, we get: 
    \begin{align*}
        \ln(1-x) &= \sum_{t=1}^\infty -\frac{x^t}{t!} \\
        \implies &\ln\left(1-\frac{1}{8n}\right) \geq -\frac{1}{8n} \\
        \implies &-\frac{1}{\ln\left(1-\frac{1}{8n}\right)} \geq 8n
    \end{align*}
    \[ \]
    Thus, overall, we get: 
    \[ k \geq -\frac{\ln(3)}{\ln\left(1-\frac{1}{8n}\right)} \geq (8\ln 3)n\]
    Thus, our chosen $k$ suffices to prove the required bounds.
\end{solution}





\end{document}