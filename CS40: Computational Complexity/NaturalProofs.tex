

\documentclass[12pt]{article}


\usepackage{fullpage}
\usepackage{mdframed}
\usepackage{colonequals}
\usepackage{algpseudocode}
\usepackage{algorithm}
\usepackage{tcolorbox}
\usepackage[all]{xy}
\usepackage{proof}
\usepackage{mathtools}
\usepackage{bbm}
\usepackage{amssymb}
\usepackage{amsthm}
\usepackage{amsmath}
\usepackage{amsxtra}
\newcommand{\bb}{\mathbb}


\newtheorem{theorem}{Theorem}[section]
\newtheorem{corollary}{Corollary}[theorem]
\newtheorem{lemma}{Lemma}

\newcommand{\mathcat}[1]{\textup{\textbf{\textsf{#1}}}} % for defined terms

\newenvironment{problem}[1]
{\begin{tcolorbox}\noindent\textbf{Problem #1}.}
{\vskip 6pt \end{tcolorbox}}

\newenvironment{enumalph}
{\begin{enumerate}\renewcommand{\labelenumi}{\textnormal{(\alph{enumi})}}}
{\end{enumerate}}

\newenvironment{enumroman}
{\begin{enumerate}\renewcommand{\labelenumi}{\textnormal{(\roman{enumi})}}}
{\end{enumerate}}

\newcommand{\defi}[1]{\textsf{#1}} % for defined terms

\theoremstyle{remark}
\newtheorem*{solution}{Solution}

\setlength{\hfuzz}{4pt}

\newcommand{\calC}{\mathcal{C}}
\newcommand{\calF}{\mathcal{F}}
\newcommand{\C}{\mathbb C}
\newcommand{\N}{\mathbb N}
\newcommand{\Q}{\mathbb Q}
\newcommand{\R}{\mathbb R}
\newcommand{\Z}{\mathbb Z}
\newcommand{\F}{\mathbb F}
\newcommand{\br}{\mathbf{r}}
\newcommand{\RP}{\mathbb{RP}}
\newcommand{\CP}{\mathbb{CP}}
\newcommand{\nbit}[1]{\{0, 1\}^{#1}}
\newcommand{\bits}{\{0, 1\}^{n}}
\newcommand{\bbni}{\bigbreak \noindent}
\newcommand{\norm}[1]{\left\vert\left\vert#1\right\vert\right\vert}
\newcommand{\dbar}{\overline{\partial}}
\let\d\relax
\let\calF\relax
\newcommand{\d}{\partial}
\newcommand{\calO}{\mathcal{O}}
\newcommand{\calF}{\mathcal{F}}
\newcommand{\calG}{\mathcal{G}}
\newcommand{\calH}{\mathcal{H}}
\newcommand{\calE}{\mathcal{E}}

\let\1\relax
\newcommand{\1}{\mathbf{1}}
\newcommand{\fr}[2]{\left(\frac{#1}{#2}\right)}

\newcommand{\vecz}{\mathbf{z}}
\newcommand{\vecr}{\mathbf{r}}
\DeclareMathOperator{\Cinf}{C^{\infty}}
\DeclareMathOperator{\Id}{Id}

\DeclareMathOperator{\Alt}{Alt}
\DeclareMathOperator{\ann}{ann}
\DeclareMathOperator{\codim}{codim}
\DeclareMathOperator{\End}{End}
\DeclareMathOperator{\Hom}{Hom}
\DeclareMathOperator{\id}{id}
\DeclareMathOperator{\M}{M}
\DeclareMathOperator{\Mat}{Mat}
\DeclareMathOperator{\Ob}{Ob}
\DeclareMathOperator{\opchar}{char}
\DeclareMathOperator{\opspan}{span}
\DeclareMathOperator{\rk}{rk}
\DeclareMathOperator{\sgn}{sgn}
\DeclareMathOperator{\Sym}{Sym}
\DeclareMathOperator{\tr}{tr}
\DeclareMathOperator{\img}{img}
\DeclareMathOperator{\CandE}{CandE}
\DeclareMathOperator{\CandO}{CandO}
\DeclareMathOperator{\argmax}{argmax}
\DeclareMathOperator{\first}{first}
\DeclareMathOperator{\last}{last}
\DeclareMathOperator{\cost}{cost}
\DeclareMathOperator{\dist}{dist}
\DeclareMathOperator{\path}{path}
\DeclareMathOperator{\parent}{parent}
\DeclareMathOperator{\argmin}{argmin}
\DeclareMathOperator{\excess}{excess}
\let\Pr\relax
\DeclareMathOperator{\Pr}{\mathbf{Pr}}
\DeclareMathOperator{\Exp}{\mathbb{E}}
\DeclareMathOperator{\Var}{\mathbf{Var}}
\let\limsup\relax
\DeclareMathOperator{\limsup}{limsup}
%Paired Delims
\DeclarePairedDelimiter\ceil{\lceil}{\rceil}
\DeclarePairedDelimiter\floor{\lfloor}{ \rfloor}


\newcommand{\dagstar}{*}

\newcommand{\tbigwedge}{{\textstyle{\bigwedge}}}
\setlength{\parindent}{0pt}
\setlength{\parskip}{5pt}



\begin{document}


\title{CS 40: Computational Complexity}

\author{Sair Shaikh}
\maketitle

Collaboration Notice: Talked to Henry Scheible '26 to discuss ideas.


\subsection*{Background: [1 slides]}
There is an earlier proof that $P \neq NP$ does not relativize, as in there exist oracles A and A' such that in the relativized world you can prove $P \neq NP$ or $P = NP$. Thus, any proof technique that relativizes is not effective against $P \neq NP$. \bbni
This paper says you cant use ``natural" circuit lower bound arguments to decide $P \neq NP$. Thus, similarly, if your proof ``naturalizes'', it is fruitless against $P \neq NP$ and you may as well abandon it.

\subsection*{Idea: [1 Slide]}
Here's a common way in which you may hope to show $P \neq NP$ using circuit lower bounds.
\begin{enumerate}
    \item Formulate some measure of ``variation" or ``scatter" on Boolean functions. 
    \item Show that polynomial sized circuits can only compute functions of low ``variation". This is some combinatorial property $C_n$. 
    \item Show that SAT or some other function in NP has high "variation". 
\end{enumerate}
This paper shows that if PRGs exist and $C_n$ is ``natural" then such a proof cannot work.


\subsection*{Natural Combinatorial Properties: [2 Slides]} 
A combinatorial property is some subset $\bigcup_n \{ C_n \subset F_n  \}$. 
A function $f_n$ is said to have the combinatorial property if $f_n \in C_n$. 

The combinatorial property $C_n$ is $\Gamma$-\textbf{natural} if it contains $C^*_n$ satisfying the following: 
\begin{enumerate}
    \item Constructivity: $f_n \in C^*_n$ can be decided in $\Gamma$. Note that the ``input sizes'' here are $2^n$, i.e. $f_n$ as a truth-table. 
    \item Largeness: $|C^*_n| \geq 2^{-O(n)}*|F_n| = |F_n|/N^k$ where $N = 2^n$. Intuitively, there is a non-negligible chance for a random $f_n$ to have this property. 
\end{enumerate}
When we say ``natural'' without qualification, we mean P/poly-natural. \bbni
A property is \textbf{useful} against P/poly if every sequence of functions ${f_n}$ with $f_n \in C_n$ is superpolynomial. That is, for any $k$, there exists an $n$, such that $f_n$ has a circuit lower bound $\geq n^k$. 


\subsection*{Natural Proofs: [2 slides]} 
A proof is natural against P/poly if it contains some ``natural combinatorial property" that is "useful" against P/poly.
This definition is a little vague and you might have to work to show some property used in some proof is ``natural". Towards the end of the talk, we will see that most things we can consider are natural.

Main Theorem: Such a property can be used to break any PRG. 

Any proof that some function $f_n$ does not have small circuits must either:
\begin{enumerate}
    \item Use some very specialized property of $f_n$, one shared by only a negligible fraction of functions. 
    \item Must define a property so complicated that it is outside the bounds of mathematical experience. 
\end{enumerate}
That is, the proof must be unnatural by violating either ``largeness" or ``constructivity." \\
The authors claim there are no such examples in the literature, and there are theoretical reasons for why finding these is difficult. 


\subsection*{Theorem Idea: [1 Slide]}
A natural proof that some function $f$ is not in P/poly must distinguish $f$ from a pseudo-random function in P/poly. This can be converted into an algorithm that can tell $f$ from a pseudorandom function in P/poly. This can be used to break a PRG. 

\subsection*{Theorem Statement: [1 Slide]} 
Define the hardness $H(G_k)$ of a pseudorandom generator $G_k: {0,1}^k \to {0,1}^{2k}$ as the minimal $S$ for which there exists a circuit of size $\leq S$:
\[ Pr[C(G_k(x)) = 1 ] - Pr[C(y) = 1] \geq 1/S \]

\begin{theorem*}  
There is no P/poly-natural proof against P/poly unless $H(G_k) \leq 2^{k^{o(1)}}$ for every psedorandom generator $G_k$. 
In particular, if $2^{n^\epsilon}$-hard functions exist, then there is no P/poly-natural proof against P/poly. 
\end{theorem*}

\subsection*{Proof: [7 Slides]}

[1st Slide]:\\
Assume for the sake of contradiction such a natural proof exists. Let $C_n$ be the associated large and constructible property useful against P/poly. Let $G: \{0,1\}^k \to \{0,1\}^{2k}$ be a pseudo-random generator. Let $\epsilon > 0$ and $n = \ceil{k^\epsilon}$. We will do this in $3$-steps: 
\begin{enumerate}
    \item Use $G$ to construct ``pseudo-random" boolean function in $F_n$.
    \item Use $C_n$ to provide a statistical test distinguishing random functions in $F_n$ from ``psuedo-random" these functions.
    \item Convert this statistical test to a statistical test against $G$.  
\end{enumerate}

[2nd Slide]: \\
\textbf{Step 1:} Use $G$ to construct ``pseudo-random" boolean function in $F_n$. \bbni
Let $G_0, G_1: \{0, 1\}^k \to \{0, 1\}^k$ be the first and last $k$ bits of $G$. For $y \in \{0, 1\}^n$, let $G_y = G_{y_n} \circ G_{y_{n-1}} \circ \cdots \circ G_{y_0}$. Let $f(x)(y)$ be the first bit of $G_y(x)$. Then, $f(x)$ is a Boolean function. \bbni 
Visual: For some fixed $x$, take input $y$. Send $x$ through $G$, take the first or second half based on $y_0$, send this through $G$, take first or second half using $y_1$, and so on. \bbni
Thus, $f(x)(y)$ is in $P$, and in particular, computable by poly-sized circuits. \bbni

[3rd Slide]: \\
\textbf{Step 2:} Use $C_n$ to provide a statistical test distinguishing random functions in $F_n$ from ``psuedo-random" functions $f(x)$, $x \in \{0, 1\}^k$.  \bbni

As $C_n$ is useful against $P/poly$, $f(x)$ is not in $C_n$ for any fixed $x \in \{0, 1\}^k$ and sufficiently large $k$. \\
Explanation: For fixed $x$, $f(x)$ defines a family of functions as you can input any length $y$. These have circuits upperbounded in size by some polynomial in $n$ ($n$ and $k$ are polynomially related). Thus, we can go far enough to get some instance of $f(x)$ that's not in $C_n$. \bbni

Thus, $C_n$ has empty intersection with $\{f(x) : x \in {0, 1}^k\}$. This provides a statistical test:
\[ |\Pr[C_n(\textbf{f}_n) = 1] - \Pr[C_n(f(\textbf{x})) = 1]| \geq 2^{-O(n)} \]
which is computable by circuits of size $2^{O(n)}$. \bbni
Note: By largeness, $\Pr[C_n(\textbf{f}_n)] \geq 2^{-O(n)}$ and the second term is $0$. By constructivity, this is computable by circuits of polynomial size in $2^n$, i.e. in $2^{O(n)}$. \bbni

[4th Slide]: \\
\textbf{Step 3:} Convert this statistical test to a statistical test against $G$. \bbni
We represent the computation of $f(x)$ for all $y \in \{0, 1\}^n$ as the binary tree $T$, i.e. the split at the $i$th level represents each successive bit of $y$, and the leafs represent the outputs. \todo{visual here}.  \bbni

[5th Slide]:\\
We order the internal vertices ``children first", i.e. if $v_i \to v_j$ than $j < i$. Thus, we get $v_1, \cdots, v_{2^n-1}$. \bbni

Then, we look at all the sub-trees of the form $T_i$ which contain all vertices upto $v_i$ and all leaves. Let $v_i(y)$ be the root of the subtree containing $y$ and $h_{v_i}(y)$ be the height of this subtree. Define $G_{i, y} = G_{y_n} \circ \cdots \circ G_{i, y_{n-h_{v_i}(y)+1}}$. \todo{Visual here.}\bbni

[6th Slide]: \\
Define random collection $\textbf{f}_{i, n}$ by setting $\textbf{f}_{i, n}(y)$ to the first bit of $G_{i,y}(\textbf{x}_{v_i(y)})$ for $\textbf{x}_{v_i(y)}$ being uniformly and independently random. \bbni
Notice that $\textbf{f}_{0, n} = \textbf{f}_n$ and $\textbf{f}_{2^n-1, n} = f(\textbf{x})$. Note: $h_{v_0}(y) = 1$ and $h_{v_{2^n-1}}(y) = n$. \bbni 
So, we get for some $i$, 
\[ |\Pr[C_i(f_{i+1, n}(x)) = 1] - \Pr[C_{i}(f_{i,n}(x)) = 1]| \geq \frac{2^{-O(n)}}{2^n} = 2^{-O(n)}\]

[7th Slide]: \\
Next, fix $\textbf{x}_v$ for roots $v$ of all subtrees in $T_{i+1}$ except $v_{i+1}$. Then, we have a statistical test to distinguish $G(\textbf{x}_{v_{i+1}})$ from $(\textbf{x}_{v_i'}, \textbf{x}_{v_i''})$ where $v_i', v_i''$ are the children of $v_i$. \bbni
Then, $H(G) \leq 2^{O(n)} \leq 2^{O(k^\epsilon)}$. As $\epsilon$ was arbitrary, we have broken $G$. 

\subsection*{More general comments: [1 Slide] }
More generally, if a complexity class $\Lambda$ contains a pseudo-random function generator $G$ that are sufficiently secure against $\Gamma$ -- i.e. any algorithm in $\Gamma$ can only break $G$ with exponentially small probability, then there is no $\Gamma$-natural proof against $\Lambda$.


\subsection*{Formal complexities measures are large: [3 Slides] }
[1st Slide] \\
Define a formal complexity measure $\mu: F_n \to \Z$ to be an integer-valued function to satisfy the following: 
\begin{enumerate}
    \item $\mu(f) \leq 1$ if $f$ is a $x_i$ or $\lnot x_i$. 
    \item $\mu(f \land g) \leq \mu(f) + \mu(g)$
    \item $\mu(f \lor g) \leq \mu(f) + \mu(g)$
\end{enumerate}
That is to say, $\mu(f)$ is a lower bound on the formula size for $f$. \bbni 
Examples: 
\begin{enumerate}
    \item Circuit complexity.
    \item The degree or number of monomials of the polynomial obtained by arithmetization.
\end{enumerate}

[2nd Slide] \\
Any combinatorial property based on $\mu$ already satisfies the ``largeness'' condition. In particular, we have the following theorem: 
\begin{theorem*}
    Let $\mu$ be a formal complexity measure on $F_n$ and let $\mu(f) = t$ for some $f$. Then, 
    \begin{enumerate}
        \item For at least $\frac{1}{4}$ of all functions $g \in F_n$, $\mu(g) \geq \frac{t}{4}$.
        \item For any $\epsilon = \epsilon(n)$, for at least $(1-\epsilon)$ fraction of $g \in F_n$, 
        \[ \mu(g) \geq \Omega\left(\frac{t}{\left(n+\log\left(\frac{1}{\epsilon}\right)\right)^2}\right)-n\]
    \end{enumerate}
\end{theorem*}
If you plug in $\epsilon = 1-2^{-O(n)}$, you get that for at least $2^{-O(n)}$ fraction of $g \in F_n$ have complexity $\frac{\alpha t}{n^2}$ for some $\alpha > 0$. Thus, for this to prove $P \neq NP$ the cutoff on what circuits can compute will have to be between $t/\text{poly}(n)$ and $t$. \bbni 

[3rd Slide] \\
Proof of (a): \\
Let $g$ be a uniformly random function in $F_n$. Write $f = h \oplus g$ with $h = f \oplus g$. Then, 
    \[ f = (\lnot h \land g) \lor (h \land \lnot g) \]
Note that $g, h, \lnot g, \lnot h$ are all uniformly random. Now imagine $\{g : \mu(g) \leq t/4\}$ at least $\frac{3}{4}$ of all functions. Then, by the union bound, 
\[ \Pr[\mu(g), \mu(h), \mu(\lnot g), \mu(\lnot h) < t/4] > 0 \] 
But then $\mu(f) < t$ for this choice of $g$. Contradiction. \bbni
The general idea for $(b)$ is similar, i.e. write $f$ as a boolean combination of a small number combinations involving random variables, then use union bound and the probabilistic method.  


\textbf{Total: 20 slides}



\end{document}


