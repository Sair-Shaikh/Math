\documentclass[12pt]{article}

\usepackage{fullpage}
\usepackage{mdframed}
\usepackage{colonequals}
\usepackage{algpseudocode}
\usepackage{algorithm}
\usepackage{tcolorbox}
\usepackage[all]{xy}
\usepackage{proof}
\usepackage{mathtools}
\usepackage{bbm}
\usepackage{amssymb}
\usepackage{amsthm}
\usepackage{amsmath}
\usepackage{amsxtra}
\newcommand{\bb}{\mathbb}


\newtheorem{theorem}{Theorem}[section]
\newtheorem{corollary}{Corollary}[theorem]
\newtheorem{lemma}{Lemma}

\newcommand{\mathcat}[1]{\textup{\textbf{\textsf{#1}}}} % for defined terms

\newenvironment{problem}[1]
{\begin{tcolorbox}\noindent\textbf{Problem #1}.}
{\vskip 6pt \end{tcolorbox}}

\newenvironment{enumalph}
{\begin{enumerate}\renewcommand{\labelenumi}{\textnormal{(\alph{enumi})}}}
{\end{enumerate}}

\newenvironment{enumroman}
{\begin{enumerate}\renewcommand{\labelenumi}{\textnormal{(\roman{enumi})}}}
{\end{enumerate}}

\newcommand{\defi}[1]{\textsf{#1}} % for defined terms

\theoremstyle{remark}
\newtheorem*{solution}{Solution}

\setlength{\hfuzz}{4pt}

\newcommand{\calC}{\mathcal{C}}
\newcommand{\calF}{\mathcal{F}}
\newcommand{\C}{\mathbb C}
\newcommand{\N}{\mathbb N}
\newcommand{\Q}{\mathbb Q}
\newcommand{\R}{\mathbb R}
\newcommand{\Z}{\mathbb Z}
\newcommand{\br}{\mathbf{r}}
\newcommand{\RP}{\mathbb{RP}}
\newcommand{\CP}{\mathbb{CP}}
\newcommand{\nbit}[1]{\{0, 1\}^{#1}}
\newcommand{\bits}{\{0, 1\}^{n}}
\newcommand{\bbni}{\bigbreak \noindent}
\newcommand{\norm}[1]{\left\vert\left\vert#1\right\vert\right\vert}

\let\1\relax
\newcommand{\1}{\mathbf{1}}
\newcommand{\fr}[2]{\left(\frac{#1}{#2}\right)}

\newcommand{\vecz}{\mathbf{z}}
\newcommand{\vecr}{\mathbf{r}}
\DeclareMathOperator{\Cinf}{C^{\infty}}
\DeclareMathOperator{\Id}{Id}

\DeclareMathOperator{\Alt}{Alt}
\DeclareMathOperator{\ann}{ann}
\DeclareMathOperator{\codim}{codim}
\DeclareMathOperator{\End}{End}
\DeclareMathOperator{\Hom}{Hom}
\DeclareMathOperator{\id}{id}
\DeclareMathOperator{\M}{M}
\DeclareMathOperator{\Mat}{Mat}
\DeclareMathOperator{\Ob}{Ob}
\DeclareMathOperator{\opchar}{char}
\DeclareMathOperator{\opspan}{span}
\DeclareMathOperator{\rk}{rk}
\DeclareMathOperator{\sgn}{sgn}
\DeclareMathOperator{\Sym}{Sym}
\DeclareMathOperator{\tr}{tr}
\DeclareMathOperator{\img}{img}
\DeclareMathOperator{\CandE}{CandE}
\DeclareMathOperator{\CandO}{CandO}
\DeclareMathOperator{\argmax}{argmax}
\DeclareMathOperator{\first}{first}
\DeclareMathOperator{\last}{last}
\DeclareMathOperator{\cost}{cost}
\DeclareMathOperator{\dist}{dist}
\DeclareMathOperator{\path}{path}
\DeclareMathOperator{\parent}{parent}
\DeclareMathOperator{\argmin}{argmin}
\DeclareMathOperator{\excess}{excess}
\let\Pr\relax
\DeclareMathOperator{\Pr}{\mathbf{Pr}}
\DeclareMathOperator{\Exp}{\mathbb{E}}
\DeclareMathOperator{\Var}{\mathbf{Var}}
\let\limsup\relax
\DeclareMathOperator{\limsup}{limsup}
%Paired Delims
\DeclarePairedDelimiter\ceil{\lceil}{\rceil}
\DeclarePairedDelimiter\floor{\lfloor}{ \rfloor}


\newcommand{\dagstar}{*}

\newcommand{\tbigwedge}{{\textstyle{\bigwedge}}}
\setlength{\parindent}{0pt}
\setlength{\parskip}{5pt}


\begin{document}

\title{CS 40: Computational Complexity}

\author{Sair Shaikh}
\maketitle

% Collaboration Notice: Talked to Henry Scheible '26 to discuss ideas.


\begin{problem}{30}
    Suppose that $X$ and $Y$ are normed vector spaces. 
    \begin{enumerate}
        \item Show that $\mathcal{L}(X, Y)$ is a normed vector space with respect to the the operator norm defined in lecture such that: 
        \[ ||T(x)|| \leq ||T|| ||x|| \]
        \item Show that if $S \in \mathcal{L}(Y, Z)$. Then, 
        \[  ||ST|| \leq ||S||||T|| \]
        \item Show that: 
        \[ ||T|| = \inf\{ a \geq 0 : ||T(x)|| \leq a||x|| \quad \forall x \in X\}\]
    \end{enumerate}
\end{problem}
\begin{solution} 
    \bbni 
    \begin{enumerate}
        \item First, note that $\mathcal{L}(X, Y)$ is a vector space, through pointwise addition and scalar multiplication defined in $Y$, i.e. for $T, S \in \mathcal{L}(X, Y)$ and $\alpha \in \mathbb{F}$ we let:
        \[(\alpha T + S)(x) := \alpha T(x) + S(x)\]
        for all $x \in X$. Thus, we only need to show that the operator norm is a norm and satisfies the given property. Recall the definition of the operator norm for $T \in \mathcal{L}(X, Y)$:
        \[ ||T|| = \sup_{||x|| \leq 1} ||T(x)||\] 
        \begin{enumerate}
            \item[Non-Neg.] Let $T \in \mathcal{L}(X, Y)$ be arbitrary. Then, for all $x \in X$ with $||x|| \leq 1$, $||T(x)|| \geq 0$ by the non-negativity of the norm on $Y$. Thus, $||T|| \geq 0$. 
            \item[Homogeneity.] Let $T \in \mathcal{L}(X, Y)$ and $\alpha \in \mathbb{F}$. Then, for every $x \in X$, with $||x|| \leq 1$, we have that: 
            \begin{align*}
                ||(\alpha T)(x)|| &= ||\alpha T(x)|| \\
                &= |\alpha|\cdot ||T(x)|| \\
                &\leq |\alpha| \cdot ||T||
            \end{align*} 
            using the homogeneity of the norm on $Y$. Thus, 
            \[ ||\alpha T|| \leq |\alpha| \cdot ||T(x)||\]
            Similarly, we also have: 
            \begin{align*}
                |\alpha| \cdot ||T(x)|| &= ||\alpha T(x) || \\
                &= ||(\alpha T)(x)|| \\
                &\leq ||\alpha T||
            \end{align*}
            Thus, we have:
            \[ |\alpha| \cdot ||T|| \leq ||\alpha T||\]
            Thus, we have shown that:
            \[ ||\alpha T|| = |\alpha| \cdot ||T||\]
            \item[$\triangle$ ineq.] Let $T, S \in \mathcal{L}(X, Y)$. For every $x \in X$, with $||x|| \leq 1$, we have: 
            \begin{align*}
                ||(T+S)(x)|| &= ||T(x) + S(x)||  \\
                &\leq ||T(x)|| + ||S(x)|| \\
                &\leq ||T|| + ||S||
            \end{align*}
            using the triangle inequality for the norm in $Y$. Thus, 
            \[ ||T+S|| \leq ||T|| + ||S||\]
            \item[Pos. Def.] To prove this, we first will show that $||T(x)|| \leq ||T||||x||$ for all $x \in X$. For $x \in X$, by homogeneity (and non-negativity) of the norm on $X$: 
            \[  \left\lvert\left\lvert\frac{1}{||x||} x\right\rvert\right\rvert = \frac{1}{||x||}\cdot ||x|| = 1 \]
            Thus, we note that:
            \[  \left\lvert\left\lvert T\left(\frac{1}{||x||} x\right)\right\rvert\right\rvert \leq ||T|| \]
            By the linearity of $T$ and the homogeneity of the norm in $Y$, this implies:
            \begin{align*}
                \frac{1}{||x||} ||T(x)||  &= 
                \left\lvert\left\lvert \frac{1}{||x||} T(x)\right\rvert\right\rvert \\
                &=  \left\lvert\left\lvert T\left(\frac{1}{||x||} x\right)\right\rvert\right\rvert \\
                &\leq ||T|| \\
            \end{align*}
            Thus, 
            \[ ||T(x)|| \leq ||T|| \cdot ||x||\]
            Now, let $T \in \mathcal{L}(X, Y)$ be such that $||T|| = 0$. Then, for all $x \in X$, we have that: 
            \begin{align*}
                ||T(x)|| &\leq ||T|| \cdot ||x|| \\
                &= 0
            \end{align*}
            However, by the non-negativity of the norm in $Y$, we must have that $||T(x)|| = 0$ for all $x \in X$. Then, by positive definiteness of the norm in $Y$, we have that $T(x) = 0$ for all $x \in X$. Thus, $T$ is the zero map. \\
            Conversely, if $T$ is the zero map, then for all $x \in X$ with $||x|| \leq 1$, we have that:
            \[ ||T|| = ||0 \cdot T|| = 0 \cdot ||T|| = 0\]
            by homogeneity. Thus, the norm is positive definite.
        \end{enumerate}
        Therefore, we have shown that the operator norm is a norm on $\mathcal{L}(X, Y)$ and satisfies for all $T \in \mathcal{L}(X, Y)$ and $x \in X$:
        \[ ||T(x)|| \leq ||T|| \cdot ||x||\]
        \item For any $x \in X$, with $||x|| = 1$, by applying the property from part 1 twice, we have that: 
        \begin{align*}
            ||ST(x)|| &= ||S(T(x))|| \\
            &\leq ||S|| \cdot ||T(x)|| \\
            &\leq ||S|| \cdot ||T|| \cdot ||x|| \\
            &= ||S|| \cdot ||T||
        \end{align*}
        Thus,
        \[ ||ST|| \leq ||S|| \cdot ||T||\]
        \item Let $\alpha(T)$ be the defined infimum. \bbni
        Since we have that $||T(x)|| \leq ||T||\cdot ||x||$ for all $x \in X$, $||T||$ is in the set we are taking the infimum over. Thus, $\alpha(T) \leq ||T||$. \bbni
        Moreover, by the definition of $\alpha(T)$, we have that for all $x \in X$ with $||x|| \leq 1$,
        \[ ||T(x)|| \leq \alpha(T)||x|| = \alpha(T) \]
        Thus, $\alpha(T)$ is an upperbound on $||T(x)||$ with $||x|| \leq 1$. Therefore, by the definition of the supremum, we have that:
        \[ ||T|| \leq \alpha(T)\]
        Thus, we have shown that:
        \[ ||T|| = \alpha(T) = \inf\{ a \geq 0 : ||T(x)|| \leq a||x|| \quad \forall x \in X\}\]
    \end{enumerate}
\end{solution}
\newpage 

\begin{problem}{31}
    Suppose that $X$ and $Y$ are Banach spaces with $T \in \mathcal{L}(X, Y)$. Suppose that $E$ is a closed proper subspace of $X$ such that $E \subset \ker(T)$. Show that there is a unique operator $\overline{T} \in \mathcal{L}(X/E, Y)$ such that $\overline{T}(q(x)) = T(x)$ for all $x \in X$ where $q: X \to X/E$ is the quotient map. Moreover, $||\overline{T}|| = ||T||$.
\end{problem}
\begin{solution} 
    We claim the map $\overline{T}: X/E \to Y$ given by: 
    \[ \overline{T}([x]) = T(x)\]
    satisfies the desired properties. Clearly, $\overline{T}(q(x)) = T(x)$ for all $x \in X$ by definition. We need to show that $\overline{T}$ is well-defined, linear, and satisfies $||\overline{T}|| = ||T||$ (hence is bounded/continous). \bbni
    Let $x, y \in X$ be such that $[x] = [y]$. Then, note that $x - y \in E \subset \ker(T)$. Thus, $T(x)-T(y) = T(x-y) = 0$. Thus, $T(x) = T(y)$. Then, by the definition of $\overline{T}$, we have $\overline{T}([x]) = \overline{T}([y])$. Thus, $\overline{T}$ is well-defined. \bbni
    Next, let $[x], [y] \in X/E$ and $\alpha \in \mathbb{F}$ be arbitrary. Then, noting the linearity of $q$ and $T$, we have that:
    \begin{align*}
        \overline{T}(\alpha[x] + [y]) &= \overline{T}([\alpha x + y]) \\
        &= T(\alpha x + y) \\
        &= \alpha T(x) + T(y) \\
        &= \alpha \overline{T}([x]) + \overline{T}([y])
    \end{align*}
    Thus, $\overline{T}$ is linear. \bbni
    Finally, we need to show that $||\overline{T}|| = ||T||$. First, we show that $||q|| = 1$. We already showed that $||q|| \leq 1$. Next, let $x \in X$ such that $||x||\leq 1$. Then, if $x \in E$, $q(x) = 0$. Thus, we consider $x \in X\setminus E$. Since $q$ is bounded, for all $e \in E$: 
    \begin{align*}
        ||q(x)||_{X/E} &= ||q(x+e)||_{X/E} \\
        &\leq ||q|| \cdot ||x+e||_{X}
    \end{align*}
    Thus,
    \[ ||q(x)||_{X/E} \leq ||q|| \cdot \inf_{e\in E} ||x+e||_X = ||q|| \cdot ||q(x)||_{X/E} \]
    Dividing through by $||q(x)|| > 0$ (as $x \not \in E$ closed), we have:
    \[ ||q|| \geq 1\]
    Thus, $||q|| = 1$. Next, note that for all $[x] \in X/E$ and for all  $e \in E \subset \ker(T)$,
    \begin{align*}
        ||\overline{T}([x])|| &= ||T(x)|| \\
        &= ||T(x+e)|| \\
        &\leq ||T|| \cdot ||x+e||_X \\
    \end{align*}
    Thus, we have that: 
    \[||\overline{T}([x])|| \leq ||T|| \inf_{e\in E} ||x+e||_X = ||T||\cdot \norm{[x]}_{X/E}\]
    Thus, by the definiton of the operator norm in Part 3 of Problem 30, we note that since $||T||$ is a bound for $\overline{T}$, and $||\overline{T}||$ is the infimum of these bounds, $||\overline{T}|| \leq ||T||$. This also shows $\overline{T}$ is bounded, thus, $\overline{T} \in \mathcal{L}(X/E, Y)$. Finally, from Problem 30 Part 2, we also have:
    \begin{align*}
        ||T|| \leq ||\overline{T}||||q|| = ||\overline{T}||
    \end{align*}
    Thus, $||\overline{T}|| = ||T||$. 
\end{solution}
\newpage 


\begin{problem}{33}
    Let $E$ and $X$ be Banach spaces with $E$ finite dimensional. 
    \begin{enumerate}
        \item Show that every linear map $S: E \to X$ is bounded. 
        \item Show that a linear map $T: X \to E$ is bounded if and only if $\ker(T)$ is closed. 
    \end{enumerate}
\end{problem}
\begin{solution} 
    \bbni
    \begin{enumerate}
        \item Let $e_1, \ldots, e_n$ be a basis for $E$. Then $S(e_1), \ldots, S(e_n)$ span the image of $S$. Let $B = \max\{ ||S(e_i)|| : 1 \leq i \leq n\}$. Then, let $x = \sum_{i = 1}^n a_ie_i \in E$ be arbitrary (where $a_i \in \mathbb F$). Then, using the triangle inequality, 
        \begin{align*}
            ||S(x)|| &= \norm{S\left(\sum_{i = 1}^n a_ie_i\right)} \\
            &= ||\sum_{i = 1}^n a_i S(e_i)|| \\
            &\leq \sum_{i = 1}^n |a_i| \cdot ||S(e_i)|| \\
            &\leq B \sum_{i = 1}^n |a_i|\\
            &= B \cdot ||x||_\infty
        \end{align*} 
        where we are using the identification of $E$ with $\mathbb{F}^n$ to define the $\norm{\cdot}_\infty$ (i.e. its defined with respect to our picked basis). However, as $E$ is finite dimensional, we use the fact that all norms are equivalent to obtain a constant $C > 0$ such that: 
        \[ ||x||_\infty < C||x|| \]
        Thus, we have that: 
        \[ ||S(x)|| \leq BC ||x||\]
        Thus, $S$ is bounded.
        \item Assume that $T$ is bounded. Since $X$ and $E$ are Banach spaces ($E$ is Banach since all finite dimensional spaces are Banach), we have that $T$ is continuous. Thus, the preimage of a closed set is closed. Since $E$ is a normed vector space, it is a metric space, and hence Hausdorff. Thus the singleton $\{0\}$ is closed in $E$. Thus, $T^{1}(\{0\}) = \ker(T)$ is closed in $X$. \bbni
        Next, assume that $\ker(T)$ is closed. Then, the quotient space $X/\ker(T)$ is a Banach space with norm given by the quotient norm. Then using results from Problem 31, we have the map $\overline{T}: X/\ker(T) \to E$ given by:
        \[ \overline{T}([x]) = T(x)\]
        is well-defined and linear. Moreover, if $\overline{T}$ is bounded, then so is $T$ as $||\overline{T}|| = ||T||$. Thus, we need to show that $\overline{T}$ is bounded. \bbni
        However, by the first isomorphism theorem for vector spaces, $X/\ker(T)$ is isomorphic to $\img(T) \subset E$, which is finite dimensional. Thus, $\overline{T}$ is a map from a finite dimensional Banach space, and is thus bounded by part 1. Thus, since $||\overline{T}|| = ||T||$, we have that $T$ is bounded.     \end{enumerate}
\end{solution}
\newpage 

\begin{problem}{34}
    Supposed that $E$ and $M$ are closed subspaces of a Banach space $X$. If $E$ is finite dimensional, show that $E+M = \{x+y: x \in E, y \in M\}$ is closed. 
\end{problem}
\begin{solution}
    Let $q: X \to X/M$ be the quotient map. Since $M$ is closed, $X/M$ is a Banach space. Since $\dim(q(E)) \leq \dim(E) < \infty$ (rank-nullity applied to $q|_E$) so $q(E)$ is a finite dimensional subspace of $X/M$, thus Banach, thus closed. Then, since $q$ is continous, the pre-image of closed sets is closed. However, note that $q^{-1}(q(E)) = E + M$. Thus, $E + M$ is closed in $X$.
\end{solution}
\newpage 

\begin{problem}{35}
    Suppose that $X$ and $Y$ are Banach spaces for $T \in \mathcal{L}(X, Y)$. Show that $T$ is injective with closed range if and only if: 
    \[ \inf \{||T(x)|| : ||x|| = 1\} > 0 \]
\end{problem}
\begin{solution} 
    Let $m := \inf \{||T(x)|| : ||x|| = 1\}$. \bbni
    First, assume that $T$ is injective with closed range. Then, $\img(T)$ is Banach as it is closed. Moreover, as $T$ is injective, it is a bijection onto its image. Thus, $T$ is a surjective continuous linear map between Banach spaces, and is thus open (Open Mapping Theorem). Since $T$ is a continous open bijection, it is a homeomorphism and has a continous inverse $T^{-1}: \img(T) \to X$. Thus, $T^{-1}$ is bounded. Hence, there exists a constant $C > 0$ such that $\forall x \in X$:
    \[ ||x|| \leq C||T(x)||\]
    Thus, $\forall x \in X$ with $||x|| = 1$, we rearrange to get:
    \[ ||T(x)|| \geq \frac{1}{C}\]
    Thus, $m \geq \frac{1}{C} > 0$. \bbni
    Next, assume that $m > 0$. Then for all $x \in X$ with $||x|| = 1$, we have that: 
    \[ ||T(x)|| \geq m\]
    Then, for all $x \in X$, since $\frac{x}{\norm{x}}$ has norm $1$, we have that (by homogeneity of the norm on $Y$ and linearity of $T$): 
    \begin{align*}
        \frac{1}{\norm{x}}\norm{T\left(x\right)} &= \norm{T\left(\frac{x}{\norm{x}}\right)} \\
        &\geq m 
    \end{align*}   
    Thus, 
    \[||T(x)|| \geq m||x||\]  
    Now, if $T(x) = 0$, then $||T(x)|| = 0$. Then, since $m > 0$, the inequality above implies that $||x|| = 0$. Then, by positive definiteness of the norm, we have $x = 0$. Thus, $T$ is injective. \bbni
    As $T$ is injective, it is a bijection onto its image. Thus, we can define $T^{-1}: \img(T) \to X$ (as a linear map). Then, for all $y \in \img(T)$, there exists $x \in X$ such that $T(x) = y$. Then, we have: 
    \begin{align*}
        ||T^{-1}(y)|| = ||T^{-1}(T(x))|| &= ||x|| \\
        &\leq \frac{1}{m} ||T(x)|| = \frac{1}{m} ||y||
    \end{align*}
    Thus, $T^{-1}$ is bounded. \bbni
    Now, let $(y_n) \subset \img(T)$ be a Cauchy sequence. Then, since $T^{-1}$ is bounded, $(T^{-1}(y_n))$ is also a Cauchy sequence. To see this, let $\epsilon > 0$. Then, since $(y_n)$ is Cauchy, there exists $N \in \N$ such that for all $m, n \geq N$, we have that: 
    \[ ||y_m - y_n|| < m \epsilon\]
    Then, we note that: 
    \begin{align*}
        ||T^{-1}(y_m) - T^{-1}(y_n)|| &= ||T^{-1}(y_m - y_n)|| \\
        &\leq \frac{1}{m} ||y_m - y_n|| \\
        &< \frac{1}{m} \cdot m\epsilon \\
        &= \epsilon
    \end{align*}
    Thus, $(T^{-1}(y_n))$ is Cauchy. Since $X$ is complete, $(T^{-1}(y_n)) \to x \in X$. Let $y = T(x) \in \img(T)$. Then, since $T$ is continous, we have that: 
    \[ (y_n) \to y\]
    Thus, $\img(T)$ is complete (hence Banach). Thus, $\img(T)$ is a closed subspace of $Y$. Thus, $T$ is injective with closed range.
\end{solution}
\newpage


\begin{problem}{38}
    Let $X$ be a normed vector space. A Banach space $\tilde{X}$ is called a completion of $X$ is there is an isometric isomorphism $\iota: X \to \tilde{X}$ onto a dense subspace of $\tilde{X}$. Show that any two completions $(\tilde{X}_1, \iota_1)$ and $(\tilde{X}_2, \iota_2)$ are isometrically isomorphic by an isomorphism: 
    \[ \Phi: \tilde{X}_1 \to \tilde{X}_2\]
    such that $\Phi(\iota_1(x)) = \iota_2(x)$ for all $x \in X$. 
\end{problem}
\begin{solution}
    \begin{lemma}
        If $\phi: X \to Y$ is an isometric isomorphism of normed vector spaces, then $\phi^{-1}: Y \to X$ is also an isometric isomorphism.
    \end{lemma}
    \begin{proof}
        Since $\phi$ is an isomorphism, we know that $\phi^{-1}$ is a well-defined linear isomorphism. Thus, we only need to show that $\phi^{-1}$ is an isometry. \bbni
        Let $y \in Y$ be arbitrary. Then, since $\phi$ is an isometry, we have that:
        \begin{align*}
            ||\phi^{-1}(y)||_X &= ||\phi (\phi^{-1}(y))||_Y \\
            &= ||y||_Y
        \end{align*}
        Thus, we have shown that $\phi^{-1}$ is an isometry.
    \end{proof}
    \begin{lemma}
        Let $X$ and $Y$ be Banach spaces and $D$ a dense subspace of $X$. If $T_0 \in \mathcal{L}(D, Y)$, then there exists a unique bounded linear operator $T \in \mathcal{L}(X, Y)$ such that $T(x) = T_0(x)$ for all $x \in D$.
    \end{lemma}
    \begin{proof}
        (This was optional question 32. Please move past this proof if we are allowed to use these without proof.) \\
        Since $D$ is a dense subspace of $X$, for all $x \in X$, there exists a Cauchy sequence $(x_n) \subset D$ such that $(x_n) \to x$ (we can construct this since $D$ meets every open set around $x$, and we can pick points in $B_{1/n}(x) \cap D$, which is obviously Cauchy and converges to $x$). Then, $(T_0(x_n))$ is a Cauchy sequence in $Y$ (since $T$ is bounded). Thus, $T_0(x_n) \to y \in Y$. We can then define $T: X \to Y$ by:
        \[T(x) = y\]
        where $y$ is the limit of $(T_0(x_n))$. \bbni
        We check that this is well-defined. Let $(x_n)$ and $(x'_n)$ be two Cauchy sequences in $D$ converging to $x \in X$. Then, 
        \begin{align*}
            \lim_{n\to \infty} ||T(x_n) - T(x'_n)|| &= \lim_{n \to \infty} ||T(x_n - x'_n)|| \\
            &\leq ||T||\cdot \lim_{n \to \infty} ||x_n - x'_n||
        \end{align*}
        Now, let $\epsilon > 0$. Then, since $(x_n)$ is Cauchy, there exists $N_1 \in \N$ such that for all $m, n \geq N_1$, we have that:
        \[ ||x_n - x_m|| < \frac{\epsilon}{2}\]
        Taking the limit as $m \to \infty$, we have that: 
        \[ ||x_n - x|| < \frac{\epsilon}{2}\]
        Similarly, there exists $N_2 \in \N$ such that $\forall n > N_2$: 
        \[||x'_n - x|| < \frac{\epsilon}{2}\]
        Thus, letting $N > \max\{N_1, N_2\}$, we have that for all $n \geq N$:
        \begin{align*}
            ||(x_n - x'_n) - 0|| &\leq ||x_n - x|| + ||x-x'_n|| \\
            &= ||x_n - x|| + ||x'_n-x|| \\
            &< \frac{\epsilon}{2} + \frac{\epsilon}{2} \\
            &= \epsilon 
        \end{align*}
        Thus, $\lim_{n \to \infty} ||x_n - x'_n|| = 0$. Thus, we have that:
        \begin{align*}
            \lim_{n\to \infty} ||T(x_n) - T(x'_n)|| &\leq ||T||\cdot \lim_{n \to \infty} ||x_n - x'_n|| = 0
        \end{align*}
        Since $||\cdot||$ is non-negative, by the Squeeze Theorem, we have that $\lim_{n \to \infty} ||T(x_n) - T(x'_n)|| = 0$. Thus, the sequences converge to the same limit. Thus, $T$ is well-defined. \bbni
        Moreover, if $x_n \to x \in D$, then $T(x) = \lim_{n \to \infty} T_0(x_n) = T_0(x)$ . \bbni 
        Next, note that $T$ is linear as a linear combination of two Cauchy sequences converges to the the same linear combination of their limits (easy via triangle inequality and picking $\epsilon/2$ and $\epsilon/2\alpha$). Thus, if $(T_0(x_n)) \to y$ and $(T_0(x'_n)) \to y'$, then $(T_0(x_n + \alpha x'_n)) = (T_0(x_n)+\alpha T_0(x'_n)) \to y + \alpha y'$. Thus, $T(x+\alpha x') = T(x) + \alpha T(x')$. \bbni
        Next, we need to show that $T$ is bounded. If $(x_n) \subset D$ is Cauchy and converges to $x \in X$, then we have, by continuity of norm: 
        \begin{align*}
            ||T(x)|| &= \norm{\lim_{n \to \infty} T_0(x_n)} \\
            &= \lim_{n \to \infty} ||T_0(x_n)|| \\
            &\leq \lim_{n \to \infty} ||T_0|| \cdot ||x_n|| \\
            &\leq ||T_0|| \cdot \lim_{n \to \infty} ||x_n|| \\
            &\leq ||T_0|| \cdot ||x||
        \end{align*}
        Thus, $T \in \mathcal{L}(X, Y)$. \bbni
        Finally, to show that $T$ is unique, let $T_1, T_2 \in \mathcal{L}(X, Y)$ with $T_1 = T_2$ on $D$. Then, for all $x \in X$, if $(x_n) \subset D \to x$, then, by continuity, we have that: 
        \[T_1(x)  = \lim_{n\to \infty} T_1(x_n) = \lim_{n\to \infty} T_2(x_n) = T_2(x)  \] 
        Thus, $T_1 = T_2$ on $X$. Hence, $T$ is unique.
    \end{proof}
    Now, for the main proof: Since $\iota_1$ is an isometric isomorphism onto $\iota_1(X)$, then, $\iota_1^{-1}: \iota_1(X) \to X$ is also an isometric isomorphism. Thus, we can define the map $\Phi_0: \iota_1(X) \to \tilde{X}_2$ given by:
    \[ \Phi_0(x) = \iota_2 \circ \iota_1^{-1}(x)\]
    Since $\iota_2$ is an isometric isomorphism, and a composition of isometric isomorphisms is an isometric isomorphism, we have that $\Phi_0$ is an isometric isomorphism. \bbni
    Since $\iota_1(X)$ is a dense subspace of $\tilde{X}_1$, and $\tilde{X}_2$ is Banach, by Lemma 2, we can extend $\Phi_0$ uniquely to a bounded linear operator $\Phi: \tilde{X}_1 \to \tilde{X}_2$ such that $\Phi(x) = \Phi_0(x)$ for all $x \in \iota_1(X)$. \bbni
    We need to show that $\Phi$ is an isometric isomorphism. \bbni
    First, we show that $\Phi$ is an isometry. Let $x \in \tilde{X}_1$ be arbitrary. Then, there exists a Cauchy sequence $(x_n) \subset \iota_1(X)$ such that $(x_n) \to x$. Then, by the isometry of $\Phi_0$ and continuity of the norm:
    \begin{align*}
        ||\Phi(x)|| &= \norm{\lim_{n \to \infty} \Phi(x_n)} \\
        &= \norm{\lim_{n \to \infty} \Phi_0(x_n)} \\
        &= \lim_{n \to \infty} ||\Phi_0(x_n)|| \\
        &= \lim_{n \to \infty} ||x_n|| \\
        &= \norm{\lim_{n \to \infty} x_n} \\
        &= ||x|| 
    \end{align*}
    Next, we show that $\Phi$ is injective. Let $x \in \tilde{X}_1$ be such that $\Phi(x) = 0$. Then, take a Cauchy sequence $(x_n) \subset \iota_1(X)$ such that $(x_n) \to x$. Then, we have by continuity of the norm and isometry of $\Phi$, we have:
    \begin{align*}
        ||x|| = ||\Phi(x)|| = 0
    \end{align*}
    Thus, by positive definiteness of the norm, we have that $x = 0$. Thus, $\Phi$ is injective. \bbni
    Finally, we show that $\Phi$ is surjective. Let $y \in \tilde{X}_2$ be arbitrary. Then, since $\img(\Phi) = \img(\iota_2)$ is dense in $\tilde{X}_2$ , there exists a Cauchy sequence $(y_n) \to y$ such that $(y_n) \subset \img(\Phi_0)$. Then, since $\Phi_0$ is an isometric isomorphism, it has an inverse $\Phi_0^{-1}$ that is an isometric isomorphism. Then, we have a Cauchy sequence $(\Phi_0^{-1}(y_n)) \subset \iota_1(X) = \img(\Phi_0^{-1})$ converging to $x$. Since $\Phi$ is continous, we have that:
    \begin{align*}
        \Phi(x) &= \lim_{n \to \infty} \Phi(\Phi_0^{-1}(y_n)) \\
        &= \lim_{n \to \infty} y_n = y
    \end{align*}
    as $\Phi = \Phi_0$ on $\iota_1(X)$. Thus, $\Phi$ is surjective. \bbni
    Thus, we have shown that $\Phi$ is an isometric isomorphism. \bbni
\end{solution}
\newpage 

\begin{problem}{39}
    Lets find a use for a genuine Minkowski functional. In this problem, we'll let $l_\R^\infty$ be the real Banach space of bounded sequences in $\R$. Define $m$ on $l_\R^\infty$: 
    \[ m(x) = \limsup_n x_n\]
    We clearly have $m(tx) = tm(x)$ if $t \geq 0$ and it is not hard to check that $m(x+y) \leq m(x) + m(y)$ for all $x, y \in l_\R^\infty$. We want to show that there are Banach limits or what I prefer to call a generalized limit on $l_\R^\infty$. This is we want to show that there is a functional $L \in l_\R^{\infty^*}$ such that:
    \[ L(S(x)) = L(x)\] 
    where $S \in \mathcal{L}(l_\R^\infty)$ is given by $S(x)_n = x_{n+1}$ and such that $\liminf_n x_n \leq L(x) \leq \limsup_n x_n$. (Hint provided).
\end{problem}
\begin{solution} 
    \begin{lemma}
        If $x_n$ is a sequence in $\R$, and $A_n = \frac{1}{n} \sum_{i=1}^n x_i$ converge, then $\lim_n A_n \leq \limsup_n x_n$ and $\lim_n A_n \geq \liminf_n x_n$.
    \end{lemma}
    \begin{proof}
        Let $k \in \N$ be arbitrary. Then, $\forall n \geq k$ note that: 
        \begin{align*}
            A_n &= \frac{1}{n} \sum_{i=1}^k x_k + \frac{1}{n} \sum_{i = k + 1}^n x_i \\
            &\leq \frac{1}{n} \sum_{i=1}^k x_k + \sup_{i \geq k} x_i
        \end{align*}
        Taking the $\limsup$ with respect to $n$ on both sides, we have that: 
        \begin{align*}
            \limsup_n A_n &\leq \limsup_n \left(\frac{1}{n}\sum_{i=1}^k x_i + \sup_{i \geq k} x_i\right) \\
            &= 0 + \sup_{i \geq k} x_i 
        \end{align*}
        Since this is true for arbitrary $k$, we have that: 
        \[ \limsup_n A_n \leq \lim_{k\to \infty} \sup_{i\geq k} x_i = \limsup_k x_k\]
        However, since $A_n$ converges, we have that $\limsup_n A_n = \lim_n A_n$. Thus, we have shown that:
        \[ \lim_n A_n \leq \limsup_n x_n\]
        The proof for $\liminf$ follows similarly by flipping the inequalities. 
    \end{proof}
    Define $m_n(x) = \frac{1}{n}( x_1 + \ldots + x_n)$. Then define $Y = \{ x \in l^\infty_\R : \lim_n m_n(x) \text{ exists } \}$. Then, $Y$ is a subspace as for any $x, y \in Y$ and $\alpha \in \R$, if $m_n(x) \to a$ and $m_n(y) \to b$, then: 
    \[ \lim_{n \to \infty} m_n(x + \alpha y) = \lim_{n\to \infty} \frac{1}{n}\left(\sum_{i =1}^n x_i + \alpha y_i\right) = \lim_{n\to \infty}  \frac{1}{n} \sum_{i=1}^n x_i + \alpha \lim_{n\to \infty}\frac{1}{n} \sum_{i=1}^n y_i = a + \alpha b \]
    Thus, $x+\alpha y \in Y$ and $Y$ is a subspace (this also shows $m_n$ is linear). \bbni
    Define $L_0: Y \to \R$ by: 
    \[ L_0(x) = \lim_{n \to \infty} m_n(x)\]
    Then, for $y \in Y$, we use the lemma to show that:
    \[ L_0(x) \leq m(x)\]
    where $m(x) = \limsup_n x_n$. \bbni
    Thus, by the Basic Extension Lemma, we can extend $L_0$ to a linear functional $L: l^\infty_\R \to \R$ such that $L(x) = L_0(x)$ for all $x \in Y$ and $L(x) \leq m(x)$ for all $x \in l^\infty_\R$. Moreover, for all $x \in l^\infty_\R$, we have that: 
    \begin{align*}
        L(x) &= -L(-x) \\
        &\geq -m(-x) \\
        &= \limsup_n (-x_n) \\
        &= -\liminf_n x_n
    \end{align*}
    Thus, 
    \[ \liminf_n x_n \leq L(x) \leq \liminf_n x_n\]
    Next, we show that $x-S(x) \in Y$. To see this, we unpack the definition to note that:
    \[ (x-S(x))_n = x_n - x_{n+1}\]
    Thus, we get that:
    \begin{align*}
        m_n(x-S(x)) &= \frac{1}{n}( (x_1-x_2) + (x_2-x_3) + \cdots + (x_{n}-x_{n+1})) \\
        &= \frac1n (x_1 - x_{n+1})
    \end{align*}
    Taking the limit as $n \to \infty$, we have that:
    \[ \lim_{n \to \infty} m_n(x-S(x)) = 0\]
    Thus, $x-S(x) \in Y$. \bbni
    Thus, we have that: 
    \[L(x-S(x)) = 0 \implies L(x) = L(S(x))\]
    by linearity of $L$. Thus, $L(x)$ is a generalized limit of $x$. Moreover, if $x$ converges, then $\limsup_n x_n = \liminf_n x_n$, thus, by the squeeze theorem, $L(x)$ equals this limit. 
\end{solution}
\newpage 

\begin{problem}{40}
    Prove the following Lemma from lecture. Let $X$ be a complex vector space. Every real linear functional of $X$ is the real part of a complex linear functional on $X$. In fact, if $\phi = \Re(\psi)$ then $\psi(x) = \phi(x) - i\phi(ix)$. 
\end{problem}
\begin{solution} 
    Let $\phi$ be a real linear functional on $X$. We prove existence and uniqueness separately. \bbni
    \textbf{Existence:} Let $\psi: X \to \C$ be given by: 
    \[ \psi(x) = \phi(x) - i\phi(x)\]
    Clearly, $\phi(x) = \Re(\psi)$. We need to show that $\psi$ is complex linear. \bbni
    Let $x, y \in X$ and $\alpha \in \C$ be arbitrary. Then, we have that:
    \begin{align*}
        \psi(\alpha x + y ) &= \phi(\alpha x + y) - i \phi(\alpha x + y) \\
        &= \alpha\phi(x)-i\alpha\phi(x) + \phi(y) - i\phi(y) \\
        &= \alpha(\phi(x) - i\phi(x)) + (\phi(y) - i\phi(y)) \\
        &= \alpha\psi(x) + \psi(y)
    \end{align*}
    Thus, $\psi$ is complex linear. \bbni
    \textbf{Uniqueness:} Let $\psi$ be a linear functional such that $\phi = \Re(\psi)$. Then, we need to show that $\psi(x) = \phi(x) - i\phi(ix)$. \\
    Let $g(x) = \Im(\psi)$. Then, since $\psi$ is complex linear, we have that: 
    \begin{align*}
        \phi(ix) + ig(ix) &= \psi(ix) \\ 
        &= i\psi(x) \\
        &= i\phi(x) - g(x)
    \end{align*}
    Then, comparing real parts, we have that: 
    \[ g(x) = -\phi(ix)\]
    Thus, 
    \[ \psi(x) = \phi(x) - i\phi(ix)\]
    Thus, every real linear function $\phi$ is the real part of a unique complex linear functional $\psi = \phi(x) -i\phi(ix)$ on $X$.
\end{solution}
\newpage 

\begin{problem}{41}
    Suppose that $X$ is a normed vector space such that $X^*$ is seperable. Show that $X$ is seperable. (Hint provided).
\end{problem}
\begin{solution} 
    Since $X^*$ is separable, there exists a countable dense subset $\{f_n\}_{n \in \N} \subset X^*$. Then, for each $n \in \N$, note that: 
    \[ ||f_n|| = \inf\{a : |f_n(x)| \leq a||x|| \quad \forall x \in X\}\]
    Noting that norms are non-negative and homogeneous, and $f_n$ is linear, we divide through by $||x||$ to get: 
    \[ ||f_n|| = \inf\{a : |f_n(x)| \leq a \quad \forall x \in X, ||x|| = 1\}\]
    Thus, for $\frac{1}{2}||f_n|| > 0$, we can find $x_n$, with $||x_n||=1$ such that:
    \[|f_n(x'_n)| \geq ||f_n|| - \frac{1}{2}||f_n|| = \frac{1}{2} ||f_n|| \] 
    Define $S = \opspan\{x_n : n \in \N\}$. Clearly, $S$ is a subspace. We claim that $S$ is dense in $X$. \bbni
    For contradiction, assume that $S$ is not dense in $X$. Then, there exists a (non-empty) open set that $S$ does not meet, thus $\overline{S} \neq X$ (we use the definition of the closure of $A$ to contain all points in $X$ such that any open set around that point meets $A$). Thus, $\overline{S}$ is a proper closed subspace of $X$. \bbni
    Then, by a corollary of the Hahn-Banach Theorem, we can find a functional $f \in X^*$ such that $f(x) = 0$ for all $x \in \overline{S}$ and $||f|| = 1$. We show that this contradicts the denseness of $\{f_n\}_{n \in \N}$. \bbni
    For all $n \in \N$, we have the following cases: 
    \begin{itemize}
        \item Assume $||f_n|| \geq \frac{1}{2}$. Then, for $x_n$, with $||x_n|| = 1$, we have:
        \[|f_n(x_n)-f(x_n)| = |f_n(x_n)| \geq \frac12||f_n|| \geq \frac{1}{4}\]
        Thus, $||f_n-f|| \geq 1/4$ is bounded away from $0$.
        \item Assume $||f_n|| < \frac{1}{2}$. Then, 
        \begin{align*}
            ||f|| &= ||f-f_n+f_n||  \\
            &\leq ||f-f_n|| + ||f_n||
        \end{align*}
        Thus, 
        \begin{align*}
            ||f-f_n|| \geq ||f|| - ||f_n|| \geq 1-\frac12 = \frac12
        \end{align*}        
        Thus, $||f_n-f|| \geq 1/2$ is bounded away from $0$. 
    \end{itemize}
    Thus, there exists an open neighborhood of $f$ (or radius $< \frac14$) that does not meet $\{f_n\}_{n\in\N}$, which is a contradiction since this set is dense in $X^*$.\bbni
    Thus, $\opspan\{x_n : n \in \N\}$ is dense in $X$. Moreover, since $\opspan_\Q\{x_n : n \in \N\}$ is a countable dense subset of $\opspan\{x_n : n \in \N\}$, $X$ is separable (we made a remark about this in class).
\end{solution}
\newpage 

\end{document}