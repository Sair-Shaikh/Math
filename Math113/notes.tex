\documentclass[12pt]{article}

\usepackage{fullpage}
\usepackage{mdframed}
\usepackage{colonequals}
\usepackage{algpseudocode}
\usepackage{algorithm}
\usepackage{tcolorbox}
\usepackage[all]{xy}
\usepackage{proof}
\usepackage{mathtools}
\usepackage{bbm}
\usepackage{amssymb}
\usepackage{amsthm}
\usepackage{amsmath}
\usepackage{amsxtra}
\newcommand{\bb}{\mathbb}


\newtheorem{theorem}{Theorem}[section]
\newtheorem{corollary}{Corollary}[theorem]
\newtheorem{lemma}{Lemma}

\newcommand{\mathcat}[1]{\textup{\textbf{\textsf{#1}}}} % for defined terms

\newenvironment{problem}[1]
{\begin{tcolorbox}\noindent\textbf{Problem #1}.}
{\vskip 6pt \end{tcolorbox}}

\newenvironment{enumalph}
{\begin{enumerate}\renewcommand{\labelenumi}{\textnormal{(\alph{enumi})}}}
{\end{enumerate}}

\newenvironment{enumroman}
{\begin{enumerate}\renewcommand{\labelenumi}{\textnormal{(\roman{enumi})}}}
{\end{enumerate}}

\newcommand{\defi}[1]{\textsf{#1}} % for defined terms

\theoremstyle{remark}
\newtheorem*{solution}{Solution}

\setlength{\hfuzz}{4pt}

\newcommand{\calC}{\mathcal{C}}
\newcommand{\calF}{\mathcal{F}}
\newcommand{\C}{\mathbb C}
\newcommand{\N}{\mathbb N}
\newcommand{\Q}{\mathbb Q}
\newcommand{\R}{\mathbb R}
\newcommand{\Z}{\mathbb Z}
\newcommand{\F}{\mathbb F}
\newcommand{\br}{\mathbf{r}}
\newcommand{\RP}{\mathbb{RP}}
\newcommand{\CP}{\mathbb{CP}}
\newcommand{\nbit}[1]{\{0, 1\}^{#1}}
\newcommand{\bits}{\{0, 1\}^{n}}
\newcommand{\bbni}{\bigbreak \noindent}
\newcommand{\norm}[1]{\left\vert\left\vert#1\right\vert\right\vert}
\newcommand{\dbar}{\overline{\partial}}
\let\d\relax
\let\calF\relax
\newcommand{\d}{\partial}
\newcommand{\calO}{\mathcal{O}}
\newcommand{\calF}{\mathcal{F}}
\newcommand{\calG}{\mathcal{G}}
\newcommand{\calH}{\mathcal{H}}
\newcommand{\calE}{\mathcal{E}}

\let\1\relax
\newcommand{\1}{\mathbf{1}}
\newcommand{\fr}[2]{\left(\frac{#1}{#2}\right)}

\newcommand{\vecz}{\mathbf{z}}
\newcommand{\vecr}{\mathbf{r}}
\DeclareMathOperator{\Cinf}{C^{\infty}}
\DeclareMathOperator{\Id}{Id}

\DeclareMathOperator{\Alt}{Alt}
\DeclareMathOperator{\ann}{ann}
\DeclareMathOperator{\codim}{codim}
\DeclareMathOperator{\End}{End}
\DeclareMathOperator{\Hom}{Hom}
\DeclareMathOperator{\id}{id}
\DeclareMathOperator{\M}{M}
\DeclareMathOperator{\Mat}{Mat}
\DeclareMathOperator{\Ob}{Ob}
\DeclareMathOperator{\opchar}{char}
\DeclareMathOperator{\opspan}{span}
\DeclareMathOperator{\rk}{rk}
\DeclareMathOperator{\sgn}{sgn}
\DeclareMathOperator{\Sym}{Sym}
\DeclareMathOperator{\tr}{tr}
\DeclareMathOperator{\img}{img}
\DeclareMathOperator{\CandE}{CandE}
\DeclareMathOperator{\CandO}{CandO}
\DeclareMathOperator{\argmax}{argmax}
\DeclareMathOperator{\first}{first}
\DeclareMathOperator{\last}{last}
\DeclareMathOperator{\cost}{cost}
\DeclareMathOperator{\dist}{dist}
\DeclareMathOperator{\path}{path}
\DeclareMathOperator{\parent}{parent}
\DeclareMathOperator{\argmin}{argmin}
\DeclareMathOperator{\excess}{excess}
\let\Pr\relax
\DeclareMathOperator{\Pr}{\mathbf{Pr}}
\DeclareMathOperator{\Exp}{\mathbb{E}}
\DeclareMathOperator{\Var}{\mathbf{Var}}
\let\limsup\relax
\DeclareMathOperator{\limsup}{limsup}
%Paired Delims
\DeclarePairedDelimiter\ceil{\lceil}{\rceil}
\DeclarePairedDelimiter\floor{\lfloor}{ \rfloor}


\newcommand{\dagstar}{*}

\newcommand{\tbigwedge}{{\textstyle{\bigwedge}}}
\setlength{\parindent}{0pt}
\setlength{\parskip}{5pt}


\begin{document}

\title{CS 40: Computational Complexity}

\author{Sair Shaikh}
\maketitle

Collaboration Notice: Talked to Henry Scheible '26 to discuss ideas.


\begin{itemize}
    \item[Defn.] Let $\{U_\alpha\}_{\alpha \in A}$ be an open cover of $(X, \rho)$. We say that $d > 0$ is a Lebesgue number for the cover if given any $d$-ball $B_d(x_0)$ with $x_0 \in X$, there exists $a_0 \in A$ such that $B_d(x_0) \subseteq U_{a_0}$. 
    \item[Ex] $X = \R$. $U_1 = (-\infty, 1)$, $U_2 = (0, 2)$, and $U_3 = (1, \infty)$. Here $d = 1/2$ is a LN for $\{U_1, U_2, U_3\}$. This is clear if $x_0 \in (1/2, 3/2)$. 
    \item[Ex. (Hwk.)] Given $x \in (0, 1)$, $\exists \delta_x > 0$ such that:
    \[ y \in B_{\delta_x}(x) = \{y \in (0, 1): |y-x| \leq \delta_x\} \]
    \[ \implies |1/x - 1/y| < 1 \]
    Then, 
    \[ (0, 1) = \bigcup_{x \in (0, 1)} B_{\delta_x}(x) \]
    has no Lebesgue number.
    \item[Lemma] (Lebesgue Covering Lemma) Every open cover of a compact metric space has a Lebesgue number. \\ 
    Proof. Pictures. Apr 9.
    \item[Thm.] Suppose $(X, \rho)$ is compact, and $F: (X, \rho) \to (Y, \sigma)$ is continuous. Then $F$ is uniformly continuous.
    % Proof. Let $\epsilon > 0$. We need to find $\delta > 0$ such that $\forall x, y \in X$: 
    % \[ \rho(x, y) < \delta \implies \sigma(F(x), F(y)) < \epsilon \]
    % Since $F$ is continous, $\forall z \in X$, $\exists \delta_z > 0$ such that:
    % \[ \rho(x, z) < \delta_z \implies \sigma(F(x), F(z)) < \epsilon/2 \]
    % That is, 
    % \[ F(B_{\delta_z}(z)) \subseteq B_{\epsilon/2}(F(z)) \]
    % Let $\delta > 0$ be a Lebesgue number for the cover $\{B_{\delta_z}(z)\}_{z \in X}$. \\
    % Now supposed $\rho(x,y) < \delta$. Then $\exists z \in X$ such that:
    % \[ B_\delta(x) \subseteq B_{\delta_z}(z) \]
    % and 
    % \[\sigma(F(x), F(y)) \leq \sigma(F(x), F(y)) + \sigma(F(z), F(y)) < \epsilon/2 + \epsilon/2 = \epsilon \]
    \item[Defn. ] Let $(X, \rho)$ be a metric space and $C(X)$ the $\C$-vector space of continous functions on $X$. We say $\mathcal{J} \subset C$ is equicontinous at $x \in X$ if $\forall \epsilon > 0, \exists \delta > 0$ such that:
    \[ \forall F \in \mathcal{J}, F(B_\delta(x)) \subseteq B_\epsilon(F(x)) \]
    We say $\mathcal{J}$ is equicontinous on $X$ if $\forall x \in X$, $\mathcal{J}$ is equicontinous at $x$.
    \item[Ex.] Let $X = [0, 1] \subset \R$. Let $F_n(x) = x^n \forall n \geq 1$. Let:
    \[\mathcal{J} = \{F_n: n \in \mathbb{N}\}\]
    Let $x_n = \frac{1}{2}^{1/n}$. Then, $x_n$ arrow up to $1$. Then, 
    \[ |F_n(x_n) - F_n(1)| = |1/2 -1| = 1/2\]
    Thus, $\mathcal{J}$ is not equicontinous at $1$.
    \item[Ex. (Hwk)] Show that $\mathcal{J}$ is equicontinous on $[0, 1)$. 
    \item[Defn.] Let $(F_n)$ be a sequence of ($\C$-valued) functions on $X$. Then, $(F_n)$ is uniformly bounded if $\exists M > 0$ such that $\forall n \geq 1, \forall x \in X$:
    \[ |F_n(x)| < M \]
    We say that $(F_n)$ is pointwise bounded if $\forall x \in X$, $\exists M_x > 0$ such that:
    \[ |F_n(x)| < M_x \]
    \item[Defn.] A metric space (top. space) is seperable if there is a countable dense subset $D \subset X$. 
    \item[Ex.] Since $\Q^n \subset \R^n$ is dense, $(\R^n, ||\cdot||_p)$ is separable.  
    \item[Lemma.] (Arzelà-Ascoli) Let $(X, \rho)$ is a seperable metric space and that $(F_n)$ is pointwise bounded and equicontinous in $C(X)$. Then, there is subsequence $(F_{n_k})$ such that:
        \[ \lim_{x\to\infty} F_{n_k}(x)\]
    exists $\forall x \in X$.
    \item[Yap.] Given a sequence $(x_n)$, we get subsequence by finding $n_k \in \mathbb{N}$ such that $n_{k+1} > n_k$ and $(x_{n_k})_{k=1}^\infty \to x$ is a sequence. \\
    A subsubsequence is determined by finding $n_{k_1} < n_{k_2} < \cdots$ and then we write: 
    \[ (x_{n_{k_j}})_{j=1}^\infty\]
    A subsequence is determined by an infinite subset $S_1 = \{n_1 \leq n_2 \leq \cdots\} \subset \mathbb{N}$. A subsubsequence is determined by an infinite subset $S_2 \subset S_1$, 
    \[ S_2 = \{n_{k_1} < n_{k_2} < \cdots \} \subset S_1\]
    Now, we write: 
    \[ \lim_{n \in S_1} x_n = a \text{ instead of } \lim_{k \to \infty} x_{n_k}\]
    Note that $\lim_{n \in S_1} x_n = a$ if:
    \[ \forall \epsilon > 0\, \exists N: n \geq N, n \in S' \implies |x_n - a| < \epsilon\] 
    \item[Rmk.] Suppose $S_1 \subset \mathbb{N}$ determines a subsequence as above. Suppose $S' \subset \mathbb{N}$ is infinite and:
    \[\{n \in S' : n \not \in S_1\}\]
    is finite. The $\lim_{n \in S_1} x_n = a$ then $\lim_{n \in S'} x_n = a$ as well. \\
    Proof of the AA Lemma: Pictures. Apr 9 and 10.
    \item[Rmk.] If $X$ is compact, then $C(x) = C_b(X)$ is a complete metric space with respect to the uniform norm $||\cdot||_\infty$. 
    \item[Thm.] (Arzelà-Ascoli) Let $(X, \rho)$ be a compact metric space and $(F_n) \subset C(X)$ be a sequence of functions that are point-wise bounded and equicontinous. Then $(F_n)$ has a subsequence converging uniformly to some function $F \in C(X)$.
    Proof. Pictures.
    \item[Lemma.] Suppose $X$ is compact and that $\mathcal{J} \subset C(X)$ is equicontinous on $X$. Then, $\mathcal{J}$ is uniformly equicontinous on $X$, in that for all $\epsilon > 0 \exists \delta > 0$ such that for all $x, y \in X$ and all $F  \in \mathcal{J}$, 
        \[ \rho(x,y) < \delta \implies |F(x) - F(y)| < \epsilon \]
    Rewriting, 
    \[ F(B_\delta(x)) \subseteq B_\epsilon(F(x))\]
    Proof left as homework.  
    \item[Corr.] Let $X$ be a compact metric space. Let $\mathcal{J} \subset C(X)$ be a closed subset such that $\mathcal{J}$ is equicontinous and pointwise bounded. Then $\mathcal{J}$ is compact and uniformly bounded.
    \item[Thm.] Suppose $X$ is a compact metric space. Then $\mathcal{J} \subset C(X)$ is compact if and only if $\mathcal{J}$ is closed, uniformly bounded, and equicontinous on $X$.
    Proof. Pictures.   
\end{itemize}



\end{document}