% \documentclass[12pt]{amsart}
\documentclass[12pt]{article}

\usepackage{fullpage}
\usepackage{mdframed}
\usepackage{colonequals}
\usepackage{algpseudocode}
\usepackage{algorithm}
\usepackage{tcolorbox}
\usepackage[all]{xy}
\usepackage{proof}
\usepackage{mathtools}
\usepackage{bbm}
\usepackage{amssymb}
\usepackage{amsthm}
\usepackage{amsmath}
\usepackage{amsxtra}
\newcommand{\bb}{\mathbb}


\newtheorem{theorem}{Theorem}[section]
\newtheorem{corollary}{Corollary}[theorem]
\newtheorem{lemma}{Lemma}

\newcommand{\mathcat}[1]{\textup{\textbf{\textsf{#1}}}} % for defined terms

\newenvironment{problem}[1]
{\begin{tcolorbox}\noindent\textbf{Problem #1}.}
{\vskip 6pt \end{tcolorbox}}

\newenvironment{enumalph}
{\begin{enumerate}\renewcommand{\labelenumi}{\textnormal{(\alph{enumi})}}}
{\end{enumerate}}

\newenvironment{enumroman}
{\begin{enumerate}\renewcommand{\labelenumi}{\textnormal{(\roman{enumi})}}}
{\end{enumerate}}

\newcommand{\defi}[1]{\textsf{#1}} % for defined terms

\theoremstyle{remark}
\newtheorem*{solution}{Solution}

\setlength{\hfuzz}{4pt}

\newcommand{\calC}{\mathcal{C}}
\newcommand{\calF}{\mathcal{F}}
\newcommand{\C}{\mathbb C}
\newcommand{\N}{\mathbb N}
\newcommand{\Q}{\mathbb Q}
\newcommand{\R}{\mathbb R}
\newcommand{\Z}{\mathbb Z}
\newcommand{\br}{\mathbf{r}}
\newcommand{\RP}{\mathbb{RP}}
\newcommand{\CP}{\mathbb{CP}}
\newcommand{\nbit}[1]{\{0, 1\}^{#1}}
\newcommand{\bits}{\{0, 1\}^{n}}
\newcommand{\bbni}{\bigbreak \noindent}
\newcommand{\norm}[1]{\left\vert\left\vert#1\right\vert\right\vert}

\let\1\relax
\newcommand{\1}{\mathbf{1}}
\newcommand{\fr}[2]{\left(\frac{#1}{#2}\right)}

\newcommand{\vecz}{\mathbf{z}}
\newcommand{\vecr}{\mathbf{r}}
\DeclareMathOperator{\Cinf}{C^{\infty}}
\DeclareMathOperator{\Id}{Id}

\DeclareMathOperator{\Alt}{Alt}
\DeclareMathOperator{\ann}{ann}
\DeclareMathOperator{\codim}{codim}
\DeclareMathOperator{\End}{End}
\DeclareMathOperator{\Hom}{Hom}
\DeclareMathOperator{\id}{id}
\DeclareMathOperator{\M}{M}
\DeclareMathOperator{\Mat}{Mat}
\DeclareMathOperator{\Ob}{Ob}
\DeclareMathOperator{\opchar}{char}
\DeclareMathOperator{\opspan}{span}
\DeclareMathOperator{\rk}{rk}
\DeclareMathOperator{\sgn}{sgn}
\DeclareMathOperator{\Sym}{Sym}
\DeclareMathOperator{\tr}{tr}
\DeclareMathOperator{\img}{img}
\DeclareMathOperator{\CandE}{CandE}
\DeclareMathOperator{\CandO}{CandO}
\DeclareMathOperator{\argmax}{argmax}
\DeclareMathOperator{\first}{first}
\DeclareMathOperator{\last}{last}
\DeclareMathOperator{\cost}{cost}
\DeclareMathOperator{\dist}{dist}
\DeclareMathOperator{\path}{path}
\DeclareMathOperator{\parent}{parent}
\DeclareMathOperator{\argmin}{argmin}
\DeclareMathOperator{\excess}{excess}
\let\Pr\relax
\DeclareMathOperator{\Pr}{\mathbf{Pr}}
\DeclareMathOperator{\Exp}{\mathbb{E}}
\DeclareMathOperator{\Var}{\mathbf{Var}}
\let\limsup\relax
\DeclareMathOperator{\limsup}{limsup}
%Paired Delims
\DeclarePairedDelimiter\ceil{\lceil}{\rceil}
\DeclarePairedDelimiter\floor{\lfloor}{ \rfloor}


\newcommand{\dagstar}{*}

\newcommand{\tbigwedge}{{\textstyle{\bigwedge}}}
\setlength{\parindent}{0pt}
\setlength{\parskip}{5pt}


\begin{document}

\title{CS 40: Computational Complexity}

\author{Sair Shaikh}
\maketitle

% Collaboration Notice: Talked to Henry Scheible '26 to discuss ideas.


\begin{problab}{1}
    Let $p: \R^{n+1} \setminus \{0\} \to \bb{RP}^n$ be the quotient map from HW1. For $n \geq 2$, show that $p|_{S^n}$ is a degree two cover and deduce that $\pi_1(\bb{RP}^n) \simeq \bb{Z}/2\bb{Z}$. (The fundamental group is $\bb{Z}$ for $n=1$ as $S^1 \cong \bb{RP}^1$.)
\end{problab}
\begin{solu}
    Recall the quotient map identifies points in $\R^{n+1} \setminus \{0\}$ as follows: 
    \[ \forall \lambda \neq 0 \in \R: \, (x_0, \cdots, x_n) \sim \lambda (x_0, \cdots, x_n)\]
    Let $(x_0, \cdots, x_n) \in S^n$ be a point on the sphere. Then, we additionally know that:
    \[ x_0^2 + x_1^2 + \cdots + x_n^2 = 1\]
    The equivalence class of this point on the sphere contains all points such that $\lambda(x_0, \cdots, x_n)$ such that:
    \[ (\lambda x_0)^2 + \cdots + (\lambda x_n)^2 = \lambda^2 (x_0^2 + \cdots + x_n^2) = 1\]
    Thus, we have $\lambda = \pm 1$. Thus, restricting $p$ to $S^n$, we get a quotient map that identifies antipodal points on the sphere, i.e. $x \sim -x$. \\
    Thus, each $[p] \in \RP^n$ has exactly two pre-images in $S^n$, i.e. $p|_{S^n}$ is a two to one mapping. \bbni
    Moreover, let $[x] \in \RP^n$ be a point. Then, $p|_{S^n}^{-1}([x]) = \{x, -x\}$. Let $V$ and $V'$ be two disjoint open sets in $S^n$ around $x$ and $-x$ ($S^n$ is Hausdorff). Let $W = V \cap -V'$ and $W' = V' \cap -V$. Then, $W$ and $W'$ are still disjoint. Moreover, $W$ and $W'$ are such that they contain antipodal points (by definition). \\
    Let $U = p(W)$. By the definition of $p$, we note that:
    \[ \rho|_{S^n}^{-1}(U) = W \bigsqcup W'\]
    Since $W$ and $W'$ are open, $U$ is open in $\RP^n$. Moreover, $p|_W(W) = U$ is a continous bijection of compact Hausdroff spaces, and is thus a homeomorphism. Similarly for $W'$. Thus, as $[x]$ was arbitrary, we have found an open set $U$ around $x$ whose pre-image is a disjoint union of two isomorphic copies of $U$. Thus, $p|_{S^n}$ is a degree $2$ covering map. \bbni 
    Moreover, we note that for $n \geq 2$, $S^n$ is simply connected. Thus, for any point $x \in \RP^n$, the map: 
    \[ \pi_1(\RP^n, x) \to p|_{S^n}^{-1}(x)\]
    is a bijection. Thus, $|\pi_1(\RP^n, x)| = 2$. Thus, 
    \[ \pi_1(\RP^n, x) = \Z/2\Z\]

\end{solu}
\newpage

\begin{problab}{2}
    Let $g\colon S^1 \to S^1$ be the covering map $g(z) = z^n$ for $n \in \bb{Z}$. Under the isomorphism $\pi_1(S^1, (1,0)) \cong \bb{Z}$ that we proved, compute $g_* \colon \bb{Z} \to \bb{Z}$ and the map $\phi_g: \bb{Z} \to g^{-1}(1,0)$ defined by lifting loops so that they still start at $(1,0)$.
\end{problab}
\begin{solu}

\end{solu}
\newpage

\begin{problab}{3}
    Show that there are no retractions $r \colon X \to A$ in the following cases:
    \begin{enumerate}
    \item $X = \R^3$ and $A$ is any subspace homeomorphic to $S^1$.
    \item $X = S^1 \times D^2$ and $A$ is its boundary torus $S^1 \times S^1$.
    \item $X$ is the M{\" o}bius band and $A$ is its boundary circle.
    \end{enumerate}
\end{problab}
\begin{solu}

\end{solu}
\newpage

\begin{problab}{4}
    Use the intermediate value theorem to prove the $1$-dimensional version of the Brouwer fixed point theorem: If $f \colon I \to I$ is continuous, there is a point $x \in I$ such that $f(x) = x$. 
\end{problab}
\begin{solu}

\end{solu}
\newpage

\begin{problab}{5}
    Use the intermediate value theorem to prove the $1$-dimensional version of the Borsuk-Ulam theorem: If $f: S^1 \to \R$ is continuous, there is a point $x \in S^1$ such that $f(x) = f(-x)$.
\end{problab}
\begin{solu}

\end{solu}

\end{document}