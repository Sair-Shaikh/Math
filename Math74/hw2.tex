\begin{problab}[1]
    (0.10) A space is called contractible if the identity map is nullhomotopic. Show that \( X \) is contractible if and only if for every space \( Y \), every map \( f : X \to Y \) is nullhomotopic. Similarly, show that \( X \) is contractible if and only if for any \( Y \), every map \( f : Y \to X \) is nullhomotopic.
    \end{problab}
    
    \begin{problab}[2]
    Let \( A \subset X \) and suppose \( r : X \to A \) is a continuous map such that \( r(a) = a \) for all \( a \in A \) (i.e., \( r \) is a retraction of \( X \) onto \( A \)). If \( a_0 \in A \), show that
    \[
    r_* : \pi_1(X, a_0) \to \pi_1(A, a_0)
    \]
    is surjective. (Hint: Consider also the inclusion map of \( A \) into \( X \).)
    \end{problab}
    
    \begin{problab}[3]
    (1.1.3) If \( X \) is a path-connected space, show that \( \pi_1(X) \) is abelian if and only if every change-of-basepoint isomorphism (\( \hat{\alpha} \) in class or \( \beta_h \) in Hatcher) depends only on the endpoints of the path.
    \end{problab}
    
    \begin{problab}[4]
    (1.1.6) Note that a loop based at \( x_0 \) can be regarded as a continuous map of pointed spaces
    \[
    (S^1, (1, 0)) \to (X, x_0).
    \]
    Let \( [S^1, X] \) be the set of homotopy classes of maps from \( S^1 \) to \( X \) without conditions on basepoints. Then, there is a map \( \Phi : \pi_1(X, x_0) \to [S^1, X] \) that forgets the base points.
    \begin{itemize}
      \item[(a)] Show that \( \Phi([f]) = \Phi([g]) \) if and only if \( [f] \) and \( [g] \) are conjugate in \( \pi_1(X, x_0) \).
      \item[(b)] Deduce that if \( X \) is path connected, then \( [S^1, X] \) is in bijection with conjugacy classes of \( \pi_1(X, x_0) \).
    \end{itemize}
    \end{problab}
    
    \begin{problab}[5]
    Suppose that \( p : E \to B \) is a covering map where \( B \) is connected. Show that if \( p^{-1}(b_0) \) has \( k \) elements for some \( b_0 \in B \), then \( p^{-1}(b) \) has \( k \) elements for every \( b \in B \).
    \end{problab}
    
    \begin{problab}[6]
    Let \( q : X \to Y \) and \( r : Y \to Z \) be covering maps such that \( r \) has finite degree. Show that \( p = r \circ q \) is a covering map.
    \end{problab}