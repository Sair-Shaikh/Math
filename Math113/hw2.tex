\documentclass[12pt]{article}

\usepackage{fullpage}
\usepackage{mdframed}
\usepackage{colonequals}
\usepackage{algpseudocode}
\usepackage{algorithm}
\usepackage{tcolorbox}
\usepackage[all]{xy}
\usepackage{proof}
\usepackage{mathtools}
\usepackage{bbm}
\usepackage{amssymb}
\usepackage{amsthm}
\usepackage{amsmath}
\usepackage{amsxtra}
\newcommand{\bb}{\mathbb}


\newtheorem{theorem}{Theorem}[section]
\newtheorem{corollary}{Corollary}[theorem]
\newtheorem{lemma}{Lemma}

\newcommand{\mathcat}[1]{\textup{\textbf{\textsf{#1}}}} % for defined terms

\newenvironment{problem}[1]
{\begin{tcolorbox}\noindent\textbf{Problem #1}.}
{\vskip 6pt \end{tcolorbox}}

\newenvironment{enumalph}
{\begin{enumerate}\renewcommand{\labelenumi}{\textnormal{(\alph{enumi})}}}
{\end{enumerate}}

\newenvironment{enumroman}
{\begin{enumerate}\renewcommand{\labelenumi}{\textnormal{(\roman{enumi})}}}
{\end{enumerate}}

\newcommand{\defi}[1]{\textsf{#1}} % for defined terms

\theoremstyle{remark}
\newtheorem*{solution}{Solution}

\setlength{\hfuzz}{4pt}

\newcommand{\calC}{\mathcal{C}}
\newcommand{\calF}{\mathcal{F}}
\newcommand{\C}{\mathbb C}
\newcommand{\N}{\mathbb N}
\newcommand{\Q}{\mathbb Q}
\newcommand{\R}{\mathbb R}
\newcommand{\Z}{\mathbb Z}
\newcommand{\br}{\mathbf{r}}
\newcommand{\RP}{\mathbb{RP}}
\newcommand{\CP}{\mathbb{CP}}
\newcommand{\nbit}[1]{\{0, 1\}^{#1}}
\newcommand{\bits}{\{0, 1\}^{n}}
\newcommand{\bbni}{\bigbreak \noindent}
\newcommand{\norm}[1]{\left\vert\left\vert#1\right\vert\right\vert}

\let\1\relax
\newcommand{\1}{\mathbf{1}}
\newcommand{\fr}[2]{\left(\frac{#1}{#2}\right)}

\newcommand{\vecz}{\mathbf{z}}
\newcommand{\vecr}{\mathbf{r}}
\DeclareMathOperator{\Cinf}{C^{\infty}}
\DeclareMathOperator{\Id}{Id}

\DeclareMathOperator{\Alt}{Alt}
\DeclareMathOperator{\ann}{ann}
\DeclareMathOperator{\codim}{codim}
\DeclareMathOperator{\End}{End}
\DeclareMathOperator{\Hom}{Hom}
\DeclareMathOperator{\id}{id}
\DeclareMathOperator{\M}{M}
\DeclareMathOperator{\Mat}{Mat}
\DeclareMathOperator{\Ob}{Ob}
\DeclareMathOperator{\opchar}{char}
\DeclareMathOperator{\opspan}{span}
\DeclareMathOperator{\rk}{rk}
\DeclareMathOperator{\sgn}{sgn}
\DeclareMathOperator{\Sym}{Sym}
\DeclareMathOperator{\tr}{tr}
\DeclareMathOperator{\img}{img}
\DeclareMathOperator{\CandE}{CandE}
\DeclareMathOperator{\CandO}{CandO}
\DeclareMathOperator{\argmax}{argmax}
\DeclareMathOperator{\first}{first}
\DeclareMathOperator{\last}{last}
\DeclareMathOperator{\cost}{cost}
\DeclareMathOperator{\dist}{dist}
\DeclareMathOperator{\path}{path}
\DeclareMathOperator{\parent}{parent}
\DeclareMathOperator{\argmin}{argmin}
\DeclareMathOperator{\excess}{excess}
\let\Pr\relax
\DeclareMathOperator{\Pr}{\mathbf{Pr}}
\DeclareMathOperator{\Exp}{\mathbb{E}}
\DeclareMathOperator{\Var}{\mathbf{Var}}
\let\limsup\relax
\DeclareMathOperator{\limsup}{limsup}
%Paired Delims
\DeclarePairedDelimiter\ceil{\lceil}{\rceil}
\DeclarePairedDelimiter\floor{\lfloor}{ \rfloor}


\newcommand{\dagstar}{*}

\newcommand{\tbigwedge}{{\textstyle{\bigwedge}}}
\setlength{\parindent}{0pt}
\setlength{\parskip}{5pt}


\begin{document}

\title{CS 40: Computational Complexity}

\author{Sair Shaikh}
\maketitle

% Collaboration Notice: Talked to Henry Scheible '26 to discuss ideas.


\begin{problab}{1}
    Show that $X$ is compact if and only if given any family $\mathcal{F}$ of closed sets of $X$ with the finite intersection property, we have $\bigcap_{F \in \mathcal{F}} F \neq \emptyset$.
\end{problab}
\begin{solu}
    \bbni
    \begin{itemize}
        \item[$(\implies$)] Assume $X$ is compact. We will prove that contrapositive. Let $\{F_\alpha\}$ be a family of closed sets such that: 
        \[ \bigcap_{\alpha} F_\alpha = \emptyset \]
        Let $\{U_\alpha\}$ be a family of open sets such that $U_\alpha = X \setminus F_\alpha$. Then, we have:
        \[ \bigcup_\alpha U_\alpha = X \setminus \bigcap_\alpha F_\alpha = X\]
        Thus, $\{U_\alpha\}$ is an open cover of $X$. Since $X$ is compact, there exists a finite subcover of $\{U_\alpha\}$. Let $\{U_1, \ldots, U_n\}$ be the finite subcover. Then, we have:
        \[ \bigcap_{i=1}^n F_i = X \setminus \bigcap_{i=1}^n U_i = \emptyset\]
        Thus, we have found a finite subset of $\{F_\alpha\}$ with empty intersection. Thus, $\{F_\alpha\}$ does not have the finite intersection property. Therefore, by the contrapositive, any family of closed sets with the finite intersection property has non-empty intersection.
        \item[($\impliedby$)] Assume any family of closed sets with the finite intersection property has non-empty intersection. By the contrapositive, this implies that if the family has empty intersection, it cannot have the finite intersection property. \bbni 
        Let $\{U_\alpha\}$ be an open cover of $X$. To show that $X$ is compact it suffices to show that there exists a finite subcover of $\{U_\alpha\}$. Let $\{F_\alpha\}$ be the family of closed sets such that $F_\alpha = X \setminus U_\alpha$. Then, 
        \[ \bigcap_{\alpha} F_\alpha = X \setminus \bigcup_\alpha U_\alpha  = \emptyset \]
        Thus, $\{F_\alpha\}$ does not have the finite intersection property. Therefore, there exists a finite subset of $\{F_\alpha\}$, call it $\{F_1, \ldots, F_n\}$, such that: 
        \[ \bigcap_{i = 1}^n F_i = \emptyset \]
        However, we know that:
        \[ \bigcap_{i = 1}^n F_i = X \setminus \bigcup_{i=1}^n U_i \]
        Thus, we conclude that:
        \[X \setminus \bigcup_{i=1}^n U_i = \emptyset\]
        Thus, $\bigcup_{i=1}^n U_i = X$. Therefore, $\{U_1, \ldots, U_n\}$ is a finite subcover of $\{U_\alpha\}$. Therefore, $X$ is compact.
    \end{itemize}
\end{solu}

\newpage

\begin{problab}{2}
    Let $X$ be a metric space. 
    \begin{enumerate}
        \item Show that if $E$ is a compact subspace of $X$, then $E$ is closed. 
        \item Show that if $X$ is compact and $E$ is closed in $X$, then $E$ is compact.
    \end{enumerate}
\end{problab}
\begin{solu}
    \bbni
    \begin{enumerate}
        \item Let $E$ be a compact subspace of $X$. To show that $E$ is closed, it suffices to show that $X \setminus E$ is open. Let $x \in X \setminus E$. Then, we define to families of open sets, $\{U_e\}_{e \in E}$ and $\{V_e\}_{e \in E}$, such that:
        \begin{align*}
            U_e &= B_{\epsilon}(e) \text{ where } \epsilon < \rho(x,e)/2 \\
            V_e &= B_{\epsilon}(x) \text{ where } \epsilon < \rho(x,e)/2
        \end{align*} 
        Note that by definition, $x \in V_e$ for all $e \in E$ and that $U_e \cap V_e = \emptyset$. Moreover, by definition, $\{U_e\}_{e\in E}$ is an open cover of $E$. Since $E$ is compact, there exists a finite subcover of $\{U_e\}_{e \in E}$. Let $\{U_{e_1}, \ldots, U_{e_n}\}$ be the finite subcover. Then, since $\{V_{e_1}, \ldots, V_{e_n}\}$ is a finite collection of open sets, $V = \bigcap_{i=1}^n V_{e_n}$ is an open set that contains $x$. However, since $U_{e_i} \cap V_{e_i} = \emptyset$, and $V \subseteq V_{e_i}$, we have that $V \cap U_{e_i} = \emptyset$ for all $i$. Thus,
        \[  V \cap E = V \cap \bigcup_{i=1}^n U_{e_i} = \emptyset\]
        Thus, there exists an open set $V$ containing $x$ disjoint from $E$, thus contained in $X \setminus E$. Therefore, $X \setminus E$ is open. Thus, $E$ is closed.
        \item Assume $X$ is compact and $E$ is closed in $X$. If $E = X$, then we are done. Thus, assume $X \neq E$. Let $\{U_\alpha\}$ be an open cover of $E$ in $E$. Then, there exists a family of open sets $\{V_\alpha\}$ in $X$ where, for all $\alpha$,
        \[ U_\alpha = V_\alpha \cap E \]
        Note that since $U_\alpha \subseteq V_\alpha$ for all $\alpha$, $\{V_\alpha\}$ is an open cover for $E$ in $X$. Since $E$ is closed, $X \setminus E$ is open. Thus, $\{V_\alpha\} \cup \{X \setminus E\}$ is an open cover of $X$. Since $X$ is compact, there exists a finite subcover of $\{V_\alpha\} \cup \{X \setminus E\}$. \bbni 
        Let $\{V_{1}, \ldots, V_{n}, X \setminus E\}$, be the finite subcover. If the provided finite subcover does not contain $X \setminus E$, we can just add it while maintaining a finite subcover, we assume it does. Then, $V_1, \cdots, V_n$ is an open cover for $E$ in $X$. Then, we have:
        \[ \bigcup_{i=1}^n U_i = \bigcup_{i=1}^n (V_i \cap E) = \left(\bigcup_{i=1}^n V_i\right) \cap E = E \]
        Thus, $U_1, \ldots, U_n$ is a finite subcover of $U_\alpha$ in $E$. Therefore, $E$ is compact.
    \end{enumerate}
\end{solu}

\newpage

\begin{problab}{3}
    We say that $D$ is dense in $X$ if $\overline{D} = X$. Show that $D$ is dense if and only if $D$ meets every non-empty open set in $X$. 
\end{problab}

\begin{solu}
    We use the fact that $x \in \overline{D}$ (is a point of closure) if and only if every open neighborhood of $x$ intersects $D$. This is in Royden-Fitzpatrick as the definition of closure, with Ch9.2, Proposition 3 and 4 proving that the closure is closed and the complement of an open.
    \begin{itemize}
        \item[($\implies$)] Assume $D$ is dense in $X$. Let $U$ be a non-empty open set in $X$. Then, there exists a point $x \in U$. Since $\overline{D} = X$, $x \in \overline{D}$. Thus, every open neighborhood of $x$ intersects $D$. Thus, $U$ intersects $D$.
        \item[($\impliedby$)] Assume $D$ meets every non-empty open set in $X$. Let $x \in X$ be arbitrary. Then, $D$ meets every non-empty open set containing $x$. Thus, $x$ is a point of closure of $D$. Thus, $x \in \overline{D}$. Since $x$ is arbitrary, we have that $\overline{D} = X$. Thus, $D$ is dense in $X$.
    \end{itemize}
\end{solu}
\newpage


\begin{problab}{4}
    Show that a compact metric space has a countable dense subset. It is enough for the space to be totally bounded.
\end{problab}

\begin{solu}
    Let $(X, \rho)$ be a compact metric space. We define a countable set of open covers, indexed by $\mathbb{N}$, where $\mathcal{U}_n$ is the open cover: 
    \[ \mathcal{U}_n = \{ B_{1/n}(x) : x \in X\} \] 
    Then, for each $n$, the cover $\mathcal{U}_n$ has a finite subcover. Let $C_n$ be the centers of the balls in the finite subcover, for each $n$. The union of these sets over $n$ is countable, as it is a countable union of finite sets. Call this set $C$. We will show that $C$ is dense in $X$. \bbni
    Let $U$ be a non-empty open set in $X$. Then, there exists a point $x \in U$. Since $U$ is open, there exists an $\epsilon$, such that $B_\epsilon(x) \subseteq U$. Pick an $n$ such that $1/n < \epsilon$. Then, there exists a center $c \in C_n$ such that $x \in B_{1/n}(c)$ as $C_n$ is the set of centers of a finite cover made of $1/n$-balls. Thus, we have:
    \[\rho(x, c) < 1/n < \epsilon\]
    Thus, $c \in B_\epsilon(x) \subseteq U$. Therefore, $U$ intersects $C$. Since $U$ is arbitrary, every non-empty open set in $X$ intersects $C$. Thus, $C$ is dense in $X$. Therefore, $C$ is a countable dense subset of $X$.
\end{solu}
\newpage

\begin{problab}{5}
    Show that an equicontinous family of functions on a compact metric space is uniformly equicontinous as in lecture. (Some texts do not define equicontinous at a point. Instead, whether $X$ is compact or not, equicontinuity is what we have called uniformly equicontinuous. Fortunately, there is no distinction for compact spaces.)
\end{problab}

\begin{solu}
    % Equicontinous Family + Compact Space = Uniformly Equicontinous Family. \bbni
    Let $(X,\rho)$ be a compact metric space. Let $\mathcal{J}$ be an equicontinous family of functions from $(X, \rho)$ to $(Y, \sigma)$. Let $\epsilon > 0$. We need to show that there exists a $\delta > 0$, such that for all $F \in \mathcal{J}$, $x, y \in X$:
    \[\rho(x,y) < \delta \implies \sigma(F(x), F(y)) < \epsilon \]
    Since $\mathcal{J}$ is equicontinous, for every $x \in X$, there exists a $\delta_x > 0$ such that for all $F \in \mathcal{J}$, $y \in X$:
    \[\rho(x,y) < \delta_x \implies \sigma(F(x), F(y)) < \epsilon/2 \]
    Let $U_x = B_{\delta_x}(x)$. Then, $\{U_x\}_{x \in X}$ is an open cover of $X$. Since $X$ is compact, this open cover has a Lebesgue number $\delta > 0$. Thus, for all $F \in \mathcal{J}$, $x, y \in X$, if $\rho(x,y) < \delta$, there exists a $z$ such that:
    \[B_\delta(x) \subseteq B_{\delta_z}(z)\]
    Thus, we have:
    \[\rho(x,z) < \delta_z \qquad \rho(y,z) < \delta_z \]
    Finally, we compute:
    \begin{align*}
        \sigma(F(x), F(y)) &\leq \sigma(F(x), F(z)) + \sigma(F(z), F(y)) \\
        &< \epsilon/2 + \epsilon/2 \\
        &= \epsilon
    \end{align*}
    Thus, $\mathcal{J}$ is uniformly equicontinous.
\end{solu}
\newpage 


\begin{problab}{6} 
    Show that if $X$ a metric space which is not totally bounded, then there is an unbounded continous function $f: X \to \R$. (Hints provided).
\end{problab}

\begin{solu}
    Assume $(X, \rho)$ is not totally bounded. Then, there exists a $r > 0$ such that there is no finite collection of open balls of radius $r$ that covers $X$. We will use this to construct a sequence $(x_n)$ where $\rho(x_i, x_j) > r$ for all $i \neq j$. \bbni
    Since $X$ is non-empty, we can pick a point $x_1 \in X$. Moreover, if $x_1, \cdots, x_k$ are already picked points with $\rho(x_i, x_j) > r$ for all $i \neq j$ and $1 \leq i,j \leq k$, we can pick a point $x_{k+1} \in X \setminus \bigcup_{i=1}^k B_r(x_i)$, as the finite collection of open balls of radius $r$ centered at $x_1, \cdots, x_k$ does not cover $X$. Moreover, this means that $\rho(x_{k+1}, x_i) > r$ for all $i = 1, \cdots, k$. Thus, $\rho(x_i, x_j) > r$ for all $i\neq j$ where $1 \leq i, j \leq k+1$. Thus, by induction, we can construct a sequence $(x_n)$ such that $\rho(x_i, x_j) > r$ for all $i \neq j$. \bbni
    Secondly, we claim that for $i \neq j$, we have $B_{\frac r2}(x_i) \cap B_{\frac r2}(x_j) = \emptyset$. To see this, assume that there exists a $y \in B_{\frac r2}(x_i) \cap B_{\frac r2}(x_j)$. Then, we have:
    \[ \rho(x_i, x_j) \leq \rho(x_i, y) + \rho(y, x_j) < \frac r2 + \frac r2 < r\]
    which is a contradiction. \bbni
    Next, we define a sequence of functions $f_n: X \to \R$ such that $f_n(x_n) = 1$ and $f_n(x) = 0$ for all $x \in X \setminus B_{\frac r2}(x_n)$. Consider the following: 
    \[ x \mapsto \begin{cases}
        e^{1-\frac{1}{1-\left(\frac{2\rho(x,x_n)}{r}\right)^2}} & \text{if } x \in B_{\frac r2}(x_n) \\
        0 & \text{if } x \not \in B_{\frac r2}(x_n)
    \end{cases}\]
    Notice that if $x = x_n$ then $\rho(x, x_n) = 0$ and thus $f_n(x) = e^0 = 1$. Moreover $f_n$ is evidently continous in $B_{\frac r2}(x_n)$, as it is a composition of continous functions, as well as on $X \setminus \overline{B_{\frac r2}(x_n)}$. Thus, we only need to check the boundary. \\
    Let $x$ be a point on the boundary. Then, notice that $f_n$ is a monotonically decreasing function of $|\rho(x, x_n)|$ in $B_{\frac r2}(x_n)$ and $0$ outside. Moreover, notice that $\lim_{y \to x} \rho(y,x_n) = \frac r2$. Thus, we have that $\lim_{y\to x} 1-\left(\frac{2\rho(y,x_n)}{r}\right)^2 \to 0$. Thus, $\lim_{y\to x} f_n(y) \to 0$ from inside the ball. Since the function is identically $0$ outside, the limit is well-defined. Moreover, since for $x \not \in B_{\frac r2}(x_n)$, $f_n(x) = 0$. Thus, the limit equals the value and $f_n$ is continous on $X$. \bbni
    Next, we consider the function: 
    \[f(x) = \sum_{n \in \N} nf_n(x)\]
    since the sets $B_{\frac r2}(x_i)$ are disjoint, we have that at most one $f_n(x)$ is non-zero for every $X$. Thus, the function is continous. Moreover, the function is unbounded, as for any $M > 0$, we can pick $n > M$, and then $f(x_n) = nf_n(x_n) > M$. Thus, we are done.
\end{solu}
\newpage

\begin{problab}{7}
    Let $X$ be a metric space such that every continous function $f: X \to \R$ attains its minimum value. Show that $X$ is complete. (Hints provided).
\end{problab}

\begin{solu}
    Let $(x_n)$ be a Cauchy sequence in $X$. We will show that $(x_n)$ converges to a point in $X$. \bbni
    Since every function attains its minimum value, every function also attains its maximum value, as this is the minimum value for $-f$. Thus, every function $X \to \R$ is bounded. By the contrapositive of the previous question, this implies that $X$ is totally bounded. \bbni
    For any $x \in X$, we claim that $(\rho(x, x_n))_n$ is Cauchy in $\R$. Let $\epsilon > 0$. Since $(x_n)$ is Cauchy, there exists an $N$ such that for all $m, n > N$, we have: 
    \[ \rho(x_n, x_m) < \epsilon\]
    Then, we have:
    \begin{align*}
       &\rho(x, x_n) \leq \rho(x, x_m) + \rho(x_m, x_n) \\
       \implies &\rho(x, x_n) - \rho(x, x_m) \leq \rho(x_n, x_m) < \epsilon
    \end{align*}
    Symmetrically, we have:
    \[\rho(x, x_m) - \rho(x, x_n) < \epsilon\]
    Thus, for all $n, m > N$, we have:
    \[|\rho(x, x_n) - \rho(x, x_m)| < \epsilon \]
    Thus, $(\rho(x, x_n))_n$ is Cauchy in $\R$. \bbni 
    Let $f(x) = \lim_{n \to \infty} \rho(x, x_n)$. We show that $f$ is continous. Let $\epsilon > 0$. We need to find a $\delta > 0$ such that for all $x, y \in X$: 
    \[\rho(x, y) < \delta \implies |f(x) - f(y)| < \epsilon \]
    Pick $\delta = \epsilon$, and notice:
    \begin{align*}
        \rho(x, x_n) &\leq (\rho(x, y) + \rho(y, x_n) ) \\ 
        \rho(x, x_n) - \rho(y, x_n) &\leq \rho(x, y) < \delta = \epsilon \\
    \end{align*}
    Symmetrically, we have:
    \[\rho(y, x_n) - \rho(x, x_n) < \epsilon \]
    Thus, we have:
    \[ |\rho(x, x_n) - \rho(y, x_n)| < \epsilon\]
    Taking the limit as $n \to \infty$, we have:
    \[|f(x) - f(y) | < \epsilon\]
    Thus, $f$ is continous. \bbni
    Moreover, notice that $f$ is bounded below by $0$, as it is a limit of a sequence of non-negative functions. We prove that its infimum is $0$. Let $\epsilon > 0$. Since $(x_n)$ is Cauchy, there exists an $N$ such that for all $m, n > N$, we have: 
    \[\rho(x_m, x_n) < \epsilon\]
    Letting $n \to \infty$, we have:
    \[|f(x_m) - 0| < \epsilon \]
    Thus, the infimum of $f$ is $0$. Since $f$ attains its minimum value, there exists a $x_0 \in X$ such that $f(x_0) = 0$. \bbni 
    Lastly, we show that $(x_n) \to x_0$. Since $f(x_0) = 0$, we have:
    \begin{align*}
        \lim_{n\to \infty} \rho(x_n, x_0) = 0
    \end{align*}
    which is precisely equivalent to the definition of convergence in metric spaces.
\end{solu}
\newpage

\begin{problab}{8}
    Show that a metric space is compact if and only if every continous real-valued function on $X$ attains its maximum value. (Note that every real-valued function attains its maximum if and only if every real-valued function attains its minimum. Consider $-f$.)
\end{problab}
\begin{solu}
    \bbni
    \begin{itemize}
        \item[$(\implies$)] We did this in class. Here's a reproduction of the argument. Let $X$ be a compact metric space and $f: X \to \R$ be a continous function. As $X$ is  Since $X$ is compact, $f$ is bounded.  Let $M := \sup f(x) \in (-\infty, \infty]$. Thus, for $M \neq \infty$, there exists a sequence $(x_n)$ such that: 
        \[ |f(x_n) - M| < 1/n\]
        and for $M = \infty$, there exists a sequence $(x_n)$ such that:
        \[|f(x_n)| > n\]
        That is, in both cases, there exists a sequence such that:
        \[\lim_{n \to \infty} f(x_n) = M \]
        Since $X$ is compact, it is sequentially compact. Thus, there exists a subsequence $(x_{n_k})$ that converges to a point $x_0 \in X$. \bbni 
        Then, since $f$ is continous, we have: 
        \[ f(x_0) = \lim_{n\to\infty} f(x_n) = M \]
        Thus, $f$ attains its maximum value. Considering $-f$, the same argument shows that $f$ attains its minimum value. 
        \item[$(\impliedby$)] Assume every real-valued continous function on $X$ attains its maximum value. By the previous problem (Problem 7), we know that $X$ is complete. Moreover, as every real-valued function attains its maximum value, there is no unbounded continous function on $X$. Thus, by the contrapositive of the problem before the previous problem (Problem 6), we have that $X$ is totally bounded. Finally, we claimed in class that a metric space is compact if and only if it is complete and totally bounded. Thus, $X$ is compact.        
    \end{itemize}
\end{solu}


\end{document}