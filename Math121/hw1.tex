% \documentclass[12pt]{amsart}
\documentclass[12pt]{article}

\usepackage{fullpage}
\usepackage{mdframed}
\usepackage{colonequals}
\usepackage{algpseudocode}
\usepackage{algorithm}
\usepackage{tcolorbox}
\usepackage[all]{xy}
\usepackage{proof}
\usepackage{mathtools}
\usepackage{bbm}
\usepackage{amssymb}
\usepackage{amsthm}
\usepackage{amsmath}
\usepackage{amsxtra}
\newcommand{\bb}{\mathbb}


\newtheorem{theorem}{Theorem}[section]
\newtheorem{corollary}{Corollary}[theorem]
\newtheorem{lemma}{Lemma}

\newcommand{\mathcat}[1]{\textup{\textbf{\textsf{#1}}}} % for defined terms

\newenvironment{problem}[1]
{\begin{tcolorbox}\noindent\textbf{Problem #1}.}
{\vskip 6pt \end{tcolorbox}}

\newenvironment{enumalph}
{\begin{enumerate}\renewcommand{\labelenumi}{\textnormal{(\alph{enumi})}}}
{\end{enumerate}}

\newenvironment{enumroman}
{\begin{enumerate}\renewcommand{\labelenumi}{\textnormal{(\roman{enumi})}}}
{\end{enumerate}}

\newcommand{\defi}[1]{\textsf{#1}} % for defined terms

\theoremstyle{remark}
\newtheorem*{solution}{Solution}

\setlength{\hfuzz}{4pt}

\newcommand{\calC}{\mathcal{C}}
\newcommand{\calF}{\mathcal{F}}
\newcommand{\C}{\mathbb C}
\newcommand{\N}{\mathbb N}
\newcommand{\Q}{\mathbb Q}
\newcommand{\R}{\mathbb R}
\newcommand{\Z}{\mathbb Z}
\newcommand{\br}{\mathbf{r}}
\newcommand{\RP}{\mathbb{RP}}
\newcommand{\CP}{\mathbb{CP}}
\newcommand{\nbit}[1]{\{0, 1\}^{#1}}
\newcommand{\bits}{\{0, 1\}^{n}}
\newcommand{\bbni}{\bigbreak \noindent}
\newcommand{\norm}[1]{\left\vert\left\vert#1\right\vert\right\vert}

\let\1\relax
\newcommand{\1}{\mathbf{1}}
\newcommand{\fr}[2]{\left(\frac{#1}{#2}\right)}

\newcommand{\vecz}{\mathbf{z}}
\newcommand{\vecr}{\mathbf{r}}
\DeclareMathOperator{\Cinf}{C^{\infty}}
\DeclareMathOperator{\Id}{Id}

\DeclareMathOperator{\Alt}{Alt}
\DeclareMathOperator{\ann}{ann}
\DeclareMathOperator{\codim}{codim}
\DeclareMathOperator{\End}{End}
\DeclareMathOperator{\Hom}{Hom}
\DeclareMathOperator{\id}{id}
\DeclareMathOperator{\M}{M}
\DeclareMathOperator{\Mat}{Mat}
\DeclareMathOperator{\Ob}{Ob}
\DeclareMathOperator{\opchar}{char}
\DeclareMathOperator{\opspan}{span}
\DeclareMathOperator{\rk}{rk}
\DeclareMathOperator{\sgn}{sgn}
\DeclareMathOperator{\Sym}{Sym}
\DeclareMathOperator{\tr}{tr}
\DeclareMathOperator{\img}{img}
\DeclareMathOperator{\CandE}{CandE}
\DeclareMathOperator{\CandO}{CandO}
\DeclareMathOperator{\argmax}{argmax}
\DeclareMathOperator{\first}{first}
\DeclareMathOperator{\last}{last}
\DeclareMathOperator{\cost}{cost}
\DeclareMathOperator{\dist}{dist}
\DeclareMathOperator{\path}{path}
\DeclareMathOperator{\parent}{parent}
\DeclareMathOperator{\argmin}{argmin}
\DeclareMathOperator{\excess}{excess}
\let\Pr\relax
\DeclareMathOperator{\Pr}{\mathbf{Pr}}
\DeclareMathOperator{\Exp}{\mathbb{E}}
\DeclareMathOperator{\Var}{\mathbf{Var}}
\let\limsup\relax
\DeclareMathOperator{\limsup}{limsup}
%Paired Delims
\DeclarePairedDelimiter\ceil{\lceil}{\rceil}
\DeclarePairedDelimiter\floor{\lfloor}{ \rfloor}


\newcommand{\dagstar}{*}

\newcommand{\tbigwedge}{{\textstyle{\bigwedge}}}
\setlength{\parindent}{0pt}
\setlength{\parskip}{5pt}


\begin{document}

% \title{CS 40: Computational Complexity}

\author{Sair Shaikh}
\maketitle

% Collaboration Notice: Talked to Henry Scheible '26 to discuss ideas.




\begin{problab}{1}
    Let $U \subset \mathbb{C}^n$ be a domain (i.e., a non-empty connected subset) and let $f : U \to \mathbb{C}$ be a holomorphic function.
    \begin{enumerate}
        \item Prove that $f$ satisfies analytic continuation: if it vanishes on an open subset of $U$, then it vanishes everywhere.
        \item Prove that $f$ satisfies the maximum principle: if $|f|$ admits a local maximum in $U$, then $f$ is constant.
        \item Let $M$ be a compact complex manifold. Prove that any holomorphic function on $M$ is constant.
    \end{enumerate}    
\end{problab}
\newpage


\begin{problab}{2}
    \begin{enumerate}
        \item Show that the assignment $L \mapsto (L_{\mathbb{R}}, \text{mult}(i))$ gives an equivalence between the category of complex vector spaces and the category of pairs $(V, J)$, where $V$ is a real vector space, $J : V \to V$ is an $\mathbb{R}$-linear operator satisfying $J^2 = -\mathrm{Id}_{L_\mathbb{R}}$, and a morphism $(V, J) \to (W, K)$ is defined as an $\mathbb{R}$-linear map $f : V \to W$ that intertwines $J$ and $K$, i.e., such that $K \circ f = f \circ J$.

        \item Let $(V, J)$ be a pair as above:
        \begin{enumerate}
            \item Let $\iota : v \mapsto v$ be the $\mathbb{R}$-linear automorphism of $V \otimes_{\mathbb{R}} \mathbb{C}$ induced from conjugation on the second factor. Show that $V$, seen inside $V \otimes_{\mathbb{R}} \mathbb{C}$ via $v \mapsto v \otimes 1$, is isomorphic to the fixed locus of $\iota$.
            \item By diagonalizing $J$ over $V \otimes_{\mathbb{R}} \mathbb{C}$, show that we have a decomposition $V_{\mathbb{C}} = W \oplus \bar{W}$ where $J$ acts on $W$ by multiplication by $i$ and $\bar{W} = \iota(W)$.
            \item Show that the projection map $V \to W$ is an $\mathbb{R}$-linear isomorphism that intertwines $J$ and multiplication by $i$ on $W$.
            \item Each element $v \in V_{\mathbb{C}}$ can be written as $v = v_1 + v_2$ along the above decomposition. Show that $v \in V$ if and only if $v_2 = \bar{v}_1$.
        \end{enumerate}
    \end{enumerate}
\end{problab}
\newpage


\begin{problab}{3}
    Let $E, F$ be real vector spaces and let $G$ be a complex vector space, all finite-dimensional.
    \begin{enumerate}
        \item Show that $E \otimes_{\mathbb{R}} G$ admits a natural structure of a complex vector space. Construct a basis in terms of bases of $E$ and $G$. Show that $E \otimes_{\mathbb{R}} G = (E \otimes_{\mathbb{R}} \mathbb{C}) \otimes_{\mathbb{C}} G$.
        \item Show that $\mathrm{Hom}_{\mathbb{R}}(E, G) = \mathrm{Hom}_{\mathbb{C}}(E \otimes \mathbb{C}, G)$.
        \item Show that
        \[
            (E \oplus F) \otimes_{\mathbb{R}} \mathbb{C} = (E \otimes_{\mathbb{R}} \mathbb{C}) \oplus (F \otimes_{\mathbb{R}} \mathbb{C}), \quad (E \otimes_{\mathbb{R}} F) \otimes_{\mathbb{R}} \mathbb{C} = (E \otimes_{\mathbb{R}} \mathbb{C}) \otimes_{\mathbb{C}} (F \otimes_{\mathbb{R}} \mathbb{C})
        \]
        and
        \[
            \Lambda^n E \otimes_{\mathbb{R}} \mathbb{C} = \Lambda^n(E \otimes_{\mathbb{R}} \mathbb{C}).
        \]
        \item Let $f : E \to F$ be a linear map and let $f_{\mathbb{C}} = f \otimes_{\mathbb{R}} \mathbb{C} : E \otimes_{\mathbb{R}} \mathbb{C} \to F \otimes_{\mathbb{R}} \mathbb{C}$ be the induced map. Show that
        \[
            \ker(f_{\mathbb{C}}) = \ker(f) \otimes_{\mathbb{R}} \mathbb{C}, \quad \mathrm{Im}(f_{\mathbb{C}}) = \mathrm{Im}(f) \otimes_{\mathbb{R}} \mathbb{C}.
        \]
    \end{enumerate}
\end{problab}
\newpage



\begin{problab}{4}
    Let $U \subset \mathbb{C}$ be an open subset and let $D \subset \Omega$ be a closed disk.

    \begin{enumerate}
        \item Let $f : U \to \mathbb{C}$ be a $\mathcal{C}^1$ function. Show that for all $z \in D$, we have:
        \[
            f(z) = \frac{1}{2\pi i} \int_{\partial D} \frac{f(\xi)}{\xi - z} \, d\xi + \frac{1}{2\pi i} \int_D \frac{\partial f}{\partial \bar{z}}(\xi) \frac{d\xi \wedge d\bar{\xi}}{\xi - z}.
        \]
        \textit{Hint:} Apply Stokes’ theorem to $\frac{f(\xi)}{\xi - z} d\xi$ on $D \setminus B(z, \varepsilon)$ and let $\varepsilon \to 0$.
        
        \item Let $g$ be a $\mathcal{C}^1$ function on $\mathbb{C}$ with compact support and define
        \[
            f(z) = \frac{1}{2\pi i} \int_{\mathbb{C}} \frac{g(\xi)}{\xi - z} d\xi \wedge d\bar{\xi}.
        \]
        Show that $f$ is $\mathcal{C}^1$ and $\frac{\partial f}{\partial \bar{z}} = g$. \textit{Hint:} Differentiate under the integral sign after substituting $\xi' = \xi - z$.
        
        \item Show that for any function $g$ on $U$ which is $\mathcal{C}^1$, there exists a function $f$ on $U$, also $\mathcal{C}^1$, such that $\frac{\partial f}{\partial \bar{z}} = g$ on $D$.
        
        \item In the previous question, show that if $g$ is $\mathcal{C}^\infty$, then $f$ can also be chosen to be $\mathcal{C}^\infty$.
    \end{enumerate}
\end{problab}
\newpage

\begin{problab}{5}
    Let $E$ and $F$ be two holomorphic vector bundles on a complex manifold $X$. Given an open cover $\{ U_\alpha \}$ of $X$ that trivializes $E$, the vector bundle $E$ is described on overlaps $U_\alpha \cap U_\beta$ by holomorphic transition functions:
    \[
        \rho_{\alpha \beta} : U_\alpha \cap U_\beta \to \mathrm{GL}_n(\mathbb{C}).
    \]
    \begin{enumerate}
        \item Prove the cocycle condition: $\rho_{\alpha \gamma} = \rho_{\beta \gamma} \circ \rho_{\alpha \beta}$.
        \item Let $E'$ be the quotient of $\bigsqcup_\alpha U_\alpha \times \mathbb{C}^n$ by the equivalence relation on $U_\alpha \cap U_\beta \times \mathbb{C}^n$ given by
        \[
            (x, v) \sim (x, \rho_{\alpha \beta}(x)(v)).
        \]
        Prove that $E'$ is a holomorphic vector bundle and that it is isomorphic to $E$ as vector bundles over $X$, i.e., there exists a biholomorphism $f : E \to E'$ commuting with projection to $X$.
        
        \item Conversely, assume that $E$ and $F$ are isomorphic as holomorphic vector bundles. How are their transition functions related?
        
        \item Using the transition maps of $E$ and $F$, construct the following vector bundles by writing down explicitly their transition functions: $E \otimes_{\mathbb{C}} F$, $E \oplus F$, $\Lambda^n E$.
    \end{enumerate}
\end{problab}
\begin{solu}
    \bbni
    \begin{enumerate}
        \item If suffices to show the condition over fibers, as the transition functions preserve fibers. Let $x \in U_\alpha \cap U_\beta \cap U_\gamma$ and let $E_x = \pi^{-1}(a)$ be the fiber over $x$ in $E$. \\
        Thoughts on this: Perhaps pick basis in each of the three trivializations. Then the transition functions are change of basis matrices. Then the change of basis matrices satisfy the cocycle condition. 
        \item What do we need to prove that $E'$ is a holomorphic vector bundle? We need to show that $E'$ has a projection map $\pi'$ to $X$ and that there are local trivializations $\tau_{\alpha}: \pi^{-1}(U_\alpha) \to U_\alpha \times \C^n$ whose transition maps are holomorphic. \\
        To show that these vector bundles are $E'$ and $E$ are isomorphic, we need to find a biholomorphism $f : E \to E'$ such that: 
        \[ \pi' \circ f = \pi \qquad \qquad \pi \circ f^{-1} = \pi' \]
        and also that $f$ and $f'$ are holomorphic.
        \item If $E$ and $F$ are isomorphic as holomorphic vector bundles. Then there exists a biholomorphism $f : E \to F$ with $\pi' \circ f = \pi$. Take charts $U_\alpha$ and $U_\beta$. Then, we have the following to compare: 
        \[ (x, v) \qquad (x, f(v)), \qquad (x, f(\rho(x)(v))) \qquad (x, \rho(x)(f(v))) \]
        Do $f$ and $\rho$ commute? We need to think about the biholomorphism for this. 
        \item Suppose I have the transition functions $\rho$ and $\rho'$ for $E$ and $F$ respectively. What would the transition functions for each of those look like?
        \begin{enumerate}
            \item Apply $\rho$ to the first components of the bundle and $\rho'$ to the second components of each coordinate of the bundle. 
            \item Wait, I think you can just take the $\rho$ on each component. But what if there is any intersection? Does direct sum have any intersection?
            \item Apply to each and then take the wedge product of them?
        \end{enumerate}
        
    \end{enumerate}
\end{solu}



\end{document}