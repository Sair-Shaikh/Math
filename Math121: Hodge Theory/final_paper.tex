\documentclass[12pt]{article}


\usepackage{fullpage}
\usepackage{mdframed}
\usepackage{colonequals}
\usepackage{algpseudocode}
\usepackage{algorithm}
\usepackage{tcolorbox}
\usepackage[all]{xy}
\usepackage{proof}
\usepackage{mathtools}
\usepackage{bbm}
\usepackage{amssymb}
\usepackage{amsthm}
\usepackage{amsmath}
\usepackage{amsxtra}
\newcommand{\bb}{\mathbb}


\newtheorem{theorem}{Theorem}[section]
\newtheorem{corollary}{Corollary}[theorem]
\newtheorem{lemma}{Lemma}

\newcommand{\mathcat}[1]{\textup{\textbf{\textsf{#1}}}} % for defined terms

\newenvironment{problem}[1]
{\begin{tcolorbox}\noindent\textbf{Problem #1}.}
{\vskip 6pt \end{tcolorbox}}

\newenvironment{enumalph}
{\begin{enumerate}\renewcommand{\labelenumi}{\textnormal{(\alph{enumi})}}}
{\end{enumerate}}

\newenvironment{enumroman}
{\begin{enumerate}\renewcommand{\labelenumi}{\textnormal{(\roman{enumi})}}}
{\end{enumerate}}

\newcommand{\defi}[1]{\textsf{#1}} % for defined terms

\theoremstyle{remark}
\newtheorem*{solution}{Solution}

\setlength{\hfuzz}{4pt}

\newcommand{\calC}{\mathcal{C}}
\newcommand{\calF}{\mathcal{F}}
\newcommand{\C}{\mathbb C}
\newcommand{\N}{\mathbb N}
\newcommand{\Q}{\mathbb Q}
\newcommand{\R}{\mathbb R}
\newcommand{\Z}{\mathbb Z}
\newcommand{\br}{\mathbf{r}}
\newcommand{\RP}{\mathbb{RP}}
\newcommand{\CP}{\mathbb{CP}}
\newcommand{\nbit}[1]{\{0, 1\}^{#1}}
\newcommand{\bits}{\{0, 1\}^{n}}
\newcommand{\bbni}{\bigbreak \noindent}
\newcommand{\norm}[1]{\left\vert\left\vert#1\right\vert\right\vert}

\let\1\relax
\newcommand{\1}{\mathbf{1}}
\newcommand{\fr}[2]{\left(\frac{#1}{#2}\right)}

\newcommand{\vecz}{\mathbf{z}}
\newcommand{\vecr}{\mathbf{r}}
\DeclareMathOperator{\Cinf}{C^{\infty}}
\DeclareMathOperator{\Id}{Id}

\DeclareMathOperator{\Alt}{Alt}
\DeclareMathOperator{\ann}{ann}
\DeclareMathOperator{\codim}{codim}
\DeclareMathOperator{\End}{End}
\DeclareMathOperator{\Hom}{Hom}
\DeclareMathOperator{\id}{id}
\DeclareMathOperator{\M}{M}
\DeclareMathOperator{\Mat}{Mat}
\DeclareMathOperator{\Ob}{Ob}
\DeclareMathOperator{\opchar}{char}
\DeclareMathOperator{\opspan}{span}
\DeclareMathOperator{\rk}{rk}
\DeclareMathOperator{\sgn}{sgn}
\DeclareMathOperator{\Sym}{Sym}
\DeclareMathOperator{\tr}{tr}
\DeclareMathOperator{\img}{img}
\DeclareMathOperator{\CandE}{CandE}
\DeclareMathOperator{\CandO}{CandO}
\DeclareMathOperator{\argmax}{argmax}
\DeclareMathOperator{\first}{first}
\DeclareMathOperator{\last}{last}
\DeclareMathOperator{\cost}{cost}
\DeclareMathOperator{\dist}{dist}
\DeclareMathOperator{\path}{path}
\DeclareMathOperator{\parent}{parent}
\DeclareMathOperator{\argmin}{argmin}
\DeclareMathOperator{\excess}{excess}
\let\Pr\relax
\DeclareMathOperator{\Pr}{\mathbf{Pr}}
\DeclareMathOperator{\Exp}{\mathbb{E}}
\DeclareMathOperator{\Var}{\mathbf{Var}}
\let\limsup\relax
\DeclareMathOperator{\limsup}{limsup}
%Paired Delims
\DeclarePairedDelimiter\ceil{\lceil}{\rceil}
\DeclarePairedDelimiter\floor{\lfloor}{ \rfloor}


\newcommand{\dagstar}{*}

\newcommand{\tbigwedge}{{\textstyle{\bigwedge}}}
\setlength{\parindent}{0pt}
\setlength{\parskip}{5pt}



\begin{document}


\title{CS 40: Computational Complexity}

\author{Sair Shaikh}
\maketitle

% Collaboration Notice: Talked to Henry Scheible '26 to discuss ideas.



These notes cover some of the theory of complex multiplication (CM) for elliptic curves and briefly mentions the generalization to abelian varieties towards the end. The key result that we will prove is the following:

\begin{theorem*}{1}
    Let $E/\C$ be an elliptic curve with complex multiplication by $O_{K}$, the ring of integers, of an imaginary quadratic field $K$. Then, there exists an elliptic curve $E'/\overline{\Q}$ that is isomorphic to $E$ over $\C$. 
\end{theorem*}


\section{Definitions}

We start by defining the objects and morphisms in the category of elliptic curves over a field $K$. 

\begin{definition}
    \begin{enumerate}
        \item An elliptic curve is a pair $(E, O)$ where $E$ is a non-singular curve of genus $1$ and $O \in E$. We write $E/K$ if $E$ is defined over $K$. We also note that we have an isomorphism from $E/\C$ to a curve given by the Weierstrass equation:
        \[ Y^2 = X^3 +AX + B\]
        i.e.
        \[ \phi: E \to \P^2 \qquad \phi = [x,y, 1]\]
        with $\phi(0) = [0:1:0]$. Every elliptic curve is given in this way. The elliptic curve $E$ comes with an abelian group structure with identity $O$. 
        \item We define the discriminant $\Delta$ and the $j$-invariant of $E$ as follows:
        \[ \Delta(E) = -16(4A^3+27B^2) \qquad j(E) = -1728\fr{(4A)^3}{\Delta}\]
    \end{enumerate} 
\end{definition}

The $j$-invariant, as the name suggests, is invariant under change of variables that preserve the underlying variety. The next proposition describes some of the power the $j$-invariant offers:

\begin{prop}
    \begin{enumerate}
        \item Two elliptic curves defined over $\overline{K}$ are isomorphic over $\overline{K}$ if and only if they have the same $j$-invariant.
        \item Let $j_0 \in \overline{K}$. Then, there exists an elliptic curve $E/\Q(j_0)$ with $j(E) = j_0$.
    \end{enumerate}
\end{prop}

\begin{proof}
    For $(1)$, the proof is by direct computation and change-of-variables. For $(2)$, the proof is by explicit construction. We defer to [AEC III.1.4] for the details.
\end{proof}

\begin{definition}
    \begin{enumerate}
        \item Let $E_1/\C$ and $E_2/\C$ be elliptic curves. An isogeny $\phi: E_1 \to E_2$ is a morphism of varieties with $\phi(O_1) = O_2$. Two curves are isogenous if there exists an non-zero isogeny between them. We state, but not prove, that isogeny is an equivalence relation.
        \item An isogeny from $E$ to itself is called an endomorphism. The set of endomorphisms $\End(E)$ is a ring with the addition and multiplication given by the pointwise addition and composition, respectively.
    \end{enumerate}    
\end{definition}


Since an elliptic curve has a group structure, it naturally has endomorphisms given by the "multiplication-by-$m$" maps:
\begin{definition}
    For $m \in \Z$, the multiplication-by-$m$ map is given by:
    \[ [m]: E \to E, \qquad [m](P) = mP\]
    for all $P \in E$. One can show that $[m]$ is non-constant for $m \ne 0$, thus $[m] \in \End(E)$.
\end{definition}

Thus, naturally, we have that $\Z \subset \End(E)$. It will turn out that generetically $\End(E) \cong \Z$, but there are special cases where $\End(E)$ is larger. In these cases, we define: 

\begin{definition}
    An elliptic curve $E/\C$ has complex multiplication (CM) if $\End(E) \not \cong \Z$.
\end{definition}

\section{Equivalence of Categories}
To prove the main theorem, we need to understand elliptic curves from multiple perspectives. In this section, we will show that the following categories are equivalent:
\begin{enumerate}
    \item Elliptic curves over $\C$ with isogenies.
    \item Complex tori $\C/\Lambda$ with holomorphic maps taking $O$ to $O$.
    \item Lattices $\Lambda \subset \C$ up to homothety with morphisms given by:
    \[ \Hom(\Lambda_1, \Lambda_2) = \{\alpha \in \C: \alpha \Lambda_1 \subset \Lambda_2\}\]
\end{enumerate}

To prove this equivalence, we introduce a special type of function, called elliptic functions.

\begin{definition}
    An elliptic function (relative to a lattice $\Lambda$) is a meromorphic function $f(z)$ on $\C$ that satisfies: 
    \[ f(z+\omega) = f(z) \]
    for all $z \in \C$ and $\omega \in \Lambda$.
\end{definition}

The elliptic function we care about is the Weierstrauss $\wp$-function, defined as follows:

\begin{definition}
    Let $\Lambda \subset \C$ be a lattice. The Weierstrauss $\wp$-function relative to $\Lambda$ is defined by the series: 
    \[ \wp(z; \Lambda) = \frac{1}{z^2} + \sum_{\substack{w \in \Lambda \\ w \ne 0}} \left( \frac{1}{(z-w)^2} - \frac{1}{\omega^2}\right)\]
    The Eisenstein series $G_{2k}(\Lambda)$ of weight $2k$ is the series:
    \[ G_{2k}(\Lambda) = \sum_{\substack{w \in \Lambda \\ w \ne 0} }w^{-2k}\]
\end{definition}

We prove the following propositions about the Weierstrauss $\wp$-function:

\begin{prop}
    The Weierstrass $\wp$-function converges absolutely and uniformly on every compact subset of $\C\setminus \Lambda$, with double poles at points in the lattice, and no poles elsewhere. \\
    The series $G_{2k}(\Lambda)$ converges absolutely for all $k > 1$.
\end{prop}
\begin{proof}
    We refer to AEC VI.3.1. 
\end{proof}

The derivative of $\wp(Z)$ is given by:
\[\wp'(P) = -2\sum_{i=1}^\infty \frac{1}{(z-w)^2}\]

Let $\C(\Lambda)$ be the elliptic functions relative to $\Lambda$. Then, we have the proposition that:
\begin{prop}\label{:allWeier}
    Let $\Lambda \subset \C$ be a lattice. Then, 
    \[ \C(\Lambda) = \C(\wp(z), \wp'(z))\]
    i.e. every elliptic function is a rational combination of $\wp(z)$ and $\wp'(z)$.
\end{prop}
\begin{proof}
    We refer to AEC VI.3.2.
\end{proof}

\begin{theorem}
    For all $z \in \C \setminus \Lambda$, we have the relation: 
    \[ \wp'(z)^2 = 4\wp(z)^3 - g_2\wp(z) - g_3\]
    Let $g_2 = 60G_4(\Lambda)$ and $g_3 = 140G_6(\Lambda)$. Moreover,
    \[ f(x) = 4x^3 - g_2x - g_3\]
    has distinct roots, so its discriminant:
    \[ \Delta(\Lambda) = g_2^3 - 27g_3^3\]
    is nonzero. 
\end{theorem}
\begin{proof}
    We claim that the Laurent series for $\wp(z)$ around $z = 0$ is given by: 
    \[ \wp(z) = \frac{1}{z^2} + \sum_{k=1}^\infty (2k+1)G_{2k+2}z^{2k} \]
    For $|z| < |w|$, we have:
    \begin{align*}
        \frac{1}{(w-z)^2} - \frac{1}{w^2} &= \frac{1}{w^2}\left(\frac{1}{(1-z/w)^2} - 1\right) \\
        &= \sum_{k=1}^\infty (k+1)z^k/w^{2+k}
    \end{align*}
    For odd $k$, the terms cancel by symmetry of the lattice. We get the desired series by substituting this in.  \\
    Next, we compute the following Laurent series:
    \begin{align*}
        \wp'(z)^2 &= 4z^{-6} -24G_4z^{-2} -80G_6 + \cdots \\
        \wp(z)^3 &= z^{-6} + 9G_4z^{-1} _ 15G_6 + \cdots \\
        \wp(z) &= z^{-2} + 3G_4z^2 + \cdots 
    \end{align*}
    Then, define: 
    \[ f(z) = \wp'(z)^2-4\wp(z)^3+60G_4\wp(z)+140G_6\]
    This is an elliptic function (relative to $\Lambda$) that is holomorphic at $z = 0$, thus must be constant by a compactness argument. Since $f(0) = 0$, we have:
    \[\wp'(z)^2 = 4\wp(z)^3-60G_4\wp(z)-140G_6 \]
    as desired. \\
    Next, let $\omega_1, \omega_2$ be a basis for $\Lambda$ and $\omega_3 = \omega_1 + \omega_2$. Then, $\fr{\omega_i}{2} \cong \fr{\omega_i}{2} \pmod{\Lambda}$. Then note since $\wp'$ is odd:
    \[ \wp'\fr{\omega_i}{2} = -\wp'\fr{-\omega_i}{2} = -\wp'\fr{\omega_i}{2} \]
    Thus, $\wp'\fr{\omega_i}{2} = 0$ for $i = 1,2,3$. Then, since $\wp, \wp'$ satisfy the Weierstrass equation, we note $\wp\fr{\omega_i}{2}$ are roots for $f(x)$. \\
    Next, note that $\wp(z) - \wp\fr{\omega_i}{2}$ is even, and thus has at least a double zero at $z = \fr{\omega_i}{2}$. Since it is an elliptic of order $2$, and these are the only zeros in a fundamental parallelogram, we must have $\fr{\omega_i}{2} \neq \fr{\omega_j}{2}$ for $i \neq j$. Thus, $f$ has distinct roots and non-zero discriminant.
\end{proof}

In fact, we can prove a much stronger statement:
\begin{prop}
    Let $E/\C$ be the Elliptic curve:
    \[E : y^2 = 4x^3 - g_2x -g_3 \]
    Then the map: 
    \[ \phi: \C/\Lambda \to E(\C) \subset \bb P^2(\C), \qquad z \mapsto [\wp(z), \wp'(z), 1]\]
    is a complex analytic isomorphism of complex Lie groups, i.e. it is an isomorphism of Riemann surfaces that is also a group homomorphism. 
\end{prop}
\begin{proof}
    By the previous proposition, we have that $\img(\phi) \subset E(\C)$. We need to show that $\phi$ is a bijection. \\
    Let $(x, y) \in E(\C)$. Then $\wp(z)-x$ is a nonconstant elliptic function, so it has a zero, say at $z = a$. Plugging in $z = a$ gives:
    \[ \wp^{'2}(a) = \wp(a)^3 -g_2\wp(a) -g_3 = x^3-g_2x-g_3 = y^2 \]
    Replacing $a$ by $-a$ if necessary, we obtain $\wp'(a) = y$. Then, $\phi(a) = (x,y)$. \\
    Next suppose that $\phi(z_1) = \phi(z_2)$. Then, $\wp(z)-\wp(z_1)$ is an elliptic function of order $2$ that vanishes at $z_1$, $-z_1$, and $z_2$. Thus, two of them are congruent modulo $\Lambda$. Then, if $2z_1 \not \in \Lambda$, we have $z_1 \equiv \pm z_2 \pmod{\Lambda}$, and we fix the sign since $\wp'$ is odd. If $2z_1 \in \Lambda$, then this has a double $0$ at $z_1$ and is zero at $z_2$, hence $z_1 \equiv z_2 \pmod{\Lambda_2}$. Thus, $\phi$ is injective.  \\
    Next, note that at every point of $E(\C)$, the differential form $dx/y$ is holomorphic and nonvanishing. Further, we see that:
    \[ \phi^*\fr{dx}{dy} = \frac{d\wp(z)}{\wp'(z)} = dz\]
    is also holomorphic and nonvanishing at every point of $\C/\Lambda$. Since $\phi$ is bijective, these are defined everywhere. Thus, $\phi$ is an isomorphism in the complex geometry sense.
\end{proof}

We also have a converse that we provide without proof. Interested readers should refer to [ATAEC I.3.1] for details.

\begin{theorem}(Uniformization Theorem)
    Let $A, B \in \C$ satisfy $4A^3 + 27B^2 \ne 0$. Then, there is a unique lattice $\Lambda \subset \C$ satisfying:
    \[ g_2(\Lambda) = A \qquad g_3(\Lambda) = B \]
\end{theorem}

\begin{theorem}
    Let $\phi_\alpha: \C/\Lambda_1 \to \C/\Lambda_2$ be $\phi_\alpha(z) = \alpha z \pmod{\Lambda_2}$. This map is holomorphic. Moreover,   
        \begin{enumerate}
        \item The association: 
        \begin{align*}
            \{\alpha \in \C: \alpha \Lambda_1 \subset \Lambda_2\} &\to \left\{ \substack{ \text{ holomorphic maps } \\ \phi: \C/\Lambda_1 \to \C/\Lambda_2 \\
            \text{ with } \phi(0) = 0}\right\} \\
            \alpha &\mapsto \phi_\alpha 
        \end{align*}
        is a bijection. 
        \item Let $E_1$ and $E_2$ be elliptic curves corresponding to lattices $\Lambda_1$ and $\Lambda_2$, respectively. Then the natural inclusion: 
        \[ \{\text{ isogenies } \phi: E_1 \to E_2 \} \to \left\{ \substack{ \text{ holomorphic maps } \\ \phi: \C/\Lambda_1 \to \C/\Lambda_2 \\
            \text{ with } \phi(0) = 0}\right\}\]
        is a bijection.
    \end{enumerate}
\end{theorem}
\begin{proof}
    \bbni
    \begin{enumerate}
        \item Assume $\phi_\alpha = \phi_\beta$. Then, $\alpha z \equiv \beta z \pmod{\Lambda_2}$ for all $z \in \C$. Thus, $z \mapsto (\alpha-\beta)z \equiv 0$. As $\Lambda_2$ is a discrete group, the map must be constant, hence $\alpha = \beta$. Thus, the map is injective. \\
        Next, let $\phi: \C/\Lambda_1 \to \C/\Lambda_2$ be a holomorphic map with $\phi(0) = 0$. Then, as $\C$ is simply connected, we lift to $\phi:\C \to \C$ satisfying: 
        \[ f(z+w) \equiv f(z) \pmod{\Lambda_2} \]
        for all $w \in \Lambda_1$. As $\Lambda_2$ is discrete, we have $f(z+w)-f(z)$ is constant. Moreover, we have:
        \[ f'(z+w) = f'(z)\]
        so $f'(z)$ is a holomorphic elliptic function. It follows from [AEC VI.2.1] that $f'(z)$ is constant, so $f(z) = \alpha z + \gamma$. But $\gamma = 0$ as $f(0) = 0$. Then, we note $f(\Lambda_1) \subset \Lambda_2$ implies $\alpha\Lambda_1 \subseteq \Lambda_2$. Hence, $\phi = \phi_\alpha$.
        \item Note that an isogeneny is given locally by everywhere defined rational functions, thus the map induced between the corresponding complex tori is holomorphic. Thus, the association is well-defined and injective. \\
        To show surjectivity, consider a map $\phi_\alpha$, where $\alpha \in \C^*$ with $\alpha \Lambda_1 \subseteq \Lambda_2$. The induced map on Weierstrass $\wp$-functions satisfies:
        \[ \wp(\alpha(z+w), \Lambda_2) = \wp(\alpha z + \alpha w, \Lambda_2) = \wp(\alpha z, \Lambda_2)\]
        and similarly for $\wp'$. Thus, $\wp(\alpha z, \Lambda_2)$ and $\wp'(\alpha z, \Lambda_2)$ are in $\C(\Lambda_1)$. By Proposition \ref{:allWeier}, it can be written as a rational function of $\wp(z; \Lambda_1)$ and $\wp'(z; \Lambda_1)$, i.e. an isogeny. Thus, the map is surjective.
    \end{enumerate}
\end{proof}

Note that we get the following corollary for free:

\begin{corollary}
    Let $E_1/\C$ and $E_2/\C$ be elliptic curves corresponding to lattices $\Lambda_1$ and $\Lambda_2$, respectively. Then, $E_1$ and $E_2$ are isomorphic over $\C$ if and only if $\Lambda_1$ and $\Lambda_2$ are homothetic, i.e. $\Lambda_1 = \alpha \Lambda_2$. 
\end{corollary}

Putting these results together, we have shown that:

\begin{theorem}
    The following categories are equivalent: 
    \begin{enumerate}
        \item Elliptic curves over $\C$ with isogenies.
        \item Complex tori $\C/\Lambda$ with holomorphic maps taking $0$ to $0$. 
        \item Lattices $\Lambda \subset \C$ up to homothety with maps:
        \[ \Hom(\Lambda_1, \Lambda_2) = \{\alpha \in \C: \alpha \Lambda_1 \subset \Lambda_2\}\]
    \end{enumerate}
\end{theorem}


\section{Complex Multiplication for Elliptic Curves}

Now that we have established the equivalence, we can tackle the problem directly.
\begin{definition}
    Let $K$ be a number field. An order $R$ of $K$ is a subring of $K$ that is a finitely generated $\Z$-module and satisfies $R \otimes \Q = K$. 
\end{definition}

\begin{prop}
    Let $E/\C$ be an elliptic curve with associated lattice $\Lambda = \Z \oplus \tau\Z$. Then, one of the following is true:
    \begin{enumerate}
        \item $\End(E) = \Z$. 
        \item The field $\Q(\tau)$ is an imaginary quadratic extension of $\Q$ and $\End(E)$ is isomorphic to an order in $\Q(\tau)$. 
    \end{enumerate}
\end{prop}
\begin{proof}
    Note that we have:
    \[ \End(E) \cong R := \{\alpha: \alpha \Lambda \subset \Lambda\}\]
    Thus, there are integers $a, b, c, d$ such that:
    \[ \alpha = a + b \tau \qquad \alpha\tau = c+d\tau\]
    Eliminating $\tau$, we get: 
    \[\alpha^2 - (a+d)\alpha + (ad-bc) = 0\]
    Thus, $R$ is an integral extension of $\Z$. \\
    If $R \ne \Z$, pick $\alpha \in R\setminus \Z$. Then, $b \neq 0$, so eliminating $\alpha$ gives: 
    \[ b\tau^2 - (a-d)\tau -c = 0\]
    Thus, $\Q(\tau)$ is an imginary quadratic extension of $\Q$ as $\tau \not \in \R$. Then, since $R$ is integral over $\Z$, $R$ is an order in $\Q(\tau)$.
\end{proof}

Thus, our definition for complex multiplication becomes:

\begin{definition}
    An elliptic curve $E/\C$ is said to have complex multiplication (CM) if $\End(E)$ is an order $R$ of an imaginary quadratic field.
\end{definition}

Next, we consider all elliptic curves with isomorphic endomorphism rings.

\begin{definition}
    We denote by $\Ell(R)$ the set of elliptic curves $E$ with $\End(E) \cong R$ up to isomorphism.
\end{definition}

It is natural to ask whether one can construct an elliptic curve in $\Ell(R)$ for a given $R$. Let $\alpha$ be a non-zero fractional ideal of $K$, an imginary quadratic extension of $\Q$, ($\Z$ module of rank $2$ not contained in $\R$), then $\alpha$ is a lattice in $\C$. Then, we have:
\begin{align*}
    \End(E_\alpha) &\cong \{ z \in \C : z\alpha \subset \alpha \} \\
    &= \{ z \in K : z\alpha \subset \alpha \} \\
    &= O_K \qquad (\text{$\alpha$ is a fractional ideal})
\end{align*}
Since we care about lattices up to homothety, we define: 
\begin{definition}
    Let:
    \begin{align*}
        \text{CL}(O_K) &= \frac{\{ \text{non-zero fractional ideals of } O_K \}}{ \{\text{non-zero principal ideals of $O_K$}\}}
    \end{align*}
\end{definition}

Then, there is a map: 
\begin{align*}
    \text{CL}(O_K) &\to \Ell(O_K) \\
    \alpha &\mapsto E_\alpha
\end{align*}
Moreover, define: 
\[\alpha\Lambda = \left\{\sum_{i=1}^n\alpha_i\lambda_i : \alpha_i \in \alpha, \lambda_i \in \Lambda\right\}\]


\begin{prop}
    There is a well-defined simply transitive action $[\alpha] \ast \Lambda = E_{\alpha^{-1}\Lambda}$ of $\text{CL}(O_K)$ on $\Ell(O_K)$. In particular, 
    \[ \#\text{CL}(O_K) = \#\Ell(O_K) \]
\end{prop}
\begin{proof}
    Lots of definition chasing. Look at [ATAEC II.1.2] for details.
\end{proof}

\begin{lemma}
    Let $E/\C \in \Ell(O_K)$. Then $j(E) \in \overline{\Q}$. 
\end{lemma}
\begin{proof}
    First, note that if $\phi: E \to E$ is an endomorphism of $E$, then $\phi^\sigma: E^\sigma \to E^\sigma$ is an endomorphism of $E^\sigma$. Thus, \[ \End(E^\sigma) \cong \End(E)\] 
    Next, let $\sigma \in \Aut(\C)$. Then $E^\sigma$ is obtained by letting $\sigma$ act on the coefficients of a Weierstrauss equation for $E$, and $j(E)$ is a rational combination of those coefficients, so it is clear that: 
    \[ j(E^\sigma) = j(E)^\sigma\]
    The previous theorem implies that $\End(E^\sigma) \cong O_K$. Then, the last proposition implies that $E^\sigma$ is in one of finitely many $\C$-isomorphism classes of elliptic curves. Thus, $j(E^\sigma)$ can take on finitely many values as $\sigma$ ranges over $\Aut(\C)$. Therefore, $[\Q(j(E)):\Q]$ is finite, and $j(E) \in \overline{\Q}$. (Why?)
\end{proof}

Finally, we prove the theorem that we set out to prove:

\begin{theorem}
    \[\Ell(O_K) \cong \Ell_{\overline{\Q}}(O_K) := \frac{\{E/\overline{\Q} : \End(E) \cong O_K\}}{\text{isomorphism over $\overline{\Q}$}}\]
\end{theorem}

\begin{proof}
    Fixing an embedding $\overline{\Q} \subset \C$, there is a natural map: 
    \[ \epsilon : \Ell_\Q(O_K) \to \Ell_\C(O_K)\]
    We need to show that $\epsilon$ is a bijection. \bbni
    To show surjection, note the following: 
    \begin{itemize}
        \item $j(E) \in \overline{\Q}$ from (b). 
        \item There is an elliptic curve $E'/\Q(j(E))$ with $j(E') = j(E)$ (Proposition 1.2).
        \item $E'$ is isomorphic to $E$ over $\C$ from [AEC III.1.4b].
    \end{itemize}
    Thus, $\epsilon(E') = E$. \bbni
    Next let $E', E \in \Ell_\Q(O_K)$ be such that $\epsilon(E') = \epsilon(E)$. Then, $j(E') = j(E)$. Thus, $E_1$ and $E_2$ are isomorphic over $\overline{\Q}$ (Proposition 1.2). \\
    Thus, $\epsilon$ is bijective. Hence, every elliptic curve $E/\C$ with $\End(E) \cong O_K$ is isomorphic to an elliptic curve $E'/\overline{\Q}$ with $\End(E') \cong O_K$.
\end{proof}


\section{CM for Abelian Varieties}
The following discussion is just me listing theorems I've read in Shimura's book, in trying to understand the general picture of CM for Abelian varieties. I will complete this at some point over the summer. 
\begin{definition}
    A morphism between abelian varieties $f: A \to B$ is a rational map that respects the group structure. If $f$ is birrational, it is also biregular, hence an isomorphism.
\end{definition}

\begin{definition}
    For an abelian variety $A/k$, $\End(A)$ is a free $\Z$-module of finite rank and every element is defined over a separably algebraic extension of $k$. Let $\End_\Q(A) = \End(A) \otimes_\Z \Q$. Then, $\End_\Q(A)$ is a $\Q$-algebra and $\End(A)$ is an order in $\End_\Q(A)$.
\end{definition}

The category of abelian varieties over $K$ is a semisimple category, i.e. every abelian variety is isogenous to a direct sum of simple abelian varieties.

\begin{prop}
    Let $B$ be a simple abelian variety and $K$ the center of $\End_\Q(B)$. Then $K$ is a totally real number field or a totally imaginary quadratic extention of a totally real number field.
\end{prop}

\begin{definition}
    Let $R$ be an algebra over $\Q$ with an identity element $1$. An abelian variety of type $R$ is a pair $(A, \iota)$ where $\iota$ is an isomorphism of $R$ into $\End_\Q(A)$ such that $\iota(1) = 1_A$.    
\end{definition}

Let $(A, \iota)$ be an abelian variety of type $F$ and $n = \dim(A)$. Then, one can show that $[F:\Q]$ divides $2n$. Assume $[F:\Q] = 2n$ and that $A$ is defined over $\C$. Then, as $A$ is a complex torus, we have a rational representation $M$ of degree $2n$ and an analytic representation $S$ of degree $n$ of $\End_\Q(A)$. Restricting to $F$, $M$ is the direct sum of $\phi_1, \cdots, \phi_{2n}$, all the representations of $F$ in $\C$. $S$ is equivalent to the direct sum of $\phi_1, \cdots \phi_n$, and $\overline{S}$ is equivalent to the direct sum of $\overline{\phi_1}, \cdots, \overline{\phi_n}$ which are $\phi_{n+1}, \cdots, \phi_{2n}$ in some order. In this case, we say that $(A, \iota)$ is of type $(F, \{\phi_i\})$.

\begin{prop}
    In order for $(F, \{\phi_i\})$ to be a CM-type, it is necessary and sufficient that $F$ contains two subfields $K$ and $K_0$ satisfying:
    \begin{enumerate}
        \item $K_0$ is totally real and $K$ is totally imginary quadratic extension of $K_0$. 
        \item There are no two isomorphisms among the $\phi_i$ which are complex conjugate to each other on $K$. 
    \end{enumerate}
\end{prop}

% For elliptic curve $E$ over $\C$, we noted that $E \cong \C/\Lambda$ for some $\Lambda \subset \C$. For abelian variety $A$, we note that $A \cong \C^g/\Lambda$ for some $\Lambda \subset \C^g$, but not every $\C^g/\Lambda$ is an abelian variety. However, $\C^n/\Lambda$ with a polarization is an abelian variety.

% The analytic representation of $\End^0(A)$ is given by: 
% \[ \End^0(A) \cong \{M \in \mathcal M_g(\C) : M\Q\Lambda \subset \Q\Lambda\}\]
% as it is a $g$-dimensional complex representation of $\End^0(A)$. Since $\R\Lambda = \C^n$, any $\C$-linear endomorphism  that is identity on $\Q\Lambda$ is identity on the whole of $\C^n$. Hence, $\Q\Lambda$ is a faithful $\End^0(A)$-module. Thus, 
% \[ [\End^0(A): \Q]_{red} \leq \dim_\Q \Q\Lambda = 2\dim A\]
% \begin{definition}
%    An abelian variety $A/\C$ has complex multiplication if:
%    \[ [\End^0(A): \Q]_{red} = 2\dim A\]
% \end{definition}





% \begin{prop}
%     If $\dim(A) = \dim(B)$ then there exists a morphism $A \to B$ if and only if there exists a morphism $B \to A$. We call such a morphism an isogeny. 
% \end{prop}






% For elliptic curve $E$ over $\C$, we noted that $E \cong \C/\Lambda$ for some $\Lambda \subset \C$. For abelian variety $A$, we note that $A \cong \C^g/\Lambda$ for some $\Lambda \subset \C^g$, but not every $\C^g/\Lambda$ is an abelian variety. An complex torus admitting a polatization is an abelian variety.



\begin{theorem}(Shimura, 12, Prop. 26)
    Let $(F; \{\phi\})$ be a CM-type and $(A, \iota)$ ab abelian variety of type $(F; \{\phi\})$. Then, there exists an abelian variety of type $(F; \{\phi_i\})$ isomorphic to $(A, \iota)$, defined over an algebraic number field of finite degree.
\end{theorem}


% Lets fill out the proofs in reverse. Writing notes here.
% \begin{enumerate}
%     \item The isomorphism theorem directly requires [AEC III.1.4bc] and $j(E) \in \overline{\Q}$.
%     \item $j(E) \in \overline{\Q}$ has one ??? and requires [AEC III.1.4b] and $(1.2b)$.
%     \item (1.2b) requires defining class field, understanding fractional/principle ideals, and the action of $\text{CL}(O_K)$ on $\Ell(O_K)$. This is a somewhat longer proof. Relies on [AEC VI.4.1.1], [AEC VI.5.5].
% \end{enumerate}




% \begin{theorem}
%     For a lattice $\Lambda \subset \C$, the curve $E_\Lambda$ is given by: 
%     \[ E_\Lambda = \begin{cases}
%         [\wp(z), \wp'(z): 1] & z \not \in \C/\Lambda \\
%         [0:1:0] & z \in \C/\Lambda
%     \end{cases} \]
%     where:
%     \[ \wp(z) = \frac{1}{z^2} + \sum_{\omega \in \Lambda^*} \left(\frac{1}{(z-\omega)^2} - \frac{1}{\omega^2}\right)\]
%     and if $E_\Lambda := y^2 = 4x^3 + g_2x + g_3$, then:
%     \[ g_2 = 60\sum_{\omega \in \Lambda^*} \frac{1}{\omega^4}, \quad g_3 = 140\sum_{\omega \in \Lambda^*} \frac{1}{\omega^6} \] 
% \end{theorem}

% \begin{theorem}
%     Let $\Lambda \subset \C$ be a lattice. 
%     \begin{itemize}
%         \item The Eisenstein series $G_{2k}(\Lambda)$ is absolutely convergent for all $k \ge 1$. 
%         \item The series definiting $\wp$-function converges absolutely and uniformly on every compact subset of $\C \setminus \Lambda$. The series defines a meromorphic function on $\C$ having a double pole with residue $0$ at each lattice point and no other poles. 
%         \item The Weierstrauss $\wp$-function is an even elliptic function. 
%     \end{itemize}
% \end{theorem}

% \begin{theorem}
%     Let $\Lambda \subset \C$ be a lattice. Then: 
%     \[ \C(\Lambda) = \C(\wp(z), \wp'(z))\]
%     i.e. every elliptic function is a rational combination of $\wp(z)$ and $\wp'(z)$.
% \end{theorem}

% \begin{remark}
%     The multiplication by $m$ map $[m]: E \to E$ is an endomorphism of $E$ for any $m \in \Z$. This corresponds to scaling the complex torus by $n$, as we will see. 
% \end{remark}


% \begin{definition}
%     If an elliptic curve $E/\C$ has complex multiplication $\phi(z) = mz$ for some $m \in \H$, then we call $m$ a CM point of $E$.
% \end{definition}

% \begin{theorem}
%     If $m$ is a CM point of $E_\Lambda/\C$, then $m\Lambda \subset \Lambda$ and $m^2\Lambda \subset \Lambda$. Thus, $m$ satisfies a degree $2$ polynomial.   
% \end{theorem}



\end{document}