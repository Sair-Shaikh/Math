\documentclass[12pt]{article}

\usepackage{fullpage}
\usepackage{mdframed}
\usepackage{colonequals}
\usepackage{algpseudocode}
\usepackage{algorithm}
\usepackage{tcolorbox}
\usepackage[all]{xy}
\usepackage{proof}
\usepackage{mathtools}
\usepackage{bbm}
\usepackage{amssymb}
\usepackage{amsthm}
\usepackage{amsmath}
\usepackage{amsxtra}
\newcommand{\bb}{\mathbb}


\newtheorem{theorem}{Theorem}[section]
\newtheorem{corollary}{Corollary}[theorem]
\newtheorem{lemma}{Lemma}

\newcommand{\mathcat}[1]{\textup{\textbf{\textsf{#1}}}} % for defined terms

\newenvironment{problem}[1]
{\begin{tcolorbox}\noindent\textbf{Problem #1}.}
{\vskip 6pt \end{tcolorbox}}

\newenvironment{enumalph}
{\begin{enumerate}\renewcommand{\labelenumi}{\textnormal{(\alph{enumi})}}}
{\end{enumerate}}

\newenvironment{enumroman}
{\begin{enumerate}\renewcommand{\labelenumi}{\textnormal{(\roman{enumi})}}}
{\end{enumerate}}

\newcommand{\defi}[1]{\textsf{#1}} % for defined terms

\theoremstyle{remark}
\newtheorem*{solution}{Solution}

\setlength{\hfuzz}{4pt}

\newcommand{\calC}{\mathcal{C}}
\newcommand{\calF}{\mathcal{F}}
\newcommand{\C}{\mathbb C}
\newcommand{\N}{\mathbb N}
\newcommand{\Q}{\mathbb Q}
\newcommand{\R}{\mathbb R}
\newcommand{\Z}{\mathbb Z}
\newcommand{\F}{\mathbb F}
\newcommand{\br}{\mathbf{r}}
\newcommand{\RP}{\mathbb{RP}}
\newcommand{\CP}{\mathbb{CP}}
\newcommand{\nbit}[1]{\{0, 1\}^{#1}}
\newcommand{\bits}{\{0, 1\}^{n}}
\newcommand{\bbni}{\bigbreak \noindent}
\newcommand{\norm}[1]{\left\vert\left\vert#1\right\vert\right\vert}
\newcommand{\dbar}{\overline{\partial}}
\let\d\relax
\let\calF\relax
\newcommand{\d}{\partial}
\newcommand{\calO}{\mathcal{O}}
\newcommand{\calF}{\mathcal{F}}
\newcommand{\calG}{\mathcal{G}}
\newcommand{\calH}{\mathcal{H}}
\newcommand{\calE}{\mathcal{E}}

\let\1\relax
\newcommand{\1}{\mathbf{1}}
\newcommand{\fr}[2]{\left(\frac{#1}{#2}\right)}

\newcommand{\vecz}{\mathbf{z}}
\newcommand{\vecr}{\mathbf{r}}
\DeclareMathOperator{\Cinf}{C^{\infty}}
\DeclareMathOperator{\Id}{Id}

\DeclareMathOperator{\Alt}{Alt}
\DeclareMathOperator{\ann}{ann}
\DeclareMathOperator{\codim}{codim}
\DeclareMathOperator{\End}{End}
\DeclareMathOperator{\Hom}{Hom}
\DeclareMathOperator{\id}{id}
\DeclareMathOperator{\M}{M}
\DeclareMathOperator{\Mat}{Mat}
\DeclareMathOperator{\Ob}{Ob}
\DeclareMathOperator{\opchar}{char}
\DeclareMathOperator{\opspan}{span}
\DeclareMathOperator{\rk}{rk}
\DeclareMathOperator{\sgn}{sgn}
\DeclareMathOperator{\Sym}{Sym}
\DeclareMathOperator{\tr}{tr}
\DeclareMathOperator{\img}{img}
\DeclareMathOperator{\CandE}{CandE}
\DeclareMathOperator{\CandO}{CandO}
\DeclareMathOperator{\argmax}{argmax}
\DeclareMathOperator{\first}{first}
\DeclareMathOperator{\last}{last}
\DeclareMathOperator{\cost}{cost}
\DeclareMathOperator{\dist}{dist}
\DeclareMathOperator{\path}{path}
\DeclareMathOperator{\parent}{parent}
\DeclareMathOperator{\argmin}{argmin}
\DeclareMathOperator{\excess}{excess}
\let\Pr\relax
\DeclareMathOperator{\Pr}{\mathbf{Pr}}
\DeclareMathOperator{\Exp}{\mathbb{E}}
\DeclareMathOperator{\Var}{\mathbf{Var}}
\let\limsup\relax
\DeclareMathOperator{\limsup}{limsup}
%Paired Delims
\DeclarePairedDelimiter\ceil{\lceil}{\rceil}
\DeclarePairedDelimiter\floor{\lfloor}{ \rfloor}


\newcommand{\dagstar}{*}

\newcommand{\tbigwedge}{{\textstyle{\bigwedge}}}
\setlength{\parindent}{0pt}
\setlength{\parskip}{5pt}


\begin{document}

\title{CS 40: Computational Complexity}

\author{Sair Shaikh}
\maketitle

Collaboration Notice: Talked to Henry Scheible '26 to discuss ideas.


\begin{problem}{1}
    (1.3.18) For a path-connected, locally path-connected, and semilocally simply connected space $X$, call a path-connected covering $p \colon E \to X$ \emph{abelian} if it is normal and has abelian deck transformation group. Show that $X$ has an abelian covering space that is a covering space of every other abelian covering space of $X$ and that such a `universal' abelian covering space is unique up to equivalence. Describe this covering space explicitly for $X = S^1 \vee S^1$ and $S^1 \vee S^1 \vee S^1$. 
\end{problem}

\begin{solution}
    Since $X$ is path-connected, locally path-connected, and semilocally simply connected, we note that it it has a universal cover $\tilde\rho: \tilde{B} \to X$. Let $H \subseteq G := \pi_1(X, x_0)$ be the commutator (i.e. generated by elements $[g, h]$ for $g, h \in G$). By the existence of covers theorem, there exists a covering space $\rho: (E, e_0) \to (X, x_0)$ such that $\rho_*(\pi_1(E, e_0)) = H$. We claim that $(E, \rho)$ is the unique universal abelian covering space of $X$. \bbni
    Note that since $H$ is the commutator subgroup, it is normal. To see this, let $[a,b] \in H$ be a generator, and $g \in G$. Then, 
    \begin{align*}
        g[a,b]g^{-1} &= ga^{-1}b^{-1}abg^{-1} \\
        &= ga^{-1}(g^{-1}g)b^{-1}(g^{-1}g)a(g^{-1}g)bg^{-1} \\
        &= (ga^{-1}g^{-1})(gb^{-1}g^{-1})(gag^{-1})(gbg^{-1}) \\
        &= (gag^{-1})^{-1}(gbg^{-1})^{-1}(gag^{-1})(gbg^{-1}) \\
        &= [gag^{-1}, gbg^{-1}] \in H
    \end{align*}
    Thus, $H$ is normal in $G$ and $(E, \rho)$ is a normal covering space. \bbni
    Moreover, by the normal covering theorem, we know that the deck transformation group is equal to $G/H$. However, since $H$ is the commutator, we have that $G/H$ is abelian (by definition of the commutator). Thus, $(E, \rho)$ is an abelian covering space. \bbni
    Moreover, if $(E', \rho')$ was another normal cover corresponding to $H' \subseteq G$ with abelian deck transformation group $G/H'$, then since $G/H'$ is abelian, we must have that $H \subseteq H'$ (the commutator must be quotiented out for the result to be abelian). Thus, we have that:
    \[ \rho_*(\pi_1(E, e_0)) \subseteq \rho_*'(\pi_1(E', e_0'))\]
    Since $E$ is path-connected and locally path-connected (as it is a cover of locally path-connected $X$), we can apply the general lifting theorem to get a map $f: (E, e_0) \to (E', e_0')$ such that:
    \[ \rho' \circ f = \rho \]
    By functoriality, we have that:
    \[ \rho'_* \circ f_* = \rho_* \]
    Since $\rho_*$ is injective, we have that $f_*$ is injective. Thus, 
    \[ f_*(\pi_1(E, e_0)) \subseteq \pi_1(E', e_0')\]
    is a subgroup. Thus, by the Galois correspondence, we have that $f: (E, e_0) \to (E', e_0')$ is a covering map. \bbni 
    Uniqueness follows directly from the universal property. If $A, B$ are two universal abelian covers, then by the universal property, there exists unique covering maps $f: A \to B$ and $g: B \to A$ that commute with the covering maps of $A$ and $B$. However, then $g \circ f$ is a covering map from $A$ to itself. By the uniqueness of lifts, we must have $g \circ f = \id_A$. Similarly, we have that $f \circ g = \id_B$. Thus, $f$ and $g$ are homeomorphisms, and the universal abelian cover is unique up to equivalence. \bbni 
    For $X = S^1 \vee S^1$, we have that $\pi_1(S^1 \vee S^1) = \Z * \Z$. The commutator subgroup $H$ is generated by the element $[a, b]$. Thus, we want a cover of $S^1 \vee S^1$ corresponding to $H$. Let $E = \R^2$, where we each integer interval on the $x$-axis corresponds to $a$ and on the $y$-axis corresponds to $b$. 
\end{solution}
\newpage

\begin{problem}{2}
    (1.3.20) Construct non-normal covering spaces of a Klein bottle by a Klein bottle and by a torus. 
\end{problem}

\begin{solution}
    
\end{solution}
\newpage

\begin{problem}{3}
    (1.3.29) Let $Y$ be path-connected, locally path-connected, and simply connected. Let $G_1$ and $G_2$ be two subgroups of $\mathrm{Homeo}(Y)$ defining covering space actions on $Y$. Show that the orbit spaces $Y/G_1$ and $Y/G_2$ are homeomorphic if and only if $G_1$ and $G_2$ are conjugate subgroups of $\mathrm{Homeo}(Y)$. 
\end{problem}

\begin{solution}
\end{solution}
\newpage

\begin{problem}{4}
    (2.1.10) Show that if $A$ is a retract of $X$, then the map $H_n(A) \to H_n(X)$ induced by the inclusion of $A$ in $X$ is injective for all $n$. 
\end{problem}

\begin{solution}
\end{solution}
\newpage

\begin{problem}{5}
    (2.1.11) Show that chain homotopy is an equivalence relation on the set of chain maps between two chain complexes.
\end{problem}

\begin{solution}
\end{solution}

\end{document}