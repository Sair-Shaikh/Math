\documentclass[12pt]{article}


\usepackage{fullpage}
\usepackage{mdframed}
\usepackage{colonequals}
\usepackage{algpseudocode}
\usepackage{algorithm}
\usepackage{tcolorbox}
\usepackage[all]{xy}
\usepackage{proof}
\usepackage{mathtools}
\usepackage{bbm}
\usepackage{amssymb}
\usepackage{amsthm}
\usepackage{amsmath}
\usepackage{amsxtra}
\newcommand{\bb}{\mathbb}


\newtheorem{theorem}{Theorem}[section]
\newtheorem{corollary}{Corollary}[theorem]
\newtheorem{lemma}{Lemma}

\newcommand{\mathcat}[1]{\textup{\textbf{\textsf{#1}}}} % for defined terms

\newenvironment{problem}[1]
{\begin{tcolorbox}\noindent\textbf{Problem #1}.}
{\vskip 6pt \end{tcolorbox}}

\newenvironment{enumalph}
{\begin{enumerate}\renewcommand{\labelenumi}{\textnormal{(\alph{enumi})}}}
{\end{enumerate}}

\newenvironment{enumroman}
{\begin{enumerate}\renewcommand{\labelenumi}{\textnormal{(\roman{enumi})}}}
{\end{enumerate}}

\newcommand{\defi}[1]{\textsf{#1}} % for defined terms

\theoremstyle{remark}
\newtheorem*{solution}{Solution}

\setlength{\hfuzz}{4pt}

\newcommand{\calC}{\mathcal{C}}
\newcommand{\calF}{\mathcal{F}}
\newcommand{\C}{\mathbb C}
\newcommand{\N}{\mathbb N}
\newcommand{\Q}{\mathbb Q}
\newcommand{\R}{\mathbb R}
\newcommand{\Z}{\mathbb Z}
\newcommand{\br}{\mathbf{r}}
\newcommand{\RP}{\mathbb{RP}}
\newcommand{\CP}{\mathbb{CP}}
\newcommand{\nbit}[1]{\{0, 1\}^{#1}}
\newcommand{\bits}{\{0, 1\}^{n}}
\newcommand{\bbni}{\bigbreak \noindent}
\newcommand{\norm}[1]{\left\vert\left\vert#1\right\vert\right\vert}

\let\1\relax
\newcommand{\1}{\mathbf{1}}
\newcommand{\fr}[2]{\left(\frac{#1}{#2}\right)}

\newcommand{\vecz}{\mathbf{z}}
\newcommand{\vecr}{\mathbf{r}}
\DeclareMathOperator{\Cinf}{C^{\infty}}
\DeclareMathOperator{\Id}{Id}

\DeclareMathOperator{\Alt}{Alt}
\DeclareMathOperator{\ann}{ann}
\DeclareMathOperator{\codim}{codim}
\DeclareMathOperator{\End}{End}
\DeclareMathOperator{\Hom}{Hom}
\DeclareMathOperator{\id}{id}
\DeclareMathOperator{\M}{M}
\DeclareMathOperator{\Mat}{Mat}
\DeclareMathOperator{\Ob}{Ob}
\DeclareMathOperator{\opchar}{char}
\DeclareMathOperator{\opspan}{span}
\DeclareMathOperator{\rk}{rk}
\DeclareMathOperator{\sgn}{sgn}
\DeclareMathOperator{\Sym}{Sym}
\DeclareMathOperator{\tr}{tr}
\DeclareMathOperator{\img}{img}
\DeclareMathOperator{\CandE}{CandE}
\DeclareMathOperator{\CandO}{CandO}
\DeclareMathOperator{\argmax}{argmax}
\DeclareMathOperator{\first}{first}
\DeclareMathOperator{\last}{last}
\DeclareMathOperator{\cost}{cost}
\DeclareMathOperator{\dist}{dist}
\DeclareMathOperator{\path}{path}
\DeclareMathOperator{\parent}{parent}
\DeclareMathOperator{\argmin}{argmin}
\DeclareMathOperator{\excess}{excess}
\let\Pr\relax
\DeclareMathOperator{\Pr}{\mathbf{Pr}}
\DeclareMathOperator{\Exp}{\mathbb{E}}
\DeclareMathOperator{\Var}{\mathbf{Var}}
\let\limsup\relax
\DeclareMathOperator{\limsup}{limsup}
%Paired Delims
\DeclarePairedDelimiter\ceil{\lceil}{\rceil}
\DeclarePairedDelimiter\floor{\lfloor}{ \rfloor}


\newcommand{\dagstar}{*}

\newcommand{\tbigwedge}{{\textstyle{\bigwedge}}}
\setlength{\parindent}{0pt}
\setlength{\parskip}{5pt}



\begin{document}


\title{CS 40: Computational Complexity}

\author{Sair Shaikh}
\maketitle

% Collaboration Notice: Talked to Henry Scheible '26 to discuss ideas.




\begin{problem}{1}
\begin{defn}
    A \emph{topological group}  $G$ is a group with a topology such that the maps $G \times G \to G$ given $(g,h) \mapsto g * h$ where $*$ is the group operation and $G \to G$ given by $g \mapsto g^{-1}$ are continuous. 
\end{defn}
    Let $G$ be a topological group with identity element $x_0$.
    \begin{enumerate}
    
    \item Let $C$ be the connected component of $G$ containing the identity element $x_0$. Show that $C$ is a normal subgroup and every other connected component is homemorphic to $C$. (\emph{Hint:} Show that if $g \in G$, then $gC$ is the component of $G$ containing $g$.)

    \item Let $\Omega(G,x_0)$ be the set of all loops in $G$ based at $x_0$. Define an operation $f \times g$ for $f, g \in \Omega(G,x_0)$ by
    \[ (f \times g)(s) = f(s) \ast g(s). \]
    Show that the operation $\times$ makes $\Omega(G,x_0)$ into a group. 

    \item Show that $\times$ induces a group operation on $\pi_1(G,x_0)$.

    \item Show that $\times$ agrees with the usual concatenation group operation $\cdot$ on $\pi_1(G,x_0)$. (\emph{Hint:} Compute $(f \cdot e_{x_0}) \ast (e_{x_0} \cdot g)$. )

    \item Show that $\pi_1(G,x_0)$ is an abelian group.

    \item Let $A$ be a (possibly empty) finite set of points in $\R^2$. For what values of $|A|$ can $\R^2 \setminus A$ be given the structure of a topological group? 
    \end{enumerate}
\end{problem}
\newpage


\begin{problem}{2}(50 points) 
\begin{defn}
    The \emph{real Grassmannian}  $\mathrm{Gr}(k,n)$ is the space of $k$-dimensional subspaces of $\R^n$. More precisely, let $M^*(k,n)$ be the set of $k \times n$ matrices of rank $k$ with the subspace topology in the space of all $k \times n$ matrices $M(k,n) \cong \R^{kn}$. Then, we define $\mathrm{Gr}(k,n) = M^*(k,n)/\sim$ where $A \sim B$ if and only if $A$ and $B$ have the same row space. 
\end{defn}
    Note that $\mathrm{Gr}(1,n) = \mathbb{RP}^{n-1}$. 
    \begin{enumerate}
    \item Show that $\mathrm{Gr}(k,n)$ is homeomorphic to $\mathrm{Gr}(n-k,n)$ for all $0 \leq k \leq n$.
    \item Show that $\mathrm{Gr}(2,4)$ is homeomorphic to $S^2 \times S^2/\sim$ where $(x,y) \sim (-x, -y)$ for all $(x,y) \in S^2 \times S^2$. 
    \item Describe a CW complex structure on $\mathrm{Gr}(2,4)$. (\emph{Hint:} You may wish to use the previous part.)
    \item Compute $\pi_1(\mathrm{Gr}(2,4))$. 
    \item Compute the homology of $\mathrm{Gr}(2,4)$. 
    \end{enumerate}
\end{problem}
\newpage

\begin{problem}{3}(50 points) 
    
    \begin{defn} Let $\mathcal{C}, \mathcal{D}$ be (locally small) categories and let $F, G: \mathcal{C} \to \mathcal{D}$ be functors. 
    \begin{itemize}

    \item A \emph{natural transformation} $\eta$ from $F$ to $G$ is the data of a morphism $\eta_X  \in \hom_{\mathcal{D}} (F(X), G(X))$ for every object $X$ in $\mathcal{C}$ such that for every $f \in \hom_\mathcal{C}(X,Y)$, $\eta_Y \circ F(f) = G(f) \circ \eta_X$. 

    \item A \emph{natural isomorphism} $\eta$ from $F$ to $G$ is a natural transformation such that $\eta_X$ is an isomorphism for every object $X$ of $\mathcal{C}$. If such an $\eta$ exists, the functors $F$ and $G$ are said to be \emph{naturally isomorphic}. 

    \item The functor $F:  \mathcal{C} \to \mathcal{D}$ is an \emph{equivalence of categories} if there is a functor $F':\mathcal{D} \to \mathcal{C}$ such that $F' \circ F$ is naturally isomorphic to the identity functor on $\mathcal{C}$ and $F \circ F'$ is naturally isomorphic to the identity functor on $\mathcal{D}$. If such a functor exists, $\mathcal{C}$ and $\mathcal{D}$ are said to be equivalent.
    \end{itemize}

    \end{defn}
    
    For a topological space $X$, let $\Pi(X)$ be the fundamental groupoid of $X$. 
    \begin{enumerate}
 
    \item Show that if $F: \mathcal{C} \to \mathcal{D}$ is an equivalence of categories and $X$ and $Y$ are objects of $\mathcal{C}$ such that $F(X)$ and $F(Y)$ are isomorphic, then $X$ and $Y$ are isomorphic. Deduce that the forgetful functor $F: \text{Top} \to \text{Sets}$ and the fundamental group functor $\pi_1: \text{Top} \to \text{Groups}$ are not equivalences of categories.
 
    \item Show that if $X$ is path connected, the inclusion $\pi_1(X,x) = \hom_{\Pi(X)}(x,x) \to \Pi(X)$ is an equivalence of categories for every $x \in X$.
 
    \item Show that any continuous function $f: X \to Y$ induces a functor $\Pi(f): \Pi(X) \to \Pi(Y)$.
 
    \item Show that if $f: X \to Y$ and $g: X \to Y$ are continuous and homotopic, then the functors $\Pi(f)$ and $\Pi(g)$ are naturally isomorphic. 

    \item Deduce that if $X$ and $Y$ are homotopy equivalent, then $\Pi(X)$ and $\Pi(Y)$ are equivalent. Use this to reprove the fact from class that if $f: X \to Y$ is a homotopy equivalence, then $f_*: \pi_1(X, x) \to \pi_1(Y,f(x))$ is an isomorphism.

    \item Give an example of homotopic continuous functions $f$ and $g$ such that $\Pi(f)$ and $\Pi(g)$ are not equal.
    \end{enumerate}
\end{problem}
\newpage
 
\begin{problem}{4}(50 points) 
    \begin{defn}
        Given an open cover $\mathcal{U} = \{ U_i \}_{i \in I}$ of a topological space $X$, we define the \v{C}ech complex as
        \[ \check{C}^k(X, \mathcal{U}) = \prod_{(i_0, \hdots, i_k) \in I^{k+1}} \mathcal{F}(U_{i_0} \cap \hdots \cap U_{i_k}) \]
        where $\mathcal{F}(Y) = \{ f \colon Y \to \Z : f \text{ is continuous}\}$ where $\Z$ has the discrete topology. For $f \in \check{C}^k(X, \mathcal{U})$, denote by $f_{i_0, \hdots, i_k}$ the component in $\mathcal{F}(U_{i_0} \cap \hdots \cap U_{i_k})$. Then, define 
        \[ d \colon \check{C}^k(X, \mathcal{U}) \to \check{C}^{k+1} (X, \mathcal{U}) \]
        by
        \[ (df)_{i_0, \hdots, i_{k+1}} = \sum_{j=0}^{k+1} (-1)^j f_{i_0, \hdots, \hat{i_j}, \hdots i_k}|_{U_{i_0} \cap \hdots \cap U_{i_k}} \]
    \end{defn}
    \begin{enumerate}
    
    \item Verify that the \v{C}ech complex is a cochain complex for any $X, \mathcal{U}$. 

    \item Denoting the cohomology groups of the \v{C}ech complex by $\check{H}^k(X, \mathcal{U})$, describe $\check{H}^0(X, \mathcal{U})$ for any $X, \mathcal{U}$.

    \item Find two covers of $S^1$ whose \v{C}ech cohomology groups do not coincide. 

    \item A cover $\mathcal{V} = \{V_j\}_{j \in J}$ is a \emph{refinement} of $\mathcal{U} = \{ U_i \}_{i \in I}$ if for every $j \in J$, there is an $i \in I$ such that $V_j \subseteq U_i$. Show that if $\mathcal{V}$ is a refinement of $\mathcal{U}$, then there is a chain map $\check{C}^\bullet (X, \mathcal{U} ) \to \check{C}^\bullet (X, \mathcal{V})$ given by restriction. 
    \end{enumerate}

    We say that $\mathcal{U}$ is a \emph{good cover} if every non-empty intersection $U_{i_0} \cap \hdots \cap U_{i_k}$ is contractible. Here is a theorem that can be proved using a similar argument that we deployed for barycentric subdivision in singular homology once one proves that the \v{C}ech cohomology of a contractible space with respect to any cover agrees with the \v{C}ech cohomology of point, e.g., by proving homotopy invariance.

    \begin{theorem} 
    If $\mathcal{U}$ is a good cover of $X$ and the $\mathcal{V}$ is a refinement of $\mathcal{U}$, the chain map from the previous part induces an isomorphism on \v{C}ech cohomology. Moreover, $\check{H}^k(X, \mathcal{U}) \cong H^k(X)$ for all $k$. 
    \end{theorem} 

    \begin{enumerate}[resume]
        \item Find good covers and explicitly compute the cohomology of a sphere $S^n$ and an orientable genus $g$ surface $M_g$ using the previous theorem. \\
        (\emph{Remark/challenge:} These computations give you explicit generators. There is a natural way to phrase the cup product in \v{C}ech cohomology -- can you compute it in these examples? To be clear, I'm not grading you on this remark, but just leaving something to think about.)
    \end{enumerate}
    
\end{problem}
% \item(








\end{document}