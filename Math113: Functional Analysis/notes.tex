\documentclass[12pt]{article}

\usepackage{fullpage}
\usepackage{mdframed}
\usepackage{colonequals}
\usepackage{algpseudocode}
\usepackage{algorithm}
\usepackage{tcolorbox}
\usepackage[all]{xy}
\usepackage{proof}
\usepackage{mathtools}
\usepackage{bbm}
\usepackage{amssymb}
\usepackage{amsthm}
\usepackage{amsmath}
\usepackage{amsxtra}
\newcommand{\bb}{\mathbb}


\newtheorem{theorem}{Theorem}[section]
\newtheorem{corollary}{Corollary}[theorem]
\newtheorem{lemma}{Lemma}

\newcommand{\mathcat}[1]{\textup{\textbf{\textsf{#1}}}} % for defined terms

\newenvironment{problem}[1]
{\begin{tcolorbox}\noindent\textbf{Problem #1}.}
{\vskip 6pt \end{tcolorbox}}

\newenvironment{enumalph}
{\begin{enumerate}\renewcommand{\labelenumi}{\textnormal{(\alph{enumi})}}}
{\end{enumerate}}

\newenvironment{enumroman}
{\begin{enumerate}\renewcommand{\labelenumi}{\textnormal{(\roman{enumi})}}}
{\end{enumerate}}

\newcommand{\defi}[1]{\textsf{#1}} % for defined terms

\theoremstyle{remark}
\newtheorem*{solution}{Solution}

\setlength{\hfuzz}{4pt}

\newcommand{\calC}{\mathcal{C}}
\newcommand{\calF}{\mathcal{F}}
\newcommand{\C}{\mathbb C}
\newcommand{\N}{\mathbb N}
\newcommand{\Q}{\mathbb Q}
\newcommand{\R}{\mathbb R}
\newcommand{\Z}{\mathbb Z}
\newcommand{\br}{\mathbf{r}}
\newcommand{\RP}{\mathbb{RP}}
\newcommand{\CP}{\mathbb{CP}}
\newcommand{\nbit}[1]{\{0, 1\}^{#1}}
\newcommand{\bits}{\{0, 1\}^{n}}
\newcommand{\bbni}{\bigbreak \noindent}
\newcommand{\norm}[1]{\left\vert\left\vert#1\right\vert\right\vert}

\let\1\relax
\newcommand{\1}{\mathbf{1}}
\newcommand{\fr}[2]{\left(\frac{#1}{#2}\right)}

\newcommand{\vecz}{\mathbf{z}}
\newcommand{\vecr}{\mathbf{r}}
\DeclareMathOperator{\Cinf}{C^{\infty}}
\DeclareMathOperator{\Id}{Id}

\DeclareMathOperator{\Alt}{Alt}
\DeclareMathOperator{\ann}{ann}
\DeclareMathOperator{\codim}{codim}
\DeclareMathOperator{\End}{End}
\DeclareMathOperator{\Hom}{Hom}
\DeclareMathOperator{\id}{id}
\DeclareMathOperator{\M}{M}
\DeclareMathOperator{\Mat}{Mat}
\DeclareMathOperator{\Ob}{Ob}
\DeclareMathOperator{\opchar}{char}
\DeclareMathOperator{\opspan}{span}
\DeclareMathOperator{\rk}{rk}
\DeclareMathOperator{\sgn}{sgn}
\DeclareMathOperator{\Sym}{Sym}
\DeclareMathOperator{\tr}{tr}
\DeclareMathOperator{\img}{img}
\DeclareMathOperator{\CandE}{CandE}
\DeclareMathOperator{\CandO}{CandO}
\DeclareMathOperator{\argmax}{argmax}
\DeclareMathOperator{\first}{first}
\DeclareMathOperator{\last}{last}
\DeclareMathOperator{\cost}{cost}
\DeclareMathOperator{\dist}{dist}
\DeclareMathOperator{\path}{path}
\DeclareMathOperator{\parent}{parent}
\DeclareMathOperator{\argmin}{argmin}
\DeclareMathOperator{\excess}{excess}
\let\Pr\relax
\DeclareMathOperator{\Pr}{\mathbf{Pr}}
\DeclareMathOperator{\Exp}{\mathbb{E}}
\DeclareMathOperator{\Var}{\mathbf{Var}}
\let\limsup\relax
\DeclareMathOperator{\limsup}{limsup}
%Paired Delims
\DeclarePairedDelimiter\ceil{\lceil}{\rceil}
\DeclarePairedDelimiter\floor{\lfloor}{ \rfloor}


\newcommand{\dagstar}{*}

\newcommand{\tbigwedge}{{\textstyle{\bigwedge}}}
\setlength{\parindent}{0pt}
\setlength{\parskip}{5pt}


\begin{document}

\title{CS 40: Computational Complexity}

\author{Sair Shaikh}
\maketitle

% Collaboration Notice: Talked to Henry Scheible '26 to discuss ideas.


\begin{itemize}
    \item[Defn.] Let $\{U_\alpha\}_{\alpha \in A}$ be an open cover of $(X, \rho)$. We say that $d > 0$ is a Lebesgue number for the cover if given any $d$-ball $B_d(x_0)$ with $x_0 \in X$, there exists $a_0 \in A$ such that $B_d(x_0) \subseteq U_{a_0}$. 
    \item[Ex] $X = \R$. $U_1 = (-\infty, 1)$, $U_2 = (0, 2)$, and $U_3 = (1, \infty)$. Here $d = 1/2$ is a LN for $\{U_1, U_2, U_3\}$. This is clear if $x_0 \in (1/2, 3/2)$. 
    \item[Ex. (Hwk.)] Given $x \in (0, 1)$, $\exists \delta_x > 0$ such that:
    \[ y \in B_{\delta_x}(x) = \{y \in (0, 1): |y-x| \leq \delta_x\} \]
    \[ \implies |1/x - 1/y| < 1 \]
    Then, 
    \[ (0, 1) = \bigcup_{x \in (0, 1)} B_{\delta_x}(x) \]
    has no Lebesgue number.
    \item[Lemma] (Lebesgue Covering Lemma) Every open cover of a compact metric space has a Lebesgue number. \\ 
    Proof. Pictures. Apr 9.
    \item[Thm.] Suppose $(X, \rho)$ is compact, and $F: (X, \rho) \to (Y, \sigma)$ is continuous. Then $F$ is uniformly continuous.
    \bbni Proof. Let $\epsilon > 0$. We need to find $\delta > 0$ such that $\forall x, y \in X$: 
    \[ \rho(x, y) < \delta \implies \sigma(F(x), F(y)) < \epsilon \]
    Since $F$ is continous, $\forall z \in X$, $\exists \delta_z > 0$ such that:
    \[ \rho(x, z) < \delta_z \implies \sigma(F(x), F(z)) < \epsilon/2 \]
    That is, 
    \[ F(B_{\delta_z}(z)) \subseteq B_{\epsilon/2}(F(z)) \]
    Let $\delta > 0$ be a Lebesgue number for the cover $\{B_{\delta_z}(z)\}_{z \in X}$. \\
    Now supposed $\rho(x,y) < \delta$. Then $\exists z \in X$ such that:
    \[ B_\delta(x) \subseteq B_{\delta_z}(z) \]
    and 
    \[\sigma(F(x), F(y)) \leq \sigma(F(x), F(y)) + \sigma(F(z), F(y)) < \epsilon/2 + \epsilon/2 = \epsilon \]
    \item[Defn. ] Let $(X, \rho)$ be a metric space and $C(X)$ the $\C$-vector space of continous functions on $X$. We say $\mathcal{J} \subset C$ is equicontinous at $x \in X$ if $\forall \epsilon > 0, \exists \delta > 0$ such that:
    \[ \forall F \in \mathcal{J}, F(B_\delta(x)) \subseteq B_\epsilon(F(x)) \]
    We say $\mathcal{J}$ is equicontinous on $X$ if $\forall x \in X$, $\mathcal{J}$ is equicontinous at $x$.
    \item[Ex.] Let $X = [0, 1] \subset \R$. Let $F_n(x) = x^n \forall n \geq 1$. Let:
    \[\mathcal{J} = \{F_n: n \in \mathbb{N}\}\]
    Let $x_n = \frac{1}{2}^{1/n}$. Then, $x_n$ arrow up to $1$. Then, 
    \[ |F_n(x_n) - F_n(1)| = |1/2 -1| = 1/2\]
    Thus, $\mathcal{J}$ is not equicontinous at $1$.
    \item[Ex. (Hwk)] Show that $\mathcal{J}$ is equicontinous on $[0, 1)$. 
    \item[Defn.] Let $(F_n)$ be a sequence of ($\C$-valued) functions on $X$. Then, $(F_n)$ is uniformly bounded if $\exists M > 0$ such that $\forall n \geq 1, \forall x \in X$:
    \[ |F_n(x)| < M \]
    We say that $(F_n)$ is pointwise bounded if $\forall x \in X$, $\exists M_x > 0$ such that:
    \[ |F_n(x)| < M_x \]
    \item[Defn.] A metric space (top. space) is seperable if there is a countable dense subset $D \subset X$. 
    \item[Ex.] Since $\Q^n \subset \R^n$ is dense, $(\R^n, ||\cdot||_p)$ is separable.  
    \item[Lemma.] (Arzelà-Ascoli) Let $(X, \rho)$ is a seperable metric space and that $(F_n)$ is pointwise bounded and equicontinous in $C(X)$. Then, there is subsequence $(F_{n_k})$ such that:
        \[ \lim_{x\to\infty} F_{n_k}(x)\]
    exists $\forall x \in X$.
    \item[Yap.] Given a sequence $(x_n)$, we get subsequence by finding $n_k \in \mathbb{N}$ such that $n_{k+1} > n_k$ and $(x_{n_k})_{k=1}^\infty \to x$ is a sequence. \\
    A subsubsequence is determined by finding $n_{k_1} < n_{k_2} < \cdots$ and then we write: 
    \[ (x_{n_{k_j}})_{j=1}^\infty\]
    A subsequence is determined by an infinite subset $S_1 = \{n_1 \leq n_2 \leq \cdots\} \subset \mathbb{N}$. A subsubsequence is determined by an infinite subset $S_2 \subset S_1$, 
    \[ S_2 = \{n_{k_1} < n_{k_2} < \cdots \} \subset S_1\]
    Now, we write: 
    \[ \lim_{n \in S_1} x_n = a \text{ instead of } \lim_{k \to \infty} x_{n_k}\]
    Note that $\lim_{n \in S_1} x_n = a$ if:
    \[ \forall \epsilon > 0\, \exists N: n \geq N, n \in S' \implies |x_n - a| < \epsilon\] 
    \item[Rmk.] Suppose $S_1 \subset \mathbb{N}$ determines a subsequence as above. Suppose $S' \subset \mathbb{N}$ is infinite and:
    \[\{n \in S' : n \not \in S_1\}\]
    is finite. The $\lim_{n \in S_1} x_n = a$ then $\lim_{n \in S'} x_n = a$ as well. \\
    Proof of the AA Lemma: Pictures. Apr 9 and 10.
    \item[Rmk.] If $X$ is compact, then $C(x) = C_b(X)$ is a complete metric space with respect to the uniform norm $||\cdot||_\infty$. 
    \item[Thm.] (Arzelà-Ascoli) Let $(X, \rho)$ be a compact metric space and $(F_n) \subset C(X)$ be a sequence of functions that are point-wise bounded and equicontinous. Then $(F_n)$ has a subsequence converging uniformly to some function $F \in C(X)$.
    Proof. Pictures.
    \item[Lemma.] Suppose $X$ is compact and that $\mathcal{J} \subset C(X)$ is equicontinous on $X$. Then, $\mathcal{J}$ is uniformly equicontinous on $X$, in that for all $\epsilon > 0 \exists \delta > 0$ such that for all $x, y \in X$ and all $F  \in \mathcal{J}$, 
        \[ \rho(x,y) < \delta \implies |F(x) - F(y)| < \epsilon \]
    Rewriting, 
    \[ F(B_\delta(x)) \subseteq B_\epsilon(F(x))\]
    Proof left as homework.  
    \item[Corr.] Let $X$ be a compact metric space. Let $\mathcal{J} \subset C(X)$ be a closed subset such that $\mathcal{J}$ is equicontinous and pointwise bounded. Then $\mathcal{J}$ is compact and uniformly bounded.
    \item[Thm.] Suppose $X$ is a compact metric space. Then $\mathcal{J} \subset C(X)$ is compact if and only if $\mathcal{J}$ is closed, uniformly bounded, and equicontinous on $X$.
    Proof. Pictures.   
    \item[Defn.] A topological space is called a Baire space if the countable intersection of dense open sets is dense. 
    \item[Rmk.] If $\rho$ and $\sigma$ are equivalent metrics on $X$ then $(X, \rho)$ is a Baire space if and only if $(X, \sigma)$ is a Baire space.
    \item[Defn.] If $S \subset X$ then the interior of $S$ is: 
    \[ Int(S) = \bigcup \{U \subset S : U \text{ is open in } X\} \]
    \item[Rmk.] One can write $Int_X(S)$ as interior of $S$ in $X$. Where you take the interior matters. 
    \item[Lemma.] A space $X$ is a Baire space if and only if given a countable $\{F_n\}_{n = 1}^\infty$ of closed sets, such that:
    \[ \bigcup_{n = 1}^\infty F_n\]    
    has non-empty interior, then at least one $F_n$ has (non-empty) interior. 
    \item[Hwk. ] Write $O_n = F_n^C = X \setminus F_n$. $O_n$ dense $\iff Int(F_n) = \emptyset$. 
    \item[Thm.] Barise Category Thm. Every complete metric space is a Baire space.
    \item[Rmk.] Note that $(0, 1)$ is homeomorphic to $\R$. Hence, $(0, 1)$ is a Baire space. 
    \item[Hwk.] More generally, every non-empty open subset $V$ of a complete metric space admits an equivalent complete metric (Hwk). Hence $V$ is a Baire space. Thus, if:
    \[ V = \bigcup_{n=1}^\infty V \cap F_n \]
    of each $F_n$ closed in $X$, thus $V \cap F_n$ is closed in $V$. Hence, at least one $V \cap F_n$ has non-empty interior in $V$. But $V$ is open, so it has interior in $X$.
    \item[Thm.] Suppose that $X$ is a Baire space and that $(F_n) \subset C(X)$ such that $F_n \to F$ pointwise. Then,
    \[ A = \{x \in X : F \text{ is cts at $x$}\}\]
    is dense in $X$. \\
    Proof. Pictures. 
    \item[Rmk.] If $X = \R$, then we can assume: 
    \[ A = \{x \in X: F \text{ is cts at $x$}\}\]
    is uncountable. 
    \newpage
    \item[Defn.] A normed vector space $(V, ||\cdot||)$ is called a Banach space if $V$ is complete in in the induced metric. Recall that $\mathbb{F}$ is always $\C$ or $\R$.  
    \item[Yap.] Normed vector spaces are special. 
    \begin{enumerate}
        \item $B_r(x) = x + B_r(0)$. The topology is homogenous. 
        \item $| ||v|| - ||w|| | \leq ||v - w|| \implies v \to ||v||$ is continous.
        \item $\overline{B_r(v)} = \{x \in V : ||x-v|| \leq r\}$. 
        \item $\epsilon B_r(0) = B_{\epsilon r}(0)$. 
        \item $\epsilon \overline{B_r(0)} = \overline{B_{\epsilon r}(0)}$.
    \end{enumerate} 
    \item[Thm.] Supposed that $X$ and $Y$ are normed vector spaces and $T: X \to Y$ is linear. Then, the following are equivalent: 
    \begin{enumerate}
        \item $T$ is continuous.
        \item $T$ is continous at a single point.
        \item $\exists \alpha \geq 0$ such that $||T(x)|| \leq \alpha ||x||$ for all $v \in X$. 
    \end{enumerate}
    Proof. \\
    (1) $\implies$ (2) is trivial. \\
    (2) $\implies$ (3). Since $T$ is continous at $x_0$, $\exists \delta > 0$ such that: 
    \[ T(\overline{B_\delta(x_0)}) \subset B_1(T(x_0))\]
    LHS $ = T(B_\delta(x_0)) + T(x_0)$. \\
    RHS $ = B_1(0) + T(x)$. \\
    This implies, $T(\overline{B_\delta(0)}) \subset B_1(0)$. \\
    Now if $z \neq 0$, 
    \[  ||T(z) || = || \frac{||z||}{\delta} T(\delta \cdot \frac{z}{||z||})|| \leq \frac{||z||}{\delta}\]
    Let $\alpha = \frac{1}{\delta}$. \\
    $(3) \implies (1)$. Erased. 
    \item[Rmk.] Suppose that $||\cdot||_1$ and $||\cdot ||_2$ (two different norms) induce equivalent metrics on $V$. Then, take the $\id: (V, ||\cdot||_1) \to (V, ||\cdot||_2)$ is continous (same topology). Hence, $\exists c \geq 0$ such that: 
    \[ ||x_2|| \leq c \cdot ||x_1||\]  
    Clearly, that means $c > 0$. By symmetry, there exists a $d > 0$, such that:
    \[ ||x_1|| \leq d \cdot ||x_2||\]
    Thus, the metrics and the norms are strongly equivalent.
    \item[Defn.] If $X$ and $Y$ are normed vector spaces, then $\mathcal{L}(X, Y)$ is the vector space of continous linear maps $T: X \to Y$. Define $||T|| = \sup_{||x|| \leq 1} ||T(x)||$. If $X = Y$, then we write $\mathcal{L}(X)$ instead of $\mathcal{L}(X, X)$.
    \item[Lem.] With $||T||$ as above, $\mathcal{L}(X,Y)$ is a normed vector space with:
    \[ ||T(x)|| \leq ||T||||x|| \forall x \in X\]
    If $S \in \mathcal{L}(Y, Z)$, then we write $ST$ in place of $S \circ T$ and: 
    \[ ||ST|| \leq ||S||||T||\]
    \item[Defn.] An algebra over $\mathbb{F}$ is a vector space $A$ over $\mathbb{F}$ with a ring structure, with $\lambda(xy) = (\lambda x)y = x(\lambda y)$ for all $\lambda \in \mathbb{F}$ and $x, y \in A$.
    \item[Ex. ] $M_n(\mathbb{F})$, $\R[x]$, $\C[x]$, $C(X)$. 
    \item[Defn.] If $||\cdot||$ is a norm on an algebra $A$, then we call $(A, ||\cdot||)$ a normed algebra if $\forall x,y \in A$: 
    \[ ||xy|| \leq ||x||||y|| \]
    We call $(A, ||\cdot||)$ a Banach algebra if $(A, ||\cdot||)$ is a normed algebra and $A$ is complete with respect to $||\cdot||$.
    \item[Prop.] If $X$ and $Y$ are normed vector spaces, and $Y$ is a Banach space, then $\mathcal{L}(X,Y)$ is a Banach space. If $X$ is a Banach space, then $\mathcal{L}(X)$ is a Banach algebra. \\
    Proof. Suppose that $(T_n)$ is a Cauchy sequence in $\mathcal{L:}(X,Y)$. Then, for each $z \in X$, $T_n(X)$ is a Cauchy sequence in $Y$. Hebcem $\exists T(x) \in Y$ such that:
    \[ T_n(X) \to T(X) \]
    It is not hard to see that $T: X \to Y$ is linear. \\
    Since $(T_n)$ is Cauchy in norm, its bounded i.e. $\exists m > 0$ such that:
    \[||T_n|| \leq M \forall n \geq 1 \]
    Now if $||x|| \leq 1$, then:
    \[ ||T(x)|| = \lim_{n\to \infty} ||T_n(x)|| \leq \limsup_{n} ||T_n|| ||x|| \leq M||x||\]
    This implies $T \in \mathcal{L}(X,Y)$. \\
    Let $\epsilon > 0$. Let $N$ be such that $m, n \geq N$. Thus, 
    \[ ||T_n - T_m|| \leq \frac{\epsilon}{2} \]
    Now if $||x|| \leq 1$, and if $n \geq N$, 
    \begin{align*}
        ||(T-T_n)(x)|| & = ||T(x) - T_n(x)|| \\
        &= \lim_{m\to \infty} ||T_m(x) - T_n(x)|| \\
        &\leq \limsup_{m} ||T_m - T_n|| \cdot ||x|| \\
        &\leq \frac{\epsilon}{2} < \epsilon
    \end{align*}
    \item[Yap.] Let $Y \subset X$ be a subsoace of a normed vector space $X$. Then we can form the quotient vector space $X / Y = \{x +Y : x\in X\}$ with $q: X \to X/Y$ the quotient map. \\
    If $x \in X$, then: 
     \[ \inf \{||x-y|| : y \in Y\} \] 
    depends only on $q(x)$. We call: 
    \[||q(x)|| := \inf\{||x-y|| : y \in Y\}\]
    the quotient norm on $X/Y$.
    \item[Rmk.] As $Y$ is a subspace, 
    \begin{align*}
        ||q(x)|| &= \inf\{||x+y|| : y \in Y\} \\
        &= \inf\{||x+\alpha y|| : y \in Y\} \, \forall \alpha \in \mathbb{F}\setminus\{0\}\\
    \end{align*}
    \item[Thm.] If $Y$ be a subspace of $(X, ||\cdot||)$. Then, 
    \[ ||q(x)|| = \inf\{||x-y|| : y \in Y\} \]
    is a seminorm on $X/Y$ which is a norm exactly when $Y$ is closed. If $X$ is a Banach space, and $Y$ is closed in $X$, then $X/Y$ is a Banach space. \\
    Proof. Note that for $\alpha \neq 0$: 
    \[ ||\alpha q(x) || = || q(\alpha x) || = \inf \{||\alpha x + \alpha y|| : y \in Y\} = |\alpha| \inf\{||x+y||: y \in Y\} = |\alpha|||q(x)||\]
    Next, fix $x_1, x_2 \in X$. Then given $\epsilon > 0$, $\exists y_1, y_2 \in Y$ such that:
    \begin{align*}
        ||q(x_1)|| + ||q(x_2)|| + \epsilon &\geq ||x_1 - y_1|| + ||x_2 - y_2|| \\
        &\geq ||x_1 - y_1 + x_2 - y_2|| \\
        &\geq ||q(x_1 + x_2)|| \\
        &=  ||q(x_1)|| + ||q(x_2)|| 
    \end{align*}
    Since $\epsilon > 0$ is arbitrary, $||q(x_1) + q(x_2)|| \leq ||q(x_1)|| + ||q(x_2)||$. Thus, it is a semi-norm. \\
    Sketch for closed. If quotient norm is $0$, there exists a sequence $y_n$ converging to $x$. If $Y$ closed, $x \in Y$ and $g(x) = 0$. Conversely, if quotient norm is a norm and $y_n \to x$. Then, $||q(x)|| = 0$. Then, $q(x) = 0$ and $x \in X$. \\
    Now suppose $X$ is a Banach space and $Y$ is closed in $X$. Let $(q(z_0))$ be Cauchy in $X/Y$. Then we can pass to a subsequence and assume: 
    \[ ||q(z_{n+1}) - q(z_0)|| \leq 1/2^n\]
    Let $x_1 = z_1$. Since: 
    \[ ||q(z_2) - q(x_1)|| < 1/2 \]
    Thus, 
    \[\inf\{||z_2 - x_1 - y || : y \in Y \} < 1/2\]
    Hence, we can find $x_2$ such that $q(x_1) = q(z_2)$ and 
    \[||x_2-x_1|| < 1/2 \]
    Continuing, we get $(x_n)$ such that $q(x_n) = q(z_n)$ and: 
    \[ ||x_{n+1} - x_n|| < 1/2^n\]
    (Hwk) $(x_n)$ is Cauchy, thus, $x_n \to x$ in $X$. \\
    But $||q(z)|| \leq ||z||$. Hence, $q$ is continous. Thus, 
    \[q(z_n) = q(x_n) \to q(x)\]
    \item[Rmk.] Note that $q : X \to X/Y$ is a bounded linear map of norm at most $1$. 
    \item[Thm.] Suppose that $Y$ is a closed subspace of $(X, ||\cdot||)$. Then $X$ is a Banach space if and only if both $Y$ and $X/Y$ are Banach spaces. \\
    Proof. Pictures. 4/16
    \item[Thm.] Every finite-dimensinal subspace $Y$ of a normed vector space $X$ is a Banach space and hence closed in $X$. If $\dim(Y) = n$, then every linear isomorphism $\Phi: \mathbb{F}^n \to Y$ is a homeomorphism. \\
    Proof. Pictures. 4/16.
    \item[Corr.] If $Y$ is finite-dimensional vector space, then all norms $||\cdot||$ are strongly equivalent. \\
    Proof. $\id:( Y, ||\cdot||_1) \to (Y, ||\cdot||_2)$ is a homeomorphism and continous linear maps are bounded.
    \item[Recall.]  Recall the topological definition of continous functions and open maps. If $F: X\to Y$ is a bijection, then $F^{-1}: Y \to X$ is continous if and only if $F$ is open.
    \item[Thm.] (The Big Three) (Open Mapping Theorem) Suppose that $X$ and $Y$ are Banach spaces and that $T \in \mathcal{L}(X,Y)$ is a surjection. Then $T$ is an open map.
    \item[Lem.] It will suffice to find $r > 0$ such that: 
    \[ B_r(0) \subseteq T(B_1(0)) \]
    Proof. By homogeneity, $T(B_\delta(0))$ is a neighborhood of $0_Y$ for all $\delta > 0$. By linearity, $T(B_\delta(x))$ is a neighborhood of $T(x)$ for all $x \in X$ and $\delta > 0$. Thus, if $V$ is open in $X$ and $x \in V$, then $\exists \delta > 0$ such that:
    \[ B_\delta(x) \subseteq V\]
    Then, $T(B_\delta(x))$ is a neighborhood of $T(x)$ in $T(Y)$. Thus, $T$ is an open map. Thus, $T(V)$ is open.
    \item[Lem.] It will suffice to find $r > 0$ such that: 
    \[ B_r(0) \subseteq \overline{T(B_1(0))}\]
    Proof. Assume $r > 0$ is such that the property holds. Let $y \in B_r(0)$. Then, 
    \[ \exists y_1 \in T(B_1(0)): ||y-y_1|| < r/2\]
    Then, $y-y_1 \in B_{r/2}(0)$. \\
    Hence, $\exists y_2 \in 1/2T(B_1(0)) = T(B_{1/2}(0))$ such that: 
    \[ || y- y_1 -y_2|| < r/2^2\]
    Continue to get a sequence $(y_n)$ such that: 
    \[y_n \in 2^{-n+1}T(B_1(0)) = T(B_{2^{-n+1}}(0))\]
    and: 
    \[||y - \sum_{i = 1}^n y_i|| < 2^{-n}r\]
    By construction, $\exists x_n \in X$ such that:
    \[ T(x_n) = y_n \qquad ||x_n|| < 2^{-n+1} \]
    Since $X$ is a Banach space, $x = \sum_{n=1}^\infty x_n$ converges and since $T$ is continous (hence bounded), and also: 
    \[ ||x|| < \sum_{n = 1}^\infty 2^{-n+1} = 2 \]
    and
    \[T(x) = y\] 
    Since $y$ was arbitrary, we have shown that:
    \[B_r(0) \subset T(B_2(0))\]
    This implies: 
    \[ B_{r/2}(0) \subset T(B_1(0)) \]
    This suffices by lemma $1$.
    \item[Lem. 3] If suffices to see that: 
    \[ \overline{T(B_n(0))}\] 
    has interior for some $n \geq 1$. \\
    Proof. By homogeneity, we can assume that $\overline{T(B_1(0))}$ has interior. \\
    Thus, $\exists \epsilon > 0$ such that: 
    \[B_\epsilon(y) \subset \overline{T(B_1(0))} \]
    Let $z \in B_\epsilon(0)$. Write
    \[ z = z/2 + y - (y-z/2) \in B_{\epsilon/2}(y)-B_{\epsilon/2}(y) \subseteq \overline{T(B_{1/2}(0))}- \overline{T(B_{1/2}(0))}\]
    Rest in pictures.
    \item[Proof.] Of the Open Mapping Theorem. Pictures. 4/18.
    \item[Ex.] Let $1 \leq p < \infty$. Then, let: 
    \[ l_0^p = \opspan\{e_n : n \geq 1\} = \{x \in l^p: x(n) = 0 \text{ for all but finitely many $n$}\}\]
    Notte $l_0^p$ is dense in $l^p$. \\
    Define: $T_0: l_0^p \to l_0^p$ by:
    \[T_0(e_n) = 1/n e_n\]
    You can check that $||T|| = 1$. Also $T_0$ is a bijection. 
    \[T_0^{-1}(e_n) = ne_n\]
    but $T_0^{-1} \not\in \mathcal{L}(l_0^p)$
    \item[Thm.] Suppose that $X$ and $Y$ are Banach spaces and $T \in \mathcal{L}(X,Y)$ is a bijection. Then $T^{-1} \in \mathcal{L}(Y,X)$. \\
    Proof. $T^{-1}$ is linear by general nonsense and $T$ is open by the Open Mapping Theorem. Hence $T^{-1}$ is continous hence bounded.
    \item[Yap.] If $X$ and $Y$ are Banach spaces, then I can give $X \times Y$ a norm by: 
    \[ ||(x,y|| = max\{||x||, ||y||\})\]
    This makes $X \times Y$ a Banach space.
    \newpage 
    [Missed 3 classes due to Covid. Did Open Mapping Theorem, Closed Graph Theorem, Principle of Uniform Boundedness, defined norm-dual, and did Hahn-Banach with 3 important corollaries.]
    \newpage
    \item[Thm.] The map:
    \[ \Phi: l^q \to (l^p)^\vee \]
    given by $\Phi(y) = \phi_y$ is an isometric isomorphism. That is, $\Phi$ is a isomorphism such that:
    \[ ||\Phi(y)|| = ||y|| \]
    \item[Rmk.] If $z \in \C$, then: 
    \[ \sgn(z) = \frac{z}{|z|} \text{ if } z \neq 0 \text{ else } 0 \]
    Then, $z = |z| \sgn(z)$ and $|z| = \sgn(z) z$.
    \item[Lemma.] If $y \in l^q$, then $||\phi_y|| = ||y||_q$.   
    Proof. Pictures. 4/28.
    \item[Lemma.] If $\phi \in (l^p)^\vee$, then $\exists y \in l^q$ such that:
    \[ \phi = \phi_y\]
    Proof. Pictures. 4/28.
    Let $y_n = \phi(e_n)$. Then, if $x \in l^p$, then: 
    \[ x = \sum_{n=1}^\infty x_ne_n\]
    Note, 
    \[ || x = \sum_{n=1}^N x_n e_n||_p^p = \sum_{n=N+1}^\infty |x_n|^p \]
    converges in the Banach space $l^p$. \\
    Hence, 
    \[ \phi(x) = \sum_{n=1}^\infty x_ny_n \]
    Then, if we can show $y \in l^q$, then $\phi = \phi_y$, and we are done. \\
    Let $y^N \in l^q$ be given by:
    \[ y^N(n) = \begin{cases}
        y_n \text{ if } 1\leq n \leq N \\
        0 \text{ otherwise }
    \end{cases}\] 
    Then, $\phi_{y^N} \in (l_p)^\vee$ and $||\phi_{y^N}|| \leq ||y^N||_q$. \\
    If $x \in l^p$, then,
    \[\phi_{y^N}(x) = \sum_{n=1}^N x_ny_n \to \sum_{n=1}^\infty x_yy_n = \phi(x) \]
    Hence, by the Principle of Uniform Boundedness, $\exists M > 0$ such that:
    \[ ||\phi_{y^N}|| \leq M  \qquad \forall N \geq 1\]
    Thus, if $q < \infty$, then, 
    \[ \sum_{n=1}^N |y_n|^q \leq M^q < \infty \]
    \item[Rmk.] If $X$ is a normed vector space, then we know that $X^*$ is a Banach space. Then, we can form the bidual $(X^*)^* = X^{**}$. We get a natural map $\iota$
    \[ \iota: X \to X^{**}\]
    $\iota(x)(\phi) = \phi(x)$ is evaluation at $x$. \\
    Now, 
    \begin{align*}
        ||\iota(x)|| = \sup_{||\phi||\leq 1} | \iota(x)(\phi)|
        &= \sup_{||\phi||\leq 1} |\phi(x)|  (*)
    \end{align*}
    Then, $(*) \leq ||x||$ since $||\phi||\leq 1$. \\
    Then, by Hahn-Banach, $\exists \phi \in X^*$ such that $||\phi|| = 1$ and $\phi(x) = ||x||$. Thus, 
    \[||\iota(x)|| = ||x||\]
    We can identify $X$ at $\iota(X)$ in $X^{**}$. \\
    If $X$ is a Banach space, then, $\iota(x)$ is complete, hence closed in $X^{**}$. \\
    Otherwise, $\overline{\iota(X)}$ is a Banach space containing $X$ as a dense subspace. Then, $\overline{\iota(X)}$ is the completion of $X$ as a Banach space. 
    \item[Defn.] If $X$ is a Banach space then $X$ is reflexive if $\iota(X) = X^{**}$ is onto. 
    \item[Rmk.] If $X$ is reflexive, then $X^{**}$ are isometrically isomorphic. The converse can fail. It can be that they are isometrically isomorphic but $\iota$ is not the map. [Robert James 1951].
    \item[Ex.] Let $1 < p < \infty$. Let $\frac{1}{p} + \frac{1}{q} = 1$. Then,
    \[ (l^p)^{*} = \{\phi_y^p : y \in l^q\} \] 
    and also:
    \[ (l^q)^{*} = \{\phi_x^q : x \in l^p\} \]
    If $x \in l^p$, then, $\iota(x) \in (l^p)^{**}$, and:
    \begin{align*}
        \iota(x)(\phi_y^p) &= \phi_y^p(x) \\
        &= \phi_x^q(y)
    \end{align*}  
    Thus, $\iota: l^p \to (l^p)^(**)$ is surjective and $l^p$ is reflexive for $1 < p < \infty$.
    \item[Prop.] $l^p$ is reflexive for $1 < p < \infty$.
    \item[Rmk.] Let $X$ be a normed vector space and $D = \{d_n\}_{n=1}^\infty \subset X$ be a countable subset of $X$. Then the rational space of $D$, $\opspan_\Q(D)$ is the span of $D$ viewing $X$ as a rational vector space. Thus,
    \[\opspan_\Q(D) = \bigcup_{n=1}^\infty \{\sum_{k=1}^r r_kd_k : r_k\in\Q \, d_k \in D\}\]
    Then, $\opspan_\Q(D)$ is countable. Then if $\opspan(D)$ is dense, then since $\opspan_\Q(D)$ is dense in $\opspan(D)$. Thus, $X$ is separable. \\
    Same works over $\C$ replacing $\Q$ by $\Q + i\Q$. 
    \item[Corr.] $l^p$ is seperatble for $1 \leq p < \infty$. Let $D = \{e_n\}_{n=1}^\infty$. But $l^\infty$ is not separable. 
    \item[Hwk.] If $X^*$ is separable, then $X$ is.  
    \newpage 
    \item[Thm.] If $1 \leq p \leq \infty$, and if $q$ is the conjugate exponent, then: 
    \[ \Phi: l^q \to (l^p)^* \]
    is an isometric isomorphism when $\Phi(y) = \phi_y^p$ and 
    \[ \phi_y^p = \sum_{n=1}^{\infty} x_ny_n\]
    \item[Rmk.] If $(l^p)^* = \{\phi_y^p : y \in l^q\}$ and if $q \leq \infty$, then: 
    \[ (l^q)^* = \{\phi_x^q : x \in l^p\}\] 
    \item[Defn.] Recall: If $X$ is a Banach space, then $X$ is reflexive if $\iota(X) = X^{**}$ is surjective.
    \item[Prop.] If $1 < p < \infty$, then $l^p$ is reflexive.\\ 
    Proof. COnsider $\iota: l^p \to (l^p)^{**}$. If $\phi_y^p \in (l^p)^*$, then: 
    \[ \iota(x)(\phi_y^p) = \phi_y^p(x) = \phi_x^q(y) \]
    Let $\psi \in (l^p)^{**}$. Then,
    \begin{align*}
        \psi(\phi_y^p) &= \psi(\Phi(y))
    \end{align*}
    Now $y \to \psi(\Phi(y))$ is a in $(l^q)^*$. Hence, $\exists x \in l^p$ such that:
    \[ \psi(\Phi(y)) = \phi_x^q(y) \]
    Thus, $\iota(x) = \psi$. Thus, $\iota$ is onto.
    \item[Rmk.] If $X$ is a normed vector space over $\mathbb{F}$ and $D \subset X$ is countable and $\overline{\opspan(D)} = X$, then $X$ is separable. 
    \item[Corr.] If $1 \leq p < \infty$, then $l^p$ is separable. \\
    Proof. $D = \{e_n : n \geq 1\}$.  
    \item[Ex.] $l^\infty$ is not separable. \\
    Proof. Let $A \subseteq \N$. Let: 
    \[ x_A(n) = \1_{n \in A}\]
    If $A \neq B$, then $||x_A - x_B||_\infty = 1$. But $\{x_A : A \subseteq \N\}$ is uncountable. You cannot find a countable dense subset anymore (cant fit all of them into countably many $1/4$ balls).
    \item[Hwk.] $(l^\infty)^*$ is not separable. Thus, $l^1$ is not reflexive.  
    \item[Defn.] Suppose $T \in \mathcal{L}(X, Y)$. Then we define: 
    \[ T^*: Y^* \to X^*\]
    by: 
    \[ T^*(\phi)(x) = \phi(T(x))\]
    \item[Prop.] If $X$ and $Y$ are normed vector spaces, and $T \in \mathcal{L}(X,Y)$, then $T^* \in \mathcal{L}(Y^*, X^*)$ and $||T^*|| = ||T||$. \\
    Proof. First, check that $T^*$ is linear (easy). \\
    Next, show the norm equality. Look at: 
    \begin{align*}
        ||T^*(\phi)|| &= \sup_{||x|| \leq 1} |T^*(\phi)(x)| \\
        &= \sup_{||x|| \leq 1} |\phi(T(x))| \\
        &\leq ||\phi||||T||||x|| \\
        &\leq ||\phi||||T||
    \end{align*}
    Thus, 
    \[ ||T^*|| \leq ||T||\] 
    Fix $\epsilon > 0$. Then, $\exists x_0 \in X$ such that $||x_0|| = 1$ and: 
    \[ ||T(x_0)|| > ||T|| - \epsilon\]
    But $\exists \phi \in Y^*$ such that $||\phi|| = 1$ and:
    \[ \phi(T(x_0)) = ||T(x_0)|| \]
    Then, 
    \begin{align*}
        ||T^*|| &\geq ||T^*(\phi)|| \\
        &\geq |T^*(\phi)(x_0)| \\
        &= |\phi(T(x_0))| \\
        &= ||T(x_0)||
        &> ||T|| - \epsilon
    \end{align*}
    Since $\epsilon > 0$ is arbitrary, we have:
    \[ ||T^*|| \geq ||T||\]
    Thus, $||T^*|| = ||T||$.
    \item[Thm.] Let $X$ and $Y$ be Banach spaces and suppose that: 
    \[ T: X \to Y \qquad S: Y^* \to X^* \]
    are functions (not linear or bounded) such that $\forall \phi \in Y^*$ and $x \in X$: 
    \[ S(\phi)(x) = \phi(T(x)) \]
    Then, $T \in \mathcal{L}(X,Y)$ and $S \in \mathcal{L}(Y^*, X^*)$, with $S = T^*$. \\
    Proof. Supposed that $x, y \in X$ and $\alpha \in \mathbb{F}$. Then, if $\phi \in Y^*$, we have:
    \begin{align*}
        \phi(T(\alpha x + y)) &= S(\phi)(\alpha x + y) \\
        &= \alpha(S(\phi)(x)) + S(\phi)(y) \\
        &= \alpha\phi(T(x)) + \phi(T(y)) \\
        &= \phi(\alpha T(x) + T(y)) 
    \end{align*}
    Since $\phi \in Y^*$ is arbitrary, we have $T(\alpha x + y) = \alpha T(x) + T(y)$. (something about separate points). \\
    To see that $T$ is bounded, use CGT. Suppose that $x_n \to x$ in $X$ and $T(x_n) \to y$ in $Y$. But $\forall \phi \in Y^*$, we have: 
    \begin{align*}
        \phi(y) &= \lim \phi(T(x_n)) \\
        &= \lim S(\phi)(x_n) \\
        &= S(\phi)(x) \\
        &= \phi(T(x))
    \end{align*}
    Thus, $y = T(x)$. Thus, $T$ is bounded. \\
    But $T^*(\phi) = S(\phi)$. 
    \item[Yap.] Let $(X, \tau)$ be a topological space. Then $\beta \subset \tau$ is a basis for $\tau$ if given any $U \in \tau$ and $x \in U$, then $\exists V \in \beta$ such that $x \in V \subset U$. \\
    We say $S$ is a neighborhood of $x \in X$ if $\exists U \in \tau$ such that $x \in U \subset S$. \\
    Wrote $\mathcal{N}(x)$ for the set of all neighborhoods of $x$. We say that $\alpha \subseteq \mathcal{N}(x)$ is a neighborhood basis at $x$ if $U \in \mathcal{N}(x)$, there exists $V \in \alpha$ such that: 
    \[ x \in V \subseteq U \]
    \item[Ex. ] (1) In a metric space, the collection of all open balls is a basis for the metric topology. \\
    (2) In $\R^n$, every point has a neighborhood basis consisting of compact sets. Such spaces are called locally compact.
    \item[Lemma.] Low Hanging Fruit: Let $(X , \tau)$ be a topological space and $\alpha(x)$ has a neighborhood basis at $x \in X$ consisting of open sets. Then: 
    \[ \beta = \bigcup_{x \in X} \alpha(x) \]
    is a basis for $\tau$.  
    \item[Lemma.] Low Hanging Fruit: $\beta \in \tau$ is a bssis for $\tau$ if and only if given $U \in \tau$, 
    \[ U = \bigcup_{V \in \beta \subset U} V\]
    \item[Defn.] Let $(X, \tau)$ be a topological space. Then, 
    \begin{enumerate}
        \item $(X, \tau)$ is seperable if it ihas a countable dense subset. 
        \item $(X, \tau)$ is 2nd countable if it has a countable basis.
        \item $(X, \tau)$ 1st countable if every point has a countable neighborhood basis.
    \end{enumerate}
    \item[Rmk.] \begin{enumerate}
        \item Every 2nd countable space is seperable. The converse holds in metric spaces. Hwk 8. 
        \item Metric spaces are 1st countable. 
    \end{enumerate}
    \item[Ex.] \begin{enumerate}
        \item Let $X$ be a set. Then $\tau = \mathcal{P}(X)$ is the discrete topology. This is the metric topology coming from the the discrete metric. 
        \item $\tau = \{X, \phi\}$. 
    \end{enumerate}
    \item[Lemma.] Let $S \subset \mathcal{P}(X)$. Then, there is a smallest topology $\tau(S)$ that contains $S$. \\
    Proof. Let $\tau(S) = \bigcap\{\tau' : \tau' \text{ is a top and } S \in \tau'\}$. 
    \item[Prop.] Let $\beta \subset \mathcal{P}(X)$ be a coer of $X$. Then $\beta$ is a basis for $\tau(B)$ if and only if $U, V \in \beta$ and $x \in U \cap V$, then $\exists W \in \beta$ such that $x \in W \subset U \cap V$. \\
    Proof. Hwk. 
    \newpage
    [One lecture on omitted. Defined weak topology. ]
    \newpage 
    \item[Defn.] An ordered set $(X, \leq)$ is directed if given $x, y \in X$, $\exists z \in X$ such that $x \leq z$ and $y \leq z$.
    \item[Ex.] \begin{itemize}
        \item $X = \N$. 
        \item Let $(Y, \tau)$ be a topological space and $y_0 \in Y$. Then $X = \mathcal{N}(y_0)$ be the collection of neighborhoods of $y_0$ induced by reverse inclusion. Then, $U \leq V \iff V \subset U$. Then $\mathcal{N}(y_0)$ is directed. If $U, V \in \mathcal{N}(y_0)$, then so is $U \cap V$. 
        \item We can do the same thing for open neighborhoods. 
    \end{itemize} 
    \item[Defn.] \begin{itemize}
        \item A net $x$ is a set $X$ is a function $x: \:ambda \to X$ where $\Lambda$ is a directed set. As with seqeucnes, we usually write $X_\lambda$ for $x(\lambda)$ and $(x_\lambda)_{\lambda \in \Lambda}$ in place of $x: \Lambda \to X$, or just $x$. 
        \item If $X$ is a topological space, then we say that a net $(x_\lambda)_{\lambda \in \Lambda}$ converges to $x_0 \in X$ if $(x_\lambda)$ is eventually in every neighborhood of $x_0$. That is, if $U \in \mathcal{N}(x_0)$, then $\exists \lambda_0 \in \Lambda$ such that: 
        \[ \lambda \geq \lambda_0  \implies x_\lambda \in U \]
        \item We say that $x_0 \in X$ is an accmulation point of a net $x_\lambda$ if $(x_\lambda)$ is frequently in every neighborhood of $x_0$. That is if $U$ is a neighborhood of $x_0$ and $\lambda_0 \in \Lambda$, then $\exists \lambda \geq \lambda_0$ such that $x_\lambda \in U$.
    \end{itemize}
    \item[Prop.] Suppose $X$ is a topological space and $E \subset X$. Then, $x_0 \in \overline{E}$ if and only if there exists a net $(x_\lambda) \subset E$ with $x_\lambda \to x_0$. \\
    Proof. Suppose we have $x_\lambda \subset E$ with $x_\lambda \to x_0$. If $x_0 \not \in \overline{E}$, then $\exists U \in \mathcal{O}(x_0)$ such that $U \cap E = \emptyset$. But then we would eventually have $(x_\lambda)$ in $U$. Contradiction. \bbni
    Now suppose that $x_0 \in \overline{E}$. Let $\Lambda = \mathcal{O}(x_0)$.  \\
    If $W \in \Lambda$, then $W \cap E \neq \emptyset$. Then, we let $x_W \in W \cap E$. Then, $(x_W)_{W \in \Lambda}$ converges to $x_0$. \\
    If $U_0 \in \mathcal{O}(x_0)$, then $U \geq U_0$, $x_U \in U \subset U_0$. 
    \item[Ex.] $S = \{ \sqrt{n}e_n \in l^2 : n \geq 1\}$, then $0 \in \overline{S}^w$ (weak topology).
    \item[Defn.] A subset $C$ of a vector space $V$ over $\mathcal{F}$ is convex if $x, y \in C$ and $t \in [0, 1]$, then $tx + (1-t)y \in C$.
    \item[Ex.] In a normed vector space $X$, every open ball $B_r(x_0)$ is convex if $x_0 \in X$ and $r > 0$. To see this, supposed $x, y \in B_r(x_0)$, and $t \in [0, 1]$. Then,
    \begin{align*}
        ||tx+(1-t)y - x_0|| &\leq ||t(x-x_0)|| + ||(1-t)(y-x_0)|| \\
        &< tr + (1-t)r \\
        &= r
    \end{align*}
    \item[Lemma.] Let $C$ be an open convex neighborhood of $0$ in a normed vector space $X$. Then, 
    \[m(x) = \inf\{s > 0, s^{-1}x \in C\}\]
    Then $m$ is a Minkowski functional on $X$ such that: 
    \[C = \{x \in X: m(x) < 1\}\]
    Proof. If $x \in X$, then $\frac1n x \to 0$. So $\frac1n x$ is eventually in $C$. So $m(x) < \infty$ for all $x \in X$. \bbni
    Note that if $t \geq 0$, then $m(tx) = tm(x)$. (This is one of the axioms for a Minkowski functional). \\
    Also let $s^{-1}x$, $t^{-1}y \in C$. Then: 
    \[ (s+t)^{-1}(x+y) = \frac{s}{s+t}s^{-1}x + \frac{t}{s+t}t^{-1}y \in C\]
    Thus, $m(x+y) \leq s + t$. \\
    Since $s^{-1}x, t^{-1}y \in C$ is arbitrary, we have:
    \[ m(x+y) \leq m(x) + m(y)\]
    This is the second axiom for a Minkowski functional. \\
    Thus, $m: X \to \R$ is a Minkowski functional. Now if $x \in C$, then $(1+1/n)x \to x$. Thus, for some $n$, 
    \[ m(x) = frac{1}{1+1/n} < 1\]
    If $m(x) K 1$, then $\exists s < 1$ such that $s^{-1}x \in C$. \bbni
    But $0 \in C$ and $x = (1-s)0 + s(s^{-1}x) \in C$. 
    \item[Thm.] (Hahn-Banach Seperation Theorem). Let $A$ and $B$ be disjoint non-empty convex subsets of some normed vector space $X$. If $A$ is also open, then there exists $\phi \in X^*$ and $t \in \R$ such that: 
    \[ \Re(\phi(x)) < t \leq \Re(\phi(y)) \]
    for all $x \in A$ and $y \in B$. \bbni
    Proof. We start with $\mathbb{F} = \R$. Then, we can use the Basic Extension Lemma. Fix $x_0 \in A$ and $y_0 \in B$. Let $z_0 = x_0 - y_0$. Let: 
    \[ C = A-B + z_0\]
    We can check that $C$ is convex. \\
    Since $C = \bigcup_{y \in B} A -y + z_0$. Thus, $C$ is union of open sets, hence open neighborhood of $0$. \bbni  
    Let $m: X \to \R$ be the corresponding Minkowski functional. \\
    We claim $z_0 \not \in C$. If $x-y+z_0 = z_0$, then $x = y \in A \cap B - \emptyset$. \\
    Thus, $m(z_0) \geq 1$. \\
    Define: 
    \[ \phi_0: \R z_0 \to \R\]
    by $\phi_0(\alpha z_0)  = \alpha$. \\
    Since $m(x) \geq 0$, then for all $x$ if $\alpha \leq 0$, then: 
    \[ \phi_0(\alpha z_0) \leq m(\alpha z_0)\]
    If $\alpha \geq 0$, then: 
    \[ \phi_0(\alpha z_0) = \alpha \leq \alpha m(z_0) = m(\alpha z_0)\]
    Thus, $\phi_0(z) \leq m(z)$ for all $z \in \R z_0$. \\
    Thus, the Basic Extension Lemma gives us $\phi: X \to \R$ such that, $\phi(x) \leq m(x)$. \\
    Thus, $\phi(x) < 1$ if $x \in C$. \\
    Also, $-\phi(x) = \phi(-x) < 1$ if $x \in -C$. \\
    Thus, $\forall \epsilon > 0$, 
    \[ |\phi(x)| < \epsilon \text{ if } x \in \epsilon C \cap -\epsilon C\]
    Since $\epsilon C \cap -\epsilon C$ is a neighborhood of $0$ for all $\epsilon < 0$, this means $\phi$ is continous at $0$. \\
    Thus, $\phi \in X^*$. If $x \in A$ and $y \in B$, then: 
    \[ x-y + z_0 \in C\]
    Then, 
    \[ m(x-y+z_0) < 1\]
    Thus, 
    \[ \phi(x-y+z_0) < 1\]
    Since $\phi(z_0) = \phi_0(z_0) - 1$, we have: 
    \[ \phi(x) < \phi(y)\]
    Since $\phi$ is linear and since $A$ and $B$ are convex, $\phi(A)$ and $\phi(B)$ are intervals. Since $A$ is open and $\phi$ is linear, $\phi(A)$ must be open. \\
    Then, we can let $t$ be the right-hand endpoint of $\phi(A)$. \bbni
    Now if $\mathbb{F} = \C$, then we can treat $X$ as a real space and produce a real linear functional $\psi: X \to \R$ such that: 
    \[ \psi(A) < t \leq \phi(B)\]
    Now let $\phi(x) = \psi(x) -i \phi(ix)$. $\phi$ is continous since $\psi$ is and continuity implies boundedness. 
    \newpage
    [Skipped a class of notes. Took pictures. 5/9 ]
    \newpage
    \item[Ex.] The sequence $(\sqrt{n}e_n) \subset l^2$ has no weakly convergent sequence. 
    \item[Ex.] For each $\alpha \in l^\infty$, let:
    \[ D_\alpha = \{z \in \C: |z| \leq ||\alpha||_\infty \}\]
    Then if $\alpha = (\alpha_n)$ then $\alpha_n \in D_\alpha$ for all $n \geq 1$. \\
    We now know that:
    \[Z = \Pi_{\alpha \in l^\infty} D_a\]
    is compact in the product topology. Let $(z_n) \subset Z$ be the sequence given by: 
    \[ z_n(\alpha) = \alpha_n\] 
    But we claim $(z_n)$ has no convergent subsequence. \\
    Proof. Suppose $(z_{n_k})$ converges to $z_0 \in Z$. Then it converges pointwise. Thus for all $\alpha \in l^\infty$, we have:
    \[z_{n_k}(\alpha) \to z_0(\alpha)\]
    Let $\alpha_0 \in l^\infty$ be given by: 
    \[ \alpha_0(n) = \begin{cases}
        (-1)^k \text{ if } n = n_k \\
        0 \text{ otherwise }
    \end{cases}\]
    But now:
    \[ z_{n_k}(\alpha_0) = (-1)^k\]
    But
    \[ \lim_{k \to \infty} z_{n_k}(\alpha_0)\]
    does not exist. 
    \item[Defn.] Let $x: \Lambda \to X$ be a net in a set $x$. Then a net $y: M \to X$ is a subnet of $x$ if $\exists h: M \to \Lambda$ such that $y = x \circ h$ and for all $\lambda_0 \in \Lambda$, there exists $m(\lambda_0)$ such that: 
    \[ m \geq m(\lambda_0)  \implies h(m) \geq \lambda_0\]
    \item[Rmk.] If $h: M \to \Lambda$ is monotone in that $m_1 \geq m_2$ then $h(m_1) \geq h(m_2)$, then given $\lambda_0 \in \Lambda$ it suffices to find $m(\lambda_0)$ such that such that $h(m(\lambda_0)) \geq \lambda_0$.
    \item[Yap.] We usually write: 
    \[ (y_m)_{m\in M} \text{ as } (x_{h(m)})_{m\in M} \text{ or } (x_{\lambda_m})_{m \in M}\] 
    Example: If you have a subsequence $(x_n)_{n \in \N}$ and $h: \N \to \N$ strictly increasing, we can write: 
    \[ x_k = h(k) \qquad n_{k+1} > n_k\]
    \item[Lemma.] Let $(x_\lambda)_{\lambda \in \Lambda}$ be a net $m$ a topological space. Then $x_0$ is an accumulation point of $(x_\lambda)$ if and only if there is a subnet $(x_{h(m)})_{m \in M}$ converging to $x_0$. \\
    Proof. Suppose that $(x_{h(m)})_{m \in M}$ converges to $x_0$. \\ 
    Let $U \in \mathcal{O}(x_0)$ and $\lambda_0 \in \Lambda$. \\
    Since the subnet converges to $x_0$, there is a $m_0 \in M$ such that $m \geq m_0$ implies $x_{h(m)} \in U$. \\
    But there exists $m_1 = m(\lambda_0)$ such that $m\geq m_1$ implies $h(m) \geq \lambda_0$. \\
    Let $m_2$ be such that $m_2 \geq m_1$ and $M_2 \geq m_0$. Then,
    \[ h(m_2) \geq \lambda_0 \text{ and } x_{h(m_2)} \in U\]
    Thus, $x_0$ is an accumulation point of $(x_\lambda)$. \bbni
    Now suppose that $x_0$ is an accumulation point of $(x_\lambda)$. Let: 
    \[ \mathcal M \{(U, \lambda) \in \mathcal{O}(x_0) \times \Lambda : x_\lambda \in U\}\]
    and $(U_1, \lambda_1) \leq (U_2, \lambda_2)$ if $U_2 \subset U_1$ and $\lambda_1 \geq \lambda_2$. \\
    Claim: $\mathcal{M}$ is a directed set. To see this, let $(U_1, \lambda_1), (U_2, \lambda_2) \in \mathcal{M}$. Let $\lambda_3 \geq \lambda_1$ and $\lambda_3 \geq \lambda_2$. Let $\lambda_4 \geq \lambda_4$ such that $x_{\lambda_4} \in U_1 \cap U_2$. Then, $(U_1 \cap U_2, \lambda_4) \geq (U_k, \lambda_k)$ for $k = 1, 2$. \\
    Now define $h: M \to \Lambda$ by $h(U, \lambda) = \lambda$. Clearly, 
    \[ (x_{(U, \lambda)})_{(U, \lambda)\in M} \to x_0\]
    \item[Corr.] A topological space is compact if and only if every net has a convergent subnet. 
    \item[Not.] If $X$ is a normed vector space, and $Y$ a collection of linear functional on $X$. Then we write $\sigma(X, Y)$ for the initial topology on $X$ induced by $\mathcal{F} = Y$ (smallest topology that makes all of these functional continous, likely want separation so you get Hausdorff etc). 
    \item[Ex.] $\sigma(X, X^*)$ is the weak topology on $X$. \\
    $\sigma(X^*, X^{**})$ is the weak topology on $X^*$.
    \item[Yap.] But now we want to consider: 
    \[ \sigma(X^*, X) := \sigma(X^*, \iota(X))\]
    \item[Defn.] If $X$ is a normed vector space, then: 
    \[ \sigma(X^*, X)\]
    is called the weak-* topology on $X^*$.
    \item[Lemma.] Let $X$ be a normed vector space. Then a sub-basis for the weak-* topology on $X^*$ consists of all sets of the form:
    \[  \mathcal{U}(x, \phi_0, \epsilon) = \{\phi \in X^* : |\phi(x)-\phi_0(x)| \leq \epsilon \}\]
    where $x \in X, \phi_0 \in X^*$ and $\epsilon > 0$. \\
    Then we get a neighborhood basis for $\phi_0 \in X^*$ by taking: 
    \[ U(\{x_1, \cdots, x_n\}, \phi_0, \epsilon) = \{ \phi \in X^* : |\phi(x_k)-\phi_0(x_k)| < \epsilon \, \forall k\} \]
    \item[Lemma.] A net $(\phi_\lambda)$ in $X^*$ converges in weak-* to $\phi_0$ if $\phi_\lambda(x) \to \phi_0(x)$ for all $x \in X$.
    \item[Thm.] (Alaoglu's Theorem.) If $X$ is a normed vector space and:
    \[ B^* = \{\phi \in X^*: ||\phi|| \leq 1\}\]
    is the closed unit ball, then $B^*$ is compact in the weak-* topology. \\
    Proof. Let $D_r = \{z \in \mathcal F: |z| \leq r\}$. Let: 
    \[ Z = \Pi_{x \in X} D_{||x||}\]
    Then we know that $Z$ is compact in the product topology by Tychonoff's theorem. \\
    Define: 
    \[ j: B^* \to Z \]
    by $j(\phi)(x) = \phi(x)$. Note that $j$ is injective. \\
    If $\phi_lambda \to \phi_0$ in the weak-* topology, then $\phi_\lambda(x) \to \phi_0(x)$ for all $x \in X$. Thus, $j(\phi_\lambda)(x) \to j(\phi_0)(x)$ for all $x \in X$. Thus, $j(\phi_\lambda) \to j(\phi_0)$ in $Z$. \\
    Now, claim that $j$ has closed range in $Z$. Suppose $j(\phi_\lambda) \to z_0$ in $Z$. Then, 
    \[ j(\phi_\lambda)(x) \to z_0(x) \qquad \forall x \in X \]
    That is:
    \[ \phi_\lambda(x) \to z_0(x) \]
    Then,
    \begin{align*}
        z_0(x+y) &= \lim_\lambda \phi_\lambda(x+y)  \\
        &= \lim_\lambda \phi_\lambda(x) + \phi_\lambda(y) \\
        &= z_0(x) + z_0(y) \\
    \end{align*}
    We also clearly have: 
    \[ |z_0(x)| \leq ||x|| \]
    Thus, $z_0 \in j(B^*)$. Thus, $j(B^*)$ is compact. \\
    If $j(\phi_\lambda) \to j(\phi)$ in $Z$, then:
    \[ j(\phi_\lambda)(x) \to j(\phi)(x) \qquad \forall x \in X\] 
    Then, $\phi_\lambda(x) \to \phi(x)$ for all $x \in X$. Thus, $\phi_\lambda \to \phi$ in the weak-* topology (thus inverse is continous).
    Thus, $j$ is a homeomorphism onto $j(B^*)$ and then $B^*$ is compact. 
    \newpage
    \item[Defn.] We defined the weak-* topology on $X^*$, $\sigma(X^*, X)$, to be the initial topology on $X^*$ induced by $\mathcal{J}$ a family of linear functional on $X$. 
    \item[Defn.] A subbases for $\sigma(X^*, X)$ is given by the sets: 
    \[ \mathcal{U}(x, \phi_0, \epsilon) = \{\phi \in X^* : |\phi(x) - \phi_0(x)| < \epsilon\}\]
    for all $x \in X$, $\phi_0 \in X^*$ and $\epsilon > 0$.
    \[ \mathcal{U}(x, \phi_0, \epsilon) = \omega(\iota(x), \phi_0, \epsilon) = \{ \phi \in X^* : | \iota(x)(\phi)-\iota(x)(\phi_0)| < \epsilon\}\]
    \item[Defn.] If $V$ is a vecto space over $\mathcal{F}$, then we call $(\cdot, \cdot): V\times V \to \mathcal{F}$ a sesquilinear form on $V$ if for all $x, y, z \in V$ and $\alpha \in \mathcal{F}$, we have:
    \begin{itemize}
        \item $(x+\alpha y, z) = (x, z) + \alpha(y, z)$
        \item $(x, \alpha y+z) = \overline{\alpha}(x, y) + (x, z)$
    \end{itemize}
    If we also have:
    \[ (x, y) = \overline{(y, x)}\]
    then we say that $(\cdot, \cdot)$ is a self-adjoint and if:
    \[(x, x) \geq 0\]
    then we say $(\cdot, \cdot)$ is positive. 
    \item[Lem.] (Polarization Identity). If $\mathbb{F} = \C$ and $(\cdot, \cdot)$ is a sesquilinear form on $V$, then:
    \[ (x, y) = \frac 1 4 \sum_{k=0}^3 i^k(x+i^ky, x + i^ky)\]
    Proof. Check that: 
    \begin{align}
        4(x, y) &= (x+y, x+y) + i(x+iy, x+iy) -(x-y, x-y) - i(x-iy, x-iy) \\
    \end{align}
    \item[Corr.] If $\mathbb{F} = \C$, then $(\cdot, \cdot)$ is self-adjoint if and only if $(x, x) \in \R$ for all $x \in V$. Hence on a complex vector space, a positive sesquilinear form is always self-adjoint.
    \item[Defn.] A positive self-adjoint sesquilinear form on $V$ is called a pre-inner product. If $(x, x) = 0$ if and only if $x = 0$, then we call it an inner product.  
    \item[Defn. ] If $(\cdot, \cdot)$ is a pre-inner product on $V$, then we define: 
    \[ \norm{V} = (v, v)^{frac12}\]
    for all $v \in V$.
    \item[Rmk.] The polariziation identity then becomes for $\C$: 
    \[ (x, y) = \frac14 \sum_{k=0}^3 i^k||x+i^ky||^2\]
    for for $\R$, 
    \[ (x, y) = \frac13(||x+y||^2 - ||x-y||^2)\]
    \item[Lemm.] (Cauchy-Schwarz Inequality). If $(\cdot, \cdot)$ is a pre-inner product on $V$, then $\forall x, y \in V$: 
    \[ |(x, y)| \leq \norm{x}\norm{y}  \]
    Proof. For all $\alpha \in \mathcal{F}$, 
    \begin{align*}
        0 &\leq ||\alpha x + y||^2 = (\alpha x + y, \alpha x + y) \\
        &= |\alpha|^2 ||x|| + \alpha(x, y) + \overline{\alpha}(x, y) + ||y||^2 \\
        &= |\alpha|^2 ||x||^2 + 2\Re(\alpha(x,y)) + ||y||^2
    \end{align*}
    Let $\tau \in \mathcal{F}$ be such that $\tau(x, y) = |(x, y)|$. If $\alpha = t\tau$ with $t \in \R$, then:
    \begin{align*}
        0 &\leq t^2 ||x||^2 + 2t|(x,y)| + ||y||^2  
    \end{align*}
    Then, by the quadratic forumula, 
    \begin{align*}
        &4|(x, y)|^2 - 4||x||^2||y||^2 \leq 0 \\
        \implies & |(x, y)| \leq ||x||||y||
    \end{align*}
    \item[Corr.] Let $(\cdot, \cdot)$ be a pre-inner product on $V$. Then: 
    \[ ||x|| = (x, x)^{\frac12}\]
    is a semi-norm on $V$ which is a norm exactly when $(\cdot, \cdot)$ is an inner product. \\
    Proof. The homogeneity is clear. Use Cauchy Schwarz to see: 
    \[ ||x+y||^2 \leq (||x|| + ||y||)^2\]
    \item[Defn.] A vector space $H$ with an inner product is called a Hilbert space if the normed vector space $(H, \norm{})$ is complete with respect to $||x|| = (x, x)^{\frac12}$. 
    \item[Ex.] (1) If $H = \mathbb F^n$ and
    \[ (x, y) = \sum_{k=1}^n x_k \overline{y}_k\]
    Here,
    \[ ||x|| = ||x||_2\]
    (2) Let $H = l^2$. Now, 
    \[ (x, y) = \sum_{n=1}^\infty x_n\overline{y_n}\]
    Again, 
    \[ ||x|| = ||x||_2\]
    (3) Let $H = L^2(x, \mathcal M, \mu)$ and:
    \[ (f, g) = \int_X ( f(x)\overline{g(x)} d\mu(x))\]
    \item[Thm.] (Parallelogram Law). If $V$ is a pre-inner product space, then for all $x, y \in H$, 
    \[ ||x+y||^2 + ||x-y||^2 = 2||x||^2 + 2||y||^2\]
    \item[Thm.] (Jordan von Neumann Thm.) Let $X$ be a normed vector space where norm satisfies the parallelogram law. Then $X$ admits an inner product such that $||x|| = (x, x)^{\frac12}$.
    \item[Prop.] If $H$ is an inner product space and if $x_n \to x$ and $y_n \to y$ in $H$. Then, 
    \[ (x_n, y_n) \to (x, y)\]
    Proof. Since $H$ is a normed vector space, the norm is continous. Thus, 
    \[||x_n|| \to ||x|| \qquad ||y_n|| \to ||y||\]
    Now, using Cauchy-Schwarz,
    \begin{align*}
        |(x_n, y_n) - (x, y)| &\leq |(x_n-x, y_n)| + |(x, y_n - y)| \\
        &\leq ||x_n-x|| \cdot ||y_n|| + ||x|| \cdot ||y_n - y|| \\
        &\to 0
    \end{align*}
    \item[Defn.] Let $H$ be an inner product space. We say $x \in H$ is orthogonal to $g \in H$ or just $x \perp y$ if $(x,y) = 0$. We say $E \subset H$ is orthogonal to $F \subset H$ if $\forall x \in E, y\in H$, $x \perp y$. We would write $E \perp F$. 
    \item[Thm.] (Pythagorean Theorem). Suppose $H$ is an inner product space and that $x_1, \cdots, x_n \in H$ are pairwise orthogonal. Then, 
    \[ ||\sum_{k=1}^n x_k||^2 = \sum_{k=1}^n ||x_k|| \] 
    Proof. Simple. 
    \item[Lemm.] Let $C$ be a nonempty closed, convex subset of a Hilbert space $H$. Then $\forall y \in H$, $\exists ! x \in C$ such that: 
    \[ d(y, C) = ||y-x|| \]
    Proof. We can replace $C$ by $C-\{y\}$ and assume $y = 0$. \\
    Let $\alpha = \inf\{||x|| : x \in C\} = d(0, C)$. \\
    Let $(x_n) \subset C$ be a sequence such that $||x_n|| \to \alpha$. \\
    Then, 
    \begin{align*}
        2(||x_n||^2 + ||x_m||^2) &= ||x_n+x_m||^2 + ||x_n-x_m||^2 \\
        &= 4\norm{\frac{x_n+x_m}{2}}^2 + ||x_n-x_n||^2 \\
        &\geq 4\alpha^2 + ||x_n-x_m||^2
    \end{align*}
    But LHS convers to $4\alpha^2$. Thus, $(x_n)$ is Cauchy. Then, 
    \[ x_n \to x \qquad ||x|| = \alpha\]
    Suppose $y \in C$ and $||y|| = \alpha$. Then,
    \begin{align*}
        4\alpha^2 &= 2(||x||^2 + ||y||^2) \\
        &= ||x+y||^2 + ||x-y||^2 \\
        &\geq 4\alpha^2 + ||x-y||^2 
    \end{align*}
    Thus, $||x-y||^2 = 0$, thus $x = y$.
    \newpage 
    Missed a lecture. 5/16.
    \newpage
    \item[Thm.] (Bessel's Inequality). $H$ is an inner product space $\{e_n : n \in \N\}$ be an orthonormal subset. Then $\forall x \in H$: 
    \[ ||x||^2 \geq \sum_{n=1}^\infty |(x\mid e_n)|^2\]
    \item[Corr.] Let $\{e_n : n \in \N\}$ be an orthonomal subset of a Hilber space $H$ and let:
    \[\calE = \overline{\opspan\{e_n : n \in \N\}}\] 
    Then, for all $h \in H$, 
    \[ \sum_{n=1}^\infty (h\mid e_n)e_n\]
    converges in $H$ and: 
    \[ P_\calE (h) = \sum_{n=1}^\infty (h\mid e_n)e_n\]
    Proof. Pictures. 
    \item[Defn.] An orthonormal subset $\{e_\alpha\}_{\alpha \in A}$ is a basis for $H$ if $\opspan\{e_\alpha : \alpha \in A\}$ is dense. 
    \item[Prop.] Every Hilbert space $H$ has an orthonormal basis. \\
    Proof. Let $\Lambda$ be a collection of orthonormal subsets index by inclusion. Then $\Lambda$ is inductively ordered. Then Zorn's lemma gives us a maximal element in $\Lambda$: 
    \[ \{e_\alpha : \alpha \in A\}\]
    If $w = \opspan\{e_\alpha\} \subset H$ is proper, then $\exists e_0 \in W^\perp$ with $\norm{e_0} = 1$. This is a contradiction to maximality. 
    \item[Corr.] If $H$ is a separable $\infty$-dimensional Hilbert space, then every orthonormal basis for $H$ is countably infinite. \\
    Proof. Let $\{e)\alpha : \alpha \in A\}$ be an orthonormal subset of $H$. Then the balls: 
    \[ B_{1/\sqrt2}(e_n)\]
    are pairwise disjoint. Hence there exists at most countably many $e_n$ in $H$.
    \item[Thm.] Let $\{e_n ; n \in \N\}$ be an orthonormal subset of a Hilbert space. Then the following are equivalent:
    \begin{itemize}
        \item This is a basis (i.e. it has a dense linear span).
        \item $(h|e_n) = 0$ for all $n \in \N$ implies $h = 0$.
        \item $\forall h \in H$, $h = \sum_{n=1}^\infty (h|e_n)e_n$.
        \item For all $h, k \in H$, $(h|k) = \sum_{n=1}^\infty (h|e_n)(e_n|k)$.
        \item For all $h \in H$, $||h||^2 = \sum_{n=1}^\infty |(h|e_n)|^2$.
    \end{itemize}
    Proof. (1) $\implies$ (2). Let $w_0 = \opspan\{e_n\}$. Then, 
    \[ w_0^{\perp\perp} = \overline{w_0} = H \]
    Hence, 
    \[ w_0^\perp = \overline{w_0}^\perp = \{0\}\]
    (2) $\implies$ (1). If $w_0^\perp = \{0\}$, then: 
    \[ \overline{w_0} = w_0^{\perp\perp} = H\]
    (1) $\implies$ (3). Apply the $P_\calE$ corollary with $\calE = H$. \\
    (3) $\implies$ (4). $\phi_k(\cdot) = (\cdot \mid k)$ is continous. \\
    (4) $\implies$ (5). Let $k = h$. \\
    (5) $\implies$ (1). Suppose $w_0 = \overline{\opspan\{e_n\}} \neq H$. Then, $\exists h \in H \setminus \{0\}$ such that $(h|e_n) = 0$ for all $n \in \N$. \\
    Missed one line. 
    \item[Thm.] Let $H$ be a separable $\infty$-dimensional Hilbert space. Then, there is an isometric linear isomorphism. 
    \[ U: H \to l^2\]
    such that: 
    \[ (U(h) \mid U(k))_{l^2} = (h\mid k)_H\] 
    Proof. Let $\{e_n : n \in \N\}$ be an orthonormal basis for $H$. Let $H_0 = \opspan\{e_n : n \in \N\}$. Define: 
    \begin{align*}
        U_0: H_0 \to l^2 
    \end{align*}
    by $U_0\left(\sum_{k=1}^n \alpha_k e_k\right) = (\alpha_1, \cdots, \alpha_n, 0, 0, \cdots)$. \\
    Since $\{e_n\}$ is a Hamel basis for $H_0$, $U_0$ is well-defined and isometric by Parseval. Since $H_0$ is dense in $H$ and $U_0(H_0) = c_{0,0}$ (seqs that vanish after some point) is dense in $l^2$. Thus, $U_0$ extends to an isometrity of $H$ into $l^2$ with dense range. Since $U$ is an isometry, $U(H)$ is complete. Hence its closed and $U$ is a bijection. \\
    We want to show that this preserves the inner product. Let: 
    \[ [x, y] = (U(x) \mid U(y))_{l^2}\]
    is an inner product on $H$. Then, 
    \begin{align*}
        (U x \mid U y) &= \frac14 \sum_{k=0}^3 i^k ||U(x + i^k y)||^2 \\
        &= \frac14 \sum_{k=0}^3 i^k ||x + i^k y||^2 \\
        &= (x \mid y)
    \end{align*}
    If $\mathbb F = \R$, use: 
    \begin{align*}
        (U x \mid U y) &= \frac14 (||U(x)+U(y)||^2 + ||U(x)-U(y)||^2)
    \end{align*}
    to get the same. 
    \item[Ex.] Let $H = \F^n$. If $A \in M_n(\F)$, we set: 
    \[ L_A \in \mathcal L(\F)\]
    by $L_A(h) = Ah$. where:
    \[ (Ah)_k = \sum_{j=1}^n a_{kj}h_j\]
    where: 
    \[ a_{ij} = (T(e_j)\mid e_i)\]
    Now if $A = M_n(\F)$, let $A^* = (b_{ij})$, where 
    \[ b_{ij} = \overline{a_{ji}}\]
    then, we have:
    \begin{align*}
        (L_A(h) \mid k) &= (Ah \mid k) \\
        &= (h \mid A^*k) \\
        &= (h \mid L_{A^*}(k))
    \end{align*}
    Gien $T \in \mathcal L(\F^n)$. Then, $\exists! T^* \in \mathcal L(\F^n)$ such that:
    \[ (T(h) \mid k) = (h \mid T^*(k))\]
    \item[Thmn.] Let $H$ be a Hilbert space and $T \in \mathcal L(H)$. Then $\exists! T^* \in \mathcal L (H)$ such that for all $h, k \in H$: 
    \[ (T(h) \mid k) - (h \mid T^*(k)) \]
    Furthermore, 
    \begin{itemize}
        \item $||T^*|| = ||T||$
        \item $(T+\lambda S)^* = T^* + \overline{\lambda} S^*$
        \item $T^{**} = T$
        \item $(TS)* = S^*T^*$
        \item $\norm{T^*T} = ||T||^2$
    \end{itemize}
    Proof. Clearly, if $T^*$ exists, it is unique. For existence, fix $k \in H$. Then, 
    \[ h \mapsto (T(h) \mid k)\]
    is linear and 
    \[ |(T(h)\mid k)| \leq ||T|| \cdot ||k|| \cdot ||h||\]
    Thus, this is bounded in norm by $||T|| \cdot ||k||$. Thus this is a functional in $H^*$. Thus, $\exists! z(k)$ such that $||z(k)|| \leq ||T||||k||$ and: 
    \[ (T(h) \mid k) = (h \mid z(k)) \qquad \forall h \in H\]
    Now, check that: 
    \[ z(k+\lambda k') = z(k) + \lambda z(k') \] 
    and we can define: 
    \[ T^*(k) = z(k) \qquad ||z(k)|| \leq ||T||||k||\]
    for all $k \in H$. Thus, $||T^*|| \leq ||T||$. \bbni
    Since $T^* \in \mathcal L (H)$, hence we have $T^{**} \in \mathcal L(H)$ and: 
    \[ (T^*(h) \mid k) = (h \mid T^**(k))\]
    But:
    \[ (T^*(h) \mid k) = (h \mid T(k))\]
    since this holds for all $h$ and $k$, we have that: $T = T^{**}$. \\
    Thus, $||T|| = ||T^{**}|| \leq ||T^*||$. Thus, $||T^*|| = ||T||$. \bbni 
    Next, note: 
    \[ (h \mid (TS)^*(k)) = (TS(h) \mid k) = (h \mid S^*T^*(k)) \]
    Thus, $T^*S^* = (TS)^*$. \\
    Finally, 
    \begin{align*}
        ||T(h)||^2 &= (T(h) \mid T(h))  \\
        &= (T^*T(h) \mid h) \\
        &\leq ||T^*T|| ||h||^2 \\
    \end{align*}
    Thus, $||T||^2 \leq ||T^*T|| \leq ||T^*||||T|| = ||T||^2$. 
    \item[Ex.] $H = \R^2$. Let $T \in \mathcal{L}(\R^2)$ be given by: 
    \[ T(x,y) = (-y, x)\]
    Note that $(T(h) \mid h) = 0$ for all $h \in \R^2$.
    \item[Prop.] Let $H$ be a complex Hilbert space. If $T \in \mathcal{L}(H)$ and: 
    \[ (T(h) \mid h) = 0 \qquad \forall h \in H\]
    then $T = 0$. \\
    Proof. Let $[x, y] = (T(x) \mid y)$. Then $[\cdot, \cdot]$ is a sesquilinear form on $H$. \\ 
    Thus, we have the polarization identity: 
    \begin{align*}
        (T(x) \mid y) &= [x, y] \\
        &= \frac 14 \sum_{k=0}^3 i^k (T(x + i^k y) \mid x + i^k y) \\
        &= 0
    \end{align*}
    Thus, $T = 0$. 
    \item[Defn.] We say that $T \in \mathcal{L}(H)$ is self-adjoint if $T = T^*$.
    \item[Ex.] If $H = \F^n$ and $T \in \mathcal L(\F^n)$ has a standard matrix $A = (a_{ij})$ where: 
    \[ (a_{ij}) = (T(e_j) \mid e_i)\]
    Then, $T = T^*$ if and only if $A = A^*$.
    \item[Cor.] If $H$ is a complex Hilbert space, then $T \in \mathcal L(h)$ is self-adjoint if and only if: 
    \[ (T(x) \mid x) \in \R\]
    for all $x \in H$. \\
    Proof. If $T = T^*$, then: 
    \begin{align*}
        (T(x) \mid x) &= (x \mid T^*(x)) \\
        &= (x \mid T(x)) \\
        &= \overline{(T(x) \mid x)}
    \end{align*}
    Thus, $(T(x) \mid x) \in \R$. \\
    Conversely, if $(T(x) \mid x) \in \R$, for all $x \in H$, then:
    \begin{align*}
        (T(x) \mid x) &= \overline{(x \mid T(x))} \\
        &= (x \mid T(x)) \\
        &= (T^*(x) \mid x)
    \end{align*}
    Thus, $((T-T^*)(x) \mid x) = 0$ for all $x \in H$. Thus, $T = T^*$. 
    \item[Prop.] Let $H$ be a Hilbert space and $T \in \mathcal L(H)$, then: 
    \[ \ker(T^*) = T(H)^\perp\]
    Proof. Let $y \in \ker(T^*)$. Then, $\forall x \in H$, 
    \[(T(x) \mid y) = (x \mid T^*(y)) = 0\]
    Thus, $y \in T(H)^\perp$ and $\ker(T^*) \subseteq T(H)^\perp$. \\
    Conversely, if $y \in T(H)^\perp$, then $\forall x \in H$, 
    \[(T^*(y) \mid x) = (y, T(x)) = 0\]
    Thus, $T^*(y) = 0$ and $y \in \ker(T^*)$.
    \item[Defn.] If $T: H \to H$ is linear, we that that $T$ is bounded away from $0$ if $\exists \alpha > 0$ such that: 
    \[ ||T(x)|| \geq \alpha ||x||\]
    \item[Ex.] If $T^{-1} \in \mathcal L(H)$, then: 
    \[ ||x|| = ||T^{-1}T(x)|| \leq ||T^{-1}||||T(x)||\]
    Thus, $T$ is bounded away from $0$ with $\alpha = \frac{1}{||T^{-1}||}$.
    \item[Prop.] Suppose that $H$ is a Hilbert space and that $T \in \mathcal L(H)$. Then, the following are equivalent: 
    \begin{enumerate}
        \item $T$ is invertible.
        \item $T^*$ is invertible.
        \item Both $T$ and $T^*$ are bounded away from $0$.
        \item $T$ and $T^*$ are injective and $T(H)$ is closed. 
        \item $T$ is a bijection. 
        \item $T$ and $T^*$ are both surjective.
    \end{enumerate}
    Proof. (1) $\iff$ (2). We have $I = TT^{-1} = T^{-1}T$. Then, 
    we have $I = I^* = (T^*)(T^{-1})^* = (T^{-1})^*T^*$. Thus, $T^*$ is invertible. \\
    $(1) + (2) \implies (3)$ because invertible operators are bounded away from $0$. \\
    $(3) \implies (4)$. Problem 35 on the homework. \\
    $(4) \implies (5)$. We know that: 
    \begin{align*}
        T(H) &= \overline{T(H)} \\
        &= T(H)^{\perp\perp} \\
        &= \ker(T^*)^\perp \\
        &= \{0\}^\perp \\
        &= H
    \end{align*}
    $(5) \implies (1)$. By the Open Mapping Theorem. \\
    $(6) \implies (5)$. $\ker(T) = T^*(H)^\perp = H^\perp = \{0\}$.
    $(1) + (2) \implies (6)$.
    \item[Defn.] We say that $T \in \mathcal L(H)$ is normal if $TT^* = T^*T$. 
    \item[Lemm.] Suppose $T \in \mathcal L(H)$ is normal. Then $||T(x)|| = ||T^*(x)||$ for all $x \in H$.
    \newpage
    Missed two to three lectures of notes.
    \newpage
    \item[Rmk.] We say that $T \in \mathcal{L}(H)$ is (orthgonally) diagonalizable if there is an orthonormal basis $\{e_n\}$ of $H$ consisting of eigenvectors of $T$. \item[Rmk.] If $\{e_n : n \in \N\}$ is a basis of eigenvectors for $T$ and if $T(e_n) = \lambda_n e_n$ then $(\lambda_n) \subset l^\infty$.
    \item[Prop.] Suppose that $T \in \mathcal{L}(H)$ diagonaziable with $\{e_n : n \in \N\}$ and $T(e_n) = \lambda_n e_n$. Then $T \in \mathcal{K}(H)$ (compact) if and only if $(\lambda_n) \in c_0$. \bbni
    Proof. We showed that if $T \in \mathcal{K}(H)$, then $(\lambda_n) \in c_0$. Now suppose $(\lambda_n) \in c_0$. Let:
    \[ T_n = \sum_{k=1}^n \lambda_k \Theta_{e_k, e_k}\]
    Then, $T_n \in \mathcal{L}_{\calF}(H)$. \\
    If $\norm{x} \leq 1$, 
    \begin{align*}
        \norm{(T-T_n)(x)}^2 &= \norm{\sum_{k=n+1}^\infty \lambda_k (x \mid e_k)e_k}^2 \\
        &= \lim_{N \to \infty } \norm{\sum_{k=n+1}^N \lambda_k (x, e_k)}^2 \\
        &= \lim_{n \to \infty} \sum_{k=n+1}^N |\lambda_k|^2 |(x, e_k)|^2 \\
        &\leq \sup_{k \geq n} |\lambda_k|^2 ||x||^2 \\
        &\leq \sup_{k \geq n} |\lambda_k|^2
    \end{align*}
    Thus, $\norm{T-T_n} \leq \sup_{k \geq n} |\lambda_k|^2$. Thus, $T_n \to T$ in the operator norm. Thus, $T \in \mathcal{K}(H)$ since $(T_n) \subset \mathcal{L}_{\calF}(H)$.
    \item[Defn.] If $T \in \mathcal{L}(H)$ then the numerical range of $T$ is: 
    \[ n(T) = \sup\{|(Tx \mid x)| : ||x|| \leq 1\}\]
    \item[Ex.] Let $\F = \R$ and:
    \[ A = \begin{pmatrix}
        0 & -1 \\
        1 & 0
    \end{pmatrix}\]
    Since $A^T = A$, $L_A$ is normal, even unitary. But we know that $A$ has no eigenvalues. Hence, $L_A$ is not diagonalizable. Note that: 
    \[ n(L_A) = 0 \]
    \item[Yap.] Recall that $T \geq 0$ if and only if $T = A^*A$. But then, 
    \begin{align*}
        |(Tx \mid x)| &= (Ax \mid Ax) = ||Ax||^2
    \end{align*}
    Thus, $n(T) = ||A||^2$. Thus, $n(T) = \norm{A^*A} = \norm{T}$.
    \item[Prop.] Suppose $T \in \mathcal{L}(H)$ and $T = T^*$. Then $n(T) = ||T||$. \bbni
    Proof. We have: 
    \[ n(T) = \leq \norm{T}\]
    by Cauchy-Schwarz. Thus, by homogeneity, we have:
    \[ (Tx \mid x) \leq n(T)||x||^2 \qquad \qquad (*)\]
    Now, we recall, 
    \begin{align*}
        (T(x+y) \mid x+y) - (T(x-y) \mid x-y) &= 2((Tx \mid y) + (Ty \mid x)) \\
        &= 4\Re((Tx \mid y)) \\
    \end{align*}
    Now using $(*)$, we have:
    \begin{align*}
        4\Re((Tx \mid y)) &\leq n(T)(||x+y||^2 + ||x-y||^2) \\
        &= n(T)(2||x||^2 + 2||y||^2)
    \end{align*}
    Thus, if $||x|| = ||y|| = 1$, then:
    \[ \Re(Tx \mid y) \leq n(T)\]
    Now replacing $y$ by $Ty$ with $|T| = 1$, we have:
    \[ |(Tx \mid y)| \leq n(T) \]
    Missing. 
    \item[Rmk.] If $\F = \C$ and $T \in \mathcal{L}(H)$ is normal, then $n(T) = ||T||$. 
    \item[Lem.] (Cauchy-Schwarz Rev.) If $x, y \in H \setminus \{0\}$ and: 
    \begin{align*}
        |(x \mid y)| &= \norm{x}\norm{y} 
    \end{align*} 
    Then $\exists \lambda \neq 0$ such that $x = \lambda y$. \\
    Proof. Let $\norm{x} = \norm{y} = 1$. Let $\lambda = 1$ be such that:
    \[ \overline{\lambda}(x \mid y) = |(x \mid y)|^2 = \norm{x}\norm{y} = 1\]
    Then, 
    \begin{align*}
        \norm{x-\lambda y}^2 &= (x - \lambda y \mid x - \lambda y) \\
        &= \norm{x}^2 -2 \Re(\overline{\lambda}(x \mid y)) + |\lambda|^2 \norm{y}^2 \\
        &= 1 - 2  + 1 = 0
    \end{align*}
    \item[Prop.] Suppose $T \in \mathcal{K}(H)$ and $T = T^*$. Then, $T$ has an eigenvalue $\lambda$ such that $|\lambda| = ||T||$. \bbni
    Proof. Recall that since $T$ is compact, $T|_B: B \to H$ is weak-norm continous. Thus, if $x_\lambda \to x$ weakly then $T(x_\lambda) \to T(x)$ in norm.  \\
    Then, we have:
    \begin{align*}
        |(Tx_\lambda \mid x_\lambda) -(Tx \mid x)| &\leq |(Tx_\lambda - T(x) \mid x_\lambda)| + |(Tx \mid x_\lambda - x)| \\
        &\leq \norm{T(x_\lambda)-T(x)} + |(Tx \mid x_\lambda)-(T(x)\mid x)|
    \end{align*}
    By weak-norm continuity, this goes to $0$ as $\lambda \to \infty$. Thus, 
    \[ x \mapsto |(T(x) \mid x)|\]
    is weakly continous on $B$. Since $B$ is weakly compact, $\exists x_0 \in B$ such that:
    \[ |(T(x_0) \mid x_0)| = \sup_{\norm{x} \leq 1} |(T(x) \mid x)| = n(T) = ||T||\]
    Then, 
    \begin{align*}
        \norm{T} &= |(T(x_0) \mid x_0)| \\
        &\leq \norm{Tx_0}\norm{x_0} \\
        &\leq \norm{T}
    \end{align*} 
    Thus, $|(Tx_0 \mid x_0)| = \norm{Tx_0}\norm{x_0}$. Then, there exists $\exists \lambda$ such that $T(x_0) = \lambda x_0$. Clearly, $|\lambda| = ||T||$.
    \item[Rmk.] If $T \in \mathcal{L}(H)$ is normal, then $v \neq 0$ is an eigenvector with eigenvalue $\lambda$ if and only if:
    \[\norm{(T-\lambda I)(v)} = 0\]
    But $T-\lambda I$ is normal is $T$ is. Hence, 
    \[\norm{(T^*-\overline{\lambda}I)(v)} = 0 \]
    and $v$ is an eigenvectof for $T^*$ with eigenvalue $\overline{\lambda}$.
    \item[Thm.] (Spectral Theorem for compact operators). Suppose that $H$ is a separable Hilbert space and that $T = T^*$ is compact. Then $T$ is orthogonally diagonalizable. \bbni
    Proof. Pictures. 06/02.
    \newpage
    \item[Defn.] Let $A$ be a unital complex Banach algebrai with identity $I$. Thus, the spectrum of $a \in A$ is: 
    \[ \sigma(a) = \{\lambda \in \C : a - \lambda I \text{ is not invertible }\}\]
    \item[Ex.] If $\dim(H) < \infty$, and $T \in \mathcal{L}(H)$, then $\sigma(T)$ is the set of eigenvalues of $T$. The same is true if $G$ is a normal compact operator on a seperable complex Hilbert space $H$. 
    \item[Thm.] For all $ a\in A$, $\sigma(a)$ is a nonempty closed and bounded subset of $\C$. In fact, 
    \[ \rho(a) = \sup\{|\lambda| : \lambda \in \sigma(a)\} = \lim_{n\to\infty} \norm{a^n}^{1/n} = \inf_{n\geq 1} \norm{a^n}^{1/n}\]
    Hence, $\rho(a) \leq \norm{a}$.
    If $A$ is a unital commutative Banach algebra then we can let $\Delta = \Delta(A)$ be the set of nonzero complex homomorphisms
    \[ h: A \to \C\]
    \item[Thm.] Let $\Delta = \Delta(A)$ as above then $\Delta \neq \emptyset$ and: 
    \begin{itemize}
        \item $\norm{h} = 1$ for all $h \in \Delta$.
        \item $\lambda \in \sigma(a) \iff h(a) = \lambda$ for some $h \in \Delta$.
    \end{itemize}
    \item[Rmk.] Note that $\Delta \subset A^*$. In fact, $\Delta$ is closed in $A^*$. Thus, $\Delta$ is compact in the weak-* topology. If $A$ is seperable, then $\Delta$ is metrizable. 
    \item[Rmk.] Let $A$ and $\Delta$ be as above. Then if $a \in A$, we can define: 
    \[ \hat a: \Delta \to \C\]
    by $\hat a(h) = h(a)$. Then $\hat a$ is called the Gelfand transform and $\hat a \in C(\Delta)$. Note $\hat a(\Delta) = \sigma(a)$.\\
    This gives us a map: 
    \[ \Phi: A \to C(A)\]
    given $\Phi(a) = \hat a$. This is an algebra homomorphism.  
\end{itemize}




\end{document}